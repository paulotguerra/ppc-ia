\apendice{Ementário e bibliografias}
\label{ap:ementario_bibliografia}

\newcommand{\DetalhesDisciplina}[4]{
\begin{table}[h!]
    \centering
    \small
    \renewcommand{\arraystretch}{1.5}
    \begin{tabular}{|>{\columncolor{gray!15}}p{2.8cm}|p{11.2cm}|}
        \hline        
        \textbf{Componente} & \textbf{#1} \\
        \hline        
        \textbf{Ementa} & #2 \\
        \hline        
        \textbf{Bibliografia \newline Básica} & #3 \\
        \hline        
        \textbf{Bibliografia Complementar} & #4 \\
        \hline
    \end{tabular}
\end{table}
\clearpage
}

\DetalhesDisciplina{
    <03> - Agentes Inteligentes
}{% EMENTA
    Conceitos fundamentais de agentes inteligentes. Arquitetura, Racionalidade. Natureza dos ambientes: totalmente observável, parcialmente observável, determinístico, estocástico, episódico, sequencial, estático, dinâmico, discreto, contínuo, único agente, multiagentes. Programas de agentes: agentes reflexivos simples, agentes baseados em modelos, agentes baseados em objetivos, agentes baseados em utilidade e agentes que aprendem. Introdução a sistemas multiagentes.
}{% BIBLIOGRAFIA BÁSICA
    RUSSELL, Stuart J.; NORVIG, Peter. Inteligência Artificial: Uma Abordagem Moderna. 4. ed. Rio de Janeiro: GEN LTC, 2022. E-book. p.Capa. ISBN 9788595159495. 
\newline \newline 
WOOLDRIDGE, M. Introduction to Multiagent Systems. Wiley, 2009. ISBN: 9780470519462
\newline \newline 
COPPIN, B Inteligência artificial. LTC, 2010. ISBN: 9788521617297
}{% BIBLIOGRAFIA COMPLEMENTAR
    SILVA, Fabrício M.; LENZ, Maikon L.; FREITAS, Pedro H C.; et al. Inteligência artificial. Porto Alegre: SAGAH, 2018. E-book. p.1. ISBN 9788595029392. Disponível em: https://app.minhabiblioteca.com.br/reader/books/9788595029392/. Acesso em: 22 mai. 2025.
\newline \newline 
BRAKTO, I.Prolog Programming for Artificial Intelligence. 4 ed. Addison Wesley, 2011. ISBN 0321417461
\newline \newline 
SHOHAM, Y. Multiagent systems: algorithms, game theoretic. Cambridge University, 2009. ISBN: 9780521899437
\newline \newline 
KAUFMAN, Dora. Desmistificando a inteligência artificial. São Paulo: Autêntica Editora, 2022. E-book. p.1. ISBN 9786559281596. Disponível em: https://app.minhabiblioteca.com.br/reader/books/9786559281596/. Acesso em: 30 mai. 2025.
\newline \newline 
LIMA, Isaías. Inteligência Artificial. Rio de Janeiro: GEN LTC, 2014. E-book. p.Capa. ISBN 9788595152724. Disponível em: https://app.minhabiblioteca.com.br/reader/books/9788595152724/. Acesso em: 22 mai. 2025.
}


\DetalhesDisciplina{
    QXD0116 - Álgebra Linear
}{% EMENTA
    Matrizes. Sistemas de equações lineares. Vetores. Produto interno. Dependência e independência linear. Espaços vetoriais. Bases de espaços vetoriais. Transformações lineares. Autovalores e autovetores. Diagonalização de operadores.
}{% BIBLIOGRAFIA BÁSICA
    VUJICIC, Milan; SANDERSON, Jeffrey SPRINGERLINK (ONLINE SERVICE). Linear Algebra Thoroughly
Explained. Springer e-books Berlin, Heidelberg: Springer-Verlag Berlin Heidelberg, 2008. ISBN
9783540746393
\newline \newline 
CABRAL, Marco Aurélio Palumbo; GOLDFELD, Paulo. Curso de álgebra linear: fundamentos e aplicações. 3.
ed. Rio de Janeiro: UFRJ, 2012. 257 p. Disponível em: 94 [recurso eletrônico]
\newline \newline 
LIPSCHUTZ, Seymour; LIPSON, Marc; Álgebra Linear; Coleção Schaum; Editora Bookman; 2011.E-book.
Disponível em: https://integrada.minhabiblioteca.com.br/\#/books/9788540700413. Acesso em: 12 de set.
2022
}{% BIBLIOGRAFIA COMPLEMENTAR
    BOLDRINI, José Luiz; Algebra Linear; 3. Edição; Editora Harbra
\newline \newline 
STEINBRUCH, Alfredo; Introduçao à Algebra Linear; Makron Books Editora; 1990
\newline \newline 
IEZZI, GELSON et al; Fundamentos de matemática elementar v.4: Sequências, matrizes, determinantes e
sistemas, 8 ed., 2004, Saraiva
\newline \newline 
ROMAN, Steven SPRINGERLINK (ONLINE SERVICE). Advanced Linear Algebra. Springer e-books Third
Edition. New York, NY: Springer Science+Business Media, LLC, 2008. (Graduate Texts in Mathematics, 135)
ISBN 9780387728315
\newline \newline 
ESPINOSA, Isabel Cristina de Oliveira Navarro; BISCOLLA, Laura Maria da Cunha Canto Oliva; BARBIERI
FILHO, Plinio. Álgebra linear para computação. Rio de Janeiro, RJ: LTC; 2007. x, 286p. (Fundamentos de
informática). ISBN 9788521615521 (broch.)
\newline \newline 
ANTON, Howard; RORRES, Chris. Álgebra Linear com Aplicações. São Paulo: Bookman, 2012. E-book.
Disponível em: https://integrada.minhabiblioteca.com.br/\#/books/9788540701700. Acesso em: 3 de out.
2022
}


\DetalhesDisciplina{
    <21> - Análise Exploratória de Dados
}{% EMENTA
    Tipos e organização de dados. Coleta e tratamento de Dados. Representação gráfica e tabular de dados qualitativos. Medidas descritivas de dados quantitativos. Representação gráfica de dados quantitativos. Associação entre variáveis qualitativas.Associação entre uma variável quantitativa e qualitativa. Associações entre variáveis quantitativas.Ferramentas e linguagens para análise exploratória de dados.

}{% BIBLIOGRAFIA BÁSICA
    LARSON, Ron; FARBER, Elizabeth. Estatística aplicada. 4. ed. São Paulo: Pearson Prentice Hall, 2010.
xiv,637 p. ISBN 978-85-7605-372-9
\newline \newline 
TRIOLA, Mario F.. Introdução à estatística. 10. ed. Rio de Janeiro: LTC, c2008. xxvi, 696 p. ISBN 978-85-
216-1586-6.
\newline \newline 
TRIOLA, Mario F.; FLORES, Vera Regina Lima de Farias e. Introdução à estatística: atualização da tecnologia. 11. ed. Rio de Janeiro: LTC, c2013. xxviii ; 707 p. ISBN 97788521622060
}{% BIBLIOGRAFIA COMPLEMENTAR
    BEIGHLEY, Lynn. Use a cabeça!: SQL. Rio de Janeiro: Alta Books, 2008. xxxiv, 454 p. (Use a cabeça).
ISBN 9788576082101.
\newline \newline 
BARRY, Paul. Use a cabeça: Python. Rio de Janeiro: Alta Books, 2012. xxxiv, 457 p. (Use a cabeça). ISBN 978-85-7608-743-4.
\newline \newline 
IEZZI, Gelson; HAZZAN, Samuel; DEGENSZAJN, David Mauro. Fundamentos de matemática elementar 11 : matemática comercial, matemática financeira, estatística descritiva . São Paulo, SP: Atual, 2004. 232 p.
ISBN 8535704620 (broch.)
\newline \newline 
FERREIRA, Rafael G C.; MIRANDA, Leandro B. A de; PINTO, Rafael A.; et al. Preparação e Análise Exploratória de Dados. Porto Alegre: SAGAH, 2021. E-book. p.Capa. ISBN 9786556902890. Disponível em: https://app.minhabiblioteca.com.br/reader/books/9786556902890/ . Acesso em: 03 jun. 2025.
\newline \newline 
CARVALHO, André C. P. L. F de; MENEZES, Angelo G.; BONIDIA, Robson P. Ciência de Dados - Fundamentos e Aplicações. Rio de Janeiro: LTC, 2024. E-book. p.Capa. ISBN 9788521638766. Disponível em: https://app.minhabiblioteca.com.br/reader/books/9788521638766/ . Acesso em: 30 mai. 2025.
\newline \newline 
BARBETTA, Pedro A.; REIS, Marcelo M.; BORNIA, Antonio C. Estatística para Cursos de Engenharia, Computação e Ciência de Dados. 4. ed. Rio de Janeiro: LTC, 2024. E-book. p.29. ISBN 9788521638827. Disponível em: https://app.minhabiblioteca.com.br/reader/books/9788521638827/ . Acesso em: 03 jun. 2025.
}


\DetalhesDisciplina{
    <12> - Aprendizado por Reforço
}{% EMENTA
    Fundamentos do aprendizado por reforço. Conceitos básicos: agente, ambiente, estado, ação, recompensa e política. Processos de Decisão de Markov (MDPs) e funções de valor e política. Programação dinâmica através da avaliação de política e iteração de política. Métodos de Monte Carlo. Temporal Difference Learning. Algoritmos Q-Learning e SARSA. Aproximação de funções de valor. Aprendizado por reforço profundo (Deep RL) com Deep Q-Networks (DQN) e variações. Métodos de Policy Gradient com estimação de valor. Actor-Critic Methods (A2C, A3C e PPO). Exploração versus explotação. Aprendizado por imitação. Aprendizado por reforço multi-agente. Aplicações em jogos, robótica, sistemas autônomos e otimização de estratégias.
}{% BIBLIOGRAFIA BÁSICA
    RUSSELL, Stuart J.; NORVIG, Peter. Inteligência Artificial: Uma Abordagem Moderna. 4. ed. Rio de Janeiro: GEN LTC, 2022. E-book. p.Capa. ISBN 9788595159495. Disponível em: https://app.minhabiblioteca.com.br/reader/books/9788595159495/. 
\newline \newline 
SEJNOWSKI, Terrence. A Revoluçao do Aprendizado Profundo. Rio de Janeiro: Editora Alta Books, 2020. E-book. p.176. ISBN 9788550814353. Disponível em: https://app.minhabiblioteca.com.br/reader/books/9788550814353/.
\newline \newline 
WANG, Xu; WANG, Sen; LIANG, Xingxing; ZHAO, Dawei; HUANG, Jincai; XU, Xin; DAI, Bin; MIAO, Qiguang. Deep reinforcement learning: A survey. IEEE Transactions on Neural Networks and Learning Systems, v. 35, n. 4, p. 5064–5078, 2024. DOI: 10.1109/TNNLS.2022.3207346. Disponível em: https://ieeexplore.ieee.org/document/9904958.

}{% BIBLIOGRAFIA COMPLEMENTAR
    HARIOM, Tatsat,; SAHIL, Puri,; BRAD, Lookabaugh,. Blueprints de aprendizado de máquina e ciência de dados para finanças: desenvolvendo desde estratégias de trades até robôs Advisors com Python. Rio de Janeiro: Editora Alta Books, 2024. E-book. p.i. ISBN 9788550821726. Disponível em: https://app.minhabiblioteca.com.br/reader/books/9788550821726/.
\newline \newline 
NAEEM, Muddasar; RIZVI, Syed Tahir Hussain; CORONATO, Antonio. A gentle introduction to reinforcement learning and its application in different fields. IEEE Access, New York, v. 8, p. 209 320–209 344, 17 nov. 2020. DOI: 10.1109/ACCESS.2020.3038605. Disponível em: https://ieeexplore.ieee.org/document/9261348. 
\newline \newline 
HENDERSON, P.; ISLAM, R.; BACHMAN, P.; PINEAU, J.; PRECUP, D.; MEGER, D. Deep reinforcement learning that matters. Proceedings of the AAAI Conference on Artificial Intelligence, v. 32, n. 1, 29 abr. 2018. DOI: 10.1609/aaai.v32i1.11694. Disponível em: https://ojs.aaai.org/index.php/AAAI/article/view/11694.
}


\DetalhesDisciplina{
    <09> - Aprendizado Profundo
}{% EMENTA
    Redes neurais artificiais. Aprendizado profundo: perceptron, redes multicamadas (MLPs) e arquiteturas feedforward. Backpropagation e técnicas de otimização. Funções de ativação. Técnicas de regularização (dropout, batch normalization e data augmentation). Redes neurais convolucionais (CNNs), redes neurais recorrentes (RNNs, LSTM, GRU). Arquiteturas avançadas como ResNet, autoencoders, redes generativas (GANs, VAEs) e transformadores com mecanismos de atenção. Transfer learning, fine-tuning e uso de redes pré-treinadas. Aplicações em visão computacional, processamento de linguagem natural (NLP) e outras áreas do aprendizado profundo.
}{% BIBLIOGRAFIA BÁSICA
    FACELI, Katti; LORENA, Ana C.; GAMA, João; AL, et. Inteligência Artificial - Uma Abordagem de Aprendizado de Máquina. 2. ed. Rio de Janeiro: LTC, 2021. E-book. p.Capa. ISBN 9788521637509. Disponível em: https://app.minhabiblioteca.com.br/reader/books/9788521637509/.
\newline \newline 
RUSSELL, Stuart J.; NORVIG, Peter. Inteligência Artificial: Uma Abordagem Moderna. 4. ed. Rio de Janeiro: GEN LTC, 2022. E-book. p.Capa. ISBN 9788595159495. Disponível em: https://app.minhabiblioteca.com.br/reader/books/9788595159495/.
\newline \newline 
SEJNOWSKI, Terrence. A Revoluçao do Aprendizado Profundo. Rio de Janeiro: Editora Alta Books, 2020. E-book. p.157. ISBN 9788550814353. Disponível em: https://app.minhabiblioteca.com.br/reader/books/9788550814353/. 
}{% BIBLIOGRAFIA COMPLEMENTAR
    HARIOM, Tatsat,; SAHIL, Puri,; BRAD, Lookabaugh,. Blueprints de aprendizado de máquina e ciência de dados para finanças: desenvolvendo desde estratégias de trades até robôs Advisors com Python. Rio de Janeiro: Editora Alta Books, 2024. E-book. p.i. ISBN 9788550821726. Disponível em: https://app.minhabiblioteca.com.br/reader/books/9788550821726/.
\newline \newline 
LECUN, Yann; BENGIO, Yoshua; HINTON, Geoffrey. Deep learning. Nature, Londres, v. 521, n. 7553, p. 436–444, 1 maio 2015. DOI: 10.1038/nature14539. Disponível em: https://doi.org/10.1038/nature14539. 
\newline \newline 
JANIESCH, Christian; ZSCHECH, Patrick; HEINRICH, Kai. Machine learning and deep learning. Electronic Markets, v. 31, n. 3, p. 685–695, 1 set. 2021. DOI: 10.1007/s12525-021-00475-2. Disponível em: https://doi.org/10.1007/s12525-021-00475-2. 
}


\DetalhesDisciplina{
    <06> - Aprendizado Supervisionado
}{% EMENTA
    Princípios teóricos e práticos do aprendizado supervisionado. Classificação e Regressão. Modelos lineares e não lineares, k-NN, regressão linear, regressão polinomial, regressão logística, regularização (Ridge, Lasso, Elastic Net), SVM, Naive Bayes, árvores de decisão, métodos de ensemble (Random Forest, Gradient Boosting), e redes neurais. Pipeline de modelagem: pré-processamento, particionamento de dados, Métricas de avaliação e validação cruzada, e ajuste de hiperparâmetros. Tratamento de dados desbalanceados e overfitting. Seleção e engenharia de features. Interpretabilidade de modelos. Aplicações práticas de aprendizado supervisionado.
}{% BIBLIOGRAFIA BÁSICA
    FACELI, Katti; LORENA, Ana C.; GAMA, João; AL, et. Inteligência Artificial - Uma Abordagem de Aprendizado de Máquina. 2. ed. Rio de Janeiro: LTC, 2021. E-book. p.Capa. ISBN 9788521637509. Disponível em: https://app.minhabiblioteca.com.br/reader/books/9788521637509/.
\newline \newline 
COPPIN, Ben. Inteligência Artificial. São Paulo: LTC, 2010. E-book. Disponível em: https://integrada.minhabiblioteca.com.br/\#/books/978-85-216-2936-8.
\newline \newline 
NORVIG, Peter. Inteligência Artificial. São Paulo: GEN LTC, 2013. E-book. Disponível em: https://integrada.minhabiblioteca.com.br/\#/books/9788595156104. 
}{% BIBLIOGRAFIA COMPLEMENTAR
    HARIOM, Tatsat,; SAHIL, Puri,; BRAD, Lookabaugh,. Blueprints de aprendizado de máquina e ciência de dados para finanças: desenvolvendo desde estratégias de trades até robôs Advisors com Python. Rio de Janeiro: Editora Alta Books, 2024. E-book. p.i. ISBN 9788550821726. Disponível em: https://app.minhabiblioteca.com.br/reader/books/9788550821726/.
\newline \newline 
HAYKIN, Simon. Redes neurais princípios e prática.. São Paulo: Bookman, 2001. E-book. Disponível em: https://integrada.minhabiblioteca.com.br/\#/books/9788577800865.
\newline \newline 
LIMA, Isaías. Inteligência Artificial. Rio de Janeiro: GEN LTC, 2014. E-book. p.Capa. ISBN 9788595152724. Disponível em: https://app.minhabiblioteca.com.br/reader/books/9788595152724/. 
}


\DetalhesDisciplina{
    QXD0005 - Arquitetura de Computadores
}{% EMENTA
    Sistemas numéricos. Aritmética binária: ponto fixo e ponto flutuante.
Organização de computadores: memórias, unidade central de processamento,
unidades de entrada e unidades de saída. Linguagens de montagem. Modos
de endereçamento, conjunto de instruções. Mecanismos de interrupção e de
exceção. Barramento, comunicações, interfaces e periféricos. Organização de
memória. Memória auxiliar. Arquiteturas RISC e CISC. Pipeline. Paralelismo de
baixa granularidade. Multiprocessadores. Processadoressuperescalares 
Multicomputadores. Arquiteturas e superpipeline paralelas e não convencionais.
}{% BIBLIOGRAFIA BÁSICA
    STALLINGS,
William; VIEIRA, Daniel. Arquitetura e Organização de
Computadores. São Paulo: Pearson Prentice Hall, 2010. xiv, 624 p. ISBN
9788576055648
\newline \newline 
TANENBAUM, A. Organização Estruturada de Computadores. Pearson
Prentice Hall. São Paulo. 2007. 5ª Ed ISBN 8576050676
\newline \newline 
ENGLANDER, Irv. A Arquitetura de Hardware Computacional, Software de
Sistema e Computação e Comunicação em Rede. 4 Edição. 2011. Editora
LTC. ISBN: 9788521617914
\newline \newline 
MURDOCCA, Miles J.; HEURING, Vincent P., Introdução à arquitetura de
computadores. Rio de Janeiro: ​Campus,​ 2000. 512 p. ISBN 8535206841
}{% BIBLIOGRAFIA COMPLEMENTAR
    DELGADO, José; RIBEIRO, Carlos. Arquitetura de computadores. 4. ed. rev.
atual. FCA, 2010. ISBN 9789727226665
\newline \newline 
HENNESSY, John L.; PATTERSON, David A. Arquitetura de computadores:
uma abordagem quantitativa. 4. ed. Rio de Janeiro: Elsevier, 2008.
\newline \newline 
WEBER, Raul Fernando. Fundamentos de arquitetura de computadores. 3. ed.
Porto Alegre, RS: Sagra Luzzatto, 2008.
\newline \newline 
FEDELI, Ricardo Daniel; POLLONI, Enrico Giulio Franco; PERES, Fernando
Eduardo. Introdução à ciência da computação. 2. ed. atual. Cengage Learning,
2010. 250 p. ISBN 139788522108459.
\newline \newline 
MONTEIRO, Mario. A. Introdução à Organização de Computadores. 5. ed. Rio
de Janeiro, RJ: LTC, 2007. ISBN 9788521615439.
\newline \newline 
MACHADO, Francis B; MAIA, Luiz Paulo. Arquitetura de sistemas operacionais. 4. ed. Rio de Janeiro: LTC, 2007. 308 p.
}


\DetalhesDisciplina{
    <07> - Busca e Otimização
}{% EMENTA
    Fundamentos da resolução de problemas por meio de busca. Formulação de problemas: espaço de estados, funções de custo, ações aplicáveis e testes de objetivo. Algoritmos de busca não informada: Busca em largura, Busca em profundidade, Busca de custo uniforme e Busca em profundidade iterativa. Princípios de Busca Heurística. Algoritmos de busca informada (heurística): Busca gulosa, A* (A estrela). Busca Local: Subida de encosta (Hill Climbing), Têmpera Simulada (Simulated Annealing), Busca em Feixe Local, Algoritmos genéticos. Problemas de Satisfação de restrições (CSPs).
}{% BIBLIOGRAFIA BÁSICA
    RUSSELL, Stuart J.; NORVIG, Peter. Inteligência Artificial: Uma Abordagem Moderna. 4. ed. Rio de Janeiro: GEN LTC, 2022. E-book. p.Capa. ISBN 9788595159495. 
\newline \newline 
CORMEN, Thomas H.; LEISERSON, Charles E.; Ronald L. Rivest; et al. Algoritmos. 4. ed. Rio de Janeiro: GEN LTC, 2024. E-book. p.Capa. ISBN 9788595159914. 
\newline \newline 
COPPIN, B Inteligência artificial. LTC, 2010. ISBN: 9788521617297
}{% BIBLIOGRAFIA COMPLEMENTAR
    KLEINBERG, Jon; TARDOS, Éva. Algorithm design. Boston, Massachusetts: Pearson/Addison Wesley, c2006. 838 p. ISBN 0321295358.
\newline \newline 
TOSCANI, Laira V.; VELOSO, Paulo A. S. Complexidade de algoritmos: análise, projeto e métodos. 3. ed. Porto Alegre: Sagra Luzzato, 2012. 262 p. (Serie Livros Didáticos Informática UFRGS ; 13). ISBN 9788540701380 (broch.).
\newline \newline 
ZIVIANI, Nivio; BOTELHO, Fabiano Cupertino. Projeto de algoritmos: com implementações em java e C++. São Paulo, SP: Thomson Learning, 2007. vii, 620 p. ISBN 8522105251.
\newline \newline 
DASGUPTA, Sanjoy; PAPADIMITRIOU, Christos H.; VAZIRANI, Umesh. Algoritmos . São Paulo: McGraw-Hill, c2009. xiv, 320 p. ISBN 9788577260324.
\newline \newline 
HILLIER, FREDERICK S.; LIEBERMAN, GERALD J. Introdução a Pesquisa Operacional. MCGRAW HILL. 9a edição. (ISBN: 8580551188)
}


\DetalhesDisciplina{
    QXD0006 - Cálculo Diferencial e Integral I
}{% EMENTA
    Derivada; Aplicações; Integrais.
}{% BIBLIOGRAFIA BÁSICA
    IEZZI, Gelson; MURAKAMI, Carlos; MACHADO, Nílson José. Fundamentos de matemática elementar: 8 :limites, derivadas, noções de integral . 6. ed. São Paulo, SP: Atual, 2005. 263 p. ISBN 8535705473 (broch.)
\newline \newline 
LEITHOLD, Louis. O Cálculo com geometria analítica. 3. ed. São Paulo: Harbra, c1994. 2 v. ISBN 8529400941v.1 (broch.)
\newline \newline 
GUIDORIZZI, Hamilton Luiz. Um Curso de Cálculo - Vol. 1, 6ª edição. São Paulo: LTC, 2018. E-book. Disponívelem: https://integrada.minhabiblioteca.com.br/\#/books/9788521635574. Acesso em: 3 de out. 2022
}{% BIBLIOGRAFIA COMPLEMENTAR
    DEMANA, Franklin D. et al. Pré-cálculo. São Paulo, SP: Addison-Wesley, 2009. xv, 380 p. ISBN 9788588639379
\newline \newline 
IEZZI, G. Fundamentos de matemática elementar v.10: Geometria Espacial. 6 ed. Atual, 2005
\newline \newline 
IEZZI, G. Fundamentos de matemática elementar v.7: Geometria Analítica.7ed. Atual, 2005
\newline \newline 
GILAT, A.; SUBRAMANIAM, V. Numerical Methods With Matlab. E-Wiley,2007. ISBN: 9780471734406
\newline \newline 
DEMANA, F. D. Precalculus graphical, numerical, algebraic: media update. 7 ed. Addison-Wesley, 2010. ISBN:9780321356932
\newline \newline 
THOMAS, George Brinton; WEIR, Maurice D.; HASS, Joel. Cálculo. 12. ed. São Paulo, SP: Pearson Education doBrasil, 2012. 2 v. ISBN 9788581430867 v. 1 (broch.)
}


\DetalhesDisciplina{
    QXD0134 - Cálculo Diferencial e Integral II
}{% EMENTA
    Derivadas parciais. Funções transcendentes. Funções hiperbólicas. Noções
de coordenadas polares. As técnicas de integração. Integrais impróprias.
Integrais múltiplas. Séries.
}{% BIBLIOGRAFIA BÁSICA
    THOMAS, George Brinton; WEIR, Maurice D.; HASS, Joel. Cálculo.
12. ed. São Paulo, SP: Pearson Education do Brasil, 2012. 2 v. ISBN
9788581430874 v. 2 (broch.)
\newline \newline 
LEITHOLD, Louis. O Cálculo com geometria analítica. 3. ed. São Paulo,
SP: Harbra, c1994. 2 v. 426 p. ISBN 8529402065 v.2 (broch.).
\newline \newline 
GUIDORIZZI, Hamilton Luiz. Um curso de cálculo. 5. ed. Rio de Janeiro:
Livros Técnicos e Científicos, 2002. 4 v. ISBN 9788521612803 v. 2(broch.).
}{% BIBLIOGRAFIA COMPLEMENTAR
    THOMAS, George Brinton; WEIR, Maurice D.; HASS, Joel. Cálculo.
12. ed. São Paulo, SP: Pearson Education do Brasil, 2012. 2 v. ISBN
9788581430867 v. 1 (broch.)
\newline \newline 
GUIDORIZZI, Hamilton Luiz. Um curso de cálculo. 5. ed. Rio de Janeiro:
Livros Técnicos e Científicos, c2002. v.3
\newline \newline 
BORTOLOSSI, Humberto. Cálculo diferencial a várias variáveis: uma
introdução à teoria de otimização . 5. ed. Rio de Janeiro, RJ: Ed. PUCRio; São Paulo, SP: Loyola, 2011
\newline \newline 
GUIDORIZZI, Hamilton Luiz. Um curso de cálculo. 5. ed. Rio de Janeiro:
Livros Técnicos e Científicos, 2002. 4 v. ISBN 9788521612599 v. 1(broch.)
\newline \newline 
LEITHOLD, L. O cálculo com geometria analítica v.1, 3ed. Harbra, 1994
}


\DetalhesDisciplina{
    QXD0135 - Cálculo Diferencial e Integral III
}{% EMENTA
    Estudo de vetores, retas e planos no R3. Funções de várias variáveis.
Diferenciação. Integrais múltiplas. Campos vetoriais. Jacobiano. Integrais
sobre curvas e superfícies. Teoremas de Green, Gauss e Stokes.
}{% BIBLIOGRAFIA BÁSICA
    THOMAS, George Brinton; WEIR, Maurice D.; HASS, Joel. Cálculo.
12. ed. São Paulo, SP: Pearson Education do Brasil, 2012. 2 v. ISBN
9788581430874 v. 2 (broch.)
\newline \newline 
LEITHOLD, Louis. O Cálculo com geometria analítica. 3. ed. São Paulo,
SP: Harbra, c1994. 2 v. 426 p. ISBN 8529402065 v.2 (broch.).
\newline \newline 
GUIDORIZZI, Hamilton Luiz. Um curso de cálculo. 5. ed. Rio de Janeiro:
Livros Técnicos e Científicos, 2002. 4 v. ISBN 9788521612575 v. 3 (broch.)
}{% BIBLIOGRAFIA COMPLEMENTAR
    GUIDORIZZI, Hamilton Luiz. Um curso de cálculo. 5. ed. Rio de Janeiro:
Livros Técnicos e Científicos, 2002. 4 v. ISBN 9788521612803 v. 2(broch.)
\newline \newline 
THOMAS, George Brinton; WEIR, Maurice D.; HASS, Joel. Cálculo.
12. ed. São Paulo, SP: Pearson Education do Brasil, 2012. 2 v. ISBN 9788581430867 v. 1 (broch.)
\newline \newline 
BOYCE, William E.; DIPRIMA, Richard C. Equações diferenciais elemen-
tares e problemas de valores de contorno. 8. ed. Rio de Janeiro, RJ: Livros
\newline \newline 
Técnicos e Científicos, c2006. xvi, 434 p. ISBN 8521614993 (broch.).
BORTOLOSSI, Humberto. Cálculo diferencial a várias variáveis: uma
introdução à teoria de otimização . 5. ed. Rio de Janeiro, RJ: Ed. PUC-
Rio; São Paulo, SP: Loyola, 2011
\newline \newline 
LEITHOLD, L. O cálculo com geometria analítica v.1, 3ed. Harbra, 1994
}


\DetalhesDisciplina{
    QXD0025 - Compiladores
}{% EMENTA
    Introdução à Compiladores, Análise Léxica, Análise Sintática e Abstração de Sintaxe, Análise Semântica, Tabela de Símbolos e Análise de Escopo;Seleção de Instruções;Análise de Longevidade;Seleção de Registradores;Emissão de Código, Tópicos Especiais em Compiladores.
}{% BIBLIOGRAFIA BÁSICA
    AHO, Alfred V.; LAM, Monica S.; SETHI, Ravi; ULLMAN, Jeffrey D.,
Compiladores: princípios, técnicas e ferramentas, Pearson, 2a Edição, 2007.
\newline \newline 
LOUDEN, Kenneth. Compiladores: princípios e práticas. Thomson Pioneira, 2004.
\newline \newline 
HOPCROFT, John E. Introdução à teoria de autômatos, linguagens e
computação. Rio de Janeiro: Elsevier, 2002. 560 p
}{% BIBLIOGRAFIA COMPLEMENTAR
    DELAMARO, M. E. Como Construir Um Compilador Utilizando Ferramentas Java, 1 ed, 2004 Novatec
\newline \newline 
MENEZES, Paulo Blauth. Linguagens formais e autômatos. 6. ed. Porto
\newline \newline 
Alegre: Bookman, 2011. 215 p. (Livros didáticos ; n.3 Série Livros Didáticos ; 3) ISBN: 9788577807659
\newline \newline 
SEBESTA, Robert W. Conceitos de linguagens de programação. 5. ed.
Porto Alegre: Bookman, 2003. ISBN 8536301716.
FORBELLONE, André Luiz Villar; EBERSPÄCHER, Henri Frederico.
\newline \newline 
Lógica de programação: a construção de algoritmos e estruturas de dados . 3. ed. São Paulo: Makron, 2005. xii, 218 p.
\newline \newline 
SIPSER, Michael. Introdução à teoria da computação. São Paulo: Cengage Learning, c2007.
}


\DetalhesDisciplina{
    QXD0079 - Computação em Nuvem
}{% EMENTA
    Introdução a Computação em Nuvem, Princípios da Computação em Nuvem,
Arquitetura da Computação em Nuvem, Modelos de Serviço: Infraestrutura
como um Serviço, Plataforma como um Serviço e Software como um Serviço,
Gerenciamento de Dados em Nuvem, Middlewares para a Computação em
Nuvem, Gerenciamento e Monitoramento da Nuvem, Migração de Aplicações
para Nuvem.
}{% BIBLIOGRAFIA BÁSICA
    COULOURIS, George F.; DOLLIMORE, Jean; KINDBERG, Tim. Sistemas
distribuídos: conceitos e projeto. 5 ed. Bookman, 2013. 1048 p. ISBN
9788582600535.
\newline \newline 
OZSU, M.Tamer. Principles of distributed database systems. 3nd. ed. New
York: Springer, 2011. ISBN 9781441988331
\newline \newline 
VERAS, M. Arquitetura de Nuvem - Amazon Web Services (AWS). 1a ed.
Brasport, 2013. ISBN 9788574525686 .
}{% BIBLIOGRAFIA COMPLEMENTAR
    JENNINGS, Roger. Cloud computing with the Windows Azure Platform.
Indianapolis, Indiana: Wiley Pub., 2009.
\newline \newline 
WHITE, Tom. Hadoop: the definitive guide. California: O´Reilly, 2009.
\newline \newline 
TANENBAUM, Andrew S.; STEEN, Van Maarten; MARQUES, Arlete
Simille. Sistemas distribuídos: princípios e paradigmas. 2. ed. São Paulo, SP:
Prentice Hall, 2007.
\newline \newline 
JOSUTTIS, Nicolai M. SOA na prática: a arte da modelagem de sistemas
distribuídos. Rio de Janeiro, RJ: Alta Books, 2008.
\newline \newline 
HAY, Chris; PRINCE, Brian H. Azure in action. Stamford, Ct: Manning, 2011.
\newline \newline 
TAURION, Cezar. Cloud Computing: computação em nuvem, transformando o
mundo da Tecnologia da Informação. Rio de Janeiro, RJ: Brasport, 2009.
}


\DetalhesDisciplina{
    QXD0183 - Computação Paralela
}{% EMENTA
    Conceitos de computação paralela. Modelos de computação paralela. Algoritmos paralelos. Ambientes de programação paralela.
}{% BIBLIOGRAFIA BÁSICA
    STALLINGS, William. Arquitetura e organização de computadores. 8.
ed. São Paulo: Pearson Prentice Hall, c2010.
\newline \newline 
TANENBAUM, Andrew S.; BOS, Herbert. Sistemas operacionais modernos. 4. ed. São Paulo: Pearson, 2015.
\newline \newline 
RAUBER, Thomas and Rünger Gudula; Parallel Programming: for Multicore and Cluster Systems; First Edition; Editora Springer; ISBN-10: 364204817X ISBN-13: 978-3642048173
}{% BIBLIOGRAFIA COMPLEMENTAR
    VIANA, Gerardo Valdisio Rodrigues. Meta-heuristicas e programacao paralela em otimizacao combinatoria. Fortaleza: Edições UFC, 1998.
\newline \newline 
NEVES, Julio Cezar. Programação SHELL LINUX. 9. ed. Rio de Janeiro, RJ: Brasport, 2013.
\newline \newline 
TANENBAUM, Andrew S.; STEEN, Van Maarten. Sistemas distribuídos:
princípios e paradigmas. 2. ed.São Paulo, SP: Pearson Education, 2007.
\newline \newline 
SILBERSCHATZ, Abraham; GALVIN, Peter Baer; GAGNE, Greg. Fundamentos de Sistemas Operacionais. São Paulo: LTC, 2015. Ebook. Disponível em: https://integrada.minhabiblioteca.com.br/\#/books/978-85-216-3001-2. Acesso em: 3 de out. 2022.
\newline \newline 
PATTERSON, David A.; HENNESSY, John L.. Arquitetura de computadores: uma abordagem quantitativa. Rio de Janeiro: Elsevier, c2014.
}


\DetalhesDisciplina{
    QXD0153 - Desafios de Programação
}{% EMENTA
    Estruturas de Dados Avançadas. Busca por Padrões. Combinatória. Teoria dos
Números. Backtracking. Algoritmos em Grafos. Programação Dinâmica.
Geometria Computacional.
}{% BIBLIOGRAFIA BÁSICA
    CORMEN,	Thomas	H.	Algoritmos:	teoria	e	prática.	Rio	de	Janeiro:	Elsevier,	2002.	xvii	,	916	p.	ISBN	8535209263	(broch.).	
\newline \newline 
DASGUPTA,	Sanjoy;	PAPADIMITRIOU,	Christos	H.;	VAZIRANI,	Umesh.	
Algoritmos.	São	Paulo:	McGraw-Hill,	c2009.	xiv,	320	p.	ISBN	
9788577260324	(broch.).	
\newline \newline 
KLEINBERG,	Jon;	TARDOS,	Éva.	Algorithm	design.	Boston,	Massachusetts: Pearson/Addison Wesley, c2006. 838 p. ISBN 0321295358 (enc.).
}{% BIBLIOGRAFIA COMPLEMENTAR
    ZIVIANI, Nivio; BOTELHO, Fabiano Cupertino. Projeto de algoritmos: com implementações em java e C++. São Paulo, SP: Thomson Learning, 2007. vii, 620 p. ISBN 8522105251 (broch.).
\newline \newline 
ROSEN, Kenneth H. Matemática discreta e suas aplicações. 6. ed. São
Paulo: McGraw-Hill, c2009. xxi, 982 p. ISBN 9788577260362 (broch.).
\newline \newline 
HALIM, S.; HALIM, F.; Competitive Programming. 1 ed. Ebook. Disponível
em <http://www.comp.nus.edu.sg/~stevenha/myteaching/competitive\_prog
ramming/cp1.pdf>. Acesso
em: 18 de jan de 2016.
\newline \newline 
BOAVENTURA NETTO, Paulo Oswaldo. Grafos: teoria, modelos, algoritmos. 5. ed. rev. e ampl. São Paulo: Edgard Blücher, c2012. xiii, 310
p. ISBN 9788521206804
\newline \newline 
SZWARCFITER, Jayme Luiz; MARKENZON, Lilian. Estruturas de dados e
seus algoritmos. 2. ed. rev. Rio de Janeiro: Livros Técnicos e Científicos,
c1994. 320 p.
\newline \newline 
TOSCANI, Laira V.; VELOSO, Paulo A. S. Complexidade de algoritmos:
análise, projeto e métodos. 3. ed. Porto Alegre: Sagra Luzzato, 2012. 262 p.
(Serie Livros Didáticos Informática UFRGS ; 13). ISBN 9788540701380
}


\DetalhesDisciplina{
    QXD0276 - Desenvolvimento de Software para Dispositivos Móveis
}{% EMENTA
    Visão geral sobre dispositivos móveis: Comparação entre dispositivos de
sensoriamento, celulares, tablets e computadores convencionais; Visão geral sobre
as plataformas de desenvolvimento mais utilizadas, como Android SDK, Iphone
SDK e Windows Mobile. Requisitos e desafios para computação móvel.
Arquitetura de Software Móvel. Comunicação para Software móvel. Middleware
e frameworks para Computação Móvel. Sensibilidade ao contexto e adaptação.
Plataforma Android. Activities e Intents. Interfaces e Layouts. Services.
Localização e Mapas. Sensores disponíveis.
}{% BIBLIOGRAFIA BÁSICA
    SALMRE, Ivo. Writing mobile code: essential software engineering for building
mobile applications. New Jersey: Addison-Wesley, 2005. xviii, 771p. ISBN
9780321269317 (broch.).
\newline \newline 
OLIVEIRA, Diego Bittencourt de; SILVA, Fabrício Machado da; PASSOS,
Ubiratan R C.; et al. Desenvolvimento para dispositivos móveis. [Digite o Local
da Editora]: Grupo A, 2019. E-book. ISBN 9788595029408. Disponível em:
https://app.minhabiblioteca.com.br/\#/books/9788595029408/. Acesso em: 27
mar. 2023.
\newline \newline 
DEITEL, Paul; DEITEL, Harvey; WALD, Alexander. Android 6 para
programadores. [Digite o Local da Editora]: Grupo A, 2016. E-book. ISBN
9788582604120. Disponível em:
https://app.minhabiblioteca.com.br/\#/books/9788582604120/. Acesso em: 27
mar. 2023.
}{% BIBLIOGRAFIA COMPLEMENTAR
    POSLAD, Stefan. Ubiquitous Computing: Smart Devices, Environments and
Interactions. 1 ed, Wiley Publishing, 2009. ISBN13: 9780470035603.
\newline \newline 
FREDERICK, Gail Rahn; LAL, Rajesh. Dominando o desenvolvimento web
para smartphone:construindo aplicativos baseados em JavaScript, CSS, HTML e
Ajax para iPhone, Android, Palm Pre, BlackBerry, Windows Mobile e Nokia
S60. 344 p. Rio de Janeiro: Alta Books, 2011.
\newline \newline 
LECHETA, Ricardo R. Google android: aprenda a criar aplicações para
dispositivos móveis com o Android SDK. 2. ed. rev. ampl. São Paulo, SP:
Novatec, 2010. 608 p. ISBN 9788575222447.
\newline \newline 
SIMAS, Victor L.; BORGES, Olimar T.; COUTO, Júlia M C.; et al.
Desenvolvimento para dispositivos móveis - Volume 2. [Digite o Local da
Editora]: Grupo A, 2019. E-book. ISBN 9788595029774. Disponível em: https://app.minhabiblioteca.com.br/\#/books/9788595029774/. Acesso em: 27
mar. 2023.
\newline \newline 
MORAIS, Myllena Silva de F.; MARTINS, Rafael L.; SANTOS, Marcelo da
Silva dos; et al. Fundamentos de desenvolvimento mobile. Grupo A, 2022. Ebook. ISBN 9786556903057. Disponível em:
https://app.minhabiblioteca.com.br/\#/books/9786556903057/. Acesso em: 27 mar. 2023.
}


\DetalhesDisciplina{
    QXD0253 - Desenvolvimento de Software para Web
}{% EMENTA
    A Internet e protocolos de comunicação; HTML; CSS; Acessibilidade, design responsivo e boas práticas na programação web; JavaScript; Aplicação de JavaScript no contexto do desenvolvimento web no navegador; Chamadas assíncronas com JavaScript.
}{% BIBLIOGRAFIA BÁSICA
    LUCKOW, Décio Heinzelmann; MELO, Alexandre Altair de. Programação Java para a Web. São Paulo, SP: Novatec, 2010. 638 p. ISBN 9788575222386.
\newline \newline 
GEARY, David; HORSTMANN, Cay. Core JavaServer Faces. 3. ed. Rio de Janeiro, RJ: Alta Books, 2012: ISBN: 9788576086420.
\newline \newline 
SILVA, Maurício Samy. HTML 5: a linguagem de marcação que revolucionou a web. 2. ed. São Paulo: Novatec, 2014. 335 p. ISBN 9788575224038.
\newline \newline 
FLANAGAN, David. JAVASCRIPT – O Guia Definitivo. Bookman, 6ª ED./2012, 856583719x/9788565837194.
}{% BIBLIOGRAFIA COMPLEMENTAR
    KURNIAWAN, Budi. Java para a Web com Servlets, JSP e EJB: Budi Kurniawan; tradução Savannah Hartmann; revisão técnica Alfredo Dias da Cunha Júnior. Rio de Janeiro, RJ: Ciência Moderna, 2002. xxiv, 807 p. ISBN 8573932104 (broch.).
\newline \newline 
FREEMAN, Elisabeth; FREEMAN, Eric. Use a cabeça!: HTML com CCS e XHTML. 2. ed. Rio de Janeiro, RJ: Alta Books, 2008. xxxi, 580 p. ISBN 9788576082187 (broch.).
\newline \newline 
URUBATAN, Rodrigo. Ruby on rails: desenvolvimento fácil e rápido de aplicações Web. São Paulo, SP: Novatec, 2009. 285 p. ISBN 9788575221846 (broch.).
\newline \newline 
GONÇALVES, Edson. Desenvolvendo aplicações Web com NetBeans IDE 6. Rio de Janeiro: Ciência Moderna, 2008. 581 p. : CD-ROM ISBN 97885739366742.
\newline \newline 
BASHAM, Bryan. Use a cabeça!: Servlets \& JSP. 2. ed. Rio de Janeiro, RJ: Alta Books, 2008. ISBN 9788576082941.
}


\DetalhesDisciplina{
     - Desenvolvimento de Software para Web 2
}{% EMENTA
    Arquitetura web. HTTP API. Arquitetura REST. Streaming API. Aplicações e Servidores Web. Servidores de Aplicação. Frameworks de front-end e back-end para web.
}{% BIBLIOGRAFIA BÁSICA
    ALMEIDA, Flávio. Mean: Full stack JavaScript para aplicações web com MongoDB, Express,
Angular e Node. São Paulo, SP: Casa do Código, [2015]. xxiii, 361 p. (Caelum). ISBN
9788555190469 (broch.)
\newline \newline 
OLIVEIRA, Cláudio Luís V.; ZANETTI, Humberto Augusto P. Node.js: programe de forma rápida e prática. Rio de Janeiro: Expressa, 2021. E-book. p.15. ISBN 9786558110217. Disponível em: https://app.minhabiblioteca.com.br/reader/books/9786558110217/. Acesso em: 06 nov. 2024.
\newline \newline 
JIN, Brenda; SAHNI, Saurabh; SHEVAT, Amir. Designing Web APIs: Building APIs That Developers Love. Editora O’Reilly Media, Inc., 2018.
}{% BIBLIOGRAFIA COMPLEMENTAR
    URUBATAN, Rodrigo. Ruby on rails: desenvolvimento fácil e rápido de aplicações Web. São Paulo, SP: Novatec, 2009. 285 p. ISBN 9788575221846 (broch.).
\newline \newline 
DUCKETT, Jon. PHP\&MYSQL: desenvolvimento web no lado do servidor. Rio de Janeiro: Editora Alta Books, 2024. E-book. p.173. ISBN 9786555205930. Disponível em: https://app.minhabiblioteca.com.br/reader/books/9786555205930/. Acesso em: 06 nov. 2024.
\newline \newline 
FLANAGAN, David. JAVASCRIPT – O Guia Definitivo. Bookman, 6ª ED./2012, 856583719x/9788565837194.
\newline \newline 
SMITH, Ben. JSON básico: conheça o formato de dados preferido da web. São Paulo: Novatec, 2015. 400 p. ISBN 9788575224366 (broch.).
\newline \newline 
FERRAZ, Reinaldo. Acessibilidade na web: boas práticas para construir sites e aplicações acessíveis. São Paulo: Casa do Código, 2020. 246 p. ISBN 978-65-86110-10-4.
\newline \newline 
ALVES, William P. Java para Web - Desenvolvimento de Aplicações. Rio de Janeiro: Érica, 2015. E-book. p.1. ISBN 9788536519357. Disponível em: https://app.minhabiblioteca.com.br/reader/books/9788536519357/. Acesso em: 16 jan. 2025.
}


\DetalhesDisciplina{
    QXD0232 - Educação Ambiental
}{% EMENTA
    Educação ambiental, conceitos e metodologias. Histórico da educação
ambiental (EA). Conferências e marcos legais da EA. Desenvolvimento
Sustentável. Perspectivas filosóficas do Desenvolvimento Sustentável.
Transdisciplinaridade e Educação Ambiental. A Práxis em Educação
Ambiental.
}{% BIBLIOGRAFIA BÁSICA
    REIGOTA, Marcos. O que é educação ambiental. 2.ed. São Paulo, SP:
Brasiliense, 2012. 107p. (Coleção Primeiros Passos; 292). ISBN
9788511001228 (broch.).
\newline \newline 
ESMERALDO, Gema Galgani Silveira Leite. Educação, v. 5. Fortaleza:
Fundação Demócrito Rocha/Assembléia Legislativa do Estado do Ceará/
Sistema de Transmissão Nordeste S.A, 2015. 52 p. (Convivência com o
semiárido). ISBN 9788575297063 (enc.).
\newline \newline 
OLIVEIRA NETO, João Martins de. Gestão, v. 6. Fortaleza: Fundação
Demócrito Rocha/ Assembléia Legislativa do Estado do Ceará/ Sistema de Transmissão Nordeste S.A, 2015. 50 p. + 1 DVD (Convivência com o semiárido). ISBN 9788575297070 (enc.).
}{% BIBLIOGRAFIA COMPLEMENTAR
    MANSUR, Ricardo. Governança de TI verde: o ouro verde da nova TI . Rio de
Janeiro: Ciência Moderna, 2011. 212 p. ISBN 9788539900459 (broch.).
\newline \newline 
SUPERINTENDÊNCIA ESTADUAL DO MEIO AMBIENTE. Programa de
Educação Ambiental do Ceará : PEACE . Fortaleza: SEMACE, 2009.
\newline \newline 
SUASSUNA, Joao. Água, v. 1. Fortaleza: Fundação Demócrito Rocha/
Assembléia Legislativa do Estado do Ceará/ 2015. 52 p. + 1 DVD
(Convivência com o semiárido). ISBN 9788575297025 (enc.).
\newline \newline 
MARTINS, Eduardo Sávio Passos Rodrigues; OLIVEIRA, Sônia Barreto
Perdigão de; CARVALHO, Margareth Silvia Benicio de Souza. Clima, v.3.
Fortaleza: Fundação Demócrito Rocha/ Assembléia Legislativa do Estado do
Ceará, Sistema de Transmissão Nordeste S.A, 2015.. 52 p. + 1 DVD
(Convivência com o semiárido). ISBN 9788575297049 (enc.).
\newline \newline 
FABRE, Nicolas Arnaud. Produção, v. 4. Fortaleza: Fundação Demócrito
Rocha/ Assembléia Legislativa do Estado do Ceará/ Sistema de Transmissão
Nordeste S.A, 2015.. 52 p. + 1 DVD (Convivência com o semiárido). ISBN
9788575297056 (enc.).
\newline \newline 
SILVA, José Borzacchiello da. Terra, v. 2. Fortaleza: Fundação Demócrito
Rocha/ Assembléia Legislativa do Estado do Ceará, Sistema de Transmissão
Nordeste S.A, 2015. 52 p. + 1 DVD (Convivência com o semiárido). ISBN
9788575297032 (enc.).
}


\DetalhesDisciplina{
    QXD0245 - Educação em Direitos Humanos
}{% EMENTA
    Direitos Humanos, democratização da sociedade, cultura e paz e cidadanias. O nascituro, a criança e o adolescente como sujeitos de direito: perspectiva histórica e legal. O ECA e a rede de proteção integral. Educação
em direitos humanos na escola: princípios orientadores e metodologias. O
direito à educação como direito humano potencializador de outros direitos. Movimentos, instituições e redes em defesa do direito à educação.
Igualdade e diversidade: direitos sexuais, diversidade religiosa e diversidade étnica. Os direitos humanos de crianças e de adolescentes nos meios
de comunicação e nas mídias digitais.
}{% BIBLIOGRAFIA BÁSICA
    MORAES, Alexandre D. Direitos Humanos Fundamentais. [Digite o Local
da Editora]: Grupo GEN, 2021. E-book. ISBN 9788597026825. Disponível
em: https://app.minhabiblioteca.com.br/\#/books/9788597026825.
Acesso em: 23 set. 2022.
\newline \newline 
NETO, Silvio B. Curso de Direitos Humanos. [Digite o Local da Editora]:
Grupo GEN, 2021. E-book. ISBN 9788597028249. Disponível em: https:
//app.minhabiblioteca.com.br/\#/books/9788597028249. Acesso em:
23 set. 2022.
\newline \newline 
SCARANO, Renan Costa V.; DORETO, Daniella T.; ZUFFO, Sílvia; et
al. Direitos humanos e diversidade. [Digite o Local da Editora]: Grupo
A, 2018. E-book. ISBN 9788595028012. Disponível em: https://app.
minhabiblioteca.com.br/\#/books/9788595028012. Acesso em: 23 set.
2022.
}{% BIBLIOGRAFIA COMPLEMENTAR
    BRASIL/SECRETARIA ESPECIAL DE DIREITOS HUMANOS. Estatuto da Criança e do Adolescente (Lei 8069/90). Brasília, 2008. Disponível
em: http://www.planalto.gov.br/ccivil\_03/LEIS/L8069.htm
\newline \newline 
CADERNO DE EDUCAÇÃO EM DIREITOS HUMANOS. Educação em
Direitos Humanos: Diretrizes Nacionais. Brasília, DF: Secretaria Nacional de Promoção e Defesa dos Direitos Humanos, 2013. 76p. Disponível em: https://www.gov.br/mdh/pt-br/navegue-por-temas/educacaoem-direitos-humanos/DiretrizesNacionaisEDH.pdf
\newline \newline 
COMITÊ NACIONAL DE EDUCAÇÃO EM DIREITOS HUMANOS/
SECRETARIA ESPECIAL DOS DIREITOS HUMANOS. Plano Nacional de Educação em Direitos Humanos. Brasília: MEC/MJ/UNESCO, 2009. Disponível em: http://portal.mec.gov.br/docman/2191plano-nacional-pdf/file
\newline \newline 
NOLETO, M.J. Abrindo espaços: educação e cultura para a paz. Brasília:UNESCO, 2004. Disponível em: https://unesdoc.unesco.org/ark:
/48223/pf0000131816
\newline \newline 
BRASIL. Lei Maria da Penha: um avanço no combate à violência contra a
mulher. Brasília, DF: Senado Federal / Secretaria Especial de Editoração
e Publicações, 2007. 26 p.
\newline \newline 
MONDAINI, Marco. Direitos Humanos. [Digite o Local da
Editora]: Grupo Almedina (Portugal), 2020. E-book. ISBN
9788562938368. Disponível em: https://app.minhabiblioteca.com.
br/\#/books/9788562938368. Acesso em: 23 set. 2022.
}


\DetalhesDisciplina{
    QXD0029 - Empreendedorismo
}{% EMENTA
    Conceito de empreendedorismo. A formação da personalidade. O processo
comportamental. Fatores de sucesso, o perfil do empreendedor.
Desenvolvimento de habilidades empreendedoras. Lições e práticas
internacionais. Empreendedorismo no Brasil. Importância das MPEs na
economia. Globalização dos mercados, dos negócios e das oportunidades.
Pesquisas Tecnológicas. Propriedade Intelectual. Transferência de
Tecnologia. Papel da inovação. Ambientes de pré-incubação e incubação de ideias. Incubadoras de empresas. Parques Tecnológicos. Capital de Risco.
Recursos de Fomento. Fontes de Financiamento. Fundos Setoriais.
Programas governamentais. Plano de Negócio. Ferramentas de Plano de
Negócios. Projetos.
}{% BIBLIOGRAFIA BÁSICA
    SALIM, César Simões. Construindo planos de negócios: todos os passos
necessários para planejar e desenvolver negócios de sucesso. 3. ed. rev. e
atual. Rio de Janeiro, RJ: Elsevier, 2005. xiv, 332 p. ISBN 9788535217360
\newline \newline 
DORNELAS, Jose Carlos Assis. Empreendedorismo: transformando ideias
em negócios. 3. ed. rev. atual. Rio de Janeiro: Elsevier, 2008. 232 p. ISBN
9788535232707
\newline \newline 
DOLABELA, Fernando. O segredo de Luísa: uma idéia, uma paixão e um
plano de negócios: como nasce o empreendedor e se cria uma empresa. Rio
de Janeiro: Sextante, 2008. 299 p. ISBN 9788575423387 (broch.).
}{% BIBLIOGRAFIA COMPLEMENTAR
    HISRICH, Roberto D. Empreendedorismo. 7. ed. Porto Alegre, RS: Bookman,
2009. 662 p. ISBN 9788577803460 (broch. ).
\newline \newline 
FARAH, Osvaldo Elias. Empreendedorismo estratégico. São Paulo: Cengage
Learning, 2008. 251 p. ISBN 9788522106080 (broch.).
\newline \newline 
FERRARI, Roberto. Empreendedorismo para computação: criando negócios
em tecnologia. Rio de Janeiro, RJ: Elsevier, 2010. 164 p. ISBN
9788535234176 (broch.).
\newline \newline 
CHIAVENATO, Idalberto. Empreendedorismo: dando asas ao espírito
empreendedor: empreendedorismo e viabilização de novas empresas, um
guia eficiente para iniciar e tocar seu próprio negócio. 3. ed. rev. e atual. São
Paulo, SP: Saraiva, 2008. 281 p. ISBN 9788502067448 (broch.).
\newline \newline 
FRIEDMAN, Thomas L. O mundo é plano: uma breve história do século XXI.
2. ed., rev. e atual. Rio de Janeiro: Objetiva, 2007. 557 p. ISBN
9788573028638.
}


\DetalhesDisciplina{
    QXD0019 - Engenharia de Software
}{% EMENTA
    Visão geral e introdutória dos princípios fundamentais e ético-profissionais
da Engenharia de Software. Introdução às atividades de engenharia de
requisitos; projeto de software; modelos de desenvolvimento; e gerenciamento (qualidade, estimativa de custo, configuração, etc.) na engenharia
de software.
}{% BIBLIOGRAFIA BÁSICA
    SOMMERVILLE, I. Engenharia de software. 9 ed. Addison Wesley, 2011.
ISBN: 9788579361081.
\newline \newline 
PRESSMAN, Roger; MAXIM, Bruce. Engenharia de Software. São
Paulo: AMGH, 2016. E-book. Disponível em: https://integrada.
minhabiblioteca.com.br/\#/books/9788580555349. Acesso em: 3 de
out. 2022.
\newline \newline 
LARMAN, Craig. Utilizando UML e padroes : uma introducao a analise
e ao projeto orientados a objetos. 5. ed. Porto Alegre: Bookman, 2007.
695 p. ISBN 856003152-9
}{% BIBLIOGRAFIA COMPLEMENTAR
    TELES, V. Extreme programming. Novatec. 2004. ISBN: 8575220470
\newline \newline 
MOLINARI, L. Gerência de configuração: técnicas e práticas no desenvolvimento do software. VISUAL BOOKS. ISBN: 8575022105
\newline \newline 
DELAMARO, Marcio. Introdução ao Teste de Software. São Paulo:
GEN LTC, 2016. E-book. Disponível em: https://integrada.
minhabiblioteca.com.br/\#/books/9788595155732. Acesso em: 3 de
out. 2022.
\newline \newline 
KERIEVSKY, J. Refatoração para padrões. Bookman, 2008.
ISBN:9788577802449
\newline \newline 
PEZZÉ, M.; YOUNG, M. Teste e análise de software: processos, princípios
e técnicas. Bookman, 2008. ISBN:9788577802623
\newline \newline 
PILONE, D.; MILES, R. Use a cabeça! desenvolvimento de software.
ALTA BOOKS, 2008.
}


\DetalhesDisciplina{
    QXD0179 - Estatística Multivariada
}{% EMENTA
    Introdução a análise multivariada. Análise dos componentes principais. Análise
de agrupamentos (clusters). Análise de discriminantes. Modelos de regressão.
}{% BIBLIOGRAFIA BÁSICA
    HAIR, Joseph F. et al. Análise multivariada de dados. 6. ed. Porto Alegre:
Bookman, 2009. 688 p. ISBN 9788577804023 (enc.).
\newline \newline 
CORRAR, Luiz J; PAULO, Edilson; DIAS FILHO, José Maria; (Cord.)
FUNDAÇÃO INSTITUTO DE PESQUISAS CONTÁBEIS, ATUARIAIS E
FINANCEIRAS. Análise multivariada: para os cursos de administração, ciências
contábeis e economia. São Paulo, SP: Atlas, 2007. xxiv, 541 p. ISBN
9788522447077 (broch.).
\newline \newline 
IZENMAN , Alan SPRINGERLINK (ONLINE SERVICE). Modern Multivariate
Statistical Techniques : Regression, Classification, and Manifold Learning .
Springer eBooks New York, NY: Springer-Verlag New York, 2008. (Springer
Texts in Statistics,) ISBN 9780387781891. Disponível em:
<http://dx.doi.org/10.1007/978-0-387-78189-1>. Acesso em : 21 set. 2010.
\newline \newline 
MINGOTI, Sueli Aparecida. Análise de dados através de métodos de estatística
multivariada: uma abordagem aplicada. Belo Horizonte: Editora UFMG, 2005.
295p. : (Didática.8) ISBN 857041451X (broch.)
}{% BIBLIOGRAFIA COMPLEMENTAR
    LARSON, Ron; FARBER, Betsy. Estatística aplicada. 4. ed. São Paulo, SP:
Pearson/ Prentice Hall, 2010. xiv,637 p. ISBN 9788576053729 (broch.).
OJA, H.; Multivariate Nonparametric Methods with R. Springer Ebooks New
York: Springer New York. Disponível em
<http://link.springer.com/book/10.1007/978-1-4419-0468-3>. Acesso em 19 de
janeiro de 2016.
\newline \newline 
BARBETTA, Pedro Alberto; REIS, Marcelo Menezes; BORNIA, Antonio Cezar.
Estatística para cursos de engenharia e informática. 3. ed. São Paulo, SP:
Atlas, 2010. 410 p.
\newline \newline 
WALPOLE, Ronald E. Probabilidade e estatística: para engenharia e ciências.
8. ed. São Paulo, SP: Pearson/ Prentice Hall, 2009. xiv, 491 p. ISBN
9788576051992 (broch.).
\newline \newline 
JOHNSON, Richard A.; WICHERN, Dean W. Applied Multivariate Statistical
Analysis. Phi Learning Private Limited, 2010. ISBN-10 8120345878 ISBN-13
9788120345874
\newline \newline 
FÁVERO, Luiz Paulo et al. Análise de dados: modelagem multivariada para
tomada de decisões. Rio de Janeiro, RJ: Elsevier, Campus, 2009. xx, 646 p.
ISBN 9788535230468
\newline \newline 
EVERITT, Brian; HOTHORN, Torsten. An introduction to applied multivariate
analysis with R. 978-3-642-13312-1. Springer Science \& Business Media, 2011. ISBN
Disponível em <http://dx.doi.org/10.1007/978-1-4419-9650-3>
}


\DetalhesDisciplina{
    QXD0010 - Estrutura de Dados
}{% EMENTA
    Noções de análise de algoritmos, Recursividade, Tipos Abstratos de Dados,
Algoritmos de Ordenação, Listas Sequenciais e Encadeadas, Pilhas, Filas,
Árvores.
}{% BIBLIOGRAFIA BÁSICA
    FEOFILOFF, Paulo. . Algoritmos em linguagem C. Rio de Janeiro: Elsevier,
2009. 208p. ISBN 9788535232493 (broch.).
\newline \newline 
DROZDEK, Adam. Estrutura de dados e algoritmos em C++. São Paulo:
Thomson, 2002. 579p ISBN 852210295 (broch.).
\newline \newline 
ZIVIANI, Nivio; BOTELHO, Fabiano Cupertino. Projeto de algoritmos: com
implementações em java e C++. São Paulo, SP: Thomson Learning, 2007. 620
p. ISBN 8522105251 (broch.).
}{% BIBLIOGRAFIA COMPLEMENTAR
    CORMEN, Thomas H. Algoritmos: teoria e prática. Rio de Janeiro: Elsevier,
2002. 916 p. ISBN 8535209263 (broch.)
\newline \newline 
SZWARCFITER, Jayme Luiz; MARKENZON, Lilian. Estruturas de dados e
seus algoritmos. 2. ed. rev. Rio de Janeiro: Livros Técnicos e Científicos,
c1994. 320 p. ISBN 8521610149 (broch.).
\newline \newline 
ASCENCIO, Ana Fernanda Gomes; CAMPOS, Edilene Aparecida Veneruchi
de. Fundamentos da programação de computadores: algoritmos, Pascal,
C/C++ e java. 2. ed. São Paulo, SP: Prentice Hall, 2007. viii, 434 p. ISBN
8576051480 (broch.).
\newline \newline 
AGUILAR, Luis. Fundamentos de programação: algoritmos, estrutura de dados
e objetos. São Paulo, SP: McGraw-Hill, 2008. 690 p. ISBN 9788586804960
(broch.).
\newline \newline 
CELES, Waldemar; CERQUEIRA, Renato; RANGEL, José Lucas. Introdução a
estrutura de dados: com técnicas de programação em C. Rio de Janeiro:
Elsevier, 2004. 294 p. (Campus, Sociedade Brasileira de Computação) ISBN
8535212280 (broch.).
}


\DetalhesDisciplina{
    QXD0115 - Estrutura de Dados Avançada
}{% EMENTA
    Balanceamento de árvores de busca. Filas de prioridade (​heaps)​ . Estruturas de
dados para conjuntos disjuntos (​union-find​). Grafos: representação e
caminhamento. Tabelas ​hash​ e tratamento de colisões.
}{% BIBLIOGRAFIA BÁSICA
    CORMEN, Thomas H. Algoritmos: teoria e prática. Rio de Janeiro: Elsevier,
2002. xvii , 916 p. ISBN: 8535209263
\newline \newline 
DROZDEK, Adam. Estrutura de dados e algoritmos em C++. São Paulo:
Thomson, 2002. 579p.
\newline \newline 
SZWARCFITER, Jayme Luiz; MARKENZON, Lilian. Estruturas de dados e
seus algoritmos. 2. ed. rev. Rio de Janeiro: Livros Técnicos e Científicos,
c1994. 320 p.
\newline \newline 
SEDGEWICK, Robert, WAYNE, Kevin. Algorithms (4th Edition). Addison-Wesley Professional; 4 edition, 2011. ISBN: 032157351X
}{% BIBLIOGRAFIA COMPLEMENTAR
    GOODRICH, Michael T.; TAMASSIA, Roberto. Data structures and algorithms
in Java. 5th ed. New York, NY: J. Wiley \& Sons, 2010. xxii, 714 p. ISBN
9780470383261 (enc.).
\newline \newline 
CELES, Waldemar; CERQUEIRA, Renato; RANGEL, José Lucas. Introdução a
estruturas de dados: com técnicas de programação em C. Rio de Janeiro, RJ:
Elsevier: Campus, 2004. xiv, 294 p. (Editora Campus). ISBN 8535212280
(broch.).
\newline \newline 
DASGUPTA, Sanjoy; PAPADIMITRIOU, Christos H.; VAZIRANI, Umesh.
Algoritmos. São Paulo: McGraw-Hill, c2009. xiv, 320 p. ISBN 9788577260324
(broch.).
\newline \newline 
ROSEN, Kenneth H. Matemática discreta e suas aplicações. 6. ed. São Paulo:
McGraw-Hill, c2009. xxi, 982 p. ISBN 9788577260362 (broch.).
\newline \newline 
MEHLHORN, Kurt; SANDERS, Peter SPRINGERLINK (ONLINE SERVICE).
Algorithms and Data Structures : The Basic Toolbox . Springer eBooks Berlin,
Heidelberg: Springer-Verlag Berlin Heidelberg, 2008. ISBN 9783540779780.
Disponível em : <http://dx.doi.org/10.1007/978-3-540-77978-0>. Acesso em :
21 set. 2010.
\newline \newline 
KARUMANCHI, Narasimha. Data Structures and Algorithms Made Easy.
Createspace Pub, 2011. ISBN 1468108867
}


\DetalhesDisciplina{
    QXD0508 - Ética e Legislação
}{% EMENTA
    Conceituação do direito moral e ética. Direito e garantias constitucionais. Direito do trabalho. Direito comercial. Direito do consumidor. Direito autoral. Direito civil. Ética profissional.
}{% BIBLIOGRAFIA BÁSICA
    BARGER, Robert N. Ética na computação: uma abordagem baseada em casos.
Rio de Janeiro, RJ: LTC, 2011. xiv, 226 p. ISBN 9788521617761 (broch.).
\newline \newline 
GONÇALVES, Victor Hugo P. Marco Civil da Internet Comentado. São Paulo:
Grupo GEN, 2016. E-book. ISBN 9788597009514. Disponíví el em:
https://app.minhabiblioteca.com.br/\#/books/9788597009514/. Acesso em:
06 out. 2022.
\newline \newline 
ALMEIDA, Guilherme Assis D.; CHRISTMANM, Martha O. Ética e Direito: Uma
Perspectiva Integrada, 3ª edição. São Paulo: Grupo GEN, 2009. E-book. ISBN
9788522467150. Disponíví el em:
https://app.minhabiblioteca.com.br/\#/books/9788522467150/. Acesso em:
05 out
}{% BIBLIOGRAFIA COMPLEMENTAR
    DINIZ, Debora, et al. (Org.). Ética em pesquisa: temas globais. Brasílíia, DF:
Letras Livres, Editora Universidade de Brasílíia, 2011. 404 p. ISBN
9788598070209. Disponíví el em:
<http://www.repositoriobib.ufc.br/00003e/00003e82.pdf>.
\newline \newline 
SÁ, Antônio Lopes D. Ética Profissional. São Paulo: Grupo GEN, 2019. E-book.
ISBN 9788597021653. Disponíví el em:
https://app.minhabiblioteca.com.br/\#/books/9788597021653/. Acesso em:
26 set. 2022.
\newline \newline 
LÉVY, Pierre. As tecnologias da inteligência: o futuro do pensamento na era da informática . 2. ed. Rio de Janeiro: Editora 34, 2010. 206 p. (Coleção TRANS). ISBN 9788585490157 (broch.)
\newline \newline 
FLEISHER, Soraya. SCHUCH, Patrice (Orgs.) Ética e regulamentação na
pesquisa antropológica. Brasílíia: LetrasLivres: Editora Universidade de
Brasílíia, 2010. ISBN 978-85-98070-24-7 ISBN 978-85-230-1246-5.
Disponíví el em: <http://www.repositoriobib.ufc.br/00003e/00003e7d.pdf>.
\newline \newline 
MASIERO, Paulo Cesar. Ética em computação. São Paulo, SP: EDUSP, 2008. 213
p. (Acadêmica ; 32). ISBN 9788531405754
}


\DetalhesDisciplina{
    <08> - Ética em Inteligência Artificial
}{% EMENTA
    Responsabilidade legal em algoritmos de IA (vieses, justiça). Privacidade, segurança e vigilância em sistemas inteligentes. Explicabilidade, auditabilidade em algoritmos de IA. Impactos da IA no mercado de trabalho e nas relações sociais. Regulação e governança da IA: Diretrizes éticas e iniciativas internacionais de governança (UNESCO, OCDE, UE). Lei Geral de Proteção de Dados Pessoais (LGPD). Marco legal brasileiro. Estudo de casos em cenários reais. 
}{% BIBLIOGRAFIA BÁSICA
    FEFERBAUM, Marina; SILVA, Alexandre Pacheco da; COELHO, Alexandre Z.; et al. Ética, Governança e Inteligência Artificial. São Paulo: Almedina, 2023. E-book. p.1. ISBN 9786556279145. Disponível em: https://app.minhabiblioteca.com.br/reader/books/9786556279145/. Acesso em: 22 mai. 2025.
\newline \newline 
BLACKMAN, Reid. Máquinas Éticas: seu guia conciso para uma IA totalmente imparcial, transparente e respeitosa. Rio de Janeiro: Editora Alta Books, 2024. E-book. p.1. ISBN 9788550822426. Disponível em: https://app.minhabiblioteca.com.br/reader/books/9788550822426/. Acesso em: 23 mai. 2025.
\newline \newline 
RUSSELL, Stuart J.; NORVIG, Peter. Inteligência Artificial: Uma Abordagem Moderna. 4. ed. Rio de Janeiro: GEN LTC, 2022. E-book. p.Capa. ISBN 9788595159495. 
}{% BIBLIOGRAFIA COMPLEMENTAR
    PINTO, Rodrigo Alexandre L.; NOGUEIRA, Jozelia. Inteligência Artificial e Desafios Jurídicos: Limites Éticos e Legais. São Paulo: Almedina, 2023. E-book. p.1. ISBN 9786556279268. Disponível em: https://app.minhabiblioteca.com.br/reader/books/9786556279268/. Acesso em: 23 mai. 2025.
\newline \newline 
UNESCO. Recomendação sobre a Ética da Inteligência Artificial. Paris: UNESCO, 2022. Disponível em: https://unesdoc.unesco.org/ark:/48223/pf0000381137\_por. Acesso em: 30 maio 2025.
\newline \newline 
STAHL, Bernd Carsten. Artificial Intelligence for a Better Future: An Ecosystem Perspective on the Ethics of AI and Emerging Digital Technologies. Cham: Springer, 2021. E-book. ISBN 978-3-030-69978-9. Disponível em: https://link.springer.com/book/10.1007/978-3-030-69978-9. Acesso em: 30 maio 2025.
\newline \newline 
BARGER, R. N. Ética na computação: UMA ABORDAGEM BASEADA EM CASOS. LTC, 2011. ISBN: 9788521617761
\newline \newline 
STAHL, Bernd Carsten; SCHROEDER, Doris; RODRIGUES, Rowena. Ethics of Artificial Intelligence: Case Studies and Options for Addressing Ethical Challenges. Cham: Springer, 2023. (SpringerBriefs in Research and Innovation Governance). Disponível em: https://doi.org/10.1007/978-3-031-17040-9. Acesso em: 30 maio 2025.
}

\DetalhesDisciplina{
    QXD0011 - Fundamentos de Banco de Dados
}{% EMENTA
    Visão geral do gerenciamento de banco de dados. Arquitetura de um Sistema Gerenciador de Banco de Dados. Modelagem e projeto de banco
de dados: Modelo Entidade-Relacionamento, Modelo Relacional e Projeto
de Bancos de Dados Relacionais. SQL. Projeto Avançado: Restrições de
Integridade e Normalização.
}{% BIBLIOGRAFIA BÁSICA
    SILBERSCHATZ, Abraham. Sistema de Banco de Dados. São Paulo:
GEN LTC, 2020. E-book. Disponível em: https://integrada.
minhabiblioteca.com.br/\#/books/9788595157552. Acesso em: 3 de
out. 2022.
\newline \newline 
ELMASRI, R.; NAVATHE, S. B. Sistemas de banco de dados. 6 ed.
Pearson/Addison-Wesley, 2011. ISBN: 9788579360855
\newline \newline 
HEUSER, Carlos Alberto. Projeto de banco de dados - V4 - UFRGS. São
Paulo: Bookman, 2011. E-book. Disponível em: https://integrada.
minhabiblioteca.com.br/\#/books/9788577804528. Acesso em: 3 de
out. 2022.
}{% BIBLIOGRAFIA COMPLEMENTAR
    RAMAKRISHNAN, Raghu; GEHRKE, Johannes. Sistemas de Gerenciamento de Bancos de Dados. São Paulo: AMGH, 2008. E-book. Disponível em: https://integrada.minhabiblioteca.com.br/\#/books/
\newline \newline 
9788563308771. Acesso em: 3 de out. 2022.
DATE, C.J. Introdução a Sistemas de Banco de Dados. 8 ed. Campus,
2004. ISBN. 9788535212730
\newline \newline 
OLIVEIRA, C.H. SQL: Curso prático. Novatec, 2002. ISBN:9788575220245
BEIGHLEY, Lynn. Use a cabeça! SQL. Alta Books, 2008. ISBN:
9788576022101
\newline \newline 
ALVES, William Pereira. Banco de Dados. São Paulo: Érica, 2014. Ebook. Disponível em: https://integrada.minhabiblioteca.com.br/\#/
books/9788536518961. Acesso em: 27 de set. 2022.
}


\DetalhesDisciplina{
    QXD0001 - Fundamentos de Programação
}{% EMENTA
    Algoritmos, Conceitos Fundamentais de Programação, Expressões, Controles
de Fluxo, Funções e Procedimentos, Ponteiros, Vetores e Matrizes, Cadeias
de Caracteres, Alocação Dinâmica, Tipos Estruturados e Arquivos.
}{% BIBLIOGRAFIA BÁSICA
    MEDINA, Marco; FERTIG, Cristina. Algoritmos e programação: teoria e prática
2ed. Novatec, 2004.ISBN: 9788575220733/857522073X 2. ed. São Paulo, SP:
Novatec, 2006. 384 p. ISBN 857522073X (broch.).
\newline \newline 
ASCENCIO, A. F. G.;CAMPOS, E. A. V. Fundamentos da programação de
computadores: algoritmos, Pascal, C/C++ e Java. 2 ed. Prentice Hall, 2007.
ISBN: 978576051480 3. ed. São Paulo: Pearson Education do Brasil, c2012.
x, 569 p. ISBN 9788564574168 (broch.).
\newline \newline 
CELES, W.; CERQUEIRA, R.; RANGEL, J. L. Introdução à estrutura de dados:
com técnica de programação em C. Elsevier, 2004. ISBN: 8535212280.
\newline \newline 
FORBELLONE, A.L.V.; EBERSPACHER, H.F. Lógica de programação: a
construção de algoritmos. 3 ed. Prentice Hall, 2005
}{% BIBLIOGRAFIA COMPLEMENTAR
    SCHILDT, Herbert. C completo e total. 3 ed., rev. atual. São Paulo:
Pearson/Makron Books, c1997. xx. 827 p + 1 CD-ROM ISBN 8534605955
(broch.).
\newline \newline 
DEITEL, Harvey. M.; C++ como programar. 5 ed. Prentice Hall, 2006. 5. ed.
São Paulo, SP: Pearson/Prentice Hall, 2006. xlii,1163 p. + cd-rom ISBN
8576050560 (broch.).
\newline \newline 
AGUILAR, Luis JOYANNES. Fundamentos de programação: algoritmos,
estruturas de dados e objetos. São Paulo: McGraw-Hill, c2008. xxix, 690 p.
ISBN 9788586804960 (broch.).
\newline \newline 
MONTGOMERY, Eduard. Programando em C: simples \& prático. Rio de
Janeiro, RJ: Alta Books, 2006. 157 p. ISBN 9788576081210 (bronch).
\newline \newline 
FEOFILOFF, PAULO. Algoritmos em linguagem C. Rio de Janeiro : Elsevier,
2009. ISBN : 9788535232493
}


\DetalhesDisciplina{
    QXD0254 - Gerência de Projetos
}{% EMENTA
    Conceitos, terminologia e contexto de gerência de projetos. Ciclo de vida de
produto e projeto. Interessados (stakeholders). Organização de empresas
(funcionais, matriciais e baseadas em projetos). Estratégias para seleção de
projetos. Processos de gerência de projetos. Gerência de escopo. Gerência de
tempo (definição de atividades, sequenciamento de atividades, estimativa de
recursos, estimativa de duração, desenvolvimento de cronograma e controle de
cronograma). Gerência de custos (estimativas, orçamento e controle). Gerência
de qualidade. Gerência de recursos humanos. Gerência de comunicação. Gerência
de riscos. Gerência de aquisições. Gerência de integração (desenvolver carta de
projeto, desenvolver escopo preliminar, desenvolver plano de gerência de projeto,
dirigir e gerenciar a execução de projetos, monitorar e controlar atividades de
projeto, controle de mudanças e fechamento do projeto).
}{% BIBLIOGRAFIA BÁSICA
    RUBIN, Kenneth S. Scrum essencial: um guia prático para o mais popular
processo ágil. [Digite o Local da Editora]: Editora Alta Books, 2017. E-book.
ISBN 9788550804118. Disponível em:
https://app.minhabiblioteca.com.br/\#/books/9788550804118/. Acesso em: 27
mar. 2023.
\newline \newline 
CARVALHO, Marly M. Fundamentos em Gestão de Projetos - Construindo
Competências para Gerenciar Projetos. [Digite o Local da Editora]: Grupo GEN,
2018. E-book. ISBN 9788597018950. Disponível em:
https://app.minhabiblioteca.com.br/\#/books/9788597018950/. Acesso em: 27
mar. 2023.
\newline \newline 
CAMARGO, Robson Alves de; RIBAS, Thomaz. Gestão ágil de projetos. [Digite
o Local da Editora]: Editora Saraiva, 2019. E-book. ISBN 9788553131891.
Disponível em: https://app.minhabiblioteca.com.br/\#/books/9788553131891/.
Acesso em: 27 mar. 2023.
}{% BIBLIOGRAFIA COMPLEMENTAR
    DINSMORE, Paul C.; CABANIS-BREWIN, Jeannette. AMA: manual de
gerenciamento de projetos. Rio de Janeiro, RJ: Brasport, 2009. 498p. ISBN
9788574523237 (broch.).
\newline \newline 
INSTITUTE, Project M. Um guia de conhecimento em gerenciamento de projetos (guia PMBOK®). [Digite o Local da Editora]: Editora Saraiva, 2014. E-book. ISBN 9788502223745. Disponível em:
https://app.minhabiblioteca.com.br/\#/books/9788502223745/. Acesso em: 27 mar. 2023.
\newline \newline 
PHILLIPS, Joseph. Gerência de projetos de tecnologia da informação: no
caminho certo, do início ao fim. Rio de Janeiro, RJ: Elsevier, 2003. 449 p. ISBN 9788535211832 (broch.).
\newline \newline 
VAZQUEZ, Carlos Eduardo; SIMÕES, Guilherme Siqueira.; ALBERT, Renato
Machado. Análise de pontos de função: medição, estimativas e gerenciamento de projetos de software . 10. ed.rev. e ampl. São Paulo, SP: Érica, 2013. 272p. ISBN 9788536504520 (broch.).
\newline \newline 
DINSMORE, Paul Campbell; BARBOSA, Adriane Monteiro Cavalieri. Como se tornar um profissional em gerenciamento de projetos: livro-base de 'Preparação para certificação PMP® - Project management professional'. 4. ed., rev. e ampl. Rio de Janeiro: Qualitimark, 2011. 383p. ISBN 9788573039788 (Broch.).
\newline \newline 
COHN, Mike; SILVA, Aldir José Coelho da. Desenvolvimento de software com scrum: aplicando métodos ágeis com sucesso . Porto Alegre: Bookman, 2011. 496 p. ISBN 9788577808076 (broch.).
}


\DetalhesDisciplina{
    QXD0035 - Inglês Instrumental I
}{% EMENTA
    Vocabulário ligado à Informática. Técnicas de Leitura e Compreensão.
Tradutores. Verbos e tempos verbais. Nomes e Pronomes.
}{% BIBLIOGRAFIA BÁSICA
    GALLO, L. R. Inglês instrumental para informática .Ícone Ed.2008.
\newline \newline 
THE OFFICIAL GUIDE TO the TOEFL test. 4th ed. New York: McGraw-Hill,
2012. 653 p. + 1 CD-ROM ISBN 9780071766586 (broch.).
\newline \newline 
MURPHY, Raymond. English grammar in use: a self-study reference and
practice book for intermediate students. 3rd. ed. Cambridge: Cambridge
University Press, 2007. 379 p. ISBN 052143680X.
\newline \newline 
SWAM, M. Practical English Usage. 3 ed. Oxford do Brasil, 2005. ISBN:
0194420981.
}{% BIBLIOGRAFIA COMPLEMENTAR
    COLLINS. COLLINS: dicionário escolar. Martins Fontes, 2009.
EFRAIM, T. Decision support and business. 9 ed. Prentice Hall, 2010.
\newline \newline 
LONGMAN gramática escolar da língua inglesa: gramática de referências com
exercícios e respostas . São Paulo, SP: Longman, 2004. 317 p. : ISBN
8587214470
\newline \newline 
MACMILLAN EDUCATION (EDITORA). MacMillan English Dictionary for
advanced learners of American English. Macmillan Education. ISBN
9780333966709
\newline \newline 
MARINOTTO, D. Reading on info tech: inglês para informática. 2 ed. Novatec,
2007.
}


\DetalhesDisciplina{
    QXD0036 - Inglês Instrumental II
}{% EMENTA
    Conversação, leitura de artigos e jornais da área. Escrita de trabalhos técnicos.
Apresentação de seminários. Noções avançadas de gramática e compreensão
de texto.
}{% BIBLIOGRAFIA BÁSICA
    GALLO, L. R. Inglês instrumental para informática .Ícone Ed.2008.
\newline \newline 
THE OFFICIAL GUIDE TO the TOEFL test. 4th ed. New York: McGraw-Hill,
2012. 653 p. + 1 CD-ROM ISBN 9780071766586 (broch.).
\newline \newline 
MARINOTTO, D. Reading on info tech: inglês para informática. 2 ed. Novatec,
2007.
}{% BIBLIOGRAFIA COMPLEMENTAR
    COLLINS. COLLINS: dicionário escolar. Martins Fontes, 2009.
\newline \newline 
EFRAIM, T. Decision support and business. 9 ed. Prentice Hall, 2010.
\newline \newline 
LONGMAN gramática escolar da língua inglesa: gramática de referências com
exercícios e respostas . São Paulo, SP: Longman, 2004. 317 p. : ISBN
8587214470
\newline \newline 
MACMILLAN English dictionary for advanced learners. 2 ed. Macmillan
Education, 2007. ISBN: 9780230025455
\newline \newline 
MURPHY, Raymond. English grammar in use: a self-study reference and
practice book for intermediate students. 3rd. ed. Cambridge: Cambridge
University Press, 2007. 379 p. ISBN 052143680X.
}


\DetalhesDisciplina{
    <22> - Inteligência Artificial Explicável
}{% EMENTA
    Necessidade de métodos de IA explicável e seus desafios. Interpretabilidade e explanabilidade. Modelos interpretávels vs. modelos caixa-preta. Explicabilidade pré-modelo. Técnicas de visualização de modelos. Métodos Agnósticos.
}{% BIBLIOGRAFIA BÁSICA
    RUSSELL, Stuart J.; NORVIG, Peter. Inteligência artificial. Rio de Janeiro: Elsevier, Campus, 2013. 988 p. ISBN 9788535237016 (broch.).
\newline \newline 
MOLNAR, Christoph. Interpretable machine learning: a guide for making black box models explainable. 3. ed. [S.l.: s.n.], 2025. Disponível em: https://christophm.github.io/interpretable-ml-book/. Acesso em: 23 maio 2025.
\newline \newline 
COPPIN, Ben. Inteligencia artificial. Rio de Janeiro, RJ: LTC, 2010. 636 p. ISBN 9788521617297.
}{% BIBLIOGRAFIA COMPLEMENTAR
    HOLZINGER, Andreas; GOEBEL, Randy; FONG, Ruth; MOON, Taesup; MÜLLER, Klaus-Robert; SAMEK, Wojciech (Orgs.). XXAI – Beyond Explainable AI: International Workshop, Held in Conjunction with ICML 2020, July 18, 2020, Vienna, Austria, Revised and Extended Papers. Open access. Cham: Springer, ©2022.
\newline \newline 
GENOVESI, Sergio; KAESLING, Katharina; ROBBINS, Scott (Orgs.). Recommender Systems: Legal and Ethical Issues. Open access. Cham: Springer, ©2023. Disponível em: https://link.springer.com/book/10.1007/978-3-031-28241-6. Acesso em: 6 jun. 2025.
\newline \newline 
HORA, Nina da. MyNews Explica Algoritmos. São Paulo: Edições 70, 2023. E-book. p.59. ISBN 9786554271943. Disponível em: https://app.minhabiblioteca.com.br/reader/books/9786554271943/. Acesso em: 06 jun. 2025.
\newline \newline 
BLACKMAN, Reid. Máquinas Éticas: seu guia conciso para uma IA totalmente imparcial, transparente e respeitosa. Rio de Janeiro: Editora Alta Books, 2024. E-book. p.1. ISBN 9788550822426. Disponível em: https://app.minhabiblioteca.com.br/reader/books/9788550822426/. Acesso em: 06 jun. 2025.
\newline \newline 
FEFERBAUM, Marina; SILVA, Alexandre Pacheco da; COELHO, Alexandre Z.; et al. Ética, Governança e Inteligência Artificial. São Paulo: Almedina, 2023. E-book. p.48. ISBN 9786556279145. Disponível em: https://app.minhabiblioteca.com.br/reader/books/9786556279145/. Acesso em: 06 jun. 2025.
}


\DetalhesDisciplina{
    <13> - Inteligência Artificial Generativa
}{% EMENTA
    Fundamentos, arquiteturas e aplicações da Inteligência Artificial Generativa. Redes adversariais generativas (GANs), Autoencoders Variacionais (VAEs) e modelos de difusão. Arquiteturas transformer e modelos de linguagem de grande escala (LLMs). Técnicas de prompt engineering e fine-tuning. Métricas de avaliação de modelos generativos. Implicações éticas e sociais da IA generativa. Criação e avaliação de modelos generativos. Aplicações em geração de imagens, texto, áudio e vídeo. 
}{% BIBLIOGRAFIA BÁSICA
    FEUERRIEGEL, Stefan; HARTMANN, Jonas; JANIESCH, Christian; et al. Generative AI. Business \& Information Systems Engineering, Heidelberg, v. 66, p. 111–126, fev. 2024. Disponível em: https://doi.org/10.1007/s12599-023-00834-7. 
\newline \newline 
MANDUCHI, Laura et al. On the Challenges and Opportunities in Generative AI. 2025. Pré-publicação (preprint) disponível em: arXiv:2403.00025 [cs.LG]. Disponível em: https://arxiv.org/abs/2403.00025.
\newline \newline 
RUSSELL, Stuart J.; NORVIG, Peter. Inteligência Artificial: Uma Abordagem Moderna. 4. ed. Rio de Janeiro: GEN LTC, 2022. E-book. p.Capa. ISBN 9788595159495. Disponível em: https://app.minhabiblioteca.com.br/reader/books/9788595159495/.
}{% BIBLIOGRAFIA COMPLEMENTAR
    HARIOM, Tatsat,; SAHIL, Puri,; BRAD, Lookabaugh,. Blueprints de aprendizado de máquina e ciência de dados para finanças: desenvolvendo desde estratégias de trades até robôs Advisors com Python. Rio de Janeiro: Editora Alta Books, 2024. E-book. p.i. ISBN 9788550821726. Disponível em: https://app.minhabiblioteca.com.br/reader/books/9788550821726/.
\newline \newline 
GUZDIAL, Matthew; SNODGRASS, Sam; SUMMERVILLE, Adam. Generative AI. In: Procedural Content Generation via Machine Learning: An Overview. Cham: Springer Nature Switzerland, 2025. p. 199–218. Disponível em: https://doi.org/10.1007/978-3-031-84756-1\_11.
\newline \newline 
FERRARA, Emilio. GenAI against humanity: nefarious applications of generative artificial intelligence and large language models. Journal of Computational Social Science, v. 7, n. 1, p. 549–569, 1 abr. 2024. DOI: 10.1007/s42001-024-00250-1. Disponível em: https://doi.org/10.1007/s42001-024-00250-1. 
}


\DetalhesDisciplina{
    QXD0256 - Interação Humano-Computador
}{% EMENTA
    Os conceitos de interação e interface humano-computador; Estilos e paradigmas de interação: interfaces gráficas, manipulação direta, ícones e
linguagens visuais. Teorias de IHC: Engenharia cognitiva e Engenharia
semiótica de sistemas interativos; Sistemas de Ajuda e Sistemas de Explicação; Design de Interação: modelagem de interfaces e concretização
do projeto de interface (prototipação de interfaces, ferramentas de apoio
à construção de interfaces); Avaliação de sistemas interativos: métodos
de inspeção, métodos empíricos, testes com usuários, aspectos éticos na
relação com os usuários; Acessibilidade: conceitos, Lei Nacional de Acessibilidade, recomendações W3C para um site acessível, ferramentas de apoio
ao design de sistemas acessíveis, avaliação de acessibilidade.
}{% BIBLIOGRAFIA BÁSICA
    BARBOSA, Simone D. J.; SILVA, Bruno Santana da. Interação humanocomputador. Rio de Janeiro: Elsevier, c2010. 384 p. (Série SBC, Sociedade
\newline \newline 
Brasileira de Computação). ISBN 9788535234183.
\newline \newline 
PREECE, J.; ROGERS,Y. Design de interação: além da interação homemcomputador. Porto Alegre: Bookman, 2005.ISBN: 9788536304946
\newline \newline 
BARRETO, Jeanine dos S.; JR., Paulo A P.; BARBOZA, Fabrício F M.; et al. Interface humano-computador. Grupo A, 2018. E-book. ISBN
9788595027374. Disponível em: https://app.minhabiblioteca.com.
br/\#/books/9788595027374. Acesso em: 28 set. 2022.
}{% BIBLIOGRAFIA COMPLEMENTAR
    BENYON, David. Interação humano-computador. 2. ed. São Paulo:
Pearson Prentice Hall, 2011. 442 p. ISBN 9788579361098
\newline \newline 
SANTA ROSA, José Guilherme; MORAES, Anamaria de. Design participativo: técnicas para inclusão de usuários no processo de ergodesign de
interfaces. Rio de Janeiro: Rio Book’s, 2012. 170 p. ISBN 9788561556167
\newline \newline 
CYBIS, Walter de Abreu; BETIOL, Adriana Holtz.; FAUST, Richard.
Ergonomia e usabilidade: conhecimentos, métodos e aplicações. 3. ed.
São Paulo: Novatec, 2015. 496 p. ISBN: 9788575224595
\newline \newline 
SOBRAL, Wilma S. DESIGN DE INTERFACES - INTRODUÇÃO. Editora Saraiva, 2019. E-book. ISBN 9788536532073. Disponível em: https:
//app.minhabiblioteca.com.br/\#/books/9788536532073. Acesso em:
28 set. 2022.
\newline \newline 
BATISTA, Claudia R.; ULBRICHT, Vania R.; FADEL, Luciane M. Design para acessibilidade e inclusão. Editora Blucher, 2017. E-book. ISBN
9788580393040. Disponível em: https://app.minhabiblioteca.com.
br/\#/books/9788580393040/. Acesso em: 28 set. 2022.
}

\DetalhesDisciplina{
    <02> - Introdução à Inteligência Artificial
}{% EMENTA
    Introdução aos conceitos básicos de Inteligência Artificial (IA): Fundamentos, História da IA, Estado da Arte, Riscos e Benefícios da IA. Visão geral sobre as principais áreas da IA: representação do conhecimento, planejamento automatizado, aprendizado de máquina, comunicação, percepção e atuação. Contextualização acerca da profissão, curso e interdisciplinaridade.
}{% BIBLIOGRAFIA BÁSICA
    RUSSELL, Stuart J.; NORVIG, Peter. Inteligência Artificial: Uma Abordagem Moderna. 4. ed. Rio de Janeiro: GEN LTC, 2022. E-book. p.Capa. ISBN 9788595159495. 
\newline \newline 
COPPIN, B Inteligência artificial. LTC, 2010. ISBN: 9788521617297
\newline \newline 
SHOHAM, Y. Multiagent systems: algorithms, game theoretic. Cambridge University, 2009. ISBN: 9780521899437
}{% BIBLIOGRAFIA COMPLEMENTAR
    KAUFMAN, Dora. Desmistificando a inteligência artificial. São Paulo: Autêntica Editora, 2022. E-book. p.1. ISBN 9786559281596. Disponível em: https://app.minhabiblioteca.com.br/reader/books/9786559281596/. Acesso em: 30 mai. 2025.
\newline \newline 
SANTAELLA, Lucia. A inteligência artificial é inteligente?. São Paulo: Edições 70, 2023. E-book. p.1. ISBN 9786554270588. Disponível em: https://app.minhabiblioteca.com.br/reader/books/9786554270588/. Acesso em: 22 mai. 2025.
\newline \newline 
LIMA, Isaías. Inteligência Artificial. Rio de Janeiro: GEN LTC, 2014. E-book. p.Capa. ISBN 9788595152724. Disponível em: https://app.minhabiblioteca.com.br/reader/books/9788595152724/. Acesso em: 22 mai. 2025.
\newline \newline 
SILVA, Fabrício M.; LENZ, Maikon L.; FREITAS, Pedro H C.; et al. Inteligência artificial. Porto Alegre: SAGAH, 2018. E-book. p.1. ISBN 9788595029392. Disponível em: 
https://app.minhabiblioteca.com.br/reader/books/9788595029392/pageid/0
\newline \newline 
BRACHMAN, R ; LEVESQUE, Hector. Knowledge representation and reasoning. MorganKaufmann, 2004. 
}


\DetalhesDisciplina{
    QXD0113 - Língua Brasileira de Sinais - LIBRAS
}{% EMENTA
    Fundamentos histórico culturais da Libras e suas relações com a educação
dos surdos. Parâmetros e traços linguísticos da Libras. Cultura e identidades surdas. Alfabeto datilológico. Expressões não-manuais. Uso do espaço.
Classificadores. Vocabulário da Libras em contextos diversos. Diálogos em
língua de sinais.
}{% BIBLIOGRAFIA BÁSICA
    QUADROS, Ronice M. de; KARNOPP, Lodenir B.. Língua de sinais brasileira.. São Paulo: ArtMed, 2003. E-book. Disponível em: https://integrada.minhabiblioteca.com.br/\#/books/9788536311746. Acesso
em: 3 de out. 2022.
\newline \newline 
CORRÊA, Ygor; CRUZ, Carina R. Língua Brasileira de Sinais e Tecnologias Digitais. [Digite o Local da Editora]: Grupo A, 2019. E-book.
ISBN 9788584291687. Disponível em: https://app.minhabiblioteca.
com.br/\#/books/9788584291687. Acesso em: 23 set. 2022
\newline \newline 
QUADROS, Ronice Müller de. Educação de surdos: a aquisição da linguagem.. São Paulo: ArtMed, 1997. E-book. Disponível em: https://
integrada.minhabiblioteca.com.br/\#/books/9788536316581. Acesso
em: 3 de out. 2022.
}{% BIBLIOGRAFIA COMPLEMENTAR
    GOES, Maria Cecília Rafael; SMOLKA, Ana Luiza B. A linguagem e o
outro no espaço escolar: Vygotsky e a construção do conhecimento. Campinas: Papirus, 2013.
\newline \newline 
GOLDFELD, Marcia. A Criança Surda: linguagem e cognição numa perspectiva sócio-interacionista. São Paulo: Plexus, 2002.
\newline \newline 
SOUZA, Margarida M. P. Voando com Gaivotas: um estudo das interações
na educação de surdos. Dissertação (Mestrado em Educação Brasileira).
Faculdade de Educação, UFC. 2008. 152 p. [online]
\newline \newline 
COSTA, Márcia Cunha Silva. Educação inclusiva e prática docente : tenho
um aluno surdo em minha sala. E agora? 2013. Dissertação (mestrado em
Educação) -  Universidade Federal do Ceará, Faculdade de Educação, Programa de Pós-Graduação em Educação Brasileira, Fortaleza, 2013. [online]
\newline \newline 
PINHEIRO, Kátia Lucy; LEITAO, Vanda Magalhaes. Práticas pedagógicas bilíngues para crianças do Instituto Cearense de Educação de Surdos.
2012. 164f. Dissertação (mestrado) -  Universidade Federal do Ceará, Programa de Pós-Graduação em Educação Brasileira , Fortaleza, 2012. [online]
}


\DetalhesDisciplina{
    QXD0016 - Linguagens de Programação
}{% EMENTA
    Conceitos básicos de LP: domínios de aplicação, influências no projeto,
paradigmas, métodos de implementação, critérios de avaliação, evolução das
linguagens. Análise léxica e sintática. Variáveis: identificadores, vinculações,
verificação de tipos, escopo. Tipos de dados. Expressões e a declaração de
atribuição. Abstração de processos: subprogramas. Abstração de dados e
orientação a objetos. Noções de programação funcional. Noções de
programação lógica.
}{% BIBLIOGRAFIA BÁSICA
    SEBESTA, R.W. Conceitos de linguagens de programção. 9 ed. Bookman,
2011. ISBN: 9788577807918
\newline \newline 
TUCKER, A. B.; NOONAN, R. Linguagens de programação: princípios e
paradigmas. 2 ed. McGrawHill, 2008. ISBN: 9788577260447
\newline \newline 
CHEN, Yinong; TSAI, Wei-Tek. Introduction to programming languages:
proggramming in C, C++,
Scheme, Prolog, C\#, and SOA. 2nd ed. xii, 383 p.
\newline \newline 
WATT, D.A. Programming language: concepts and paradigms. Prentice Hall,
1990.
}{% BIBLIOGRAFIA COMPLEMENTAR
    DEITEL, H. M. C++ como programar. 5 ed. Prentice Hall, 2006. ISBN:
8576050560
\newline \newline 
AHO, A. V.; SETHI, R.; ULLMAN, J. D. Compiladores: princípios, técnicas e
ferramentas. 2 ed. Pearson/Addison-Wesley, 2008. ISBN: 9788588639249
\newline \newline 
HOPCROFT, J. E.; ULLMAN, J. D.; MOTWANI, R. Introdução à teoria dos
autômatos: linguagens e computação. ​Campus,​ 2002. ISBN:9788535210729
\newline \newline 
BARWISE, J. Language, proof and logic. Seven Bridges, 2002. ISBN:
9781575863740
\newline \newline 
MENEZES, P.B. Linguagens formais e autômatos. 5 ed. Sagra Luzzato, 2008.
ISBN: 9788577807659
\newline \newline 
BARNES, D.J.; KOLLING, M. Programação orientada a objetos com Java: uma
introdução prática usando BLUEJ 4 ed. Prentice Hall Brasil, 2009. ISBN:
9788576051879
\newline \newline 
URUBATAN, R. Ruby on rails: desenvolvimento fácil e rápido. Novatec, 2009.
}


\DetalhesDisciplina{
    QXD0040 - Linguagens Formais e Autômatos
}{% EMENTA
    Introdução. Linguagens, gramáticas e expressões regulares, autômatos finitos. Linguagens e gramáticas livre-do contexto e autômatos de pilha.
Linguagens sensíveis ao contexto. Hierarquia de classes de linguagens. Tópicos especiais e aplicações das linguagens formais e autômatos.
}{% BIBLIOGRAFIA BÁSICA
    HOPCROFT, J. E.; ULLMAN, J. D.; MOTWANI, R., Introdução à teoria
dos autômatos: linguagens e computação, 1 ed, 2002, 10a tiragem Campus
\newline \newline 
MENEZES, P.B. Linguagens formais e autômatos. 5 ed. Sagra Luzzato, 2008. ISBN: 9788577807659
\newline \newline 
RAMOS, M. V.; NETO, J.J.; VEGA, I.S. Linguagens Formais: teoria,
modelagem e implementação. Bookman, 2009. ISBN: 9788577804535
}{% BIBLIOGRAFIA COMPLEMENTAR
    AHO, A. V.; SETHI, R.; ULLMAN, J. D. Compiladores: princípios,
técnicas e ferramentas. 2 ed. Pearson/Addison-Wesley, 2008. ISBN:
9788588639249
\newline \newline 
CARNIELLI, W.; EPSTEIN, R. L. Computabilidade, Funções Computáveis, Lógica e os Fundamentos da Matemática. UNESP, 2009. ISBN:
9788571398979
\newline \newline 
SEBESTA, R.W. Conceitos de linguagens de programação. 5 ed. Bookman, 2003. ISBN: 9788577807918
\newline \newline 
SIPSER, M. Introdução a teoria da computação. 2 ed. Thompson Learning, 2007. ISBN: 9788522104994
\newline \newline 
TUCKER, A. B.; NOONAN, R. Linguagens de programação: princípios e
paradigmas. 2 ed. McGrawHill, 2008. ISBN: 9788577260447
DIVERIO, Tiarajú Asmuz. Teoria da computação: máquinas universais e
computabilidade. 3. ed. Porto Alegre: Bookman, 2011. 288 p. (Livros
didáticos. n.5)
}


\DetalhesDisciplina{
    QXD0017 - Lógica para Computação
}{% EMENTA
    Lógica proposicional e de Primeira Ordem; Formalização de problemas;
Sistemas dedutivos: axiomático, natural e tableaux; Correção e Completude. Lógicas Temporais para a Validação de Sistemas.
}{% BIBLIOGRAFIA BÁSICA
    SOUZA, João Nunes de. Lógica para ciência da computação: fundamentos
de linguagem, semântica e sistemas de dedução. Rio de Janeiro: Elsevier,
2002. 309 p. ISBN 8535210938 (broch.).
\newline \newline 
SILVA, Flávio Soares Corrêa da; FINGER, Marcelo; MELO, Ana Cristina
Vieira de. Lógica para computação. São Paulo, SP: Thomson Learning,
2006. 234 p. ISBN 8522105170 (broch.).
\newline \newline 
BISPO, Carlos Alberto F.; CASTANHEIRA, Luiz B.; FILHO, Oswaldo
Melo S. Introdução à Lógica Matemática. [Digite o Local da Editora]:
Cengage Learning Brasil, 2017. E-book. ISBN 9788522115952. Disponível
em: https://app.minhabiblioteca.com.br/\#/books/9788522115952.
Acesso em: 07 out. 2022.
}{% BIBLIOGRAFIA COMPLEMENTAR
    ENDERTON, Herbert B. A mathematical introduction to logic. 2nd ed.
San Diego, California: Harcourt/Academic Press, c2001. 317 p. ; ISBN 0122384520 (enc.).
\newline \newline 
GERSTING, Judith L. Fundamentos matemáticos para a ciência da computação: um tratamento moderno de matemática discreta . 5. ed. Rio
de Janeiro: Livros Técnicos e Científicos, 2004. 597 p. ISBN 8521614225
(broch.).
\newline \newline 
CLARKE, E. M. Model checking. Cambridge: MIT Press, 1999. 314 p.
ISBN 9780262032704 (enc.).
\newline \newline 
HUTH, Michael; RYAN, Mark. . Lógica em ciência da computação: modelagem e argumentação sobre sistemas . 2. ed. Rio de Janeiro: LTC,
2008. 322 p. ISBN 9788521616108 (broch.).
\newline \newline 
ALENCAR FILHO, Edgard de. Iniciação à lógica matemática. São Paulo:
Nobel, [2002]. 203 p. ISBN 852130403X (broch).
}


\DetalhesDisciplina{
    QXD0056 - Matemática Básica
}{% EMENTA
    Lógica: conectivos lógicos, tabela verdade, fórmulas equivalentes. Conjuntos:
notação, operações, propriedades das operações, diagramas de Venn,
partição, cardinalidade, conjuntos das partes, produto cartesiano. Contagem:
princípios da multiplicação e da adição, princípio da exclusão, princípios das
casas dos pombos, permutações, combinações, teorema binominal, triângulo
de pascal. Relações: definições, terminologia, propriedades. Funções:
definições, terminologia, propriedades.
}{% BIBLIOGRAFIA BÁSICA
    GERSTING, Judith L. Fundamentos matemáticos para a ciência da
computação: um tratamento moderno de matemática discreta . 5. ed. Rio de
Janeiro: Livros Técnicos e Científicos, 2004. 597 p. ISBN 8521614225 (broch.).
\newline \newline 
SILVA, Sebastiao Medeiros da; SILVA, Elio Medeiros da; SILVA, Ermes
Medeiros da. Matemática básica para cursos superiores. São Paulo: Atlas,
2002. 227 p. ISBN 8522430357 (broch.).
\newline \newline 
IEZZI, Gelson; MURAKAMI, Carlos. Fundamentos de matemática elementar: 1:
conjuntos, funções. 410 p. 9.ed. ISBN: 9788535716801
}{% BIBLIOGRAFIA COMPLEMENTAR
    ROSEN, Kenneth H. Matemática discreta e suas aplicações. 6. ed. São Paulo,
SP: McGraw-Hill, 2009. xxi, 982 p. ISBN 9788577260362 (broch.).
\newline \newline 
ALENCAR FILHO, Edgard de. Iniciação a lógica matemática. São Paulo:
Nobel, [2002] 203p ISBN 852130403X (broch).
\newline \newline 
MENEZES, Paulo Blauth; UNIVERSIDADE FEDERAL DO RIO GRANDE DO
SUL. Matemática discreta para computação e informática. 3. ed. Porto Alegre,
RS: Bookman, 2010. 350 p (Livros didáticos. 16).
\newline \newline 
DEMANA, Franklin D. Pré-cálculo. São Paulo: Addison-Wesley, 2009. 380 p.
ISBN 9788588639379 (broch.).
\newline \newline 
MENEZES, Paulo Blauth; TOSCANI, Laira V.; GARCÍA LÓPEZ, Javier.
Aprendendo matemática discreta com exercícios. Porto Alegre, RS: Bookman,
2009. 356p. (Livros didáticos informática ufrgs; v. 19) ISBN 9788577804719
(broch.).
\newline \newline 
SCHEINERMAN, Edward R. Matemática discreta: uma introdução . São Paulo:
Cengage Learning, 2011. 573 p. ISBN 9788522107964 (broch.).
}


\DetalhesDisciplina{
    QXD0120 - Matemática Computacional
}{% EMENTA
    Noções de modelagem matemática de problemas. Noções de métodos
numéricos: erros de representação em ponto flutuante e perda de significância;
raízes de equações; integração numérica; interpolação; solução de sistemas
de equações lineares; minimização de funções. Programação linear: forma
padrão e dualidade; método simplex. Uso de pacotes computacionais.
}{% BIBLIOGRAFIA BÁSICA
    BARROSO, Leônidas Conceição et al. Cálculo numérico: (com aplicações). 2.
ed. São Paulo, SP: Harbra, c1987. 367 p. ISBN 8529400895 (broch.).
\newline \newline 
GOLDBARG, Marco Cesar. Otimização combinatória e programação linear:
modelos e algoritmos. Rio de Janeiro: Elsevier: ​Campus,​ 2005. xvi, 518 p. :
ISBN 9788535215205 (broch.)
\newline \newline 
HILLIER, FREDERICK S.; LIEBERMAN, GERALD J. Introdução a Pesquisa
Operacional. MCGRAW HILL. 9a edição. (ISBN: 8580551188)
}{% BIBLIOGRAFIA COMPLEMENTAR
    RUGGIERO, Marcia A. Gomes; LOPES, Vera Lucia da Rocha. Cálculo
numérico: aspectos teóricos e computacionais . 2. ed. Pearson, c1997. ISBN
8534602042.
\newline \newline 
PASSOS, Eduardo José Pedreira Franco dos. Programação linear como
instrumento da pesquisa operaciona: Eduardo José Pedreira Franco dos
Passos. São Paulo, SP: Atlas, 2008. xii, 451p. ISBN 9788522448395 (broch.).
\newline \newline 
GERSTING, Judith L. Fundamentos matemáticos para a ciência da
computação: um tratamento moderno de matemática discreta . 5. ed. Rio de
Janeiro: Livros Técnicos e Científicos, c2004. xiv, 597 p. ISBN: 8521614225.
\newline \newline 
CHENEY, Ward; KINCAID, David (David Ronald). Numerical mathematics and
computing. 3rd. ed. Pacific Grove, CA: Books/Cole, c1994. 578p. ISBN
0534201121
\newline \newline 
SPERANDIO, Décio; MENDES, João Teixeira; SILVA, Luiz Henry Monken e.
Cálculo numérico: características matemáticas e computacionais dos métodos
numéricos. São Paulo, SP: Prentice Hall, 2003. ix, 354 p. ISBN 8587918745
(broch.).
}


\DetalhesDisciplina{
    QXD0008 - Matemática Discreta
}{% EMENTA
    Técnicas de demonstração: exaustiva, direta, contraposição, absurdo, indução
(fraca e forte). Somatórios: notação, propriedades, séries aritméticas,
geométricas e harmônicas, algumas fórmulas de somatórios úteis. Teoria dos
números: divisibilidade, primos, teorema fundamental da aritmética, aritmética
modular, aplicações. Relações: fechos, ordem parcial e total, relações e
classes de equivalências. Grafos: terminologia, alguns grafos especiais,
isomorfismo, conectividade, árvores (definição e propriedades).
}{% BIBLIOGRAFIA BÁSICA
    GERSTING, Judith L. Fundamentos matemáticos para a ciência da
computação: um tratamento moderno de matemática discreta . 5. ed. Rio de
Janeiro: Livros Técnicos e Científicos, 2004. 597 p. ISBN 8521614225 (broch.).
\newline \newline 
MENEZES, Paulo Blauth; UNIVERSIDADE FEDERAL DO RIO GRANDE DO
SUL. Matemática discreta para computação e informática. 3. ed. Porto Alegre,
RS: Bookman, 2010. 350 p (Livros didáticos. 16).
\newline \newline 
ROSEN, Kenneth H. Matemática discreta e suas aplicações. 6. ed. São Paulo,
SP: McGraw-Hill, 2009. 982 p. ISBN 9788577260362 (broch.).
}{% BIBLIOGRAFIA COMPLEMENTAR
    ALENCAR FILHO, E. Iniciação à lógica matemática. 21. ed. São Paulo: Nobel,
2008. ISBN:9788521304036.
\newline \newline 
SILVA, Sebastiao Medeiros da; SILVA, Elio Medeiros da; SILVA, Ermes
Medeiros da. Matemática básica para cursos superiores. São Paulo: Atlas,
2002. 227 p. ISBN 8522430357 (broch.).
\newline \newline 
MENEZES, Paulo Blauth; TOSCANI, Laira V.; GARCÍA LÓPEZ, Javier.
Aprendendo matemática discreta com exercícios. Porto Alegre, RS: Bookman,
2009. 356p. (Livros didáticos informática ufrgs ; ; v. 19) ISBN 9788577804719
(broch.).
\newline \newline 
SCHEINERMAN, Edward R. Matemática discreta: uma introdução . São Paulo:
Cengage Learning, 2011. 573 p. ISBN 9788522107964 (broch.).
\newline \newline 
HUNTER, David J. Fundamentos da matemática discreta. Rio de Janeiro, RJ:
LTC, 2011. 235 p. ISBN 9788521618102 (broch.)
}


\DetalhesDisciplina{
    QXD0178 - Mineração de Dados
}{% EMENTA
    O que é ​data mining.​ Aplicações potenciais. O processo de Descoberta do
Conhecimento. ​Data mining, data warehouse e OLAP. Tarefas de mineração
de dados: classificação, agrupamento (​clustering​), regras de associação e
análise de desvios. Estudo de algoritmos para as principais tarefas de
mineração de dados. Avaliação dos resultados obtidos. Introdução às técnicas
de recuperação de informações e aplicações em mineração de textos e ​Web
mining.​
}{% BIBLIOGRAFIA BÁSICA
    TAN, Pang-Ning; Steinbach, Michael; Kumar, Vipin. Introdução ao Data Mining
- Mineração de Dados . Ciência Moderna, 2009. 900 p. ISBN-10 8573937610
ISBN-13 9788573937619.
\newline \newline 
PINHEIRO, Carlos A. R. Inteligência analítica: mineração de dados e
descoberta de conhecimento. Ciência Moderna, c2008. 397 p. ISBN
9788573937077 (broch.)
\newline \newline 
ELMASRI, Ramez; NAVATHE, Sham. Sistemas de banco de dados. 6. ed. -.
São Paulo, SP: Pearson Education do Brasil, 2011. xviii, 788 p. ISBN:
9788579360855
\newline \newline 
WITTEN, Ian H.; FRANK, Eibe; HALL, Mark A. Data Mining: Practical Machine
Learning Tools and Techniques. The Morgan Kaufmann Series in Data
Management Systems. Morgan Kaufmann, 3ª Ed., 2011. ISBN: 0123748569
}{% BIBLIOGRAFIA COMPLEMENTAR
    SILBERSCHATZ, Abraham; KORTH, Henry F.; Sudarshan, S. Sistema de
banco de dados. Rio de Janeiro: Elsevier: Campus, 2012. ISBN
9788535245356
\newline \newline 
RAMAKRISHNAN, Raghu; GEHRKE, Johannes. Sistemas de Gerenciamento
de banco de dados. São Paulo, SP: McGraw-Hill, 2008. xxvii, 884 p. ISBN:
9788577260270
\newline \newline 
DATE, C. J. Introdução a sistemas de banco de dados. 8. ed. Rio de Janeiro:
Elsevier, 2004. 865p ISBN: 8535212736
\newline \newline 
LESKOVEC, Jure; RAJARAMAN, Anand; ULLMAN, Jeffrey David. Mining of massive datasets [e-book]. 3. ed. Cambridge: Cambridge University Press, 2020. Disponível em: https://www.mmds.org/.
\newline \newline 
ZAKI, M. J.; MEIRA Jr., W. Data Mining and Analysis: Fundamental Concepts
and Algorithms. Cambridge 9780521766333. University Press, Disponível
May 2014. ISBN: em: http://www.dataminingbook.info/uploads/book.pdf
\newline \newline 
GRUS, Joel. Data Science from Scratch: First Principles with Python. O'Reilly
Media. 1st edition. 2015. ISBN 149190142X.
\newline \newline 
BISHOP, Christopher M. Pattern Recognition and Machine Learning. 1 ed.
Springer, 2007. ISBN: 0387310738
\newline \newline 
HAN, Jiawei; Kamber, Micheline; Pei, Jian. Data Mining: Concepts and
Techniques. The Morgan Kaufmann Series in Data Management Systems.
Morgan Kaufmann, 3ª Ed., 2011. ISBN: 0123814790
\newline \newline 
HASTIE, Trevor; TIBSHIRANI, Robert; Friedman, Jerome. The Elements of
Statistical Learning: Data Mining, Inference, and Prediction. Springer, 2ª Ed.,
2009. ISBN: 0387848576
}


\DetalhesDisciplina{
    <14> - MLOPs
}{% EMENTA
    Integração de práticas de engenharia de software, DevOps e aprendizado de máquina, abordando conceitos de Big Data e seus desafios. Ciclo de vida completo de projetos de aprendizado de máquina em produção, incluindo versionamento de dados e modelos. Reproducibilidade e governança de modelos. Arquiteturas e paradigmas para processamento distribuído. Técnicas de ingestão, armazenamento, processamento e visualização de dados em larga escala. Pipelines automatizados de CI/CD para aprendizado de máquina, automação. Ferramentas e ecossistemas para processamento de grandes volumes de dados. Containerização e orquestração. Computação em nuvem para Big Data. Ferramentas de deployment e plataformas de MLOps. Testes automatizados para modelos de aprendizado de máquina, monitoramento de modelos em produção.
}{% BIBLIOGRAFIA BÁSICA
    HUYEN, Chip. Projetando sistemas de Machine Learning: processo interativo para aplicações prontas para produção. Rio de Janeiro: Editora Alta Books, 2024. E-book. p.4. ISBN 9788550819648. Disponível em: https://app.minhabiblioteca.com.br/reader/books/9788550819648/.
\newline \newline 
SANTOS, Roger R.; BORDIN, Maycon V.; NUNES, Sergio E.; et al. Fundamentos de Big Data. Porto Alegre: SAGAH, 2021. E-book. p.Capa. ISBN 9786556901749. Disponível em: https://app.minhabiblioteca.com.br/reader/books/9786556901749/.
\newline \newline 
ERL, Thomas; MONROY, Eric B. Computação em Nuvem: Conceitos, Tecnologia, Segurança e Arquitetura. 2. ed. Porto Alegre: Bookman, 2024. E-book. p.i. ISBN 9788582606599. Disponível em: https://app.minhabiblioteca.com.br/reader/books/9788582606599/.
}{% BIBLIOGRAFIA COMPLEMENTAR
    KREUZBERGER, Dominik; KÜHL, Niklas; HIRSCHL, Sebastian. Machine Learning Operations (MLOps): overview, definition, and architecture. IEEE Access, v. 11, p. 31866–31879, 2023. DOI: 10.1109/ACCESS.2023.3262138. Disponível em: https://ieeexplore.ieee.org/document/10081336.
\newline \newline 
SYMEONIDIS, Georgios; NERANTZIS, Evangelos; KAZAKIS, Apostolos; PAPAKOSTAS, George A. MLOps – definitions, tools and challenges. In: IEEE 12th Annual Computing and Communication Workshop and Conference (CCWC), 2022, Las Vegas. Anais [...]. [S.l.]: IEEE, 2022. p. 453–460. DOI: 10.1109/CCWC54503.2022.9720902. Disponível em: https://ieeexplore.ieee.org/document/9720902.
\newline \newline 
TESTI, Matteo; BALLABIO, Michela; FRONTONI, Eleonora; IANNELLO, Guido; MOCCIA, Stefano; SODA, Pasquale; VESSIO, Giuseppe. MLOps: a taxonomy and a methodology. IEEE Access, v. 10, p. 63 606–63 618, 2022. DOI: 10.1109/ACCESS.2022.3181730. Disponível em: https://ieeexplore.ieee.org/document/9792270. 
}


\DetalhesDisciplina{
    <23> - Percepção e Ação Robótica
}{% EMENTA
    Tipos de sensores para robótica;
Percepção a partir de visão computacional;
Calibração de câmeras e visão estereoscópica;
Odometria visual;
Fusão sensorial e técnicas de localização robótica;
Localização e mapeamento simultâneo (SLAM);
Planejamento de rotas e navegação.
}{% BIBLIOGRAFIA BÁSICA
    NISE, Norman S.. Engenharia de Sistemas de Controle, 7a edição. São
Paulo: LTC, 2017. E-book. Disponível em: https://integrada.minhabiblioteca.
com.br/\#/books/9788521634379. Acesso em: 30 de maio de 2025.
\newline \newline 
Ben Coppin. 2004. Artificial Intelligence Illuminated. Jones and Bartlett Publishers, Inc., USA.
\newline \newline 
RUSSELL, Stuart J.; NORVIG, Peter. Inteligência artificial. Rio de Janeiro: Elsevier, Campus, 2013. 988 p. ISBN 9788535237016 (broch.).

}{% BIBLIOGRAFIA COMPLEMENTAR
    ROMERO, Roseli Aparecida F.; PRESTES, Edson; OSÓRIO, Fernando
et al. Robótica Móvel. São Paulo: LTC, 2014. E-book. Disponível em: https://integrada.minhabiblioteca.com.br/\#/books/978-85-216-2642-8. Acesso em: 30 de maio 2025.
\newline \newline 
MATARIC, Maja J.. Introdução á robótica. São Paulo: Editora Blucher, 2014. E-book. Disponível em: https://integrada.minhabiblioteca.com.br/\#/books/9788521208549. Acesso em: 30 de maio de 2025.
\newline \newline 
ANDRADE-CETTO, Juan; SANFELIU, Alberto. Environment learning for indoor mobile robots: a stochastic state estimation approach to simultaneous localization and map building.. Berlim; New York, NY: Springer, 2006. (Springer tracts in advanced robotics ; 23). Disponível em: https://link.springer.com/book/10.1007/11418382. Acesso em 30 de maio de 2025.
\newline \newline 
JAZAR, Reza N; SPRINGERLINK (ONLINE SERVICE). Theory of Applied Robotics : Kinematics, Dynamics, and Control (2nd Edition) . Springer eBooks 2nd. XXIII, 883p. 400 illus., 200 illus. in color ISBN 9781441917508. 
\newline \newline 
ZHANG, Dan; SPRINGERLINK (ONLINE SERVICE). Parallel Robotic Machine Tools. Springer eBooks ISBN 9781441911179.
}


\DetalhesDisciplina{
    <24> - Pesquisa e Inovação
}{% EMENTA
    Pesquisa em acervos físicos e virtuais: títulos, base de dados, periódicos, patentes, marcas, desenhos industriais. Método científico. Conceituação, tipos e metodologia de pesquisa. Elaboração de artigos técnico científicos, projetos de pesquisa e projetos de inovação.

Tipologias da inovação: industrial, em serviços, tecnológica, organizacional, aberta e fechada. Sistema de inovação, políticas de ciência, tecnologia e inovação (CT&I)
no Brasil e no mundo.
}{% BIBLIOGRAFIA BÁSICA
    WAZLAWICK, Raul Sidnei. Metodologia de pesquisa para ciência da computação. Rio de Janeiro : Elsevier.
2008.
\newline \newline 
DEZ tipos de inovação: Dez tipos de inovação** a disciplina de criação de avanços de ruptura. São Paulo: DVS Editora, 2015. 263 p. ISBN 978-85-8289-084-4.
\newline \newline 
VALLE, Rogério ; OLIVEIRA, Saulo Barbará de (org.). Análise e modelagem de processos de negócio:
foco na notação BPMN (Business Process Modeling Notation). São Paulo: Atlas, 2013. 207 p. ISBN 978-85-224-5621-5.
\newline \newline 

\newline \newline 


}{% BIBLIOGRAFIA COMPLEMENTAR
    A guide to the project management body of knowledge (PMBOK GUIDE)** A GUIDE to the project management body of knowledge (PMBOK GUIDE). 4rd. ed. Newtown Square, Pa: Project Management Institute, c2008. xxvi, 467 p. ISBN 9781933890517 (broch.).
\newline \newline 
FERREIRA, Valdinéia Barreto. E-science e políticas públicas para ciência, tecnologia e inovação no Brasil. Salvador: EDUFBA, 2018. 250 p. ISBN 978-85-2321707-5.
\newline \newline 
KALBACH, James. Mapeamento de experiências: um guia para criar valor por meio de jornadas, blueprints e diagramas. Rio de Janeiro: Alta Books, 2017. 362 p. ISBN 978-85-508-0061-5. Classificação: 658.4063 K21m (BCQ) Ac.217555
\newline \newline 
CHESBROUGH, Henry William. Modelos de negócios abertos: como prosperar no novo cenário da inovação. Porto Alegre: Bookman, 2012. xvi, 220 p. ISBN 978-85-7780-955-4.
\newline \newline 
EDLER, Jakob; WALZ, Rainer. Systems and Innovation Research in Transition: Research Questions and Trends in Historical Perspective. Springer Nature, 2024. DOI: https://doi.org/10.1007/978-3-031-66100-6 Acesso em: 3 de jun, 2025.
}


\DetalhesDisciplina{
    QXD0181 - Pesquisa Operacional
}{% EMENTA
    Introdução à Pesquisa Operacional e aos Sistemas de Apoio à Decisão.
Programação linear. Modelos de programação linear. Método simplex.
Problema do transporte. Dualidade. Técnicas avançadas em Pesquisa
Operacional.
}{% BIBLIOGRAFIA BÁSICA
    HILLIER, Frederick S.; LIEBERMAN, Gerald J.. Introdução à Pesquisa
Operacional. São Paulo: AMGH, 2013. E-book. Disponível em: https://
integrada.minhabiblioteca.com.br/\#/books/9788580551198. Acesso
em: 3 de out. 2022.
\newline \newline 
PASSOS, Eduardo J. P. F.. Programação linear como instrumento da
pesquisa operacional. Atlas, 2008. 451p. ISBN 9788522448395
\newline \newline 
COLIN, Emerson C.. Pesquisa Operacional -  170 Aplicações em Estratégia, Finanças, Logística, Produção, Marketing e Vendas, 2a edição. São Paulo: Atlas, 2017. E-book. Disponível em: https://integrada.
minhabiblioteca.com.br/\#/books/9788597014488. Acesso em: 3 de
out. 2022.
}{% BIBLIOGRAFIA COMPLEMENTAR
    ANDRADE, Eduardo Leopoldino de. Introdução à Pesquisa Operacional
- Método e Modelos para Análise de Decisões, 5a edição. São Paulo: LTC,
2015. E-book. Disponível em: https://integrada.minhabiblioteca.
com.br/\#/books/978-85-216-2967-2. Acesso em: 29 de set. 2022.
\newline \newline 
VIANA, Gerardo Valdisio Rodrigues. Meta-heurísticas e programação paralela em otimização combinatória. Fortaleza: Edições UFC, 1998. 250p.
\newline \newline 
ISBN 8572820396
CHAN, Alan H. S; AO, Sio-Iong SPRINGERLINK (ONLINE SERVICE).
Advances in Industrial Engineering and Operations Research. Springer
eBooks Boston, MA: Springer Science+Business Media, LLC, 2008. (Lecture Notes in Electrical Engineering ; 5) ISBN 9780387749051. Disponível
em: http://dx.doi.org/10.1007/978-0-387-74905-1. Acesso em : 21 set.
2010
\newline \newline 
QUDRAT-ULLAH, H; DAVIDSEN, P.I; SPECTOR, J.M SPRINGERLINK (ONLINE SERVICE). Complex Decision Making : Theory and
Practice . Springer e-books Berlin, Heidelberg: springer, 2008. (Understanding Complex Systems,) ISBN 9783540736653. Disponível em:
http://dx.doi.org/10.1007/978-3-540-73665-3. Acesso em: 21 set. 2010.
\newline \newline 
GOLDBARG, Marco Cesar. Otimização combinatória e programação linear: modelos e algoritmos. Rio de Janeiro: Elsevier: Campus, 2005. xvi,
518 p. : ISBN 9788535215205 (broch.).
}


\DetalhesDisciplina{
    <15> - Planejamento Automatizado
}{% EMENTA
    Modelos e linguagens de planejamento. (I) Planejamento Clássico: planejamento como satisfatibilidade (SATPLAN); grafo de planejamento (GRAPHPLAN); planejamento como busca heurística; construção de heurísticas independentes de domínio. (II) Variações e extensões de planejamento clássico: planejamento não-determinístico; busca num grafo AND-OR (LAO*); planejamento como verificação de modelos (baseado na lógica temporal CTL); planejamento contingente e conformante.
}{% BIBLIOGRAFIA BÁSICA
    RUSSELL, Stuart J.; NORVIG, Peter. Inteligência Artificial: Uma Abordagem Moderna. 4. ed. Rio de Janeiro: GEN LTC, 2022. E-book. p.Capa. ISBN 9788595159495. 
\newline \newline 
COPPIN, B Inteligência artificial. LTC, 2010. ISBN: 9788521617297
\newline \newline 
CORMEN, Thomas H.; LEISERSON, Charles E.; Ronald L. Rivest; et al. Algoritmos. 4. ed. Rio de Janeiro: GEN LTC, 2024. E-book. p.Capa. ISBN 9788595159914. 
}{% BIBLIOGRAFIA COMPLEMENTAR
    CLARKE, E. M. Model checking. Cambridge: MIT Press, 1999. 314 p. ISBN 9780262032704 (enc.).
\newline \newline 
BAIER, Christel; KATOEN, Joost-Pieter. Principles of model checking. Cambridge, Massachusetts: The Mit Press, 2008. xvii, 975 p. ISBN 9780262026499.
\newline \newline 
SILVA, Flávio Soares Corrêa da; FINGER, Marcelo; MELO, Ana Cristina Vieira de. Lógica para computação - 2ª edição. 2. ed. Porto Alegre: +A Educação - Cengage Learning Brasil, 2018. E-book. p.63. ISBN 9788522127191.
\newline \newline 
Ronald Brachman and Hector Levesque. 2004. Knowledge Representation and Reasoning. Morgan Kaufmann Publishers Inc., San Francisco, CA, USA.
\newline \newline 
Huth, Michael, and Mark Ryan. ""Lógica em ciência da computação: modelagem e argumentação sobre sistemas."" Rio de Janeiro, 2a edição edition. tradução e revisão técnica Valéria de Magalhães Iório (2008).
}

"\DetalhesDisciplina{
    QXD0109 - Pré-Cálculo
}{% EMENTA
    Funções no espaço contínuo: estudo de sinal, raízes, polinomiais, racionais, exponenciais, logarítmicas,trigonométricas.
}{% BIBLIOGRAFIA BÁSICA
    DEMANA, Franklin D. et al. Pré-cálculo . São Paulo, SP: Addison-Wesley, 2009. xv, 380 p. ISBN9788588639379 (broch.).
\newline \newline 
IEZZI, Gelson; MURAKAMI, Carlos. Fundamentos de matemática elementar: 1 : conjuntos, funções . 8. ed.,. SãoPaulo, SP: Atual, 2004. 374 p. ISBN 8570562705 (broch.) .
\newline \newline 
SILVA, Sebastiao Medeiros da; SILVA, Elio Medeiros da; SILVA, Ermes Medeiros da. Matemática básica paracursos superiores . São Paulo: Atlas, 2002. 227p.
}{% BIBLIOGRAFIA COMPLEMENTAR
    LEITHOLD, Louis. O Cálculo com geometria analítica. 3. ed. São Paulo: Harbra, c1994. 2 v. ISBN 8529400941v.1 (broch.)
\newline \newline 
IEZZI, Gelson. Fundamentos de matemática elementar, 6: complexos, polinômios, equações . 8. ed. São Paulo,SP: Atual, 2013. 250 p. ISBN 9788535717525 (broch.)
\newline \newline 
IEZZI, Gelson. Fundamentos de matemática elementar: 7 : geometria analítica. 5. ed. São Paulo, SP: Atual,2005. 282 p. ISBN 8535705465 (broch.)
\newline \newline 
IEZZI, Gelson. Fundamentos de matemática elementar, 3: trigonometria: 123 exercícios resolvidos, 385 exercícios propostos com resposta, 236 testes de vestibulares com resposta. 7. ed. São Paulo, SP: Atual, 1993.303 p. ISBN 8570562691
\newline \newline 
CALDEIRA, André Machado; SILVA, Luiza Maria Oliveira da,; MACHADO, Maria Augusta Soares. Pré-cálculo. 3.ed. rev. e ampl. São Paulo, SP: Cengage Learning; c2014. xv, 558p. ISBN 9788522116126 (broch.)
\newline \newline 
IEZZI, GELSON et al; Fundamentos de matemática elementar v.2: Logaritmos, 8 ed., 2004, Saraiva
}"

\DetalhesDisciplina{
    QXD0112 - Probabilidade e Estatística
}{% EMENTA
    Fundamentos de análise combinatória. Conceito de probabilidade e seus
teoremas fundamentais. Variáveis aleatórias. Distribuições de probabilidade.
Estatística descritiva. Noções de amostragem. Distribuições amostrais: discreta
e contínua. Inferência estatística: teoria da estimação, intervalos de confiança
e testes de hipóteses. Regressão linear simples. Correlação.
}{% BIBLIOGRAFIA BÁSICA
    LARSON, Ron; FARBER, Betsy. Estatística aplicada. 4. ed. São Paulo, SP:
Pearson/ Prentice Hall, 2010. xiv,637 p. ISBN 9788576053729 (broch.).
\newline \newline 
BARBETTA, Pedro Alberto; REIS, Marcelo Menezes; BORNIA, Antonio Cezar.
Estatística para cursos de engenharia e informática. 3. ed. São Paulo, SP:
Atlas, 2010. 410 p.
\newline \newline 
WALPOLE, Ronald E. Probabilidade e estatística: para engenharia e ciências.
8. ed. São Paulo, SP: Pearson/ Prentice Hall, 2009. xiv, 491 p. ISBN
9788576051992 (broch.).
\newline \newline 
MORETTIN, Luiz Gonzaga. Estatística básica:: probabilidade e inferência /
volume único. São Paulo, SP: Pearson Educational do Brasil. 2010. ISBN
8576053705 ISBN-13 9788576053705 (broch.).
}{% BIBLIOGRAFIA COMPLEMENTAR
    HAZZAN, Samuel. Fundamentos de matemática elementar, 5: combinatória,
probabilidade : 43 exercícios resolvidos, 439 exercícios propostos com
resposta, 155 testes de vestibulares com resposta. 7. ed. São Paulo, SP: Atual,
2004. 184 p. ISBN 8535704612 (broch.).
\newline \newline 
DANCEY, Christine P.; REIDY, John. Estatística sem matemática para
psicologia: usando SPSS para Windows. 3. ed. Porto Alegre, RS: Artmed,
2006. 608 p. (Biblioteca Artmed. Métodos de Pesquisa) ISBN 8536306882
(broch.).
\newline \newline 
HARPER, Brian D.; MERIAM, J. L; KRAIGE, L. G. Solving statistics problems in
\newline \newline 
MATLAB: engineering mechanics: statics. 6th ed. Massachusetts, [Estados
Unidos]: J. Wiley \& Sons, 2007. 139 p. ISBN 9780470099254 (broch.).
\newline \newline 
TRIOLA, Mario F. Introdução à estatística- Atualização da Tecnologia. 11. ed.
Rio de Janeiro, RJ:LTC, 2013. 740 p. ISBN 9788521622062 (broch.).
\newline \newline 
IEZZI, Gelson; HAZZAN, Samuel; DEGENSZAJN, David Mauro.Fundamentos
de matemática elementar: 11 : matemática comercial, matemática financeira,
estatística descritiva . São Paulo, SP: Atual, 2004. 232 p.
}


\DetalhesDisciplina{
    <25> - Processamento de Áudio e Voz
}{% EMENTA
    Modelos acústicos tradicionais, misturas de gaussianas e cadeias ocultas de markov. Representações de sinais para áudio e voz. Arquiteturas de redes neurais para reconhecimento de fala. Modelos de encoder, vocoder e arquiteturas para sintetização de voz. Arquiteturas para aplicações em música.
}{% BIBLIOGRAFIA BÁSICA
    HAYKIN, Simon S.. Redes neurais: princípios e prática. 2. ed. Porto Alegre: Bookman, 2001. xvii, 900 p.
ISBN 9788573077186.
\newline \newline 
Paulo S. R. Diniz; Eduardo A. B. Silva; Sergio L. Netto. Processamento digital de sinais. 
Grupo A. 2ª edição. Disponível em:
https://app.minhabiblioteca.com.br/reader/books/9788582601242 
\newline \newline 
Caseli, H.M.; Nunes, M.G.V. (org.) Processamento de Linguagem Natural: Conceitos, Técnicas e Aplicações em Português. 3 ed. BPLN, 2024. Disponível em: https://brasileiraspln.com/livro-pln/3a-edicao.
}{% BIBLIOGRAFIA COMPLEMENTAR
    RUSSELL, Stuart J.; NORVIG, Peter. Inteligência artificial. Rio de Janeiro: Elsevier, Campus, 2013. 988 p. ISBN 9788535237016 (broch.).
\newline \newline 
Júlio Serafim Martins; Maikon Lucian Lenz; Michel Bernardo Fernandes Da Silva; et al. Grupo A. Processamentos de Linguagem Natural. Disponível em: https://app.minhabiblioteca.com.br/reader/books/9786556900575
\newline \newline 
PAAß, Gerhard; GIESSELBACH, Sven. Foundation models for natural language processing: pre-trained language models integrating media. Cham: Springer, 2023. (Artificial Intelligence: Foundations, Theory, and Algorithms). DOI: https://doi.org/10.1007/978-3-031-23190-2. 
\newline \newline 
LIU, Zhiyuan; LIN, Yankai; SUN, Maosong (Ed.). Representation learning for natural language processing. 2. ed. Singapore: Springer, 2023. DOI: https://doi.org/10.1007/978-981-99-1600-9. 
\newline \newline 
AWAD, Mariette; KHANNA, Rahul. Efficient learning machines: theories, concepts, and applications for engineers and system designers. Berkeley, CA: Apress, 2015. DOI: https://doi.org/10.1007/978-1-4302-5990-9. 
}


\DetalhesDisciplina{
    <10> - Processamento de Dados em Larga Escala
}{% EMENTA
    Arquiteturas e paradigmas para manipulação e análise de grandes volumes de dados. Big Data. Arquiteturas e paradigmas para processamento de dados distribuído. Processamento de dados em lote (batch) e em fluxo (stream). Data Lakes e Data Warehouses. Estratégias de particionamento, tolerância a falhas e escalabilidade. Práticas de ETL, transformação e persistência de dados em ambientes distribuídos. Bancos de dados NoSQL e NewSQL. Computação em nuvem para Big Data. Otimização de performance e gerenciamento de recursos. Desenvolvimento de pipelines de dados escaláveis e eficientes.
}{% BIBLIOGRAFIA BÁSICA
    HUYEN, Chip. Projetando sistemas de Machine Learning: processo interativo para aplicações prontas para produção. Rio de Janeiro: Editora Alta Books, 2024. E-book. p.4. ISBN 9788550819648. Disponível em: https://app.minhabiblioteca.com.br/reader/books/9788550819648/.
\newline \newline 

PPADILHA, Juliana; SOARES, Juliane A.; ALVES, Nicolli S R.; et al. Analytics para big data. Porto Alegre: SAGAH, 2022. E-book. p.2. ISBN 9786556903477. Disponível em: https://app.minhabiblioteca.com.br/reader/books/9786556903477/.
\newline \newline 
SANTOS, Roger R.; BORDIN, Maycon V.; NUNES, Sergio E.; et al. Fundamentos de Big Data. Porto Alegre: SAGAH, 2021. E-book. p.Capa. ISBN 9786556901749. Disponível em: https://app.minhabiblioteca.com.br/reader/books/9786556901749/.

}{% BIBLIOGRAFIA COMPLEMENTAR
    ERL, Thomas; MONROY, Eric B. Computação em Nuvem: Conceitos, Tecnologia, Segurança e Arquitetura. 2. ed. Porto Alegre: Bookman, 2024. E-book. p.i. ISBN 9788582606599. Disponível em: https://app.minhabiblioteca.com.br/reader/books/9788582606599/.
\newline \newline 
RAGKOULIS, Marios; CARBONE, Paris; KALAVRI, Vasiliki; KATSIFODIMOS, Asterios. A survey on the evolution of stream processing systems. The VLDB Journal, [S.l.], v. 33, n. 2, p. 507–541, mar. 2024. DOI: 10.1007/s00778-023-00819-8. Disponível em: https://doi.org/10.1007/s00778-023-00819-8. 
\newline \newline 
LOURENÇO, João Ricardo; CABRAL, Bruno; CARREIRO, Paulo; VIEIRA, Marco; BERNARDINO, Jorge. Choosing the right NoSQL database for the job: a quality attribute evaluation. Journal of Big Data, [S.l.], v. 2, art. 18, p. 1–26, ago. 2015. DOI: 10.1186/s40537‑015‑0025‑0. Disponível em: https://journalofbigdata.springeropen.com/articles/10.1186/s40537-015-0025-0.
}


\DetalhesDisciplina{
    QXD0188 - Processamento de Imagens
}{% EMENTA
    Introdução ao processamento digital de imagens, Fundamentos sobre imagens
digitais, Formação de Imagens, Áreas de Aplicação. Transformações de
intensidade e filtragem espacial, Filtragem no domínio da frequência,
Amostragem e Quantização. Classificação de Imagens, Restauração e
reconstrução de imagens, Processamento de imagens coloridas, ​Wavelets e
processamento multiresolução, Compressão de imagens, Processamento
morfológico de imagens, Segmentação de imagens, Representação e
descrição, Reconhecimento de objetos.
}{% BIBLIOGRAFIA BÁSICA
    GONZALEZ, Rafael C.; WOODS, Richard E. Processamento digital de
imagens. 3. ed. Pearson, 2010. xv,624 p. ISBN 9788576054016 (broch.).
\newline \newline 
RUSS, John C. The Image Processing Handbook. Taylor \& Francis. 6 ed.
2010. ISBN 1439840458
\newline \newline 
PARKER, J. R. Algorithms for image Processing and Computer Vision. John
Wiley. 2 ed. 2010. ISBN 0470643854
}{% BIBLIOGRAFIA COMPLEMENTAR
    CONCI, Aura; AZEVEDO, Eduardo. Computação Gráfica, Volume 1 - Geração
de Imagens. Publicado por Elsevier. 2003. ISBN: 9788535212525, 384 páginas
GOMES, J. M.; VELHO, L. Fundamentos de computação gráfica. IMPA. 2008.
ISBN: 8524402008
\newline \newline 
BISWAS, Sambhunath; LOVELL, Brian C SPRINGERLINK (ONLINE
SERVICE). Bezier and Splines in Image Processing and Machine Vision.
Springer eBooks London: Springer-Verlag London Limited, 2008. ISBN
9781846289576. Disponível em : 
<http://dx.doi.org/10.1007/978-1-84628-957-6>. Acesso em : 21 set. 2010.
\newline \newline 
HUTCHISON, David; ELMOATAZ, Abderrahim; KANADE, Takeo; KITTLER,
Josef; KLEINBERG, Jon M; LEZORAY, Olivier; MAMMASS, Dris; MATTERN,
Friedemann; MITCHELL, John C; NAOR, Moni; NIERSTRASZ, Oscar;
\newline \newline 
NOUBOUD. Image and Signal Processing : 3rd International Conference,
ICISP 2008 Cherbourg-Octeville, France, July 1-3, 2008 Proceedings . Springer
eBooks Berlin, Heidelberg: Springer-Verlag Berlin Heidelberg, 2008. (Lecture
Notes in Computer Science, 5099) ISBN 9783540699057. Disponível em :
<http://dx.doi.org/10.1007/978-3-540-69905-7>. Acesso em : 21 set. 2010.
\newline \newline 
SARFRAZ, M SPRINGERLINK (ONLINE SERVICE). Interactive Curve
Modeling : With Applications to Computer Graphics, Vision and Image
Processing . Springer e-books London: Springer-Verlag London Limited, 2008.
ISBN 9781846288715. Disponível em :
<​http://dx.doi.org/10.1007/978-1-84628-871-5​>. Acesso em : 21 set. 2010.
\newline \newline 
CONCI, Aura; AZEVEDO, Eduardo.; LETA, Fabiana R. Computação gráfica,
v.2: teoria e prática. Rio de Janeiro, RJ: Elsevier: Campus, 2008. 407 p., [8] p.
de estampas + 1 CD-ROM ISBN 97885352232193 (broch).
\newline \newline 
JAIN, A.K. Fundamentals of Digital Image Processing. 1 ed. Prentice-Hall,
Addison-Wesley, 1988. ISBN-10: 0133361659 ISBN-13: 978-0133361650.
\newline \newline 
BOVIK, Al (ed.). Handbook of Image and Video Processing. 2 ed. Academic
Press, 2005. ISBN-10: 0121197921 ISBN-13: 9780121197926.
\newline \newline 
LIM, J. S. Two-dimensional Signal and Image Processing. Prentice Hall Press,
1990. ISBN-10: 0139353224 ISBN-13: 978-0139353222.
\newline \newline 
PETROU, Costas; PETROU, Maria. Image Processing: the fundamentals. 2 ed.
Wiley, 2010. ISBN-10: 047074586X ISBN-13: 9780470745861
\newline \newline 
JAHNE, Bernd. Practical Handbook on Image Processing for Scientific
Applications. 2 ed. CRC Press, 2004. ISBN-10: 0849319005 ISBN-13:
9780849319006.
}


\DetalhesDisciplina{
    <11> - Processamento de Linguagem Natural
}{% EMENTA
    Fundamentos e aplicações do Processamento de Linguagem Natural (PLN). Pré-processamento de texto. Representações textuais (bag-of-words, TF-IDF e representações distribuídas de palavras (word embeddings). Modelos de linguagem estatísticos e neurais. Arquiteturas baseadas em transformadores e mecanismos de atenção. Modelos pré-treinados. Classificação textual, análise de sentimentos, reconhecimento de entidades nomeadas (NER), extração de informações, tradução automática, sumarização de texto, sistemas de perguntas e respostas, chatbots e sistemas de diálogo. Métodos de avaliação de desempenho.
}{% BIBLIOGRAFIA BÁSICA
    Caseli, H.M.; Nunes, M.G.V. (org.) Processamento de Linguagem Natural: Conceitos, Técnicas e Aplicações em Português. 3 ed. BPLN, 2024. Licença Creative Commons. Disponível em: https://brasileiraspln.com/livro-pln/3a-edicao.
\newline \newline 
MCROY, Susan. Principles of Natural Language Processing. Milwaukee: Susan McRoy, 24 jul. 2021. 264 p. (Open textbook). ISBN 978‑1‑7376595‑0‑1. Disponível em: https://wisconsin.pressbooks.pub/naturallanguage/.
\newline \newline 
RUSSELL, Stuart J.; NORVIG, Peter. Inteligência Artificial: Uma Abordagem Moderna. 4. ed. Rio de Janeiro: GEN LTC, 2022. E-book. p.Capa. ISBN 9788595159495. Disponível em: https://app.minhabiblioteca.com.br/reader/books/9788595159495/. 
}{% BIBLIOGRAFIA COMPLEMENTAR
    HARIOM, Tatsat,; SAHIL, Puri,; BRAD, Lookabaugh,. Blueprints de aprendizado de máquina e ciência de dados para finanças: desenvolvendo desde estratégias de trades até robôs Advisors com Python. Rio de Janeiro: Editora Alta Books, 2024. E-book. p.i. ISBN 9788550821726. Disponível em: https://app.minhabiblioteca.com.br/reader/books/9788550821726/.
\newline \newline 
FACELI, Katti; LORENA, Ana C.; GAMA, João; AL, et. Inteligência Artificial - Uma Abordagem de Aprendizado de Máquina. 2. ed. Rio de Janeiro: LTC, 2021. E-book. p.Capa. ISBN 9788521637509. Disponível em: https://app.minhabiblioteca.com.br/reader/books/9788521637509/.
\newline \newline 
LIU, Zhiyuan; LIN, Yankai; SUN, Maosong (eds.). Representation Learning for Natural Language Processing. Singapore: Springer, 2023. eBook ISBN 978‑981‑99‑1600‑9; impresso ISBN 978‑981‑99‑1599‑6. Disponível em: https://link.springer.com/book/10.1007/978-981-99-1600-9. 

}


\DetalhesDisciplina{
    QXD0114 - Programação Funcional
}{% EMENTA
    Visão geral e motivação. Recursão sobre listas, números naturais, árvores, e
outros dados definidos recursivamente. Uso de funções como dados.
Expressões lambda. Avaliação preguiçosa. Prática de programação em
linguagem deste paradigma. Questões práticas como ​I/O​, depuração e
persistência de estruturas de dados.
}{% BIBLIOGRAFIA BÁSICA
    WAMPLER, Dean. Programação Funcional Para Desenvolvedores Java :
Ferramentas para Melhor Concorrência, Abstração e Agilidade. Novatec. 1a
ed., 2012. (ISBN 9788575223161)
\newline \newline 
SÁ, Claudio Cesar de. Haskell : uma abordagem prática. São Paulo, SP:
Novatec, 2006. 287 p.
\newline \newline 
SEIBEL, Peter. Practical common lisp . Berkeley, Ca: Apress, 2005. xxv, 499 p.
(The Expert’s voice in programming languages).
\newline \newline 
MICHAELSON, Greg. An Introduction to Functional Programming Through
Lambda Calculus . Dover Publications, 2011. ISBN: 0486478831
\newline \newline 
COUSINEAU, Guy; MAUNY, Michel; CALLAWAY, K. The Functional Approach
to Programming. Cambridge University Press; English edition, 1998. ISBN-10:
0521576814
}{% BIBLIOGRAFIA COMPLEMENTAR
    CHEN, Yinong; TSAI, Wei Tek. Introduction to programming languages:
programming in C, C++, Scheme, Prolog, C\#, and SOA. 2nd ed. xii, 383 p.
\newline \newline 
SEBESTA, Robert W. Conceitos de linguagens de programação. 9. ed. Porto
Alegre, RS: Bookman, 2011. ix, 792 p.
\newline \newline 
TUCKER, Allen B.; NOONAN, Robert. Linguagens de programação: princípios
e paradigmas. São Paulo, SP: McGraw Hill, 2009. xxi, 599p.
\newline \newline 
LEE, Kent SPRINGERLINK (ONLINE SERVICE). Programming Languages :
An Active Learning Approach . Springer eBooks Boston, MA: Springer-Verlag
US, 2008. ISBN 9780387794228. Disponível em :
<http://dx.doi.org/10.1007/978-0-387-79421-1>. Acesso em : 21 set. 2010.
\newline \newline 
GABBRIELLI, Maurizio; MARTINI, Simone; SPRINGERLINK 
(ONLINE SERVICE). Programming Languages: Principles and Paradigms . Springer
eBooks: Springer-Verlag London, 2010. ISBN 978-1-84882-914-5. Disponível em : <http://link.springer.com/book/10.1007/978-1-84882-914-5>. Acesso em :
12 jan. 2016.
\newline \newline 
EMERICK, Chas; CARPER, Brian; GRAND, Christophe. Clojure Programming.
O'Reilly Media; 1 edition, 2011. ISBN: 1449394701
\newline \newline 
LIPOVACA, Miran. Learn You a Haskell for Great Good!: A Beginner's Guide.
O'Reilly; 1 edition, 2011. ISBN: 1593272839
\newline \newline 
PETRICEK, Tomas; SKEET, Jon. Real-World Functional Programming: With
Examples in F\# and C\#. Manning Publications; 2010. ISBN-10: 1933988924
\newline \newline 
DYBVIG, R. Kent. The Scheme Programming Language, MIT Press; fourth
edition, 2009. ISBN-10: 026251298X
}


\DetalhesDisciplina{
    QXD0007 - Programação Orientada a Objetos
}{% EMENTA
    Introduzir o paradigma de Programação Orientada a Objetos (OO), juntamente
com seus conceitos de classes, objetos, herança, encapsulamento e
polimorfismo, além dos conceitos de Interfaces e exceções que são inerentes
às linguagens de programação orientadas a objetos. Desenvolvimento de um
pequeno sistema baseados no paradigma de programação OO.
}{% BIBLIOGRAFIA BÁSICA
    DEITEL, H. M. Java: como programar. 8 ed. Prentice Hall, 2010. ISBN:
9788576055631
\newline \newline 
HORSTMANN, Cay S. Core Java: volume I - fundamentos. 8. ed. São Paulo,
SP: Pearson, 2009. xiii, 383 p. ISBN 9788576053576
\newline \newline 
MCLAUGHLIN, Brett; POLLICE, Gary; WEST, David.Use a cabeça:análise e
projeto orientado ao objeto. Rio de Janeiro, RJ: Alta Books, 2007. xxviii, 441 p.
}{% BIBLIOGRAFIA COMPLEMENTAR
    BARNES, D.J.; KOLLING, M. Programação orientada a objetos com Java :
uma introdução prática usando BLUEJ. 4 ed. Prentice Hall Brasil, 2009. ISBN:
9788576051879
\newline \newline 
SIERRA, Kathy; BATES, Bert. Use a cabeça! Java. Rio de Janeiro: Alta Books,
2007. 470 p. ISBN 0596009208.
\newline \newline 
DEITEL, H. M. C++ como programar. 5 ed. Prentice Hall, 2006. ISBN:
8576050560.
\newline \newline 
BLAHA, Michael; RUMBAUGH, James. Modelagem e projetos baseados em
objetos com UML 2. 2.ed. rev. e atual. Rio de Janeiro, RJ: Campus; Elsevier,
2006. xvii, 496 p. ISBN 9788535217537 (broch.).
\newline \newline 
MANZANO, José Augusto N. G.; COSTA Jr., Roberto Affonso da. Java 7 -
Programação de Computadores - Guia Prático de Introdução, Orientação e
Desenvolvimento. 1. ed. Editora Érica, 2011. ISBN: 9788536503745.
\newline \newline 
MEYER, Bertrand. Object-Oriented Software Construction , Ed. Prentice Hall
PTR, 1997.
}


\DetalhesDisciplina{
    QXD0110 - Projeto de Pesquisa Científico-Tecnológico
}{% EMENTA
    O problema da pesquisa e sua formulação. Métodos e Técnicas de Pesquisa.
O planejamento da pesquisa. Elaboração de projeto de pesquisa referente
ao Trabalho de Conclusão de Curso.
}{% BIBLIOGRAFIA BÁSICA
    WAZLAWICK, Raul Sidnei. Metodologia de pesquisa para ciência da computação. Rio de Janeiro : Elsevier. 2008.
\newline \newline 
LAVILLE, Christian; Dionne, Jean. A Construção do Saber: Manual de
Metodologia da Pesquisa em Ciências Humanas. Porto Alegre: Artmed,
Belo Horizonte: Editora UFMG, 2008. 340 p.
\newline \newline 
MARCONI, Marina de Andrade; Lakatos, Eva Maria. Fundamentos de
Metodologia Científica.7. ed. São Paulo, SP: Atlas, 2010.
}{% BIBLIOGRAFIA COMPLEMENTAR
    YIN, Robert K. Estudo de Caso - Planejamento e Métodos. 4aed, Porto
Alegre : Bookman, 2010.
\newline \newline 
CERVO, A.; BERVIAN, P.A.; SILVA, R. Metodologia Científica. 6. ed.
2007. ISBN 8576050471
\newline \newline 
COOPER, D. R.; Schindler, Pamela S. Métodos de Pesquisa em Administração. Porto Alegre 7a ed Bookman. 2008.
\newline \newline 
FOWLER, F.J. Pesquisa de Levantamento. Porto Alegre: Pearson, 2011.
\newline \newline 
FREIRE, P. Extensão ou Comunicação.13. ed. Paz e Terra, 2006..
CHAUÍ, Marilena. Convite à Filosofia. 14 ed. Ática, 2011.
}


\DetalhesDisciplina{
    QXD0041 - Projeto e Análise de Algoritmos
}{% EMENTA
    Noções de análise de algoritmos: análise assintótica de pior caso e caso
médio; notação big-O, little-o, ômega e teta; principais classes de complexidade; medida empírica de performance; análise de algoritmos recursivos
utilizando relações de recorrência. Projeto de algoritmos: força bruta; gulosos; divisão e conquista; programação dinâmica. Algoritmos em grafos: grafos não-direcionados e direcionados; árvores; conectividade; árvores/florestas geradoras; ordenação topológica; caminho mais curto. NPcompletude: definição das classes P e NP; teorema de Cook; principais
problemas NP-completos; técnicas de redução.
}{% BIBLIOGRAFIA BÁSICA
    CORMEN, Thomas. Algoritmos - Teoria e Prática. São Paulo: GEN LTC,
2012. E-book. Disponível em: https://integrada.minhabiblioteca.
com.br/\#/books/9788595158092. Acesso em: 3 de out. 2022.
\newline \newline 
DASGUPTA, Sanjoy; PAPADIMITRIOU, Christos; VAZIRANI, Umesh.
Algoritmos.. São Paulo: AMGH, 2009. E-book. Disponível em: https://
integrada.minhabiblioteca.com.br/\#/books/9788563308535. Acesso
em: 3 de out. 2022.
\newline \newline 
ZIVIANI, Nivio. Projeto de Algoritmos: com implementações em JAVA
e C++. São Paulo: Cengage Learning Editores SA de CV, 2012. Ebook. Disponível em: https://integrada.minhabiblioteca.com.br/\#/
books/9788522108213. Acesso em: 3 de out. 2022.
}{% BIBLIOGRAFIA COMPLEMENTAR
    KLEINBERG, Jon; TARDOS, Éva. Algorithm design. Boston, Massachusetts: Pearson/Addison Wesley, c2006. 838 p. ISBN 0321295358.
\newline \newline 
GERSTING, Judith L.. Fundamentos Matemáticos para a Ciência da
Computação. São Paulo: LTC, 2016. E-book. Disponível em: https://
integrada.minhabiblioteca.com.br/\#/books/9788521633303. Acesso
em: 3 de out. 2022.
\newline \newline 
MENEZES, Paulo Blauth. Matemática Discreta para Computação e Informática - V16 - UFRGS. São Paulo: Bookman, 2013. E-book. Disponível em: https://integrada.minhabiblioteca.com.br/\#/books/
9788582600252. Acesso em: 3 de out. 2022.
\newline \newline 
ROSEN, Kenneth H.. Matemática Discreta e suas Aplicações. São
Paulo: ArtMed, 2010. E-book. Disponível em: https://integrada.
minhabiblioteca.com.br/\#/books/9788563308399. Acesso em: 3 de
out. 2022.
\newline \newline 
GOLDBARG, Marco Cesar; GOLDBARG, Elizabeth. Grafos: conceitos,
algoritmos e aplicações. Rio de Janeiro, RJ: Elsevier, 2012. 622 p. ISBN
9788535257168
\newline \newline 
TOSCANI, Laira V.; VELOSO, Paulo A. S. Complexidade de algoritmos:
análise, projeto e métodos. 3. ed. Porto Alegre: Sagra Luzzato, 2012. 262
p. (Serie Livros Didáticos Informática UFRGS ; 13). ISBN 9788540701380
(broch.).
}


\DetalhesDisciplina{
    <19> - Projeto Extensionista I
}{% EMENTA
    Introdução à extensão universitária e à política nacional de extensão. Diretrizes Nacionais da Extensão Universitária. Formação cidadã e transformação social. Planejamento e execução de ações em comunidades
de saberes.
}{% BIBLIOGRAFIA BÁSICA
    FREIRE, Paulo. Extensão ou comunicação. 13. ed. Rio de Janeiro: Paz e Terra, 2006. 93 p. (O mundo,
hoje, 24). ISBN 8521904274.
\newline \newline 
DEUS, Sandra de. Extensão universitária: trajetórias e desafios. Santa Maria, RS. Ed. PRE-UFSM, 2020.
96 p. ISBN: 978-65-87668-01-7
\newline \newline 
FORPROEX - FÓRUM DE PRÓ-REITORES DE EXTENSÃO DAS UNIVERSIDADES PÚBLICAS
BRASILEIRAS. Política Nacional de Extensão Universitária. Manaus, 2012. Disponível em:
<https://proex.ufsc.br/files/2016/04/Política-Nacional-de-Extensão-Universitária-e-book.pdf> Acesso em:
maio de 2025.
}{% BIBLIOGRAFIA COMPLEMENTAR
    COELHO, Geraldo Ceni. O papel pedagógico da extensão universitária. Revista Em Extensão,
Uberlândia, v. 13, n. 2, p. 11–24, 2015. DOI: 10.14393/REE-v13n22014\_art01. Disponível em:
https://seer.ufu.br/index.php/revextensao/article/view/26682. Acesso em: 12 maio. 2025.
\newline \newline 
LEMOS, Ronaldo.; DI FELICE, Massimo. A vida em rede. Campinas: Papirus, 2014. 142 p. (Papirus
debates). ISBN 9788561773618 (broch.).
\newline \newline 
YIN, Robert K.. Estudo de caso: planejamento e métodos. 4. ed. Porto Alegre: Bookman, 2010. xviii, 248 p.
ISBN 978-85-7780-655-3.
\newline \newline 
UNIVERSIDADE FEDERAL DO CEARÁ. Glossário: fundamentos da extensão universitária.
Fortaleza: Pró-Reitoria de Extensão, 2024. Disponível em: https://prex.ufc.br/wpcontent/uploads/2024/08/glossario-fundamentos-da-extensao-universitaria-240819-180754.pdf. Acesso em:
12 maio 2025.
\newline \newline 
ZITKOSKI, Jaime J.; STRECK, Danilo R.; REDIN, Euclides. Dicionário Paulo Freire. 2. ed. São Paulo:
Autêntica Editora, 2008. E-book. p.Cover. ISBN 9788582178089. Disponível em:
https://app.minhabiblioteca.com.br/reader/books/9788582178089/. Acesso em: 12 mai. 2025.
}


\DetalhesDisciplina{
    <20> - Projeto Extensionista II
}{% EMENTA
    Interdisciplinaridade e interprofissionalidade nas práticas extensionistas. Interação dialógica com comunidades de saberes. Formação discente integrada a contextos extensionistas. Transformação social. Indissociabilidade entre ensino, pesquisa e extensão. Práticas extensionistas.
}{% BIBLIOGRAFIA BÁSICA
    FREIRE, Paulo. Extensão ou comunicação. 13. ed. Rio de Janeiro: Paz e Terra, 2006. 93 p. (O mundo,
hoje, 24). ISBN 8521904274.
\newline \newline 
DEUS, Sandra de. Extensão universitária: trajetórias e desafios. Santa Maria, RS. Ed. PRE-UFSM, 2020.
96 p. ISBN: 978-65-87668-01-7
\newline \newline 
FORPROEX - FÓRUM DE PRÓ-REITORES DE EXTENSÃO DAS UNIVERSIDADES PÚBLICAS
BRASILEIRAS. Política Nacional de Extensão Universitária. Manaus, 2012. Disponível em:
<https://proex.ufsc.br/files/2016/04/Política-Nacional-de-Extensão-Universitária-e-book.pdf> Acesso em:
maio de 2025.
}{% BIBLIOGRAFIA COMPLEMENTAR
    ""COELHO, Geraldo Ceni. O papel pedagógico da extensão universitária. Revista Em Extensão,
Uberlândia, v. 13, n. 2, p. 11–24, 2015. DOI: 10.14393/REE-v13n22014\_art01. Disponível em:
https://seer.ufu.br/index.php/revextensao/article/view/26682. Acesso em: 12 maio. 2025.
\newline \newline 
LEMOS, Ronaldo.; DI FELICE, Massimo. A vida em rede. Campinas: Papirus, 2014. 142 p. (Papirus
debates). ISBN 9788561773618 (broch.).
\newline \newline 
YIN, Robert K.. Estudo de caso: planejamento e métodos. 4. ed. Porto Alegre: Bookman, 2010. xviii, 248 p.
ISBN 978-85-7780-655-3.
\newline \newline 
UNIVERSIDADE FEDERAL DO CEARÁ. Glossário: fundamentos da extensão universitária.
Fortaleza: Pró-Reitoria de Extensão, 2024. Disponível em: https://prex.ufc.br/wpcontent/uploads/2024/08/glossario-fundamentos-da-extensao-universitaria-240819-180754.pdf. Acesso em:
12 maio 2025.
\newline \newline 
ZITKOSKI, Jaime J.; STRECK, Danilo R.; REDIN, Euclides. Dicionário Paulo Freire. 2. ed. São Paulo:
Autêntica Editora, 2008. E-book. p.Cover. ISBN 9788582178089. Disponível em:
https://app.minhabiblioteca.com.br/reader/books/9788582178089/. Acesso em: 12 mai. 2025.""
}


\DetalhesDisciplina{
    QXD0126 - Psicologia e Percepção
}{% EMENTA
    O ser humano em sua relação com o mundo. Processos psicológicos
relacionados à percepção, sensação e cognição. Aprendizagem, memória,
motivação, emoção e linguagem.
}{% BIBLIOGRAFIA BÁSICA
    ARNHEIM, Rudolf. Arte e percepção visual: uma psicologia da visão criadora:
nova versão. São Paulo: Cengage Learning, 2017. 509 p. ISBN
9788522126002 (broch.)
\newline \newline 
GOMES FILHO, João. Gestalt do objeto: sistema de leitura visual da forma.
9.ed. São Paulo: Escrituras, 2009. ISBN: 9788586303579
\newline \newline 
MORRIS, Charles G. e MAISTO, Albert A. Introdução à psicologia. 6ª edição.
São Paulo, 2004. ISBN: 9788587918680
}{% BIBLIOGRAFIA COMPLEMENTAR
    FELDMAN, Robert S. Introdução à Psicologia. Porto Alegre: Grupo A, 2015.
E-book. ISBN 9788580554892. Disponíví el em:
https://app.minhabiblioteca.com.br/\#/books/9788580554892/. Acesso em:
26 set. 2022.
\newline \newline 
BOCK, Ana Mercês B.; FURTADO, Odair; TEIXEIRA, Maria de Lourdes T.
Psicologias: uma introdução ao estudo de psicologia. São Paulo: Editora
Saraiva, 2018. E-book. ISBN 9788553131327. Disponíví el em:
https://app.minhabiblioteca.com.br/\#/books/9788553131327/. Acesso em:
26 set. 2022.
\newline \newline 
DA MAIA, Gabriela Felten; FORECHI, Marcilene; LOPES, Daiane D.; et al.
Comunicação e Psicologia. Porto Alegre: Grupo A, 2020. E-book. ISBN
9786581492960. Disponíví el em:
https://app.minhabiblioteca.com.br/\#/books/9786581492960/. Acesso em:
26 set. 2022.
\newline \newline 
FREUD, Sigmund. Psicologia das massas e análise do eu e outros textos
(1920-1923). São Paulo: Companhia das Letras, 2011. 343 p. (Obras
completas ; v. 15). ISBN 9788535918717 (broch.).
\newline \newline 
BARBOSA, Livia. Sociedade de consumo. 2. ed. Rio de Janeiro: Zahar, 2008.
((Ciências Sociais Passo-a-Passo ; 49).). ISBN 9788571108134.
}


\DetalhesDisciplina{
    <26> - Raciocínio sob Incerteza
}{% EMENTA
    Tipos de incerteza; Teoria de probabilidade (revisão), raciocínio probabilístico, redes bayesianas, lógica probabilística, lógica fuzzy; Raciocínio Não-Monotônico, programação em lógica, raciocínio abdutivo; Problemas de decisão sob incerteza.
}{% BIBLIOGRAFIA BÁSICA
    LARSON, Ron; FARBER, Betsy. Estatística aplicada. 4. ed. São Paulo, SP: Pearson/ Prentice Hall, 2010. xiv,637 p. ISBN 9788576053729 (broch.).
\newline \newline 
COPPIN, Ben. Inteligencia artificial. Rio de Janeiro, RJ: LTC, 2010. 636 p. ISBN 9788521617297.
\newline \newline 
RUSSELL, Stuart J.; NORVIG, Peter. Inteligência artificial. Rio de Janeiro: Elsevier, Campus, 2013. 988 p. ISBN 9788535237016 (broch.).
}{% BIBLIOGRAFIA COMPLEMENTAR
    Lukasiewicz T. Probabilistic description logic programs. International Journal of Approximate Reasoning. 2007 Jul 1;45(2):288-307. Disponível em: https://www.sciencedirect.com/science/article/pii/S0888613X06000648    
\newline \newline 
LARSON, Ron; FARBER, Betsy. Estatística aplicada. 4. ed. São Paulo, SP: Pearson/ Prentice Hall, 2010. xiv,637
p. ISBN 9788576053729 (broch.)                                                                
\newline \newline 
BARBETTA, Pedro Alberto; REIS, Marcelo Menezes; BORNIA, Antonio Cezar. Estatística para cursos de
engenharia e informática. 3. ed. São Paulo, SP: Atlas, 2010. 410 p    
\newline \newline 
MORETTIN, Luiz Gonzaga. Estatística básica:: probabilidade e inferência / volume único. São Paulo, SP:
Pearson Educational do Brasil. 2010. ISBN 8576053705 ISBN-13 9788576053705 (broch.)                                                                       
\newline \newline 
WALPOLE, Ronald E. Probabilidade e estatística: para engenharia e ciências. 8. ed. São Paulo, SP: Pearson/
Prentice Hall, 2009. xiv, 491 p. ISBN 9788576051992 (broch.)                            
}


\DetalhesDisciplina{
    QXD0177 - Recuperação de Informação
}{% EMENTA
    Avaliação em recuperação de Informação; Modelos clássicos de recuperação (Booleano, Vetorial e Probabilístico); Operações sobre texto, indexação e consultas (pré-processamento); Mineração de texto; Indexação de textos na Web; Web Search; Web Crawling; Algoritmos baseados em links.
}{% BIBLIOGRAFIA BÁSICA
    BAEZA-YATES, Ricardo; RIBEIRO-NETO, Berthier. Recuperação de Informação. 2. ed. Porto Alegre: Bookman, 2013. E-book. p.1. ISBN 9788582600498. Disponível em: https://app.minhabiblioteca.com.br/reader/books/9788582600498/.
\newline \newline 
MANNING, Christopher D.; RAGHAVAN, Prabhakar; SCHÜTZE, Hinrich. Introduction to information retrieval. New York, NY: Cambridge at the University Press, 2009. xxi, 482 p. ISBN 9780521865715.
\newline \newline 
ELMASRI, R.; NAVATHE, S. B. Sistemas de banco de dados. 6 ed. Pearson/Addison-Wesley, 2011. ISBN: 9788579360855.

}{% BIBLIOGRAFIA COMPLEMENTAR
    SILBERSCHATZ, A.; SUDARSHAN, S. Sistema de banco de dados. Campus, 2006. ISBN: 9788535211078.
\newline \newline 
LARSON, Ron; FARBER, Betsy. Estatística aplicada. 4. ed. São Paulo, SP: Pearson/ Prentice Hall, 2010. xiv,637 p. ISBN 9788576053729 (broch.).
\newline \newline 
RAMAKRISHNAN, R.; GEHRKE, J.Sistemas de gerenciamento de banco de dados. McGrawHill, 2008. DOMINICH, Sendor SPRINGERLINK (ONLINE SERVICE). 
\newline \newline 
The Modern Algebra of Information Retrieval. Springer eBooks Berlin, Heidelberg: Springer-Verlag Berlin Heidelberg, 2008. (The Information Retrieval Series, 24) ISBN 9783540776598. Disponível em : <http://dx.doi.org/10.1007/978-3-540-77659-8>. Acesso em : 21 set. 2010.
\newline \newline 
HARPER, Brian D.; MERIAM, J. L; KRAIGE, L. G. Solving statistics problems in MATLAB: engineering mechanics: statics. 6th ed. Massachusetts, [Estados Unidos]: J. Wiley \& Sons, 2007. 139 p. ISBN 9780470099254 (broch.).
\newline \newline 
BARBETTA, Pedro Alberto; REIS, Marcelo Menezes; BORNIA, Antonio Cezar. Estatística para cursos de engenharia e informática. 3. ed. São Paulo, SP: Atlas, 2010. 410 p.
\newline \newline 
GROSSMAN, David A.; FRIEDER, Ophir. Information Retrieval: Algorithms and Heuristics. Springer, 2ª Ed., 2004. ISBN: 1402030045.
\newline \newline 
BUETTCHER, Stefan; Clarke, Charles L. A.; Cormack, Gordon V. Information Retrieval: Implementing and Evaluating Search Engines. MIT Press; 1ª Ed., 2010.
\newline \newline 
MORETTIN, Luiz Gonzaga. Estatística básica: probabilidade e inferência / volume único. São Paulo, SP: Pearson Educational do Brasil. 2010. ISBN 8576053705 ISBN-13 9788576053705.
}


\DetalhesDisciplina{
    QXD0021 - Redes de Computadores
}{% EMENTA
    Organização das redes de computadores. Modelos de referência OSI e
TCP/IP. Padrões de rede. Meios físicos de transmissão. Protocolos de acesso
ao meio. Interconexão de redes. Algoritmos e protocolos de roteamento.
Protocolos de redes. Protocolos de transporte TCP e UDP. Protocolos de
aplicação. Projeto e Dimensionamento de Redes.
}{% BIBLIOGRAFIA BÁSICA
    COMER, D. Redes de computadores e a internet. 4 ed. Bookman, 2007.ISBN:
9788560031368.
\newline \newline 
KUROSE, J.; ROSS, K. W. Redes de computadores e a Internet: uma
abordagem top-down, 6 ed. Pearson, 2013. ISBN: 8581436773.
\newline \newline 
FOROUZAN, Behrouz A. Comunicação de dados e redes de computadores. 4.
ed. São Paulo, SP: McGraw-Hill, 2008. xxxiv, 1134 p.
\newline \newline 
FOROUZAN, B. A.; FIROUZ, M. Redes de Computadores: Uma Abordagem
Top-down. 1a ed. 2013. McGraw-Hill. ISBN: 9788580551686
}{% BIBLIOGRAFIA COMPLEMENTAR
    TANENBAUM, Andrew S.; WETHERALL, D. Redes de computadores. 5. ed.
São Paulo: Pearson Prentice Hall, c2011. xvi, 582 p. ISBN 9788576059240.
\newline \newline 
COMER, D. Interligação de redes com TCP/IP.1. 5 ed.(vol.1). Campus, 2006.
ISBN: 9788535220179.
\newline \newline 
OLIVEIRA, Gorki Starlin da Costa. Redes de computadores comunicações de
dados TCP/IP : conceitos, protocolos e usos. Alta Books, 2004. ISBN :
8576080567
\newline \newline 
TORRES, G. Redes de computadores. Nova Terra, 2009.
\newline \newline 
MORIMOTO, C.E. Redes: guia prático. GDH Press, 2009. ISBN
9788599593097 (broch.).
\newline \newline 
PETERSON, Larry L.; DAVIE, Bruce S. Computer networks: a systems
approach . 5th ed. Amesterdam: Elsevier, c2012. xxxi, 884 p. ISBN
9780123850591
}


\DetalhesDisciplina{
    QXD0246 - Relações Étnico-Raciais e Africanidades
}{% EMENTA
    Negritude e pertencimento étnico. Conceitos de africanidades e afrodescendência. Cosmovisão africana: valores civilizatórios africanos presentes
na cultura brasileira. Ancestralidade e ensinamentos das religiosidades
tradicionais africanas nas diversas dimensões do conhecimento no Brasil.
Introdução à geografia e história da África. As origens africanas e as nações africanas representadas no Brasil. O sistema escravista no Brasil e no
Ceará. Aportes dos africanos à formação social e cultural do Brasil e do
Ceará. Personalidades africanas, afrodescendentes e da diáspora negra que
se destacaram em diferentes áreas do conhecimento. Contexto das Ações
Afirmativas hoje. Atualização do legado africano no Brasil. Desconstrução de preconceitos e desdobramentos teórico-práticos para a atuação do
profissional na sua área de inserção no mercado de trabalho.
}{% BIBLIOGRAFIA BÁSICA
    VAZZOLER por MUNANGA, Kabengele. Negritude - Nova Edição.
[Digite o Local da Editora]: Grupo Autêntica, 2019. E-book. ISBN 9788551306529. Disponível em: https://app.minhabiblioteca.com.
br/\#/books/9788551306529. Acesso em: 23 set. 2022.
OLIVA, Anderson R.; CHAVES, Marjorie N.; FILICE, Renísia Cristina G.;
\newline \newline 
NASCIMENTO, Wan. Tecendo redes antirracistas. [Digite o Local da Editora]: Grupo Autêntica, 2019. E-book. ISBN 9788551304877. Disponível
em: https://app.minhabiblioteca.com.br/\#/books/9788551304877.
Acesso em: 23 set. 2022.
\newline \newline 
ALMEIDA, Guilherme Assis de; CHRISTMANM, Martha Ochsenhofer.
Ética e Direito: Uma Perspectiva Integrada, 3a edição. São Paulo:
Grupo GEN, 2009. E-book. Disponível em: https://integrada.
minhabiblioteca.com.br/\#/books/9788522467150. Acesso em: 3 de
out. 2022.
}{% BIBLIOGRAFIA COMPLEMENTAR
    BRASIL. Síntese de indicadores Sociais: Uma análise das condições de vida
da população brasileira. Ministério do Planejamento, Orçamento e Gestão.
IBGE. Rio de Janeiro, 2013, 266p. Disponível em: https://biblioteca.
ibge.gov.br/visualizacao/livros/liv66777.pdf
\newline \newline 
CUNHA JUNIOR, H. Abolição inacabada e a educação dos afrodescendentes. Revista Espaço Acadêmico, no. 89, 2008. Disponível em:
http://www.espacoacademico.com.br/089/89cunhajr.pdf
\newline \newline 
ROMÃO, J. História da educação do negro e outras histórias. Brasília: Ministério da Educação, Secretaria de Educação Continuada, Alfabetização
e Diversidade. 2005, 278p. Disponível em: http://etnicoracial.mec.
gov.br/images/pdf/publicacoes/historia\\_educacao\\_negro.pdf
\newline \newline 
BRASIL. Constituição da República Federativa do Brasil. Biblioteca Digital da Câmara dos Deputados. 35a ed., 2012, 446p.
\newline \newline 
BRASIL. CNE. Parecer no. 03 de 10 de março de 2004. Dispõe sobre as diretrizes curriculares nacionais para a educação das relações
étnico-raciais e para o ensino de história e cultura afrobrasileira e africana. Relatora: Petronilha Beatriz Gonçalves e Silva. Ministério da
Educação. Brasília, julho de 2004. Disponível em: https://www.gov.
br/inep/pt-br/centrais-de-conteudo/acervo-linha-editorial/
publicacoes-diversas/temas-interdisciplinares/
diretrizes-curriculares-nacionais-para-a-educacao-das-relacoes-etnico-raciais-e-para-o-ensino-de-historia-e-cultura-afro-brasileira-e-africana
\newline \newline 
SANTIAGO, R. A história da educação do negro no Brasil: interdição institucional à escolarização pelo poder e seus reflexos no século XXI. Revista da ABPN. v.5, n.10, p.196-203, 2013. Disponível em: https://www.ipea.gov.br/portal/images/stories/PDFs/
politicas\\_sociais/bps\\_19\\_cap08.pdf
}


\DetalhesDisciplina{
    <04> - Representação do Conhecimento e Raciocínio
}{% EMENTA
    Lógica de primeira ordem (revisão); Representação de conhecimento em FOL: vocabulário/léxico, fatos simples, fatos compostos, consequências lógicas, fechamento dedutivo. Problema do Quadro. Cláusulas de Horn. Resolução. Árvores de Objetivos. Introdução ao Raciocínio Não-Monotônico.
}{% BIBLIOGRAFIA BÁSICA
    Ronald Brachman and Hector Levesque. 2004. Knowledge Representation and Reasoning. Morgan Kaufmann Publishers Inc., San Francisco, CA, USA.
\newline \newline 
COPPIN, Ben. Inteligencia artificial. Rio de Janeiro, RJ: LTC, 2010. 636 p. ISBN 9788521617297.
\newline \newline 
RUSSELL, Stuart J.; NORVIG, Peter. Inteligência artificial. Rio de Janeiro: Elsevier, Campus, 2013. 988 p. ISBN 9788535237016 (broch.).
}{% BIBLIOGRAFIA COMPLEMENTAR
    SILVA, Flávio Soares Corrêa da; FINGER, Marcelo; MELO, Ana Cristina Vieira de. Lógica para computação.
São Paulo, SP: Thomson Learning, 2006. 234 p. ISBN 8522105170 (broch)                                                                                                        
\newline \newline 
HUTH, Michael; RYAN, Mark. . Lógica em ciência da computação: modelagem e argumentação sobre sistemas .
2. ed. Rio de Janeiro: LTC, 2008. 322 p. ISBN 9788521616108 (broch.)
\newline \newline 
ALENCAR FILHO, Edgard de. Iniciação à lógica matemática. São Paulo: Nobel, [2002]. 203 p. ISBN
852130403X (broch)                                                                      
\newline \newline 
ENDERTON, Herbert B. A mathematical introduction to logic. 2nd ed. San Diego, California:
Harcourt/Academic Press, c2001. 317 p. ; ISBN 0122384520 (enc.)  
\newline \newline 
GERSTING, Judith L. Fundamentos matemáticos para a ciência da computação: um tratamento moderno
de matemática discreta . 5. ed. Rio de Janeiro: Livros Técnicos e Científicos, 2004. 597 p. ISBN 8521614225
(broch.).
}


\DetalhesDisciplina{
    QXD0069 - Segurança
}{% EMENTA
    Ameaças. Segurança como atributo qualitativo de projeto de software.
Autenticação. Autorização. Integridade. Confidencialidade. Criptografia (chaves
simétricas e assimétricas). Infraestrutura de chaves públicas brasileiras
(ICP-Brasil). Certificados digitais. Assinaturas digitais. Desenvolvimento de
software seguro. Noções de auditoria de sistemas. Norma NBR 27002.
}{% BIBLIOGRAFIA BÁSICA
    IMONIANA, Joshua Onome. Auditoria de sistemas de informação. 2. ed. São
Paulo: Atlas, 2008. 207 p. ISBN 9788522450022 (broch.).
\newline \newline 
STALLINGS, William. Criptografia e segurança de redes: princípios e práticas.
4. ed. São Paulo: Pearson/ Prentice Hall, 2008. 492 p. ISBN 9788576051190
(broch.).
\newline \newline 
BEAL, Adriana. Segurança da informação: princípios e melhores práticas para
a proteção dos ativos de informação nas organizações. São Paulo, SP: Atlas,
2008. 175 p. ISBN 9788522440856 (broch.).
}{% BIBLIOGRAFIA COMPLEMENTAR
    DASWANI, Neil; KERN, Christoph; KESAVAN, Anita. Foundations of security:
what every programmer needs to know . Berkeley, Ca: Apress, 2007. 290 p.
(The Expert’s voice in security) ISBN 9781590597842 (broch.).
\newline \newline 
KUROSE, James F.; ROSS, Keith W. Redes de computadores e a Internet:
uma abordagem top-down. 5. ed. São Paulo: Pearson Addison Wesley, 2010.
xxii, 614 p. ISBN 9788588639973 (broch.).
\newline \newline 
NAKAMURA, Emilio Tissato; GEUS, Paulo Lício de. Segurança de redes em
ambientes cooperativos. São Paulo: Novatec, c2007. ISBN 9788575221365
(broch.).
\newline \newline 
STATO FILHO, André. Linux: controle de redes. Florianópolis: Visual Books,
2009. 352 p. ISBN 9788575022443 (broch.).
\newline \newline 
ULBRICH, Henrique Cesar; DELLA VALLE, James. Universidade H4CK3R:
desvende todos os segredos do submundo dos hackers . 6. ed. São Paulo:
Digerati Books, 2009. 348p. (Série Universidade) ISBN 9788578730529
(broch).
\newline \newline 
ASSOCIAÇÃO BRASILEIRA DE NORMAS TÉCNICAS. ABNT NBR ISO/IEC
27001- Tecnologia da informação - técnicas de segurança - sistemas de
gestão de segurança da informação - requisitos. Rio de Janeiro, RJ, 2006. 34
p.
\newline \newline 
ASSOCIAÇÃO BRASILEIRA DE NORMAS TÉCNICAS. ABNT NBR ISO/IEC
27002- Tecnologia da informação - técnicas de segurança - código de prática
para a gestão da segurança da informação. Rio de Janeiro, RJ, 2005. 120 p.
ISBN 9788507006480.
}

\DetalhesDisciplina{
    QXD0043 - Sistemas Distribuídos
}{% EMENTA
    Introdução: caracterização de sistemas de computação distribuída; aplicações distribuídas (caracterização e aspectos de projeto); objetivos básicos de sistemas distribuídos (transparência, abertura, escalabilidade, etc.).
Modelos de sistemas distribuídos: sistemas cliente/servidor e sistemas multicamadas; sistemas peer-to-peer. Objetos distribuídos: interface versus
implementação; objetos remotos; chamadas de métodos remotos (RMI).
Processos em sistemas distribuídos: threads e seu uso em sistemas distribuídos; processos clientes e processos servidores; noções de código móvel
e agentes de software. Sincronização e Coordenação: o conceito de tempo
em sistemas distribuídos; consenso; exclusão mútua distribuída; eleição.
}{% BIBLIOGRAFIA BÁSICA
    MONTEIRO, Eduarda Rodrigues; JUNIOR, Ronaldo C. Mengato; LIMA,
Bruno Santos de et al. Sistemas Distribuídos. São Paulo: SAGAH, 2020.
E-book. Disponível em: https://integrada.minhabiblioteca.com.br/
\#/books/9786556901978. Acesso em: 29 de set. 2022.
\newline \newline 
COULOURIS, G. F.; DOLLIMORE, J.; KINDBERG, T. Sistemas distribuídos: conceitos e projetos. 4 ed. Bookman, 2007. ISBN: 9788560031498.
\newline \newline 
TANENBAUM, A.; STEEN, V. M. Sistemas distribuídos: princípios e
paradigmas. 2 ed. Prentice Hall, 2007. ISBN: 9788576051428.
}{% BIBLIOGRAFIA COMPLEMENTAR
    WHITE, T. Hadoop: the definitive guide. O’Reilly Media, 2009.
\newline \newline 
TANENBAUM, A. S. Sistemas operacionais modernos. 2 ed. Prentice
Hall, 2003.
\newline \newline 
ALONSO, G.; CASATI, F.; KUNO, K.; MACHIRAJU, V. Web Services: Concepts, Architectures and Applications. Springer, 2004. ISBN:
9783540440086
\newline \newline 
NAKAMURA, E. T.; GEUS, P.L. Segurança de redes em ambientes cooperativos. Novatec, 2007. ISBN: 9788575221365.
\newline \newline 
ERL, THOMAS. SOA. Princípios de design de serviço. Prentice Hall, 2009.
ISBN: 9788576051893.
}


\DetalhesDisciplina{
    QXD0076 - Sistemas Multiagentes
}{% EMENTA
    Agentes inteligentes: Conceitos, modelos e arquiteturas; Agentes reativos;
Agentes Deliberativos. Fundamentos da Inteligência Artificial Distribuída e, em
especial, dos Sistemas Multiagentes. Aspectos de comportamento emergente,
comunicação, negociação e coordenação entre agentes. Metodologias de
Desenvolvimento e Arquiteturas de Sistemas Multiagentes. Apresentação de
aplicações existentes e Utilização de Plataformas para o desenvolvimento de
Sistemas Multiagentes.
}{% BIBLIOGRAFIA BÁSICA
    WOOLDRIDGE, Michael J. An introduction to multiagent systems. 2. ed. New
York: J. Wiley \& Sons, 2009. 461 p. ISBN 9780470519462 (broch.).
\newline \newline 
RUSSELL, Stuart J.; NORVIG, Peter. Inteligência artificial. Rio de Janeiro:
Elsevier, Campus, 2013. 988 p. ISBN 9788535237016 (broch.).
\newline \newline 
FOWLER, Martin. UML essencial: um breve guia para a linguagem-padrão de
modelagem de objetos . 3. ed. Porto Alegre: Bookman, 2005. 160 p. ISBN
8536304545 (broch.)
}{% BIBLIOGRAFIA COMPLEMENTAR
    DEITEL, Paul J.; DEITEL, Harvey M. Java: como programar. 8. ed. São Paulo:
Pearson Prentice Hall, 2010. 1144 p. ISBN 9788576055631 (broch.).
\newline \newline 
BLAHA, Michael; RUMBAUGH, James. Modelagem e projetos baseados em
objetos com UML 2. 2.ed. rev. e atual. Rio de Janeiro, RJ: Campus; Elsevier,
2006. xvii, 496 p. ISBN 9788535217537.
\newline \newline 
FIPA. FIPA Communicative Act Library Specification. Foundation for Intelligent
Physical Agents (FIPA), 2002. Disponível em:
<www.fipa.org/specs/fipa00037/XC00037H.pdf>. Acesso em: 24 jan. 2013.
\newline \newline 
SHOHAM, Yoav; LEYTON-BROWN, Kevin. Multiagent systems: algorithmic,
game-theoretic, and logical foundations . New York, NY: Cambridge at the
University Press, 2009. xx, 483 p.
\newline \newline 
MEYER, Bertrand. Object - oriented software construction. 2nd. ed. New
Jersey: Prentice Hall PTR, 1997. 1254p ISBN 0136291554.
}


\DetalhesDisciplina{
    QXD0013 - Sistemas Operacionais
}{% EMENTA
    O histórico, o conceito e os tipos de sistemas operacionais. A estrutura de
sistemas operacionais. Conceito de processo. Gerência de processador:
escalonamento de processos, Concorrência e sincronização de processos.
Alocação de recursos e deadlocks. Gerenciamento de memória. Memória
virtual. Gerenciamento de arquivos. Gerenciamento de dispositivos de
entrada/saída.
}{% BIBLIOGRAFIA BÁSICA
    TANENBAUM, A. S. Sistemas Operacionais Modernos. São Paulo: 3a Edição,
Prentice Hall, 2010.
\newline \newline 
SILBERSCHATZ, Abraham; GAGNE, Greg; GALVIN, Peter; Fundamentos de
Sistemas Operacionais. Rio de Janeiro. 8a Edição. LTC. 2010.
\newline \newline 
OLIVEIRA, Rômulo Silva de; CARISSIMI, Alexandre da Silva; TOSCANI,
Simão Sirineo; UFRGS. Sistemas operacionais. 4. ed. Porto Alegre: Bookman,
2010. 374 p. (Livros didáticos. 11).
}{% BIBLIOGRAFIA COMPLEMENTAR
    FERREIRA, R. E. Linux: guia do administrador do sistema. 2 ed. Novatec,
2008. ISBN: 9788575221778.
\newline \newline 
MACHADO, F.B. Arquitetura de sistemas operacionais. 4 ed. LTC, 2007.
\newline \newline 
SILBERSCHATZ, A.; GALVIN, P.B.; GAGNE, G. Sistemas Operacionais com
Java. 7 ed. ​Campus,​ 2008. ISBN:9788535224061.
\newline \newline 
TANENBAUM, A. S. Sistemas operacionais: projeto e implementação. 3 ed.
Prentice Hall, 2008.
\newline \newline 
MENASCÉ, D.; ALMEIDA, V. Planejamento de capacidade para serviços na
web. ​Campus​, 2002.
}


\DetalhesDisciplina{
    QXD0162 - Sociedade, Culturas e Tecnologia
}{% EMENTA
    A Sociedade da informação. Tecnologias e sociabilidade. As tecnologias da
informação e a sociedade: aspectos econômicos, polítíicos e sociais. Internet:
informação, diversão e interação. Ciberespaço e interatividade. Novas tecnologias, novas mídí ias. Copyleft. Indústria cultural e indústria de jogos
eletrônicos.
}{% BIBLIOGRAFIA BÁSICA
    JENKINS, Henry. FORD, Sam. GREEN, Joshua. Cultura da Conexão: criando
valor e significado por meio da mídí ia propagável. 2014. São Paulo. Ed. Aleph.
ISBN: 8576571625.
\newline \newline 
RECUERO, Raquel. A conversação em rede: comunicação mediada pelo
computador e redes sociais na internet. 2012. Florianópolis. Ed. Sulina. ISBN
852050650X
\newline \newline 
SANTAELLA, Lucia. Comunicação Ubíqí ua. 2013. São Paulo. Ed. Paulus. ISBN
8534936374.
}{% BIBLIOGRAFIA COMPLEMENTAR
    MALDONADO, Tomás. Cultura, sociedade e técnica. São Paulo: Editora
Blucher, 2012. E-book. ISBN 9788521206521. Disponíví el em:
https://app.minhabiblioteca.com.br/\#/books/9788521206521/. Acesso em:
26 set. 2022.
\newline \newline 
BATISTA, Sueli Soares dos S.; FREIRE, Emerson. Sociedade e Tecnologia na Era Digital. São Paulo: Editora Saraiva, 2014. E-book. ISBN 9788536522531.
Disponíví el em:
https://app.minhabiblioteca.com.br/\#/books/9788536522531/. Acesso em:
26 set. 2022.
\newline \newline 
LIPOVETSKY, Gilles. A sociedade da sedução: democracia e narcisismo na
hipermodernidade liberal. Barueri: Editora Manole, 2019. E-book. ISBN
9788520459317. Disponíví el em:
https://app.minhabiblioteca.com.br/\#/books/9788520459317/. Acesso em:
26 set. 2022.
\newline \newline 
VIGLIAR, José Marcelo M. LGPD e a Proteção de Dados Pessoais na
Sociedade em Rede. [Digite o Local da Editora]: Grupo Almedina (Portugal),
2022. E-book. ISBN 9786556276373. Disponíví el em:
https://app.minhabiblioteca.com.br/\#/books/9786556276373/. Acesso em:
26 set. 2022.
\newline \newline 
SHIRKY, Clay. A cultura da participação : criatividade e generosidade no
mundo conectado. Rio de Janeiro: Zahar, 2001. ISBN 9788537805183.
Disponívíel em: <http://lectio.com.br/dashboard/midia/detalhe/1374>
}


\DetalhesDisciplina{
    QXD0046 - Teoria da Computação
}{% EMENTA
    Modelos computacionais universais: máquinas de Turing e funções recursivas. Tese de Church-Turing. Computabilidade. Problemas indecidíveis.
Introdução à complexidade computacional de problemas: complexidade de
tempo e espaço. Tópicos especiais em Teoria da Computação.
}{% BIBLIOGRAFIA BÁSICA
    SIPSER, M. Introdução à teoria da computação. 2 ed. Thompson Learning, 2007. ISBN: 9788522104994.
\newline \newline 
HOPCROFT, John E. Introdução à teoria de autômatos, linguagens e
computação. Rio de Janeiro: Elsevier,2003. ISBN 8535210725
\newline \newline 
DIVERIO, T.A. Teoria da computação: máquinas universais e computabilidade . 3 ed. Bookman, 2011. ISBN:9788577808243.
}{% BIBLIOGRAFIA COMPLEMENTAR
    CARNIELLI, W.; EPSTEIN, R. L. Computabilidade, Funções Computáveis, Lógica e os Fundamentos da Matemática. UNESP,2005. ISBN:
9788571398979.
\newline \newline 
ENDERTON, Herbert B. A mathematical introduction to logic. 2nd ed.
San Diego, California: Harcourt/Academic Press, c2001. xii, 317 p. ; ISBN
0122384520 (enc.) – ISBN 0122384520/9780122384523.
\newline \newline 
PAPADIMITROU, C. Computational complexity. Addison Wesley, 1994.
ISBN: 9780201530827.
\newline \newline 
MENEZES, Paulo Blauth. Linguagens formais e autômatos. 6. ed. Porto
Alegre: Bookman, 2011. 215 p. (Livros didáticos ; n.3 Série Livros Didáticos ; 3) ISBN: 9788577807659
\newline \newline 
BARKER-PLUMMER, Dave; BARWISE, Jon; ETCHEMENDY, John.
Language, proof and logic. CSLI, 2011. xiii 606 p. + 1 CD-ROM + 1
manual (CSLI lecture notes ; v 23) ISBN 9781575866321.
}


\DetalhesDisciplina{
    QXD0152 - Teoria dos Grafos
}{% EMENTA
    Grafos, subgrafos, grafos orientados, famílias de grafos. Árvores, caminhos,
ciclos. Conexidade. Grafos eulerianos. Grafos hamiltonianos.
Emparelhamento. Cliques e Conjuntos Independentes. Coloração de arestas.
Coloração de vértices. Grafos Perfeitos. Grafos planares.
}{% BIBLIOGRAFIA BÁSICA
    GOLDBARG, Marco Cesar; GOLDBARG, Elizabeth. ​Grafos: ​conceitos,
algoritmos e aplicações. Rio de Janeiro, RJ: Elsevier, 2012. 622 p. ISBN
9788535257168 (broch.).
\newline \newline 
BOAVENTURA NETTO, Paulo Oswaldo. ​Grafos: ​teoria, modelos, algoritmos.
5. ed. rev. e ampl. São Paulo, SP: Edgard Blücher, 2012. xiii, 310 p. ISBN
9788521206804 (broch.).
\newline \newline 
ROSEN, Kenneth H. ​Matemática discreta e suas aplicações. ​6. ed. São
Paulo: McGraw-Hill, c2009. xxi, 982 p. ISBN 9788577260362 (broch.).
}{% BIBLIOGRAFIA COMPLEMENTAR
    GERSTING, Judith L. Fundamentos matemáticos para a ciência da
computação: um tratamento moderno de matemática discreta . 5. ed. Rio de
Janeiro: Livros Técnicos e Científicos, c2004. xiv, 597 p. ISBN 8521614225
(broch.).
\newline \newline 
LIPSCHUTZ, Seymour; LIPSON, Marc. Teoria e problemas de matemática
discreta. 2. ed. Porto Alegre, RS: Bookman, 2004. 511p. (Coleção Schaum)
ISBN 9788536303611.
\newline \newline 
CORMEN, Thomas H. Algoritmos: teoria e prática. Rio de Janeiro: Elsevier,
2002. xvii , 916 p. ISBN 8535209263 (broch.).
\newline \newline 
DASGUPTA, Sanjoy; PAPADIMITRIOU, Christos H.; VAZIRANI, Umesh.
Algoritmos. São Paulo: McGraw-Hill, c2009. xiv, 320 p. ISBN 9788577260324
(broch.).
\newline \newline 
JUNGNICKEL, Dieter SPRINGERLINK (ONLINE SERVICE). Graphs, Networks
and Algorithms. Springer e-books Berlin, Heidelberg: Springer-Verlag Berlin
Heidelberg, 2008. (Algorithms and Computation in Mathematics, 5) ISBN
9783540727804. Disponível em:
<http://dx.doi.org/10.1007/978-3-540-72780-4>. Acesso em: 21 set. 2010.
}


\DetalhesDisciplina{
    <27> - Teoria dos Jogos
}{% EMENTA
    Introdução à teoria de decisão. Jogos. Equilíbrio de Nash. Jogos de soma zero. Jogos sequenciais. Jogos dinâmicos. Jogos com incerteza.

}{% BIBLIOGRAFIA BÁSICA
    HILLIER, Frederick S.; LIEBERMAN, Gerald J. Introdução à pesquisa operacional. 9. ed. Porto Alegre: AMGH, 2013. xxii, 1005 p. ISBN 978-85-8055-118-1.
\newline \newline 
OSBORNE, Martin J; RUBINSTEIN, Ariel. A course in game theory. Cambridge: Mit Press, c1994. 352 p. ISBN 9780262150415.
\newline \newline 
ROSEN, Kenneth H. Matemática discreta e suas aplicações. 6. ed. São Paulo: McGraw-Hill, c2009. xxi, 982 p. ISBN 9788577260362 (broch.).
}{% BIBLIOGRAFIA COMPLEMENTAR
    Prisoner’s Dilemma: John Von Neumann, Game Theory and the Puzzle of the
Bomb, by William Poundstone (1994 – MIT Press)
\newline \newline 
IEZZI, Gelson; MURAKAMI, Carlos. Fundamentos de matemática elementar: 1: conjuntos, funções. 9. ed. São Paulo: Atual, 2013. 410 p. ISBN 978-85-357-1680-1.
\newline \newline 
COPPIN, Ben. Inteligencia artificial. Rio de Janeiro, RJ: LTC, 2010. 636 p. ISBN 9788521617297.
\newline \newline 
RUSSELL, Stuart J.; NORVIG, Peter. Inteligência artificial. Rio de Janeiro: Elsevier, Campus, 2013. 988 p. ISBN 9788535237016 (broch.).
\newline \newline 
FIANI, Ronaldo. Teoria dos Jogos. 4. ed. Rio de Janeiro: GEN Atlas, 2015. E-book. p.i. ISBN 9788595156388. Disponível em: https://app.minhabiblioteca.com.br/reader/books/9788595156388/. Acesso em: 06 jun. 2025.
}


\DetalhesDisciplina{
    <28> - Tópicos Especiais em Aprendizado de Máquina
}{% EMENTA
    Estudos de tópicos especiais abordando Aprendizado Profundo OU Aprendizado por Reforço OU Aprendizado Não Supervisionado OU Aprendizado Auto-Supervisionado OU Modelos generativos OU Interpretação de Modelos de Aprendizado de Máquina OU Aprendizado de Máquina Automatizado OU Grandes Modelos de Linguagem OU MLOps
}{% BIBLIOGRAFIA BÁSICA
    COPPIN, Ben. Inteligência artificial. Rio de Janeiro: LTC, c2010. xxv, 636 p. ISBN 9788521617297.
HAYKIN, Simon. 
\newline \newline 
HAYKIN, Simon S.. Redes neurais: princípios e prática. 2. ed. Porto Alegre: Bookman, 2001. xvii, 900 p.
ISBN 9788573077186.
\newline \newline 
RUSSELL, Stuart J.; NORVIG, Peter. Inteligência artificial. Rio de Janeiro: Elsevier, Campus, 2004. 1021
p. ISBN 85-3521-177-2.

}{% BIBLIOGRAFIA COMPLEMENTAR
    LARSON, Ron; FARBER, Betsy. Estatística aplicada. 4. ed. São Paulo, SP: Pearson/ Prentice Hall, 2010. xiv,637 p. ISBN 9788576053729 (broch.).
\newline \newline 
DASGUPTA, Sanjoy; PAPADIMITRIOU, Christos H.; VAZIRANI, Umesh. Algoritmos. São Paulo:
McGraw-Hill, c2009. xiv, 320 p. ISBN 978-85-7726-032-4.
\newline \newline 
HARIOM, Tatsat,; SAHIL, Puri,; BRAD, Lookabaugh,. Blueprints de aprendizado de máquina e ciência de dados para finanças: desenvolvendo desde estratégias de trades até robôs Advisors com Python. Rio de Janeiro: Editora Alta Books, 2024. E-book. p.i. ISBN 9788550821726. Disponível em: https://app.minhabiblioteca.com.br/reader/books/9788550821726/. Acesso em: 30 mai. 2025.
\newline \newline 
HUTTER, Frank; KOTTHOFF, Lars; VANSCHOREN, Joaquin. Automated machine learning: methods, systems, challenges. Springer Nature, 2019. Dispnível em: https://link.springer.com/book/10.1007/978-3-030-05318-5 Acesso en: 30 mai. 2025.
\newline \newline 
AWAD, Mariette; KHANNA, Rahul. Efficient learning machines: theories, concepts, and applications for engineers and system designers. Berkeley, CA: Apress, 2015. DOI: https://doi.org/10.1007/978-1-4302-5990-9. 
}

\DetalhesDisciplina{
    <29> - Tópicos Especiais em Ciência de Dados
}{% EMENTA
    Tópicos especiais em Sistemas de Bancos de Dados OU Processamento de Dados OU Governança de Dados OU  Visualização de dados OU Mineração de Dados OU Engenharia de Dados e Big Data OU Detecção de Anomalias e Dados raros 
}{% BIBLIOGRAFIA BÁSICA
    TAN, Pang-Ning; STEINBACH, Michael; KUMAR, Vipin. Introdução ao DATAMINING:/ mineração de dados. Rio de Janeiro: Ciência Moderna, 2009. xxi, 900 p. ISBN 978-85-7393-761-9.
\newline \newline 
CARVALHO, André C. P. L. F de; MENEZES, Angelo G.; BONIDIA, Robson P. Ciência de Dados - Fundamentos e Aplicações. Rio de Janeiro: LTC, 2024. E-book. p.Capa. ISBN 9788521638766. Disponível em: https://app.minhabiblioteca.com.br/reader/books/9788521638766/ . Acesso em: 30 mai. 2025.
\newline \newline 
BEHRMAN, Kennedy R. Fundamentos de Python para ciência de dados. Porto Alegre: Bookman, 2023. E-book. p.i. ISBN 9788582605974. Disponível em: https://app.minhabiblioteca.com.br/reader/books/9788582605974/ . Acesso em: 30 mai. 2025.

}{% BIBLIOGRAFIA COMPLEMENTAR
    LARSON, Ron; FARBER, Elizabeth. Estatística aplicada. 4. ed. São Paulo: Pearson Prentice Hall, 2010.
xiv,637 p. ISBN 978-85-7605-372-9
\newline \newline 
MILANI, Alessandra M P.; SOARES, Juliane A.; ANDRADE, Gabriella L.; et al. Visualização de Dados. Porto Alegre: SAGAH, 2020. E-book. p.159. ISBN 9786556900278. Disponível em: https://integrada.minhabiblioteca.com.br/reader/books/9786556900278/ . Acesso em: 03 jun. 2025.
\newline \newline 
HAYKIN, Simon S.. Redes neurais: princípios e prática. 2. ed. Porto Alegre: Bookman, 2001. xvii, 900 p. ISBN 9788573077186.
\newline \newline 
RAMAKRISHNAN, Raghu; GEHRKE, Johannes. Database management systems. 3rd ed. Boston:
McGraw-Hill, 2003. xxxii, 1065 p. ISBN 13:978-0-07-246563-1.
\newline \newline 
MACHADO, Felipe Nery Rodrigues. Tecnologia e projeto de data warehouse: uma visão multidimensional.
5. ed. São Paulo, SP: Érica, 2010. 314 p. ISBN 9788536500126 (broch.).
\newline \newline 
CURRY, Edward et al. Technologies and Applications for Big Data Value. In: Technologies and Applications for Big Data Value. Cham: Springer International Publishing, 2022. p. 1-15. DOI: https://doi.org/10.1007/978-3-030-78307-5. Acesso em 3 de jun, 2025.
}


\DetalhesDisciplina{
    <30> - Tópicos Especiais em Filosofia, Ética e Humanidades
}{% EMENTA
    Conceito de ética;
O direito autoral no treinamento de IAs;
Relação entre a IA e a sociedade;
A ética do desenvolvedor e a ética da IA;
Consciência e inteligência;
O problema difícil da Inteligência Artificial.
}{% BIBLIOGRAFIA BÁSICA
    ALMEIDA, G. A.; CHRISTMANN, M. O. Ética e direito: uma perspectiva integrada.3 ed. Atlas, 2009. ISNN:9788522455072
\newline \newline 
MASIERO, P. C. Ética em computação. EDUSP, 2008 ISBN: 9788531405754
\newline \newline 
BARGER, R. N. Ética na computação: UMA ABORDAGEM BASEADA EM CASOS. LTC, 2011. ISBN: 9788521617761
}{% BIBLIOGRAFIA COMPLEMENTAR
    LIMBERGER, T. O direito à intimidade na era da informática. Livraria do Advogado, 2007.
\newline \newline 
CARBONI, G. C. O direito do autor na multimídia. Quartier Latim, 2003.
\newline \newline 
PAESANI, L. O direito na sociedade da informação III. Atlas, 2013.
\newline \newline 
LEVY, P. Tecnologias da inteligência: o futuro do pensamento na era da informática . 2. ed. Rio de Janeiro: Editora 34, 2010.
\newline \newline 
EPSTEIN, Richard G. The case of the killer robot: stories about the professional, ethical, and societal dimensions of computing. New York, NY: John Wiley \& Sons, 1997.
}


\DetalhesDisciplina{
    <31> - Tópicos Especiais em Inteligência Artificial
}{% EMENTA
    Estudos sobre Busca OU Otimização OU Problemas de Satisfação de Restrições OU Representação do Conhecimento OU Planejamento Automatizado
OU Raciocínio sob Incerteza OU  Processamento da Linguagem Natural OU Visão Computacional OU Robótica OU Sistemas Multiagentes.
}{% BIBLIOGRAFIA BÁSICA
    RUSSELL, Stuart J.; NORVIG, Peter. Inteligência Artificial: Uma Abordagem Moderna. 4. ed. Rio de Janeiro: GEN LTC, 2022. E-book. p.Capa. ISBN 9788595159495. Disponível em: https://app.minhabiblioteca.com.br/reader/books/9788595159495/. Acesso em: 22 mai. 2025.
\newline \newline 
COPPIN, B Inteligência artificial. LTC, 2010. ISBN: 9788521617297
\newline \newline 
MARTINS, Júlio S.; LENZ, Maikon L.; SILVA, Michel Bernardo Fernandes Da; et al. Processamentos de Linguagem Natural. Porto Alegre: SAGAH, 2020. E-book. p.Capa. ISBN 9786556900575. Disponível em: https://app.minhabiblioteca.com.br/reader/books/9786556900575/. Acesso em: 23 mai. 2025.
}{% BIBLIOGRAFIA COMPLEMENTAR
    Huth, Michael, and Mark Ryan. ""Lógica em ciência da computação: modelagem e argumentação sobre sistemas."" Rio de Janeiro, 2a edição edition. tradução e revisão técnica Valéria de Magalhães Iório (2008) 
\newline \newline 
SZELISKI, Richard. Computer Vision: Algorithms and Applications. Springer, 2010. ISBN 1848829345
\newline \newline 
BISHOP, CHRISTOPHER M.; Pattern Recognition and Machine Learning. SPRINGER VERLAG, 2006. (ISBN: 0387310738)
\newline \newline 
ROMERO, Roseli Aparecida F.; PRESTES, Edson; OSÓRIO, Fernando et al. Robótica Móvel. São Paulo: LTC, 2014. E-book. Disponível em: https://integrada.minhabiblioteca.com.br/\#/books/
978-85-216-2642-8. Acesso em: 26 de set. 2022.
\newline \newline 
SHOHAM, Y. Multiagent systems: algorithms, game theoretic. Cambridge University, 2009. ISBN: 9780521899437
}


\DetalhesDisciplina{
    <32> - Tópicos Especiais em Matemática Computacional
}{% EMENTA
    Programação linear ou Programação Inteira ou Pesquisa Operacional ou Cálculo Numérico e estudo de artigos e outras publicações atuais.
}{% BIBLIOGRAFIA BÁSICA
    BARROSO, Leônidas Conceição et al. Cálculo numérico: (com aplicações). 2.
ed. São Paulo, SP: Harbra, c1987. 367 p. ISBN 8529400895 (broch.).
\newline \newline 
GOLDBARG, Marco Cesar. Otimização combinatória e programação linear:
modelos e algoritmos. Rio de Janeiro: Elsevier: ​Campus,​ 2005. xvi, 518 p. :
ISBN 9788535215205 (broch.)
\newline \newline 
HILLIER, FREDERICK S.; LIEBERMAN, GERALD J. Introdução a Pesquisa
Operacional. MCGRAW HILL. 9a edição. (ISBN: 8580551188)
}{% BIBLIOGRAFIA COMPLEMENTAR
    RUGGIERO, Marcia A. Gomes; LOPES, Vera Lucia da Rocha. Cálculo
numérico: aspectos teóricos e computacionais . 2. ed. Pearson, c1997. ISBN
8534602042.
\newline \newline 
PASSOS, Eduardo José Pedreira Franco dos. Programação linear como
instrumento da pesquisa operaciona: Eduardo José Pedreira Franco dos
Passos. São Paulo, SP: Atlas, 2008. xii, 451p. ISBN 9788522448395 (broch.).
\newline \newline 
GERSTING, Judith L. Fundamentos matemáticos para a ciência da
computação: um tratamento moderno de matemática discreta . 5. ed. Rio de
Janeiro: Livros Técnicos e Científicos, c2004. xiv, 597 p. ISBN: 8521614225.
\newline \newline 
CHENEY, Ward; KINCAID, David (David Ronald). Numerical mathematics and
computing. 3rd. ed. Pacific Grove, CA: Books/Cole, c1994. 578p. ISBN
0534201121
\newline \newline 
SPERANDIO, Décio; MENDES, João Teixeira; SILVA, Luiz Henry Monken e.
Cálculo numérico: características matemáticas e computacionais dos métodos
numéricos. São Paulo, SP: Prentice Hall, 2003. ix, 354 p. ISBN 8587918745
(broch.).
}


\DetalhesDisciplina{
    QXD0182 - Visão Computacional
}{% EMENTA
    Introdução à visão computacional. Ferramentas de apoio. Formação da
imagem, dispositivos de captura e representação. Cor e textura.
Pré-processamento e Filtros. Segmentação. Rastreamento. Reconhecimento e
Classificação. Avaliação de desempenho de algoritmos de visão 
computacional. Aplicações e tópicos avançados.
}{% BIBLIOGRAFIA BÁSICA
    GONZALEZ, Rafael C.; WOODS, Richard E. Processamento digital de
imagens. 3. ed. São Paulo, SP: Pearson Education do Brasil, 2010. xv,624 p.
ISBN 9788576054016.
\newline \newline 
CONCI, A.; AZEVEDO, E.; LETA, F. Computação Gráfica, Volume 2 -
Processamento e Análise de Imagens Digitais. Publicado por Elsevier. 2007.
ISBN 97885352232193
\newline \newline 
SZELISKI, Richard. Computer Vision: Algorithms and Applications. Springer,
2010. ISBN 1848829345
\newline \newline 
PRINCE, Simon J. Computer Vision - Models, Learning and Inference. William
Morrow, 2012. ISBN 1107011795
\newline \newline 
BRADSKI, Gary; KAEHLER, Adrian. Learning OpenCV: Computer Vision in
C++ with the OpenCV Library. 2 ed. O’Reilly Media, 2012. ISBN 1449314651
}{% BIBLIOGRAFIA COMPLEMENTAR
    SARFRAZ, M SPRINGERLINK (ONLINE SERVICE). Interactive Curve
Modeling : With Applications to Computer Graphics, Vision and Image
Processing . Springer e-books London: Springer-Verlag London Limited, 2008.
ISBN 9781846288715. Disponível em :
<http://dx.doi.org/10.1007/978-1-84628-871-5>. Acesso em : 21 set. 2010.
\newline \newline 
HUTCHISON, David; ELMOATAZ, Abderrahim; KANADE, Takeo; KITTLER,
Josef; KLEINBERG, Jon M; LEZORAY, Olivier; MAMMASS, Dris; MATTERN,
Friedemann; MITCHELL, John C; NAOR, Moni; NIERSTRASZ, Oscar;
\newline \newline 
NOUBOUD. Image and Signal Processing : 3rd International Conference,
ICISP 2008 Cherbourg-Octeville, France, July 1-3, 2008 Proceedings . Springer
eBooks Berlin, Heidelberg: Springer-Verlag Berlin Heidelberg, 2008. (Lecture
Notes in Computer Science, 5099) ISBN 9783540699057. Disponível em :
<http://dx.doi.org/10.1007/978-3-540-69905-7>. Acesso em : 21 set. 2010.
\newline \newline 
CONCI, A.; AZEVEDO, E.; Computação Gráfica, Volume 1 - Geração de
Imagens. Publicado por Elsevier. 2003. ISBN: 9788535212525, 384 páginas
\newline \newline 
GOMES, J. M.; VELHO, L. Fundamentos de computação gráfica. IMPA. 2008.
ISBN: 8524402008
\newline \newline 
LIDWELL, W.; HOLDEN, K.; BUTLER, J. Princípios universais do design. Porto
Alegre: Bookman. 2010. ISBN 9788577807383
\newline \newline 
RUSS, John C. The Image Processing Handbook. Taylor \& Francis. 6 ed.
2010. ISBN 1439840458
\newline \newline 
PARKER, J. R. Algorithms for Image Processing and Computer Vision. John
Wiley. 2 ed. 2010. ISBN 0470643854
\newline \newline 
ANGEL, E. Interactive Computer Graphics: A Top-Down Approach Using
OpenGL. Edition: 5th. Published by Addison-Wesley. 2009. ISBN-10:
0321535863, ISBN-13: 9780321535863, 864 pages
}


\DetalhesDisciplina{
    <33> - Visualização de Dados
}{% EMENTA
    Introdução à Visualização de Dados. Marcas e canais. Abstração de Dados. Abstração de Tarefas. Visualização de dados tabulares. Visualização de dados espaciais. Visualização de redes e árvores. Mapeamento de cor. Estratégias para lidar com´complexidade em visualizações.
}{% BIBLIOGRAFIA BÁSICA
    MILANI, Alessandra M P.; SOARES, Juliane A.; ANDRADE, Gabriella L.; et al. Visualização de Dados. Porto Alegre: SAGAH, 2020. E-book. p.159. ISBN 9786556900278. Disponível em: https://integrada.minhabiblioteca.com.br/reader/books/9786556900278/ . Acesso em: 03 jun. 2025.
\newline \newline 
KYRAN, Dale,. Visualização de dados com Python e JavaScript: raspe, limpe, explore e transforme seus dados – Tradução da 2ª edição. Rio de Janeiro: Editora Alta Books, 2024. E-book. p.i. ISBN 9788550821801. Disponível em: https://integrada.minhabiblioteca.com.br/reader/books/9788550821801/ . Acesso em: 03 jun. 2025.
\newline \newline 
WILKE, Claus O. Fundamentals of Data Visualization: A Primer on Making Informative and Compelling Figures. 1 ed., O’Reilly, 2019. ISBN-13: 978-1492031086, ISBN-10: 1492031089. Disponível online: https://clauswilke.com/dataviz/  Acesso em: 03 jun. 2025.
}{% BIBLIOGRAFIA COMPLEMENTAR
    
AIGNER, Wolfgang et al. Visualization of time-oriented data. London: Springer, 2011. DOI: : https://doi.org/10.1007/978-1-4471-7527-8 Acesso em: 03 ju. 2025.
\newline \newline 
BURGHARDT, Dirk; DEMIDOVA, Elena; KEIM, Daniel A. Volunteered Geographic Information: Interpretation, Visualization and Social Context. Springer Nature, 2024.DOI:  https://doi.org/10.1007/978-3-031-35374-1 Acesso em: 03 ju. 2025.
\newline \newline 
Starbuck, C. (2023). Data Visualization. In: The Fundamentals of People Analytics. Springer, Cham. DOI: https://doi.org/10.1007/978-3-031-28674-2\_15
\newline \newline 

FERREIRA, Rafael G C.; MIRANDA, Leandro B. A de; PINTO, Rafael A.; et al. Preparação e Análise Exploratória de Dados. Porto Alegre: SAGAH, 2021. E-book. p.Capa. ISBN 9786556902890. Disponível em: https://app.minhabiblioteca.com.br/reader/books/9786556902890/  . Acesso em: 03 jun. 2025.
\newline \newline 
CURRY, Edward et al. Technologies and Applications for Big Data Value. In: Technologies and Applications for Big Data Value. Cham: Springer International Publishing, 2022. p. 1-15. DOI: https://doi.org/10.1007/978-3-030-78307-5 . Acesso em 3 de jun, 2025.
}