\chapter{ESTÁGIO CURRICULAR SUPERVISIONADO}
\label{cap:estagio-curricular-supervisionado}
% regis

O estágio supervisionado é parte de grande importância na estrutura curricular do curso ao inserir os alunos, de forma supervisionada, em contato direto com as práticas do mercado de trabalho. Os alunos têm a oportunidade de observar as técnicas, métodos, processos e afins, vistos ao longo do curso, sendo aplicados no dia a dia das empresas, além de observarem as dificuldades e alternativas que lhes são inerentes, contribuindo assim para a prática das competências e habilidades previstas no perfil do egresso. No âmbito do curso de \nomedocurso esse componente curricular é disciplinado e regimentado pelo Regulamento de Estágio Curricular Supervisionado, elaborado e aprovado pelo Colegiado do Curso.

O estágio supervisionado é firmado através da assinatura de um Termo de Compromisso de Estágio por parte da UFC, do estudante estagiário e da empresa concedente do estágio, e é acompanhado pelo professor orientador de estágio e por um profissional supervisor, conforme disposição da Lei n.º 11.788, de 29 de setembro de 2008 \cite{brasil2008lei11788}, da Resolução n.º 32/CEPE, de 30 de outubro de 2009 \cite{ufc2009resolucao32}. Além disso, as experiências individuais devem ser relatadas pelo estudante estagiário. Também é possível realizar o aproveitamento de atividade de estágio por parte do estudante que já possua experiência profissional como trabalhador formal ou que tenha iniciado estágio fora do período regular de matrícula, observadas as disposições vigentes.

É de responsabilidade do professor orientador de estágio acompanhar e avaliar o Plano de Trabalho fornecido pelo estudante, a Avaliação do Rendimento Discente realizada pelo profissional supervisor, o Seminário de Relato de Experiência e o Relatório Final de Estágio apresentados pelo estudante, a partir da experiência do estágio, da inter-relação teoria e prática, e do desenvolvimento de ações exitosas e inovadoras das quais participou. Esses dispositivos serão também constituintes da avaliação do Estágio Supervisionado, sendo considerado aprovado o aluno que obtiver nota superior ou igual a 7,0, com frequência mínima de 90\%, tendo respeitado todos os requisitos dispostos no Regulamento.

Ao final do estágio, são avaliadas as experiências que poderão ser sistematizadas para publicações e apresentações em eventos da comunidade acadêmica. A carga horária do estágio no curso de \nomedocurso é de 160 horas (obrigatórias), correspondendo a 5\% da carga horária do curso. O estágio é normalmente realizado nos dois últimos anos do curso e é implementado com a atividade Estágio Supervisionado, realizada na área de \nomedocurso.
%Apesar de o estágio supervisionado normalmente ser cumprido no fim do curso, muitos dos alunos de \nomedocurso, que já realizaram ou estão realizando estágio, o fizeram antes do último ano do curso, o que reflete o nível de interesse dos alunos para ingressarem no mercado e também a demanda por profissionais com o perfil do egresso, mesmo antes de se formarem.

% No que diz respeito ao aproveitamento da atuação profissional do discente como atividade de estágio, poderá ser pleiteado para atividades realizadas em uma das seguintes modalidades: estágio prévio, trabalhador formal, atuação como pessoa jurídica, empresa sediada no exterior, programa de imersão profissional e Projeto de Pesquisa e Desenvolvimento (P\&D). Essas modalidades de aproveitamento de estágio estão previstas no documento de Regulamento de Estágio Curricular Supervisionado do Curso de \nomedocurso do Campus Quixadá.

Dentro do contexto do estágio, é importante que o curso desenvolva estratégias para a gestão da integração entre ensino e mundo do trabalho. Uma das formas de se fazer isso é através da promoção dos Seminários de Oportunidades em Pesquisa, Desenvolvimento e Inovação, cujo objetivo é aproximar o Campus da UFC em Quixadá do meio empresarial local e regional. Nesses seminários, os docentes apresentam suas áreas e projetos de pesquisa, e os representantes das empresas apresentam lacunas e necessidades de pesquisa e implementação de soluções, momento em que se busca identificar oportunidades de cooperação entre o campus e as empresas. A partir desses seminários e outros contatos com profissionais, por exemplo, com a participação deles em eventos do campus, é possível identificar demandas por conhecimentos e tecnologias que podem ser discutidos e incorporados nas atividades de ensino, a fim de capacitar os alunos ao mundo do trabalho, durante o estágio e após se formarem.

A regulamentação de estágio curricular supervisionado do curso de \nomedocurso define os procedimentos, responsabilidades, obrigações, formulários, objetivos e avaliação do estágio do curso de \nomedocurso.

As atividades desenvolvidas pelos discentes durante o Estágio Supervisionado devem ser compatíveis com as áreas de atuação previstas no perfil do egresso do curso (Capítulo~\ref{cap:perfil-profissional}), e podem se enquadrar em diferentes modalidades, conforme descrito a seguir.

\section{Estágio em Empresas Conveniadas à UFC}

O estágio pode ser realizado em empresas públicas ou privadas que possuam convênio com a UFC. A formalização desses convênios e de todo o trâmite legal entre empresa, aluno e universidade é de responsabilidade da Agência de Estágios da UFC\footnote{Agência de Estágios da UFC: \url{https://estagios.ufc.br/}}, vinculada à Pró-Reitoria de Extensão (PREx). Essa agência coordena o Programa de Estágio Curricular Supervisionado nas unidades acadêmicas da UFC, sendo responsável pela articulação, agenciamento e formalização dos estágios obrigatórios e não obrigatórios com organizações conveniadas. Atua ainda como elo institucional com os ambientes de estágio e fornece subsídios para a atualização das práticas de estágio.

O Termo de Compromisso de Estágio (TCE), regulamentado pela Procuradoria Geral da União, é o documento que oficializa o vínculo entre a Instituição de Ensino (UFC), a Concedente (empresa) e o Estagiário (aluno), conferindo segurança jurídica às partes envolvidas\footnote{A lista de empresas conveniadas à UFC pode ser acessada no site da agência: \url{https://si3.ufc.br/sigaa/public/estagio/lista.jsf}}.

Além disso, a Agência de Estágios realiza a divulgação de oportunidades de estágio, emprego e trainee, e colabora com empresas no processo de recrutamento, orientando sobre os cursos mais adequados ao perfil da vaga.

\section{Aproveitamento de Experiências Profissionais e Acadêmicas}

Atividades atuais ou anteriores desenvolvidas pelos discentes na área de IA podem ser aproveitadas como carga horária total do Estágio Supervisionado, desde que estejam em conformidade com o Regulamento de Estágio Curricular Supervisionado do Curso de \nomedocurso do Campus Quixadá. As modalidades reconhecidas incluem:

\begin{itemize}
    \item \textbf{Contrato Formal de Trabalho}: Atuação na área do curso, com vínculo empregatício formal.
    \item \textbf{Atuação como Pessoa Jurídica}: Prestação de serviços na área do curso por meio de empresa própria, no Brasil ou no exterior.
    \item \textbf{Projetos de Pesquisa}: Participação em projetos de Pesquisa e Desenvolvimento (P\&D) em inteligência artificial, formalizados em instituições de ensino superior.
    \item \textbf{Programas de Imersão Profissional}: Participação em programas estruturados voltados à inserção no mercado de trabalho.
    \item \textbf{Estágio Não Obrigatório Anterior}: Solicitação de aproveitamento de atividades previamente realizadas como estágio não obrigatório.
\end{itemize}


% O discente também pode pleitear o aproveitamento de suas atividades anteriores de estágio não obrigatório, como carga horária total das atividades de Estágio Supervisionado, de acordo com as regras estabelecidas no Regulamento de Estágio Curricular Supervisionado do Curso de \nomedocurso do Campus Quixadá.

% desde que o aproveitamento seja solicitado até o semestre letivo seguinte ao desenvolvimento das atividades pleiteadas, e as atividades desenvolvidas pelo discente tenham sido realizadas em áreas de atuação afins com o perfil de egresso previsto no Projeto Pedagógico do Curso.

% \section{CONTRATO FORMAL DE TRABALHO}

% O discente que está atuando ou que já atuou na área do curso como trabalhador formal, pode pleitear o aproveitamento de suas atividades como carga horária total das atividades de Estágio Supervisionado, de acordo com as regras estabelecidas no Regulamento de Estágio Curricular Supervisionado do Curso de \nomedocurso do Campus Quixadá.

% \section{ATUAÇÃO COMO PESSOA JURÍDICA}

% O discente que está desempenhando ou que já desempenhou atividades como pessoa jurídica para empresas do Brasil ou do exterior poderá pleitear o aproveitamento de suas atividades como carga horária total de Estágio Supervisionado, de acordo com as regras estabelecidas no Regulamento de Estágio Curricular Supervisionado do Curso de \nomedocurso do Campus Quixadá.

% \section{NÚCLEO DE PRÁTICAS EM INFORMÁTICA (NPI)}

% O Núcleo de Práticas em Informática (NPI) foi criado com o objetivo de atender à comunidade acadêmica e à sociedade do sertão central com soluções de Tecnologia da Informação. As atividades do núcleo foram iniciadas em 2009, à época com o nome ``Escritório de Projetos'', através de projetos executados por docentes e alunos do grupo PET-SI, atendendo às demandas da comunidade acadêmica. Posteriormente, perceberam-se outras possibilidades para este, como, por exemplo, uma alternativa de provimento de estágio para estudantes dos cursos de graduação do campus.

% Com o estabelecimento oficial do NPI em 2011, foram iniciados projetos com alunos concludentes do curso de Sistemas de Informação. Em 2013, alunos concludentes do curso de \nomedocurso também começaram a atuar nos projetos. Também em 2013, o núcleo passou a operar em infraestrutura própria, contando com três salas equipadas com estações de trabalho. Em 2016, os alunos de Ciência da Computação também começaram a atuar nos projetos e, em 2017, foi a vez dos alunos do curso de Design Digital também começarem a participar dos projetos do NPI. A integração dos alunos de vários cursos, atuando nos mesmos projetos, permite que os projetos explorem habilidades e competências específicas de cada curso.

% O NPI é estruturado para funcionar nos moldes de uma fábrica de software, com processo definido e projetos com clientes reais. Um processo de desenvolvimento foi elaborado e implantado como forma de padronizar as atividades dos alunos no desenvolvimento de software e incorporar melhores práticas de \nomedocurso de acordo com metodologias, métodos e modelos de maturidade de processo que já são largamente utilizados na indústria de software e na academia \cite{goncalves2013}. Os alunos participantes são orientados por uma equipe composta por professores e servidores técnico-administrativos, além de fazerem registro de ponto para o acompanhamento da frequência pelo coordenador de estágio. Esta frequência deve ser, no mínimo, de 90\% para o aluno ser considerado aprovado.

% O NPI é registrado como um programa de extensão com múltiplas perspectivas, já que atende necessidades da comunidade (fornecimento de soluções de TI), dos alunos (provimento de estágio) e dos docentes, servindo a estes últimos como espaço para ampliação da experiência profissional. Em termos de infraestrutura, o núcleo conta, no início de 2025, com 27 computadores distribuídos em três salas.

% \section{PROJETOS DE PESQUISA}

% O discente que participa ou participou de projeto de Pesquisa e Desenvolvimento (P\&D) na área de inteligência artificial, formalizado em uma instituição de ensino superior, pode utilizar suas atividades como carga horária das atividades de Estágio Supervisionado, de acordo com as regras estabelecidas no Regulamento de Estágio Curricular Supervisionado do Curso de \nomedocurso do Campus Quixadá.

% Para isso, as seguintes regras devem ser observadas:

% \begin{itemize}
%     \item O discente deverá apresentar Plano de Trabalho / Lista de Tarefas a serem desenvolvidas para análise do supervisor de estágio;
%     \item O discente deverá, durante o estágio, participar do desenvolvimento de uma ou mais soluções de IA;
%     \item O discente deve entregar, até o final de estágio, artefatos como: documentação técnica, pipelines de dados, scripts de treinamento e avaliação de modelos de IA, código de API ou aplicação que faz uso de IA, Jupyter Notebooks com experimentos, roteiros de teste, testes automatizados, modelos de IA treinados e análises de interpretabilidade.
% \end{itemize}


% \section{ATUAÇÃO EM PROGRAMA DE IMERSÃO PROFISSIONAL}

% O discente que está participando ou que participou de programa(s) de imersão profissional poderá pleitear o aproveitamento de suas atividades como carga horária total das atividades de Estágio Supervisionado, de acordo com as regras estabelecidas no Regulamento de Estágio Curricular Supervisionado do Curso de \nomedocurso do Campus Quixadá.

% \section{INICIATIVA EMPREENDEDORA (IE)}

% O estágio em Iniciativa Empreendedora (IE)\footnote{Iniciativa empreendedora (IE) na UFC em Quixadá: \url{https://www.quixada.ufc.br/iniciativa-empreendedora/}} tem o objetivo de fomentar iniciativas empreendedoras voltadas para tecnologia no Campus da UFC em Quixadá. Esse tipo de estágio é possível desde que aprovado pelo professor orientador de estágio do curso. Para ser aprovado, o aluno deve submeter uma proposta de modelo de negócio que apresente um nível adequado de inovação e que possa também ser interdisciplinar.

% Assim como o estágio tradicional em empresas, a IE do aluno é acompanhada durante todo o período vigente, de acordo com as seguintes regras:
% \begin{itemize}
%     \item Encontros quinzenais de acompanhamento das atividades realizadas e planejamento das atividades da próxima quinzena;
%     \item As atividades devem ser passíveis de auditoria para comprovação de horas trabalhadas;
%     \item A carga horária mínima é de 12 horas semanais;
%     \item Acompanhamento através de ferramentas de gerenciamento de projetos / sistemas de controle de versão.
%     \item Cumprir os itens de avaliação do estágio supervisionado.
% \end{itemize}
