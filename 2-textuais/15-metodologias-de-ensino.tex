\chapter{METODOLOGIAS DE ENSINO E DE APRENDIZAGEM}
\label{cap:metodologias-de-ensino}

Os métodos de ensino e aprendizagem, efetivamente aplicados na formação proporcionada pelo curso de \nomedocurso, são definidos com base nos objetivos de aprendizagem estabelecidos nos planos de ensino de cada componente curricular, observando-se as especificidades de cada área quanto às estratégias mais adequadas e eficazes de transposição didática dos conteúdos e sua aprendizagem por parte dos alunos, visando assegurar que os correspondentes objetivos sejam alcançados.

Buscando desenvolver práticas de ensino-aprendizagem inovadoras, a escolha dos métodos é orientada por alguns princípios gerais: a) congruência entre objetivos de aprendizagem e métodos; b) factibilidade dos métodos em termos de recursos; c) observância dos aspectos de acessibilidade em todo o seu espectro.

Como forma de aprimoramento das metodologias de ensino e de aprendizagem adotadas nos cursos de graduação, a UFC investe na formação continuada dos docentes, especialmente através do projeto CASa\footnote{Site do CASa: \url{http://casa.virtual.ufc.br/}}. A Comunidade de Cooperação e Aprendizagem Significativa (CASa) é o programa de formação docente da UFC, fundado em 2009, e que hoje compõe a Escola Integrada de Desenvolvimento e Inovação Acadêmica (EIDEIA). O CASa trata-se de um espaço institucional na universidade que reúne iniciativas voltadas para a inovação de práticas de ensino e aprendizagem, e a promoção da excelência no ensino, pesquisa e extensão. Esse projeto busca, por meio de proposta dialógica e colaborativa, proporcionar aos docentes um espaço favorável à troca de experiências, de maneira que valorize o protagonismo e a trajetória de cada docente sem perder de vista os princípios de heterogeneidade, trabalho coletivo, interação, solidariedade, equidade e transformação que constituem bases sólidas a esse programa.

As metodologias de ensino e aprendizagem estão em constante atualização, como fruto das experiências acadêmicas vividas por docentes e discentes, da formação continuada dos professores e também como resultado das Autoavaliações Institucionais semestrais. Conforme será visto no Capítulo 16, na UFC, os professores são avaliados em três dimensões com pesos diferentes. A primeira delas, e com maior peso (40\%), é “Planejamento pedagógico, didático e domínio do conteúdo”, que questiona ao aluno, entre outras coisas, o quanto ele concorda ou discorda da afirmação “Utilizou metodologias de ensino que motivaram e facilitaram o aprendizado”. O instrumento de avaliação dos docentes pelos discentes também apresenta um campo aberto, por meio do qual os alunos podem se manifestar livremente, opinando, entre outras coisas, sobre o desempenho do professor na disciplina avaliada. Todas essas informações permitem à Coordenação e ao Colegiado do curso realizarem o contínuo acompanhamento das atividades realizadas e das estratégias de ensino e aprendizagem adotadas, gerando subsídios para intervenções e melhorias sempre que necessário.

O curso de \nomedocurso, conforme listado nas subseções a seguir, adota alguns procedimentos e projetos acadêmicos visando a dar suporte a fatores como: desenvolvimento de conteúdos, estratégias de aprendizagem, acessibilidade metodológica e autonomia do discente.

\section{INTERDISCIPLINARIDADE}

Uma das principais características do curso de \nomedocurso é seu caráter multidisciplinar. Assim, como foi possível ver nas seções anteriores, sua matriz curricular é composta por formação básica, tecnológica, complementar, humanística e suplementar. Para que os alunos compreendam a relação entre essas áreas para sua formação acadêmica e atuação profissional, é primordial que, ao longo do curso, eles tenham a oportunidade de trabalhar com os conceitos de forma integrada. Por isso, a interdisciplinaridade, no curso, é praticada extensivamente como um dos princípios norteadores.

Dentro desse contexto, acreditamos que o conhecimento não se constitui apenas em uma linearidade organizada por disciplinas e seus pré-requisitos, mas sim, a partir da complexidade do real e das experiências significativas, de onde se origina a necessidade de fomento à interdisciplinaridade em eixos de formação \cite{pombo1993interdisciplinaridade}.

No curso de \nomedocurso, a interdisciplinaridade é explorada sob as perspectivas de interligação entre disciplinas e atividades intercursos, conforme apresentado a seguir.

% \subsection{PROJETOS INTEGRADORES}

% Ao longo do curso, os alunos passam por três componentes curriculares do tipo projetos integradores: Projeto integrado I, Projeto integrado II e Projeto integrado III. Tais projetos são integradores ``no sentido da integração curricular'' (SANTOS \& BARRA, 2012, p. 2).

% Considerando a importância da interdisciplinaridade na formação dos alunos, os três projetos integrados promovem o trabalho em equipe e a participação de professores com diferentes saberes de forma integrada para o desenvolvimento de projetos que aliem a teoria com estudos práticos reais. As disciplinas de Projeto integrado, distribuídas ao longo do curso, visam fornecer ao aluno a oportunidade de por em prática, de forma integrada, os assuntos estudados em disciplinas de semestres anteriores, bem como nas disciplinas cursadas simultaneamente com cada disciplina de Projeto integrado.

% Os projetos integrados são disciplinas que acontecem em sala de aula com a supervisão de professores. O conteúdo dos projetos integrados é essencialmente prático e visa aplicar de forma integrada os conhecimentos adquiridos nas outras disciplinas do semestre em que o Projeto integrado acontece. Os professores das outras disciplinas do semestre participam da formatação do que se espera dos projetos elaborados pelos alunos no semestre, a partir dos conteúdos abordados em cada disciplina que ocorre em paralelo, definindo em conjunto um projeto integrado que deve ser executado em grupo pelos alunos e com a supervisão e coordenação desses professores.

% Dessa forma, os projetos integrados criam oportunidades de aprendizado que permitem ao aluno praticar os conhecimentos teóricos e técnicos adquiridos durante sua formação acadêmica através da execução de projetos, objetivando vivenciar e assim melhor compreender a realidade em que o aluno está inserido social e profissionalmente. As disciplinas de Projeto integrado visam, ainda, exercitar o trabalho em equipe, a divisão de tarefas e suas responsabilidades decorrentes, reconhecer a diferença entre ação responsável e obrigações sociais, praticar a ética e seus processos e construir produtos e objetos da prática profissional relativos ao curso de \nomedocurso.

\subsection{INTERLIGAÇÃO ENTRE AS COMPONENTES CURRICULARES DO CURSO}

Além das disciplinas de Projeto integrado, a matriz curricular para o curso de \nomedocurso, está organizada de forma a facilitar a execução de ações de interdisciplinaridade entre as diversas disciplinas que compõem a formação do aluno e entre os demais aspectos que perpassam a sua formação geral como cidadão.

Buscou-se aproximar disciplinas com potencial de interação de forma a facilitar o surgimento de ações de mútua interferência entre conteúdos e conhecimentos. Além disso, procurou-se a existência de disciplinas com natureza intrinsecamente interdisciplinar ou que resultem dessa interação.

Além disso, o próprio agrupamento de disciplinas segundo UC (Seção \ref{sec:unidades-curriculares}) favorece o surgimento e a execução continuada de metodologias ou práticas interdisciplinares quando se observa que professores de componentes curriculares afins congregam-se em um fórum com o objetivo de pensar de forma coletiva as ações e revisões nas UC das quais participam.

\subsection{ATIVIDADES INTERCURSOS}

Dada a situação de um campus temático de TI, como o Campus da UFC em Quixadá, com seis cursos da área instalados, existe ainda a possibilidade de os alunos de \nomedocurso observarem as relações entre os conhecimentos explorados de seu curso e os conhecimentos explorados nos demais cursos na oportunidade em que estiverem cursando conjuntamente disciplinas desses cursos, situação possível na forma de disciplinas optativas em comum ou de optativas livres. Essa integração os capacita a observar onde e de que formas os conhecimentos do curso estão inseridos ou são capazes de influenciar os demais cursos do campus.

Portanto, considerando a integração do curso de \nomedocurso no Campus da UFC em Quixadá, é possibilitada aos alunos a oportunidade de cursar disciplinas e realizar atividades dos demais cursos de computação do campus. Através dessas atividades intercursos, o aluno poderá desenvolver várias competências, tais como:
\begin{itemize}
    \item Conhecer e identificar soluções de software para problemas em diversificados domínios de aplicação;
    \item Compreender aspectos das tecnologias de construção de software, suas possibilidades e limitações;
    \item Trabalhar em equipe com pessoas de outra formação técnica;
    \item Conhecer o mercado de trabalho relacionado a softwares e serviços.
\end{itemize}

Da mesma forma, alunos de outros cursos também poderão desenvolver competências semelhantes em relação à área de \nomedocurso. A matriz curricular do curso prevê uma parte dos créditos reservada para disciplinas optativas livres, que podem ser usadas para as atividades intercursos.

Outra forma de integração intercursos acontece quando professores de \nomedocurso unem-se a professores dos outros cursos do campus em disciplinas, pesquisas e projetos.

Existe ainda no campus uma discussão em andamento sobre a possibilidade da criação de uma disciplina nos moldes de Projeto integrado, mas que fosse ofertada para todos os cursos, a fim de fomentar a integração dos alunos de cursos diferentes em projetos interdisciplinares envolvendo múltiplas áreas de conhecimento e atuação. É também uma forma de concretizar a vocação do campus em formar recursos humanos na área de TI, mas com especializações complementares.

O campus também promove anualmente eventos acadêmicos (Workshops, Encontros Universitários e Empreenday), cuja programação tem sido enriquecida com palestras, cursos e oficinas ligadas a temas de interesse dos discentes de \nomedocurso.

\section{INTEGRAÇÃO ENTRE TEORIA E PRÁTICA}

A formação no curso integra teoria e prática de modo essencial e mútuo: o conhecimento teórico fundamenta a aplicação em situações reais, enquanto a prática enriquece e direciona a teoria. Essa conexão constante entre saber e fazer é central para o aprendizado.
%
O projeto pedagógico valoriza a interação entre professores e alunos, reconhecendo o aprendizado como um processo coletivo \cite{morin2016sete}, e busca ampliar a participação da comunidade em atividades relevantes e inovadoras. A proposta prioriza a construção conjunta do saber e a aprendizagem colaborativa, superando um modelo puramente transmissor do conhecimento.

O aprendizado é um ciclo contínuo de teoria e prática, que se informam e complementam desde o início do curso, e não apenas no estágio final. Disciplinas introdutórias, como Fundamentos de Programação, já combinam conceitos teóricos com aplicação prática em laboratório. Essa abordagem visa consolidar o aprendizado pela experiência desde os primeiros semestres, buscando reduzir a evasão associada à concentração teórica inicial.
%
Essa dinâmica de ensino mantém o curso atualizado e focado em temas relevantes para os estudantes, ao mesmo tempo, em que garante o aprendizado das teorias essenciais para a análise crítica dos conteúdos. O objetivo é um conteúdo moderno e pertinente, alicerçado em uma base teórica sólida para uma formação completa em IA.

Em suma, a integração teoria-prática desenvolve habilidades e competências cruciais para o futuro profissional, definindo o perfil do egresso e preparando os alunos para o mercado de trabalho. Aprender fazendo e compreendendo a teoria subjacente é fundamental nesta formação.

\section{FLEXIBILIDADE NA ESTRUTURAÇÃO CURRICULAR}

A estrutura curricular do curso é planejada para ser flexível e abrangente. Com poucos pré-requisitos entre as disciplinas, e com uma ampla variedade de disciplinas optativas e optativas livres, os alunos têm autonomia para construir seu próprio percurso formativo.
%
Além disso, dada a rápida evolução da área de Inteligência Artificial, a atualização contínua do currículo e a existência de disciplinas de ementa flexível (Tópicos Especiais) permitem a rápida incorporação dos últimos avanços da área.

\section{AS TIC NO PROCESSO ENSINO-APRENDIZAGEM} \label{sec:TICEA}

Os alunos geralmente já possuem familiaridade com diversas TIC que utilizam no seu cotidiano, como computadores, celulares, redes sociais e aplicativos de conversação online. Durante a sua experiência na universidade, novos sistemas e ambientes digitais são integrados à sua rotina acadêmica.
%
Muitas disciplinas do curso com carga horária prática têm aulas em laboratório. Nesses casos, os alunos têm à disposição sistemas e ambientes específicos, de acordo com as necessidades das disciplinas, instalados nos computadores dos laboratórios, conforme solicitação dos docentes. Em alguns casos, é possível instalá-los também nos computadores pessoais dos alunos, por meio de software livre, com acesso gratuito, ou software proprietário, com licenças acadêmicas liberadas aos alunos ou licenças de teste.

Para apoiar o processo de ensino-aprendizagem, o curso adota um conjunto de sistemas com diversos fins:

\begin{itemize}
    \item Sistema Integrado de Gestão Acadêmica (SIGAA)\footnote{SIGAA: \url{https://si3.ufc.br/sigaa/}}: É a ferramenta de tecnologia da informação que a UFC disponibiliza para sua comunidade e no qual os procedimentos da área acadêmica são informatizados através de módulos e portais específicos TIC{souza2015docentes}. Do ponto de vista de ensino-aprendizagem, é no SIGAA que os docentes disponibilizam o plano de aula e consolidam o diário de aula e as informações de frequência e notas, utilizados para definir as aprovações e reprovações dos alunos. É o sistema oficial da UFC para esse tipo de informação. O sistema também oferece um ambiente completo para a disponibilidade de material de aula e comunicação entre o docente e os alunos, através de notícias e fóruns.
    \item Modular Object-Oriented Dynamic Learning Environment (Moodle)\footnote{Moodle: \url{https://moodle2.quixada.ufc.br}}: É uma plataforma de aprendizagem a distância baseada em software livre, bastante popular no Brasil e no mundo, com servidor exclusivo no campus e utilizado por grande parte dos professores e alunos do curso. Em comparação com o SIGAA, o Moodle apresenta uma variedade maior de recursos para o desenvolvimento das atividades, tais como: notícias, fóruns, eventos, chat, enquete, glossário, tarefa, questionário, wiki e até um laboratório virtual de programação, por meio do qual o professor cadastra uma tarefa de programação e pode avaliá-la automaticamente com o uso do ambiente pelo aluno. %O uso do Moodle não é obrigatório, mas muitas disciplinas do curso são gerenciadas nele.
    \item Ferramentas adicionais: dependendo das características da disciplina e do interesse dos docentes e discentes, podem ser usadas ferramentas adicionais, como, por exemplo: Discord\footnote{\url{https://discord.com/}},  Slack\footnote{\url{https://slack.com/}} (ambiente de mensagens e trabalho em grupo), Trello\footnote{\url{https://trello.com/}} (ferramenta de controle de atividades e projetos), Google Sala de Aula\footnote{\url{https://edu.google.com/intl/pt-BR_ALL/k-12-solutions/classroom}} (ambiente virtual de aprendizagem), GitHub\footnote{\url{https://github.com/}} (plataforma colaborativa de desenvolvimento), Integrated Development Environments (IDEs) (ambientes completos de apoio ao desenvolvimento de software), redes sociais (Facebook, WhatsApp, Telegram e Instagram) e ambientes de armazenamento em nuvem (Google Drive, Dropbox, Microsoft OneDrive).
\end{itemize}


\section{METODOLOGIAS ATIVAS DE APRENDIZAGEM}
% regis

De forma geral, um aspecto fundamental que norteia as metodologias adotadas para o curso de \nomedocurso é o aprendizado ativo. O método ativo ou metodologia ativa tem como princípio o protagonismo do aluno, ou seja, há um “deslocamento da perspectiva do docente (ensino) para o estudante (aprendizagem)” \cite[p. 270]{diesel2017metodologias}. Em comparação com o método tradicional, ao invés da prioridade na transmissão de informações e na centralidade na figura do docente (abordagem tradicional), no método ativo “os estudantes ocupam o centro das ações educativas e o conhecimento é construído de forma colaborativa” \cite[p. 271]{diesel2017metodologias}.

\begin{quote}
Assim, em contraposição ao método tradicional, em que os estudantes possuem postura passiva de recepção de teorias, o método ativo propõe o movimento inverso, ou seja, passam a ser compreendidos como sujeitos históricos e, portanto, a assumir um papel ativo na aprendizagem, posto que têm suas experiências, saberes e opiniões valorizadas como ponto de partida para a construção do conhecimento. \cite[p. 271]{diesel2017metodologias}.    
\end{quote}

No curso de \nomedocurso, na relação professor-aluno, busca-se adotar práticas pedagógicas de formação intelectual, técnica e profissional do aluno visando o desenvolvimento de sua consciência crítica e autonomia, em linha com os Princípios Norteadores estabelecidos neste documento. O desenvolvimento desses processos constitui-se tanto na relação professor-aluno, nos momentos de encontros presenciais (sala de aula, laboratórios e demais espaços de uso comum do curso), quanto nas relações mediadas por ferramentas tecnológicas digitais (uso de ambientes virtuais de aprendizagem, ferramentas de comunicação, dentre outras).
Isso significa estabelecer que:

\begin{itemize}
    \item Professor e aluno são coautores dos saberes a serem desenvolvidos ao longo das disciplinas e do curso, evitando-se o posicionamento do professor como único detentor e disseminador de informações e conhecimentos;

    \item A aprendizagem coletiva, em grupos, ganha destaque, haja vista a necessidade de agregação de saberes, experiências e práticas diversas para o enfrentamento de questões complexas e atuais que se colocam aos aprendizes;

    \item O professor assume o papel de mediador de tal aprendizagem coletiva e de gestor de tais espaços coletivos de aprendizagem, sejam eles presenciais ou a distância.
\end{itemize}

No campus, há ainda práticas exitosas e experimentais de técnicas específicas e inovadoras alinhadas com o método ativo.

\subsection{APRENDIZAGEM BASEADA EM PROBLEMAS}
% regis


A forma mais recorrente no curso de uso da metodologia ativa é empregando a aprendizagem baseada em problemas ou projetos (PBL – \textit{Problem/Project Based Learning}). A PBL promove o pensamento crítico, a construção coletiva do conhecimento e estimula a participação ativa do estudante na sua formação. Segundo \citeonline[p. 9]{bender2014aprendizagem}, “a aprendizagem baseada em projetos é um modelo de ensino que consiste em permitir que os alunos confrontem as questões e os problemas do mundo real que consideram significativos, determinando como abordá-los e, então, agindo de forma cooperativa em busca de soluções.”

Essa metodologia vem sendo aplicada com o objetivo de incentivar uma articulação entre teoria e prática, ao longo de todo o curso e de forma profunda. Ao associar competência prática e conhecimento teórico, desde o início do curso, fornece-se uma base para a construção da autonomia intelectual do aluno, favorecendo o desenvolvimento do interesse tanto em buscar o uso prático de conhecimentos adquiridos, como buscar a fundamentação teórica para práticas conhecidas.

Nas disciplinas extensionistas, os alunos serão estimulados a desenvolver projetos envolvendo problemas reais, muitas vezes pertencentes ao seu próprio cotidiano.

\subsection{APRENDIZAGEM ENTRE PARES OU TIMES}
% regis


Uma prática muito comum em escolas e universidades é a realização de atividades (trabalhos, projetos, pesquisas, exercícios) coletivamente (em duplas ou grupos maiores). A partir da formação de equipes, procura-se que o aprendizado seja feito em conjunto e haja compartilhamento de ideias \cite{lyceum2017metodologias}.

Segundo essa visão, quando os alunos resolvem os desafios e trabalham juntos, podem beneficiar-se na busca pelo conhecimento. Com a ajuda mútua, “eles podem aprender e ensinar ao mesmo tempo, formando o pensamento crítico, construído por meio de discussões embasadas e levando em consideração opiniões divergentes” \cite{lyceum2017metodologias}.

No curso de \nomedocurso, além das disciplinas de extensão, muitas outras adotam essa abordagem, como, por exemplo, Fundamentos de Programação e Programação Orientada a Objetos.

\section{PROGRAMAS DE ACOMPANHAMENTO E AUXÍLIO A ALUNOS COM DIFICULDADES DE APRENDIZAGEM}

O curso de \nomedocurso, assim como os outros cursos do Campus da UFC em Quixadá, conta com dois programas básicos de acompanhamento para alunos com dificuldade de aprendizagem: a) Programa de Iniciação à Docência (PID), que incentiva o interesse do estudante de graduação por atividades docentes. Nesse projeto, vinculados a disciplinas específicas do curso e orientados por um professor da área, alunos mais experientes ministram atividades de monitoria e acompanhamento dos alunos de uma determinada disciplina; b) Programa de Orientação Acadêmica (POA), que busca favorecer a integração dos alunos à vida universitária, orientando-os quanto às suas atividades acadêmicas, prioritariamente nos dois anos iniciais do curso. Contribui, portanto, para o processo de socialização e ambientação dos alunos ao campus.

%No POA, há um acompanhamento contínuo feito por professores e servidores técnico-administrativos aos alunos participantes do programa. Por conhecer as dificuldades que a maioria dos alunos enfrenta para continuarem o curso até o fim, é reconhecida a importância do acompanhamento individual, onde aluno e orientador acadêmico têm a oportunidade de estabelecer uma relação com base no diálogo, com o objetivo de contornar algumas das dificuldades enfrentadas por esses sujeitos, proporcionando um momento de fala e de escuta. Assim, todos os envolvidos nesse processo beneficiam-se: os alunos, com a oportunidade de melhorar seu rendimento acadêmico e assim concluir o curso de modo exitoso; os orientadores, com a possibilidade de conhecerem mais de perto os alunos com quem dividem o espaço da sala de aula; e, por fim, o próprio processo ensino-aprendizagem (mais informações sobre o POA podem ser consultadas na Seção \ref{sec:POA}).

%\textcolor{red}{esse parágrafo deveria ser uma nova seção pois está falando de outro tipo de assistência} 
É importante ressaltar que é uma política do curso incentivar os professores a executar esses projetos, visando a melhoria do processo de formação do estudante. Além disso, o aluno do curso de \nomedocurso pode, caso necessário, utilizar o serviço de apoio psicopedagógico do campus, disponibilizado pelo Núcleo de Atendimento Social (NAS), que conta com corpo técnico especializado composto por psicólogo, assistente social e nutricionista (mais sobre o NAS na Seção \ref{sec:NAS}).

\section{ACESSIBILIDADE METODOLÓGICA}
\label{sec:acessibilidade}

A acessibilidade metodológica diz respeito à ausência de barreiras nas metodologias e técnicas de ensino-aprendizagem, considerando sempre o aprendiz em suas necessidades individuais, sejam elas relacionadas a deficiências ou não. Assim, no curso de \nomedocurso, busca-se estimular os docentes a refletirem sobre noções de conhecimento, aprendizagem, avaliação e inclusão educacional para remover as barreiras pedagógicas.

Há pelo menos três iniciativas já em andamento no campus que favorecem a acessibilidade metodológica: a) diversificação curricular: os alunos podem cursar disciplinas optativas e optativas livres, compartilhadas ou não com outros cursos; b) flexibilização do tempo: em geral, os professores mantêm-se disponíveis fora dos horários de aula para atendimento aos alunos que necessitam de acompanhamento especial; c) utilização de recursos de acessibilidade: por ser um campus temático de TI, professores e alunos estão habituados a trabalhar com tecnologias digitais, inclusive em sala de aula. Isso favorece o acesso aos softwares e materiais didáticos por estudantes com deficiência e para aqueles que preferem estudar extraclasse. Uma ferramenta importante nesse sentido é a disponibilização de materiais didáticos no Moodle, conforme descrito na Seção \ref{sec:TICEA}.

Outra preocupação relacionada à acessibilidade metodológica é o tamanho das turmas. Em alguns casos, a necessidade de acompanhamento constante e de atividades práticas exige que a turma seja dividida. Assim como realizado nos demais cursos do Campus, há a previsão de ocorrer, principalmente nos semestres iniciais, a divisão da turma. %com um mesmo professor ou com professores diferentes. 

A partir do SiSU 2018, conforme a Lei n.º 13.409, sancionada em dezembro de 2016 \cite{brasil2016lei13409}, o preenchimento das vagas deve levar em consideração também uma reserva em cada modalidade de cota para pessoas com deficiência, no mínimo igual à proporção da população da unidade da federação onde está instalada a instituição, de acordo com o IBGE. Assim, o curso deve estar preparado para receber alunos com deficiências diversas. Com o apoio da ``Secretaria de Acessibilidade UFC Inclui''\footnote{Secretaria de Acessibilidade UFC Inclui: \url{https://acessibilidade.ufc.br/pt/sobre/}} e em parceria com os profissionais da Biblioteca Universitária\footnote{Biblioteca Acessível: \url{http://www.biblioteca.ufc.br/biblioteca-acessivel/}}, dependendo da necessidade, serão disponibilizados recursos e tecnologias assistivas para comunicação e estudo como: pranchas de comunicação, texto impresso e ampliado, softwares ampliadores de tela, softwares de comunicação alternativa, softwares leitores de tela, intérpretes de Libras, entre outros recursos. Mais aspectos relacionados à acessibilidade são apresentados na Seção \ref{sec:AssistenciaEmAcessibilidade}.