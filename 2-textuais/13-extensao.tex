\chapter{EXTENSÃO}
\label{cap:extensao}

O conceito de Extensão Universitária, instituído no I Encontro Nacional de Pró-Reitores de Extensão das Universidades Públicas Brasileiras, em 1987, e ratificado pelo Plano Nacional de Extensão Universitária em 2012, pode ser entendido como ``o processo educativo, cultural e científico que articula o ensino e a pesquisa de forma indissociável e viabiliza a relação transformadora entre a universidade e a sociedade''. Assim, a Extensão Universitária consiste em um conjunto de ações de natureza interdisciplinar e multidisciplinar, articulando os saberes produzidos na universidade com a sociedade de modo geral.

Considerando ainda o princípio da indissociabilidade entre ensino, pesquisa e extensão, previsto no artigo 207 da Constituição Federal de 1988. No mesmo sentido, considerando a concepção de currículo estabelecida na Lei nº 9.394/96, em seu capítulo IV – Da Educação Superior, expressa em seu Artigo 43, incisos VII, as seguintes finalidades da educação superior: ``VII - promover a extensão, aberta à participação da população, visando à difusão das conquistas e benefícios resultantes da criação cultural e da pesquisa científica e tecnológica geradas na instituição''.

Vale destacar ainda a Estratégia 7 da Meta 12 do Plano Nacional de Educação 2014-2024 (Lei nº 13.005/2014), estabeleceu que as Instituições de Ensino Superior (IES) devem assegurar, no mínimo, 10\% (dez por cento) do total de créditos curriculares exigidos para a graduação em programas e projetos de extensão universitária. Assim, buscando adequar-se à legislação e fortalecer a política interna de oferta da extensão na UFC, a Resolução nº 28 do CEPE-UFC, de 1º de dezembro de 2017 \cite{ufc2017resolucao28}, normatizou e estabeleceu os procedimentos pedagógicos e administrativos para os cursos procederem à inclusão das ações de extensão nos currículos dos cursos de graduação no âmbito da UFC.

A Resolução nº 28 define duas modalidades para realizar a curricularização da extensão, a saber:
\begin{itemize}
    \item Unidade Curricular Especial de Extensão (UCEE), como definido no Artigo 5º, Inciso I: ``constituída de ações de extensão, ativas e devidamente cadastradas na Pró-Reitoria de Extensão, cujas temáticas serão definidas no currículo'';
    \item Carga Horárias em Componentes Curriculares, como definido no Artigo 5º, Inciso II: ``parte de componentes curriculares com destinação de carga horária de extensão definida no currículo''.
\end{itemize}

\vspace{1em}

O curso de \nomedocurso apresenta a formalização da extensão no PPC através das duas modalidades supramencionadas, (i) por meio da UCEE e (ii) por meio dos projetos executados nas disciplinas do curso, onde as práticas de extensão compõem a carga horária da disciplina, ficando expressa na matriz curricular o quantitativo de horas que será dedicado à extensão, as cargas-horárias alocadas para a extensão por disciplina serão listadas na seção Integralização Curricular (Seção 9.3). De acordo com o Art. 13 da Resolução nº 28, os cursos de graduação da UFC deverão designar pelo menos um supervisor de extensão para analisar e validar o cumprimento das ações da extensão previstas em seus respectivos Projetos Pedagógicos.

As atividades de extensão são caracterizadas assim pela participação ativa de discentes, docentes e comunidade extra-acadêmica, que, atuando em conjunto, promovem a interação transformadora entre a Universidade e a sociedade. As atividades de extensão são acompanhadas no âmbito do curso por um supervisor de extensão designado pelo colegiado do curso, sendo este responsável por analisar e validar o cumprimento das ações da extensão.

De acordo com a resolução CEPE Nº 04, de 27 de fevereiro de 2014, as atividades de extensão da UCEE devem ser classificadas em uma área temática principal e, opcionalmente, uma área temática secundária. À vista disso, alinhando o perfil do egresso do Curso de \nomedocurso e considerando as necessidades da região, as atividades de extensão da UCEE deverão ser realizadas com projetos cuja área temática principal esteja contida na lista abaixo, selecionadas da lista apresentada no artigo 5º da referida resolução: (1) Comunicação; (2) Cultura; (3) Direitos Humanos e Justiça; (4) Educação; (5) Meio Ambiente; (6) Saúde; (7) Tecnologia e Produção; e (8) Trabalho \cite{ufc2014resolucao04}.

Foi desenvolvido um  Manual de Regulamentação da Extensão, no qual são consideradas as especificidades com relação ao cumprimento da carga horária destinada à extensão pelos estudantes, bem como regras para o cadastro e acompanhamento das ações de extensão na UCEE, conforme preconiza o Artigo 12º da Resolução nº 28.  %\textcolor{red}{tem esse manual para o curso de ES...acho que não podemos usar.}

Para facilitar o processo de gestão das Atividades de Extensão e inserção das informações, foi desenvolvido o Módulo de Registro de Ação de Extensão no Sistema Integrado de Gestão de Atividades Acadêmicas (SIGAA) com o objetivo de melhorar a experiência de acesso e uso aos servidores (docentes e TAEs) e discentes.