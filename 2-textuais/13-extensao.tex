\chapter{EXTENSÃO}
\label{cap:extensao}

O conceito de Extensão Universitária, instituído no I Encontro Nacional de Pró-Reitores de Extensão das Universidades Públicas Brasileiras, em 1987, e ratificado pelo Plano Nacional de Extensão Universitária em 2012, pode ser entendido como ``o processo educativo, cultural e científico que articula o ensino e a pesquisa de forma indissociável e viabiliza a relação transformadora entre a universidade e a sociedade''. Assim, a Extensão Universitária consiste em um conjunto de ações de natureza interdisciplinar e multidisciplinar, articulando os saberes produzidos na universidade com a sociedade de modo geral.

A Extensão Universitária na Universidade Federal do Ceará (UFC) é definida pela Resolução Nº 05/CEPE, de 07 de março de 2025, como um processo de educação superior que objetiva promover uma relação mutuamente transformadora entre a universidade e a sociedade, integrando ensino e pesquisa por meio da cultura, arte, ciência, tecnologia e inovação, tendo em vista a articulação e o desenvolvimento social. 
Neste sentido, o processo de desenvolvimento da Extensão na UFC segue os seguintes os princípios norteadores:% \cite{ufc2025resolucao05}: 

\begin{enumerate}[label=\Roman* --, leftmargin=*]
    \item alinhamento da ciência, da arte e da tecnologia às prioridades do local, região e país; sensibilidade da universidade à necessidade de interação com outros saberes e necessidades, adquiridos nas atividades de ensino, pesquisa e extensão, sem pretender ser proprietária de um saber universal e pronto; 
    \item participação nos movimentos sociais para desenvolvimento de ações que busquem a superação das desigualdades e processos excludentes; 
    \item democratização de conhecimentos acadêmicos e científicos como parte do processo de interação e inclusão social desenvolvidos pela Extensão; 
    \item prestação de serviços como desenvolvimento de trabalho social; 
    \item atuar junto aos sistemas públicos de ensino para o fortalecimento da educação básica. 
\end{enumerate}


A Resolução Nº 05/CEPE, de 07 de março de 2025, define como diretrizes gerais para orientar a formulação, execução e avaliação das atividades de extensão a interação dialógica universidade-sociedade, a indissociabilidade entre o ensino, a pesquisa e a extensão, a interdisciplinaridade e a interprofissionalidade, a formação cidadã de estudantes e o impacto e transformação social, considerando que:

\begin{enumerate}[label=\Roman* --, leftmargin=*]
    \item a interação dialógica orienta o desenvolvimento das relações entre a universidade e setores sociais, superando o discurso da hegemonia acadêmica;
    \item a indissociabilidade entre o ensino, a pesquisa e a extensão pressupõe que as atividades de extensão são mais efetivas quando vinculadas ao processo de formação, pessoal e profissional, e de geração de conhecimento, tendo como princípio o desenvolvimento de práticas que promovam ações articuladoras da interdisciplinaridade, da transdisciplinaridade entre as áreas do conhecimento, bem como da interprofissionalidade;
    \item a interdisciplinaridade e a interprofissionalidade contribuem para o desenvolvimento da cultura extensionista na universidade, atendendo à complexidade das comunidades de saberes, aqui entendidas como grupos de pessoas que se articulam com as atividades de extensão Universitária por meio de uma relação dialógica. Estas comunidades não se limitam a grupos vulneráveis, podem ser movimentos sociais, entidades profissionais e outras formas de organização;
    \item a formação cidadã de estudantes implica no desenvolvimento de uma consciência cidadã, agregada à sua contribuição para a transformação social e integrada à matriz curricular de seu curso de graduação e/ou pós-graduação;
    \item a transformação social diz respeito à forma de relacionamento da universidade com a sociedade, com objetivo de, por meio da interação dialógica, trabalhar nos interesses e necessidades da população, possibilitando melhor compreensão das problemáticas e para o aprimoramento das políticas públicas.
\end{enumerate}

A concepção de currículo estabelecida na Lei nº 9.394/96, em seu capítulo IV – Da Educação Superior, expressa em seu Artigo 43, incisos VII, as seguintes finalidades da educação superior: ``VII - promover a extensão, aberta à participação da população, visando à difusão das conquistas e benefícios resultantes da criação cultural e da pesquisa científica e tecnológica geradas na instituição''. 

A promoção da extensão na UFC e no curso ocorre por meio de ações de extensão, sendo classificadas na Resolução Nº 05/CEPE, de 07 de março de 2025, em cinco modalidades: Programa, Projeto, Curso e Oficina, Evento ou Prestação de Serviço. 

Um Programa consiste em um conjunto articulado de ações com um objetivo comum, executado em um prazo de dois a cinco anos e podendo ser de natureza estratégica, estruturante ou setorial. Já um Projeto é uma ação de caráter processual com objetivo específico e prazo determinado de quatro meses a dois anos, que dialoga com as comunidades de saberes e promove a formação dos estudantes, estabelecendo um vínculo com atividades de pesquisa científica.

Cursos e Oficinas são ações pedagógicas, teóricas ou práticas, com carga horária mínima de duas horas, podendo ser classificadas como iniciação, atualização, treinamento, qualificação profissional ou aperfeiçoamento. Os Eventos, por sua vez, são ações de curta duração que implicam na apresentação ou exibição pública de conhecimento ou produto cultural, artístico, esportivo ou científico, como congressos, seminários e exposições.

A Prestação de Serviço refere-se à aplicação do conhecimento universitário para a solução de problemas de parceiros externos, incluindo serviços técnicos, assistência e consultoria. Dentre suas classificações, destacam-se o desenvolvimento tecnológico e inovação, que abrange a criação de softwares, aplicativos e sistemas inovadores, e as atividades de empresa júnior, ambas diretamente alinhadas à área de tecnologia. Todas essas modalidades podem gerar produtos de extensão, como publicações, relatórios técnicos, softwares, sistemas tecnológicos e jogos educativos, que materializam o conhecimento produzido.

Conforme a Resolução Nº 05/CEPE, de 07 de março de 2025, todas as atividades de extensão devem ser obrigatoriamente classificadas em Áreas do Conhecimento, Áreas Temáticas e Linhas de Extensão. A classificação em uma das Áreas do Conhecimento toma por base as definições do Conselho Nacional de Desenvolvimento Científico e Tecnológico (CNPq). Adicionalmente, as atividades devem ser classificadas conforme uma área temática principal e, opcionalmente, uma secundária. Por fim, as ações também precisam ser categorizadas segundo uma linha de extensão principal e, opcionalmente, uma secundária, sendo estas importantes para a articulação das atividades em programas. A Resolução Nº 05/CEPE, de 07 de março de 2025, apresenta as listas e os detalhamentos de cada uma dessas categorias de classificação.

Alinhando o perfil do egresso do Curso de \nomedocurso e considerando as necessidades da região, as atividades de extensão da UCEE do curso deverão ser realizadas em projetos que se enquadrem, preferencialmente, nas seguintes classificações, conforme a Resolução Nº 05/CEPE, de 07 de março de 2025:

\begin{itemize}
    \item \textbf{Áreas do Conhecimento}: Ciências Exatas e da Terra ou Multidisciplinar.
    \item \textbf{Áreas Temáticas}: Comunicação; Cultura; Direitos Humanos e Justiça; Educação; Meio Ambiente; Saúde; Tecnologia e Produção; Trabalho; ou Gestão.
    \item \textbf{Linhas de Extensão}: Comunicação estratégica; Desenvolvimento de produtos; Desenvolvimento regional; Desenvolvimento tecnológico; Desenvolvimento urbano; Educação profissional; Empreendedorismo; Espaços de ciência; Formação de professores; Gestão informacional; Gestão institucional e pública; Inovação tecnológica; Metodologias e estratégias de ensino/aprendizagem; Pessoas com deficiência; Questões ambientais;  Tecnologia da informação; ou Temas específicos/desenvolvimento humano.
\end{itemize}

A Estratégia 7 da Meta 12 do Plano Nacional de Educação 2014-2024 (Lei nº 13.005/2014), estabeleceu que as Instituições de Ensino Superior (IES) devem assegurar, no mínimo, 10\% (dez por cento) do total de créditos curriculares exigidos para a graduação em programas e projetos de extensão universitária. 
%
A Resolução Nº 05/CEPE, de 07 de março de 2025, estabelece a curricularização da extensão por meio da inserção de ações curriculares de extensão nos cursos de graduação como componentes obrigatórios para sua integralização. Por essa resolução, os projetos pedagógicos dos cursos de graduação da UFC devem prever a realização de atividades de Extensão que correspondam ao percentual mínimo de 10\% do total da carga horária do curso. %As ações de extensão inseridas nos cursos deverão reforçar as diretrizes nacionais da extensão, e, quando ofertadas por disciplinas, tais elementos devem estar descritos na ementa.

O processo de curricularização da extensão poderá ser realizado de três formas, desde que sejam atendidas as diretrizes da Extensão Universitária nas atividades realizadas, identificando a importância do ensino pela extensão, destacando-se objetivos, conhecimentos, habilidades e atitudes por ela constituídos ou desenvolvidos: 

\begin{enumerate}[label=\Roman* --, leftmargin=*]
    \item Unidade Curricular Especial de Extensão (UCEE), constituída de carga horária atribuída por meio de Programas e Projetos que identifiquem no seu planejamento a importância, o desenvolvimento e a avaliação de ações formativas de ensino de extensão; 
    \item Componentes curriculares com destinação de carga horária de extensão definidas na sua criação e regulamentação; 
    \item Ações Curriculares em Comunidades de Saberes (ACCS), integradas ao Plano Pedagógico do Curso (PPC), aprovadas pela Câmara de Extensão do CEPE e/ou pela Câmara de Graduação do CEPE; 
\end{enumerate}

Em conformidade com o Art. 4º da Resolução CNE/CES Nº 7, de 18 de dezembro de 2018, e com o parágrafo 1º do Art. 15 da Resolução Nº 05/CEPE, de 07 de março de 2025, o curso de \nomedocurso destina um total de 320 horas para as atividades de extensão, o que corresponde a 10\% da carga horária total de 3.200 horas do curso. Essa carga horária é integralizada por meio da combinação das modalidades previstas no Art. 20 da Resolução Nº 05/CEPE, de 07 de março de 2025.

A distribuição dessa carga horária é realizada em dois blocos, utilizando as modalidades previstas no Art. 20 da referida resolução:

\begin{itemize}
    \item \textbf{32 horas} são alocadas em Componentes Curriculares com destinação de carga horária de extensão, onde as práticas extensionistas compõem parte da carga horária de disciplinas da matriz curricular, conforme detalhado na seção de Integralização Curricular (Seção \ref{sec:integralizacao}).
    \item \textbf{288 horas} restantes podem ser integralizadas de forma flexível pelo discente por meio de uma das outras duas modalidades: a Unidade Curricular Especial de Extensão (UCEE) , constituída pela participação em programas e projetos de extensão; ou por meio de Ações Curriculares em Comunidades de Saberes (ACCS).
\end{itemize}

Para facilitar o processo de gestão das Atividades de Extensão e inserção das informações, foi desenvolvido o Módulo de Registro de Ação de Extensão no Sistema Integrado de Gestão de Atividades Acadêmicas (SIGAA) com o objetivo de melhorar a experiência de acesso e uso aos servidores (docentes e TAEs) e discentes.

O estudante poderá solicitar aproveitamento de carga horária das ações curriculares de extensão declaradas por setores com competência de outras instituições de ensino superior no Brasil ou no Exterior. Os pedidos serão analisados pela Câmara de Extensão do CEPE, em conformidade com o Projeto Pedagógico do Curso.

Caso o discente realize uma carga horária em atividades de extensão superior à exigida para a curricularização, as horas excedentes poderão ser contabilizadas para o cumprimento do total de horas de atividades complementares.

Para a integralização da carga horária obrigatória de extensão, é vedada a duplicidade no cômputo de horas com outras atividades curriculares. Dessa forma, as cargas horárias dedicadas ao Estágio Curricular Supervisionado, ao Trabalho de Conclusão de Curso e às Atividades Complementares não podem ser contabilizadas para este fim, por se tratarem de componentes distintos da matriz curricular. O mesmo princípio de não duplicidade se aplica a outras atividades de natureza extensionista, como a participação em Empresas Juniores e no Programa de Educação Tutorial (PET), cujas horas não podem ser simultaneamente utilizadas para o cômputo da extensão e de outra atividade formativa.
 
% As atividades de extensão são caracterizadas assim pela participação ativa de discentes, docentes e comunidade extra-acadêmica, que, atuando em conjunto, promovem a interação transformadora entre a Universidade e a sociedade. As atividades de extensão são acompanhadas no âmbito do curso por um supervisor de extensão designado pelo colegiado do curso, sendo este responsável por analisar e validar o cumprimento das ações da extensão.



O Manual de Procedimentos de Integralização de Atividades de Extensão (Anexo \ref{an:manual-extensao}) detalha as especificidades para o cumprimento da carga horária destinada à extensão pelos estudantes, as regras para o cadastro e acompanhamento das ações nas diferentes modalidades, como a Unidade Curricular Especial de Extensão (UCEE), e os critérios de avaliação e validação das atividades.