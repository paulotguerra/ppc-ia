\chapter{ÁREAS DE ATUAÇÃO DO FUTURO PROFISSIONAL}
\label{cap:areas-de-atuacao}
% regis


O egresso do curso de Bacharelado em Inteligência Artificial estará apto a atuar em uma ampla gama de setores e contextos que demandam soluções inteligentes, baseadas em dados, automação de processos, aprendizado de máquina e raciocínio computacional. A formação proposta pelo curso visa preparar profissionais com sólida base teórica e prática para enfrentar desafios reais por meio do desenvolvimento, aplicação e avaliação de sistemas inteligentes, sempre com responsabilidade ética e compromisso social.

As áreas de atuação para esse profissional são diversas e em constante expansão, destacando-se:

\begin{itemize}
    \item \textbf{Tecnologia da Informação e Comunicação (TIC):} desenvolvimento de sistemas autônomos, softwares inteligentes, assistentes virtuais, soluções de recomendação e personalização, além de aplicações em computação em nuvem e computação embarcada.

    \item \textbf{Indústria e Manufatura:} aplicação de IA em processos de automação, manutenção preditiva, controle de qualidade, robótica industrial e otimização de cadeias produtivas.
    
    \item \textbf{Saúde e Biotecnologia:} uso de técnicas de IA para diagnóstico assistido por computador, análise de imagens médicas, descoberta de fármacos, modelagem de sistemas biológicos e monitoramento inteligente de pacientes.
    
    \item \textbf{Agronegócio e Meio Ambiente:} implementação de soluções para agricultura de precisão, monitoramento ambiental, previsão climática, detecção de pragas, otimização de recursos naturais e sustentabilidade.
    
    \item \textbf{Serviços Financeiros e Seguros:} desenvolvimento de modelos preditivos para análise de risco, detecção de fraudes, algoritmos de investimento, scoring de crédito e automação de processos bancários.
    
    \item \textbf{Transporte e Logística:} aplicação de IA em sistemas de roteamento inteligente, veículos autônomos, gerenciamento de frotas, previsão de demanda e logística urbana.
    
    \item \textbf{Educação e Tecnologias Educacionais:} criação de sistemas adaptativos de aprendizagem, tutores inteligentes, análise de desempenho acadêmico e personalização de conteúdos.
    
    \item \textbf{Setor Público e Políticas Públicas:} uso de IA para apoio à tomada de decisão, melhoria na prestação de serviços públicos, análise de grandes volumes de dados governamentais e formulação de políticas públicas baseadas em evidências.
    
    \item \textbf{Pesquisa Científica e Desenvolvimento:} atuação em centros de pesquisa, universidades e laboratórios de inovação, contribuindo para o avanço do conhecimento e desenvolvimento de novas técnicas, algoritmos e aplicações em IA.
\end{itemize}

Além disso, o egresso poderá empreender, criar startups ou liderar iniciativas inovadoras que envolvam a aplicação ética e eficiente da Inteligência Artificial nos mais diversos setores. Sua formação o habilita também a continuar os estudos em programas de pós-graduação, tanto acadêmicos quanto profissionais, ampliando ainda mais suas possibilidades de atuação.
