\chapter{GESTÃO ACADÊMICA DO CURSO}
\label{cap:gestao-academica-do-curso}
\section{Coordenação do Curso}

A Coordenação de Curso de graduação é regida pelo Estatuto da UFC \cite{ufc2025estatuto} e pelo Regimento Geral da UFC \cite{ufc_regimento_geral_2019}. A Coordenação de Curso de graduação é exercida, no plano deliberativo e consultivo, pelo Colegiado de curso e, no plano executivo, pelo Coordenador de curso.

O coordenador de curso é um gestor pedagógico que deve ter o compromisso com a melhoria da qualidade do curso, atuando nas dimensões didáticas, pedagógicas, administrativas e políticas, por meio do exercício da liderança democrática, desenvolvendo ações propositivas e proativas. Na UFC, será um professor associado ou titular, ou que possua o título de doutor; na inexistência ou impossibilidade destes, um professor adjunto e, em último caso, um professor assistente, eleito em escrutínio secreto pelos integrantes do Colegiado do curso, entre os seus pares representantes de UC, para um mandato de três anos, permitida uma única recondução.

Concomitantemente com a eleição do coordenador de curso e segundo as mesmas normas, é realizada a eleição do vice-coordenador, para cumprir mandato de igual duração, a quem caberá substituir o coordenador durante suas faltas e impedimentos, bem como concluir o mandato do titular nos casos de renúncia ou afastamento definitivo.

Nas faltas e impedimentos simultâneos do coordenador e do vice-coordenador, a coordenação do curso será exercida pelo professor mais antigo, entre os seus pares representantes de UC e, no caso de empate, pelo mais idoso.

O coordenador de curso exerce o seu mandato em dedicação exclusiva ou em regime de tempo integral \cite[p. 30]{ufc2025estatuto}. A coordenação do curso de \nomedocurso é assessorada diretamente pela Secretaria Acadêmica do campus, com pelo menos um secretário dedicado ao curso.

A coordenação trabalha articuladamente com o colegiado do curso, o NDE e os discentes, no compromisso com a melhoria contínua da qualidade do curso, atuando nas dimensões didáticas, pedagógicas, administrativas e políticas, desenvolvendo ações propositivas e proativas e favorecendo a integração e a melhoria contínua das atividades realizadas no curso. As atribuições mais habituais da Coordenação do curso de \nomedocurso são:
\begin{itemize}
    \item Matrícula: durante o período de matrículas (regular e ajuste), o coordenador e/ou o vice observam, acompanham e orientam os alunos, tirando dúvidas sobre os componentes curriculares, auxiliando nas suas escolhas levando em consideração seus objetivos pessoais e procurando solucionar os problemas que porventura possam surgir, como a solicitação de vagas extras em componentes curriculares de outros cursos ou o reequilíbrio de número de vagas em disciplinas do curso.

    \item Atendimento aos discentes: a coordenação do curso mostra-se disponível para atender os alunos e orientá-los sobre questões de diferentes tipos, com relação e interferência na vida acadêmica dos discentes, como a composição das disciplinas a serem matriculadas no semestre e ao longo do curso, reclamações e sugestões sobre o andamento das aulas, a relação com professores, assuntos relacionados à estrutura do campus e orientação profissional.

    \item Atendimento aos docentes: os professores têm acesso facilitado à coordenação e partilham com ela demandas relacionadas às atividades acadêmicas que desenvolvem. Para promover as potencialidades do corpo docente, na medida do possível, a coordenação consulta os professores sobre suas preferências de disciplinas a cada início de semestre e comunica sobre eventos e congressos na área, que possam incentivar o desenvolvimento de pesquisa por parte dos docentes. A coordenação também auxilia no planejamento de atividades complementares dentro e fora do campus (como viagens de estudo, reserva de laboratórios específicos, etc.), com questões pedagógicas com discentes ou turmas, com assessoria e apoio em questões administrativas (como pedido de afastamento, progressão funcional, etc.), e no desenvolvimento de trabalho específico com professores que apresentem resultados com potencial de melhora na Avaliação Institucional, ferramenta que permite aos alunos expressarem, semestralmente, suas opiniões sobre a experiência de cursar cada disciplina em dimensões apropriadas para uma avaliação pedagógica, como apresentada no Capítulo \ref{cap:procedimentos-de-acompanhamento} deste documento.

    \item Recepção e orientação de ingressantes: cada nova turma de ingressantes deve ser recepcionada com muita atenção e zelo pela coordenação. Na primeira semana de aula, a coordenação deve apresentar detalhes sobre o curso (PPC, características, perfil do egresso, matriz curricular); recomendações e conselhos sobre a vida acadêmica; informações sobre serviços de apoio ao discente, política de bolsas e auxílios (Seção \ref{sec:bolsas-e-auxilios}), entre outros. Também nesse encontro, a coordenação começa a conhecer os ingressantes (cidades de origem, interesses, motivações e expectativas). Os alunos aproveitam a participação nesses encontros como atividade complementar.

    % \item Promoção de eventos: a Coordenação tem organizado algumas palestras de diversas áreas de interesse dos alunos do curso e do campus, por exemplo, dES – Dia da \nomedocurso, evento organizado pela coordenação do curso com o objetivo de promover discussões sobre o curso e áreas temáticas da \nomedocurso entre os alunos, egressos e professores. Também participa diretamente da promoção e organização de eventos tradicionais do campus como o WTISC, os Encontros Universitários, o InfoGirl e o Empreenday, tanto sugerindo e convidando palestrantes da área, como promovendo cursos e oficinas.

    \item Outras funções administrativas: definir pautas e convocar reuniões do colegiado do curso; tratar e encaminhar processos a secretarias e coordenadorias da UFC; auxiliar a Coordenação de Programas Acadêmicos do campus na definição de componentes curriculares ofertadas e alocação de professores, horários e salas; atribuir e acompanhar tarefas da Secretaria Acadêmica; participar do Conselho do campus (que funciona como órgão deliberativo, em todos os assuntos de sua competência, e como órgão consultivo de sua Diretoria) \cite{ufc_regimento_geral_2019}.
\end{itemize}

A atuação do coordenador tem como importante insumo os resultados da Avaliação Institucional, em que discentes se autoavaliam e avaliam a atuação docente, infraestrutura e coordenação; e docentes autoavaliam-se e avaliam suas condições de trabalho. Os dados coletados subsidiam a elaboração de um documento denominado Plano de Melhorias, que tem seu conteúdo debatido no Colegiado do curso e no NDE.

Nas atividades da coordenação há o exercício constante de promoção de uma liderança democrática, onde todos os entes envolvidos têm voz. Essa característica é reforçada nos encontros promovidos pela coordenação com os discentes, na apresentação dos resultados da Avaliação Institucional, na disponibilização de horários de atendimentos para professores e alunos e no esforço constante de criação de um ambiente saudável de trabalho e aprendizagem.

\section{Colegiado do Curso}

Na UFC, o Colegiado é regido pelas resoluções CEPE/UFC, N° 03 de 29 de janeiro de 2016 \cite{ufc2016resolucao03} e CEPE/UFC, n.º 07 de 08 de abril de 1994 \cite{ufc1994resolucao07}. O Colegiado é a instância máxima no plano deliberativo e consultivo do curso, onde são propostas, apreciadas e avaliadas as políticas e ações de gestão, e compõe, junto à coordenação, a esfera administrativa do curso.

O Colegiado do curso é formado por representação docente através das UC, com os representantes eleitos pelos pares, e por representação estudantil, também com representantes eleitos por pares, estes na proporção de 1/5 do total de docentes representantes de UC. No curso de \nomedocurso, o Colegiado é constituído pelos professores titulares representantes das UCs e seus respectivos suplentes, além de um aluno titular e seu suplente, que se reúnem mensalmente, considerando-se como pré-agendadas a realização de reuniões ordinárias na última semana do mês. Caso seja necessário, reuniões extraordinárias podem ser marcadas. %Mais detalhes sobre a composição do Colegiado do curso podem ser encontrados em \url{https://es.quixada.ufc.br/coordenacao/nde-e-colegiado/}.

Todas as deliberações são registradas em ata que, juntamente com os demais documentos de trabalho do Colegiado, ficam disponíveis em arquivos online. Caso alguma deliberação necessite de aprovação superior, a Coordenação conduz a pauta, após discutida em Colegiado, para a próxima reunião do Conselho do campus, onde serão realizados os encaminhamentos necessários.

Além das reuniões mensais, o Colegiado do curso de \nomedocurso amplia suas atividades de deliberação empregando recursos colaborativos online. A utilização de grupo de e-mails e pasta compartilhada agiliza as discussões e permite um aprofundamento em questões mais sensíveis que, sem a utilização desses recursos, não teriam possibilidade de ser debatidas, na mesma profundidade, durante as reuniões.

Uma das importantes ações do Colegiado é deliberar a respeito da demanda por componentes curriculares do curso antes do início de cada semestre. Depois de definida, a Coordenação do curso cadastra as disciplinas correspondentes no sistema ``Oferta Acadêmica'', criado pelo NPI, para auxiliar neste processo. O sistema tem funcionalidades como: importação das disciplinas ofertadas em semestres anteriores; solicitação de vagas em turmas compartilhadas entre os cursos; detalhamento das necessidades de cada disciplina, como aulas em laboratório, turmas divididas, indicação de professores, turnos preferenciais. Concluída esta etapa, a Coordenadoria de Programas Acadêmicos do campus (setor que tem a responsabilidade de acompanhar a programação acadêmica da unidade) trabalha na definição de quais e como as disciplinas serão ofertadas. Para auxiliar nesse processo, um segundo sistema (Sistema de Alocação), também desenvolvido internamente, faz a alocação automática de componentes curriculares, docentes e horários, a partir daquilo que foi demandado por cada Colegiado de curso e das regras específicas cadastradas.

Os principais insumos para a atuação do Colegiado são: os resultados das Avaliações Institucionais realizadas semestralmente (informações detalhadas no Capítulo \ref{cap:processos-de-avaliacao-interna}); demandas de alunos ou docentes encaminhadas diretamente à Coordenação do curso ou a algum dos membros do colegiado; demandas oriundas da Direção ou Conselho do campus. Durante as reuniões, pode ocorrer a participação de convidados (por exemplo, grupos de estudantes) para temas específicos. A análise desses insumos leva à elaboração do relatório de gestão de melhorias (Plano de Melhorias), que pauta a atuação da Coordenação.

\section{Núcleo Docente Estruturante - NDE}
\label{sec:NDE}

Na UFC, o Núcleo Docente Estruturante (NDE) é regido pelas resoluções CEPE/UFC nº 10/2012 (UFC, 2012b) e MEC/CONAES nº 1/2010 (BRASIL, 2010a). O NDE constitui segmento da estrutura de gestão acadêmica em cada curso de graduação, com atribuições consultivas, propositivas e de assessoria sobre matéria de natureza acadêmica e pedagógica, corresponsável pela elaboração, implementação, acompanhamento, atualização e consolidação do Projeto Pedagógico do Curso.

Com caráter de instância autônoma, colegiada e interdisciplinar, vinculada à coordenação de curso, o NDE é composto pelo coordenador do curso e, no mínimo, 5 outros docentes que possuam, preferencialmente, o título de doutor, que atuam no desenvolvimento do curso e exercem liderança acadêmica, percebida na produção de conhecimentos na área, no desenvolvimento do ensino e em outras dimensões entendidas como importantes pela instituição. A escolha dos representantes docentes é feita pelo Colegiado de curso para um mandato de três anos, com possibilidade de uma recondução. A renovação dos membros dá-se pela finalização do mandato ou por necessidade individual, de modo que parte deles permaneça, como modo de preservar o espírito do curso. %Mais detalhes sobre a composição do NDE podem ser encontrados em \url{https://es.quixada.ufc.br/coordenacao/nde-e-colegiado/}.

São atribuições do NDE:
\begin{itemize}
    \item Avaliar, periodicamente, pelo menos a cada três anos no período do ciclo avaliativo do Sistema Nacional de Avaliação da Educação Superior (SINAES) e, sempre que necessário, elaborar propostas de atualização para o PPC e encaminhá-las para apreciação e aprovação do Colegiado do curso;

    \item Fazer o acompanhamento curricular do curso, tendo em vista o cumprimento da missão e dos objetivos definidos em seu Projeto Pedagógico;
    
    \item Zelar pela integração curricular interdisciplinar entre as diferentes atividades de ensino constantes no currículo;
    
    \item Contribuir para a consolidação do perfil profissional do egresso do curso;

    \item Indicar formas de incentivo ao desenvolvimento de linhas de pesquisa e extensão, oriundas de necessidades da graduação, de exigências do mundo do trabalho e afinadas com as políticas públicas relativas à área de conhecimento do curso;
    
    \item Zelar pelo cumprimento das Diretrizes Curriculares Nacionais para os cursos de graduação;

    \item Sugerir e fomentar ações voltadas para a formação e o desenvolvimento dos docentes vinculados ao curso.
\end{itemize}

No curso de \nomedocurso, o NDE realiza todas essas atividades em parceria com a Coordenação e o Colegiado do curso, sempre com a preocupação de confirmar que o trabalho desenvolvido está alinhado com o perfil do egresso e com as novas demandas do mundo do trabalho.

O NDE do curso de \nomedocurso encontra-se constituído e atuante, reunindo-se ordinariamente uma vez ao semestre, com suas deliberações registradas em ata e alinhado com a regulamentação da UFC.

% \section{Integração com as redes públicas de ensino/Integração do curso com o sistema local e regional de saúde-SUS}

\section{Apoio ao discente}
Fortalecer o vínculo institucional do estudante pelas condições de acesso, permanência, melhoria contínua e qualidade de vida é a missão da Pró-reitoria de Assuntos Estudantis (PRAE) da UFC. Suas atribuições são ampliar as condições de permanência dos estudantes em situação de vulnerabilidade socioeconômica na UFC, viabilizar a igualdade de oportunidades entre os estudantes, contribuir para a melhoria do desempenho acadêmico individual e agir, preventivamente, nas situações de retenção e evasão decorrentes da insuficiência de condições financeiras.

No Campus da UFC em Quixadá, o Núcleo de Atendimento Social (NAS) implementa e acompanha os projetos e ações promovidos pela PRAE.

\subsection{NÚCLEO DE ATENDIMENTO SOCIAL (NAS)}
\label{sec:NAS}

O NAS foi previsto no Planejamento Estratégico realizado no Campus da UFC em Quixadá em 2013. Dentre as atividades e ações de assistência estudantil, foi prevista a criação de um setor de apoio à saúde e bem-estar dos discentes para promoção de serviços de assistência psicopedagógica dentro do campus.

Grande parte dos alunos do campus é de fora da cidade de Quixadá e passa por diversos desafios no decorrer do seu percurso acadêmico. Para auxiliá-los, o NAS promove a implementação de políticas, programas e ações de acompanhamento e orientação, além de trabalharem em conjunto com as coordenações dos cursos, no desenvolvimento de estratégias de acessibilidade metodológica e instrumental. O núcleo conta com uma equipe interdisciplinar, formada por um psicólogo, um assistente social e um nutricionista, pautada no diálogo e preparada para elaborar, implementar e coordenar projetos que promovam a articulação entre as políticas de ensino superior à assistência estudantil, integrando processos psicossociais e educacionais, identificando e superando desafios; além de viabilizar ações, serviços e programas que previnam a evasão estudantil.

O núcleo também gerencia os processos de seleção e acompanhamento de Bolsa de Iniciação Acadêmica, Auxílio Moradia, Renovação do Auxílio Moradia, Auxílio Emergencial, Auxílio Creche, Isenção do RU. Dentre as atividades desenvolvidas nesses processos, estão: a formulação de edital, a definição de cronogramas, a formação de comissão, o recebimento de documentos, a realização de entrevistas, a análise documental, a divulgação dos resultados (apoiada pelo Núcleo de Comunicação do campus), o controle das listas de frequência dos bolsistas e o acompanhamento de seu desempenho acadêmico. %Em 2017, cerca de 500 alunos do campus participaram de uma ou mais dessas atividades.

Uma das atividades diretamente relacionadas com o acolhimento e a permanência de discentes é a palestra de recepção aos ingressantes, com apresentação dos serviços e pessoal do núcleo aos  alunos que chegam ao campus no início do ano letivo. %300

Com esse mesmo objetivo, a coordenação do Projeto de Orientação Acadêmica (POA) é outra atividade importantíssima promovida pelo NAS. O projeto propõe o acompanhamento dos alunos, principalmente, dos anos iniciais, com o objetivo de aprimorar sua visão sobre o ambiente universitário e a vida acadêmica, dedicando-se ao aproveitamento nas disciplinas e em atividades que complementam sua formação. %Foi iniciado também em 2017.2, o processo de Orientação Acadêmica em Grupo, baseado na orientação acadêmica individual, mas tentando atender um maior número de alunos. Em 2021, foram realizados onze encontros, com temas relacionados ao âmbito acadêmico e à assistência estudantil. %No último ano, cerca de 43 alunos foram acompanhados por um Orientador Acadêmico no campus.  % através da tutoria de um servidor experiente (na função de orientador acadêmico)

No início do ano letivo, é aplicado o Questionário sobre o Perfil do Aluno, cujo objetivo é conhecer mais profundamente os discentes recém-ingressos, abordando questões como características socioeconômicas e demográficas, perfil do estudante, motivos da escolha do curso, expectativas e receios (SOUZA, 2016) (SOUZA, 2017).

A Jornada de Iniciação Acadêmica (JOIA) é um exemplo de evento realizado pelo NAS. Anualmente, e especialmente para os alunos participantes do processo de seleção da Bolsa de Iniciação Acadêmica, o núcleo apresenta as características da bolsa e promove ações de interação entre os discentes, com o objetivo de identificar possíveis relações entre seus perfis e os projetos desenvolvidos no campus aos quais eles podem vincular-se. %Em 2021, cerca de 108 alunos participaram do JOIA.

%Em abril de 2018, foi realizada uma mesa redonda organizada pelo NAS e pelo Núcleo de Cultura e Arte do campus com o tema “Uma discussão sobre o machismo pela perspectiva de gênero”, que contou com a participação de convidados da UFC, da Universidade da Integração Internacional da Lusofonia Afro-Brasileira (UNILAB) e da FECLESC. Destacando a importância de falar sobre o machismo de maneira geral, mas também trazendo para a realidade de mulheres na área de TIC, assim como no âmbito universitário, o evento procurou esclarecer questões sobre machismo, feminismo e dar luz a situações que mulheres passam e que acabam não percebendo, dado o enraizamento do machismo na cultura.

% Existe a previsão de ampliação das atividades do NAS, com o reforço à saúde dos discentes a partir da oferta de atendimento odontológico no campus.

A seguir, são apresentados os serviços que compõem o NAS e as ações especificamente por ele promovidas.

\subsubsection{Serviço de Psicologia}

O Serviço de Psicologia do Campus da UFC em Quixadá visa à promoção de um ambiente educativo harmônico e produtivo e tem por finalidade apoiar, orientar e acompanhar o aluno nos diversos processos de adaptação frente à realidade do contexto universitário. Com esse objetivo, o serviço procura analisar e corrigir os equívocos institucionais que comprometem a qualidade do processo de ensino-aprendizagem e as relações intersubjetivas estabelecidas entre os diversos atores que compõem a vida acadêmica, contribuindo para o bem-estar e para a qualidade de vida dos alunos do campus.

O serviço oferece aconselhamento e orientação psicológica aos alunos cuja dificuldade esteja centrada nos fatores psíquicos ligados às questões acadêmicas, focando nos aspectos cognitivos, sociais e afetivos que geram resistência e dificultam o processo de aprendizagem, o desempenho acadêmico e o bem-estar estudantil.

\subsubsection{Serviço de Nutrição}

A alimentação saudável é essencial para a manutenção da saúde e deve estar baseada em práticas alimentares que tenham significado social e cultural, além de ser acessível do ponto de vista físico e financeiro. O Serviço de Nutrição tem como objetivo auxiliar os alunos a adotar hábitos mais saudáveis a partir de duas ações principais: a orientação nutricional individualizada e a coordenação do Refeitório Universitário (RU) do campus.

O RU oferece refeições balanceadas e de qualidade, a um valor monetário simbólico, subsidiado pela universidade, aos alunos (R\$1,10 na recarga via GRU e R\$3,00 no guichê de vendas), e constitui-se como espaço de convivência e integração de estudantes, docentes e servidores técnico-administrativos.

% Durante o período da pandemia, as atividades do RU foram suspensas. Em 2019 (último ano antes da pandemia), foram servidas no campus, aproximadamente, 152.919 refeições, sendo 95.172 almoços e 57.747 jantares. No segundo semestre, teve início o processo de informatização do sistema de tickets, permitindo a consulta, em tempo real, do número de refeições em toda a instituição. O dia 23/08/2017 ficou marcado por ter sido a primeira vez que um tíquete eletrônico foi utilizado no campus, durante o jantar.

O Serviço de Nutrição avalia a qualidade do alimento servido no RU, essencialmente, de duas formas. Primeiro, a partir da promoção diária de uma pesquisa de satisfação sobre as refeições, com o objetivo de recolher a opinião da comunidade acadêmica. Após a refeição, cada comensal coloca em uma urna, conforme a sua percepção, uma avaliação que varia em três níveis, separados por tipo de proteína consumida (carne vermelha, carne branca, vegetariana). O segundo método é feito por meio da pesagem dos alimentos recebidos e do lixo descartado, conhecida como relação resto-ingestão. O estudo do resultado desses métodos permite ao Serviço de Nutrição acompanhar a qualidade do alimento servido e dá informações preciosas sobre a necessidade de alterações no cardápio do RU.

%No ano de 2021, foram atendidos, de forma remota, 47 alunos nesse serviço. 
Para além das atividades de gestão do RU, o serviço oferece atendimento nutricional à comunidade acadêmica. Após o atendimento pelo nutricionista, uma nova consulta é marcada para a entrega do plano nutricional individualizado proposto. %Além dessas, outras consultas informais, não contabilizadas, foram realizadas para fins de acompanhamento, esclarecimento de dúvidas, reformulação do plano proposto, etc.

%Complementando o atendimento individualizado, desde 2018, o Serviço de Nutrição promove o grupo “Mitos e verdades sobre alimentação”. Em cada encontro é debatido um tema específico da área, com uma breve exposição seguida de debate. Além disso, em 2020, foi criado um canal no Telegram, denominado “Shots de Nutrição”, com o objetivo de ajudar a repensar a alimentação no dia a dia, com o envio de informações gerais sobre alimentação saudável.

\subsubsection{Serviço Social}

Na área de Serviço Social, o assistente social realiza suas intervenções no atendimento à população e/ou na formulação e execução de políticas públicas que possibilitam o acesso aos direitos sociais, com base em uma formação crítica, que o capacita para realizar a análise da realidade e intervir nas várias questões apresentadas.

O Serviço Social do Campus da UFC em Quixadá volta-se para o atendimento das demandas dos discentes, com base na Política Nacional de Assistência Estudantil (PNAES, Decreto n.º 7.234/2010) \cite{decreto_7234_2010}. Além do trabalho direto com os auxílios oferecidos pela PRAE, o Serviço Social realiza orientações aos alunos e encaminhamentos para a rede socioassistencial do município de Quixadá e adjacências. Realiza ainda pesquisas e ações educativas relativas às mais variadas expressões da questão social, tais como violência, vulnerabilidade socioeconômica, dentre outros. Os atendimentos para orientação são realizados através do agendamento, realizado na sala do Serviço Social ou por e-mail.

\subsection{APOIO PEDAGÓGICO E ACADÊMICO}

Todos os docentes do curso são responsáveis pelo acompanhamento e apoio pedagógico, de maneira sistemática, com horários de atendimento aos discentes fora de sala de aula. Além deles, a coordenação tem papel importante nessa área, com o acompanhamento individual de matrícula, orientação sobre carga-horária adequada ao discente, planejamento do fluxo curricular para alunos com reprovações e o planejamento e adequação da oferta de componentes curriculares, como a inclusão de turmas extras na medida da necessidade, visando minimizar o represamento curricular.

Complementando o trabalho desenvolvido por docentes e coordenação, é importante ressaltar duas outras ações desenvolvidas no campus, alinhadas às políticas institucionais descritas no PDI da UFC: o Programa de Orientação Acadêmica (POA) e a mobilidade acadêmica.

\subsubsection{Programa de Orientação Acadêmica (POA)}
\label{sec:POA}

Ao longo do percurso formativo, inúmeros aspectos podem interferir no processo de aprendizagem, dentre eles a adaptação do aluno ao contexto universitário. O ingresso na vida acadêmica pode constituir-se como momento de crise na vida do sujeito, visto que a transição do Ensino Médio para o Ensino Superior implica no aumento de responsabilidades e na necessidade de desenvolver autonomia. Trata-se de uma fase geradora de amadurecimento e, ao mesmo tempo, desencadeadora de sentimentos de vulnerabilidade e desamparo. Soma-se a isso o fato de a maior parte dos alunos do campus vir de outras cidades e terem, ainda muito novos e inexperientes, que sair da casa dos pais, para morar em Quixadá, o que os leva a assumir responsabilidades da “vida adulta” que até então não conheciam.

Diante desse cenário, foi criado o Programa de Orientação Acadêmica (POA), desenvolvido como uma política inovadora do Campus da UFC em Quixadá, que busca promover a integração dos alunos à vida universitária, orientando-os quanto às suas atividades acadêmicas, prioritariamente nos dois anos iniciais do curso, contribuindo, dessa forma, para o processo de socialização e ambientação dos alunos ao campus.

%São duas modalidades de acompanhamento. A individual, onde cada aluno participante do programa tem um orientador específico (docente ou servidor técnico-administrativo), com quem mantém encontros regulares. O planejamento é realizado de forma singular, pensado a partir da realidade de cada aluno.

As atividades ocorrem por meio de orientação grupal, nas quais, periodicamente, são realizados encontros para trabalhar temáticas relacionadas à trajetória acadêmica (exemplos de temáticas: “Mercado de Trabalho x Área Acadêmica”, “Encontros Universitários: Por que participar?”, “Desmistificando a Universidade”, dentre outras).

%No POA, articulam-se os membros da comunidade acadêmica em diferentes níveis. São atribuições dos orientadores acadêmicos: pensar, junto ao aluno, considerando a programação acadêmica do seu curso, um fluxo curricular compatível com seus interesses e possibilidades de desempenho acadêmico; orientar a tomada de decisões relativas à matrícula; apresentar aos alunos o projeto pedagógico do curso de graduação e a estrutura universitária; encontrar-se pelo menos 2 vezes por semestre com seus alunos; e entregar à secretaria acadêmica, ao final de cada semestre letivo, relatório das atividades realizadas por cada aluno.

%Os alunos que participam do programa têm o compromisso de: manter contato com o orientador para o agendamento dos encontros; participar ativamente da construção do plano de estudos e de outras atividades propostas pelo orientador, atuando como protagonista no processo; agendar novos encontros com o orientador sempre que julgar necessário.

A comunidade acadêmica compromete-se a: acolher os estudantes no contexto universitário, viabilizando a sua integração; colaborar para a promoção de estratégias dialógicas de ensino-aprendizagem; favorecer processos comunicacionais envolvendo servidores e discentes; desenvolver a autonomia e o protagonismo dos estudantes na busca de soluções para os desafios do cotidiano universitário; e sanar os fatores de retenção, desistência e abandono, promovendo ações que identifiquem e minimizem os problemas no âmbito de cada curso.

\subsubsection{Mobilidade Acadêmica}

A mobilidade acadêmica é o processo que possibilita ao discente matriculado em uma IES estudar em outra e, após a conclusão dos estudos, obter um comprovante de estudos e, possivelmente, o aproveitamento de disciplinas em sua instituição de origem.
A mobilidade acadêmica envolve a existência de condições apropriadas, que contribuem com a formação e o aperfeiçoamento dos quadros docente e discente, objetivando a aquisição de novas experiências e a interação com outras culturas.

Os discentes do campus têm acesso a duas modalidades de mobilidade acadêmica, oferecidas pela PROGRAD e pela Pró-Reitoria de Relações Internacionais (PROINTER).

O Programa Andifes de Mobilidade Acadêmica alcança somente alunos regularmente matriculados em cursos de graduação de universidades federais, que tenham concluído pelo menos 20\% (vinte por cento) da carga horária de integralização do curso de origem e com no máximo duas reprovações acumuladas nos dois períodos letivos que antecedem o pedido de mobilidade. O estudante da UFC pode solicitar a mobilidade acadêmica a qualquer tempo, mas deve buscar informações junto à IFES de seu interesse sobre seus prazos e procedimentos.

É possível também a mobilidade acadêmica entre a UFC e instituições no exterior, em programas promovidos pela PROINTER, órgão que coordena as relações da universidade com instituições estrangeiras de educação, ciência e cultura, bem como oferece o suporte necessário à execução de convênios e acordos internacionais através das atividades desenvolvidas pelas unidades que lhe são subordinadas (no campus, a Coordenadoria de Assuntos Internacionais).

Os pedidos de inscrição dos alunos que desejem participar de programas de mobilidade acadêmica são realizados mediante encaminhamento do coordenador do curso à PROINTER, juntamente com o plano de estudos elaborado pelo aluno, contendo as disciplinas que cursará na IES desejada. Cabe ao coordenador do curso analisar as solicitações de afastamento temporário, bem como os programas das disciplinas a serem cursadas, de modo a permitir, inequivocamente, a posterior e obrigatória concessão de equivalência e consequente dispensa. O coordenador emitirá parecer conclusivo sobre as solicitações e informará a PROINTER para que esta providencie junto a IES pretendida a efetivação do Intercâmbio.

A UFC é conveniada a diferentes programas de mobilidade internacional oferecidos por diversos países, como os Programas BRAFITEC e Duplo Diploma de Graduação em engenharia (com a França), e UNIBRAL e PROBRAL (com a Alemanha), além do programa Erasmus Mundus que já beneficiou vários estudantes da UFC através de projetos coordenados pela Universidade Técnica de Munique (Alemanha), pela Universidade de Santiago de Compostela (Espanha) e pela Universidade do Porto (Portugal).

A UFC também participou do PROGRAMA ALFA, programa de cooperação entre IES da União Europeia e da América Latina, e PROGRAMA ALBAN, programa de cooperação entre União Europeia e países latino-americanos, destinado a estudantes e profissionais latino-americanos e futuros acadêmicos.

\subsection{POLÍTICA DE BOLSAS E AUXÍLIOS}
\label{sec:bolsas-e-auxilios}

%A UFC disponibiliza diversos programas de auxílios financeiros para permanência e integração acadêmica do discente, como o Programa de Bolsas de Auxílio Moradia e de Iniciação Acadêmica, que constituem auxílios para alunos socialmente vulneráveis, além da oferta de bolsas de mérito acadêmico, como as dos programas de Iniciação à Docência e de Iniciação Científica. Além disso, o campus também conta com a Coordenadoria de Estágios, responsável por apoiar e acompanhar o planejamento dos estágios curriculares obrigatórios e não-obrigatórios, como apresentado no Capítulo 10. A seguir, são apresentados os programas de bolsas e auxílios que constituem a política institucional nesse âmbito.

A UFC conta com diversas políticas e programas para apoio ao estudante, visando a permanência e a integração acadêmica dos discentes. Dentre os programas existentes, alguns oferecem auxílio financeiro para alunos em vulnerabilidade socioeconômica, como o Programa de Auxílio Moradia e o Programa de Bolsas de Iniciação Acadêmica (BIA); enquanto outros oferecem bolsas por mérito acadêmico, como os Programas de Iniciação à Docência e de Iniciação Científica.  O Campus da UFC em Quixadá também conta com a Coordenadoria de Estágios, obrigatórios e não obrigatórios, como apresentado no Capítulo \ref{cap:estagio-curricular-supervisionado}. Os programas que constituem a política institucional de bolsas e auxílios são apresentados a seguir.

\subsubsection{Programa Ajuda de Custo}

%Concede apoio aos estudantes dos cursos de graduação que desejam apresentar trabalhos em eventos de naturezas diversas. Apoia o Diretório Central dos Estudantes (DCE), os Centros Acadêmicos (CA) e as Associações Atléticas na participação em eventos, com representação de delegados e equipes de modalidades esportivas e na promoção de eventos acadêmicos, políticos, culturais e esportivos.
Visa complementar as despesas referentes ao deslocamento de estudantes, concedendo apoio financeiro aos estudantes dos cursos de graduação que desejam participar de eventos de naturezas diversas, representando a Universidade. Também apoia o Diretório Central dos Estudantes (DCE), os Centros Acadêmicos (CA) e as Associações Atléticas na participação em eventos, com representação de delegados e equipes de modalidades esportivas e na promoção de eventos acadêmicos, políticos, culturais e esportivos.

\subsubsection{Programa Auxílio Moradia}


%Tem por objetivo viabilizar a permanência de estudantes matriculados nos cursos de graduação, localizados fora dos municípios de residência e que estejam em comprovada situação de vulnerabilidade socioeconômica assegurando-lhes auxílio institucional para complementação de despesas com moradia e alimentação durante todo o período do curso ou enquanto persistir a situação. A vinculação dos estudantes ao programa não os impede de receber, por mérito, qualquer uma das bolsas dos diversos programas da UFC, de agências de fomento ou de empresas.
Tem por objetivo viabilizar a permanência de estudantes matriculados nos cursos de graduação, localizados fora dos municípios de residência, e que estejam em comprovada situação de vulnerabilidade socioeconômica. Assegura a esses estudantes um auxílio institucional para complementação de despesas com moradia e alimentação durante todo o período do curso ou enquanto persistir a situação de vulnerabilidade. A vinculação dos estudantes ao programa não os impede de receber, por mérito, qualquer uma das bolsas dos diversos programas da UFC, de agências de fomento ou de empresas.

\subsubsection{Auxílio Emergencial}

É um auxílio em dinheiro destinado aos estudantes dos cursos presenciais de graduação, que apresentem vulnerabilidade socioeconômica comprovada e que atendam a um dos requisitos que estejam impedindo que o aluno frequente normalmente as aulas. São atendidos pelo auxílio estudantes que moram em cidades diferentes do Campus  em que está matriculado, com dificuldades financeiras para realizar o  deslocamento diário;  estudantes com necessidades médicas (inclusive de saúde mental) sem os quais ficará difícil frequentar normalmente as aulas, mediante apresentação de laudo médico; estudantes que apresentem dificuldades para aquisição de material acadêmico, mediante solicitação formal do professor da disciplina e orçamento, que sejam essenciais para cursar a matéria em questão, exceto para pagamento de fotocópia e compra de livros; estudantes que estejam em situação agravante de vulnerabilidade ou risco social que não conseguiram acessar nenhum outro auxílio, bolsa ou estágio, avaliados por meio de estudo social.

\subsubsection{Auxílio Creche}

O Auxílio Creche foi regulamentado por meio do anexo \textit{ad referendum} XXI – da resolução n.º 08/2013 do CEPE. Podem solicitar este benefício estudantes regularmente matriculados em cursos de graduação da UFC, com a carga horária mínima exigida em edital, que atendam aos critérios do Programa Nacional de Assistência Estudantil (PNAES) e tenham filhos de 4 meses a 6 anos incompletos, sob sua guarda e convivendo no mesmo domicílio. O auxílio creche tem por objetivo apoiar nas despesas com o cuidado da criança, permitindo que o estudante frequente as aulas regularmente, mantenha bom desempenho acadêmico e conclua o curso no tempo previsto.



\subsubsection{Programa de Promoção da Cultura Artística}

%Mais conhecido como Bolsa Arte, foi instituído pela Resolução nº 08 do CEPE – Conselho de Ensino, Pesquisa e Extensão, em sua reunião de 26 de abril de 2013. O programa tem como objetivo principal oferecer aos estudantes, servidores docentes e técnico-administrativos da UFC condições para produção, realização e fruição de bens artístico-culturais e tem duração de um ano.
%O programa é gerido pela Secretaria de Cultura Artística (Secult-Arte/UFC), que tem por objetivo trabalhar pela articulação das iniciativas relacionadas às artes na instituição, incentivando e apoiando ações e projetos. A partir do apoio dispensado em diferentes ações, visa fortalecer a cultura artística, compreendida como dimensão inalienável da vida universitária, buscando criar estratégias para o incremento da produção estética nas diversas linguagens das artes, e estimulando a reflexão crítica sobre esta mesma produção.

Considerando a necessidade de fazer com que a cultura artística esteja presente nos processos de formação universitária desenvolvidos na UFC, o CEPE instituiu  em 26 de abril de 2013 o Programa de Promoção da Cultura Artística, também conhecido como Bolsa Arte. O programa tem como objetivo principal oferecer aos estudantes, servidores docentes e técnico-administrativos da UFC condições para a produção, realização e fruição de bens artístico-culturais e tem duração de um ano.

O programa é gerido pela Secretaria de Cultura Artística (Secult-Arte/UFC), que trabalha pela articulação das iniciativas relacionadas às artes na instituição, incentivando e apoiando ações e projetos. A partir do apoio dispensado em diferentes ações, visa fortalecer a cultura artística, compreendida como dimensão inalienável da vida universitária, buscando criar estratégias para o incremento da produção estética nas diversas linguagens das artes, e estimulando a reflexão crítica sobre esta mesma produção.

\subsubsection{Programa de Educação Tutorial (PET)}

O campus conta com dois grupos PET, a saber: PET-TI (Conexão dos Saberes, Tecnologia da Informação) e PET-SI (Sistemas de Informação). Os grupos desenvolvem ações de pesquisa, ensino e extensão, contando com o apoio de um tutor, 
com o objetivo de fortalecer os vínculos entre a instituição e a comunidade de Quixadá. O PET-TI é um grupo multidisciplinar, o qual pode ser constituído por discentes de todos os cursos do campus. O PET-SI é um grupo exclusivo para alunos do curso de Sistemas da Informação. No entanto, também desenvolve atividades em colaboração com o PET-TI.  % e, em 2022, de uma equipe de oito alunos, dois bolsistas são do curso de \nomedocurso.

A equipe de bolsistas é renovada à medida que os membros mais antigos terminam o curso. O processo de seleção é realizado por uma comissão composta por membros atuais do PET, professores e o tutor, e os bolsistas, quando iniciam suas atividades, devem ter disponibilidade de participar por pelo menos dois anos do programa.

Os grupos PETs são responsáveis pela promoção de um número importante de atividades no campus, como: a recepção e orientação dos alunos ingressantes, oferecendo informação acerca do campus, do curso e da vida acadêmica; o InfoGirl, realizado anualmente, busca atrair mulheres para a área de TI, apresentando a área para meninas do ensino médio de escolas públicas por meio de palestras, oficinas, \textit{workshops} e roda de conversa com profissionais e alunas do campus; o ensino de programação nas escolas, com minicursos ofertados em escolas públicas de Ensino Fundamental (Anos Finais) e Médio; o grupo de preparação para a OBI (Olimpíada Brasileira de Informática) e Maratona de Programação, eventos promovidos pela Sociedade Brasileira de Computação, para a resolução de desafios de lógica e programação; o ``\textit{Workshop} de Tecnologia da Informação do Sertão Central (WTISC)'', realizado anualmente, promove palestras, minicursos, mesas redondas e \textit{hackathons} em diversas  áreas de Tecnologia da Informação de forma a fortalecer e motivar a formação técnico-profissional dos estudantes; e o PET-Recebe, projeto colaborativo entre os PET-TI e PET-SI, que organiza visitas guiadas de alunos de escolas de ensino médio ao Campus.

%o Ecopet, que tem o objetivo de conscientizar os alunos e servidores sobre a importância da preservação ambiental, promovendo práticas sustentáveis dentro e fora da universidade por meio de desafios, apresentações e competições que envolvam alunos e servidores; os seminários de pesquisa, visando acompanhar as atividades individuais e coletivas do grupo, são realizados seminários de pesquisas semanais; a edição regional do “Festival latino-americano de instalação de software livre – FLISoL”, evento anual cujo objetivo é promover o uso de Software Livre, mostrando ao público em geral sua filosofia, abrangência, avanços e desenvolvimento; o “Dojo de Programação” uma série de reuniões entre desenvolvedores com diferentes níveis de experiência, com a finalidade de aprimorar técnicas e metodologias de programação; e o 

%\subsubsection{Programa Institucional de Bolsas de Administração (PIBAD)}

%O PIBAD tem como finalidade promover a inserção dos estudantes nas unidades administrativas e acadêmicas da universidade, por meio da interação dos conhecimentos inerentes a rotinas administrativas necessárias à gestão destas unidades, nos moldes estabelecidos em seus editais. O programa é gerenciado pela Pró-reitoria de Pesquisa e Pós-Graduação (PRPPG).

\subsubsection{Programa de Bolsas de Extensão Universitária}

%O programa destina bolsas ao estudante de graduação vinculado a uma ação de extensão, orientado e acompanhado por um professor ou servidor técnico-administrativo de nível superior vinculado ao quadro da UFC, e tem como objetivos: apoiar, por meio da concessão de bolsas, alunos regularmente matriculados em cursos de graduação da UFC, proporcionando o desenvolvimento de ações de extensão, com vistas à formação cidadã e à transformação social; viabilizar a participação de discentes no processo de interação entre a universidade e outros setores da sociedade através de atividades acadêmicas que contribuam para a sua formação acadêmica, profissional e para o exercício da cidadania; incentivar os processos educativos, culturais, científicos e tecnológicos, como forma de aprendizagem da atividade extensionista, articulados com o ensino e a pesquisa de forma indissociável e que viabilizem a relação transformadora entre a universidade e outros setores da sociedade, contribuindo para a inclusão social; fomentar o interesse em extensão universitária e incentivar novos talentos potenciais entre estudantes de graduação, assim como contribuir para a formação e a qualificação de cidadãos socialmente comprometidos.

O programa destina bolsas como auxílio financeiro ao estudante de graduação vinculado a um Programa ou a um Projeto de Extensão, orientado e acompanhado por um professor ou servidor técnico-administrativo de nível superior vinculado ao quadro da UFC. O programa tem como objetivos: apoiar, por meio da concessão de bolsas, alunos regularmente matriculados em cursos de graduação da UFC, proporcionando o desenvolvimento de ações de extensão, com vistas à formação cidadã e à transformação social; 
possibilitar a implementação do processo de curricularização da
extensão nos cursos de graduação da UFC no âmbito da Unidade Curricular
Especial de Extensão;
viabilizar a participação de discentes no processo de interação entre a universidade e outros setores da sociedade através de atividades acadêmicas que contribuam para a sua formação acadêmica, profissional e para o exercício da cidadania; 
incentivar os processos educativos, culturais, científicos e tecnológicos, como forma de aprendizagem da atividade extensionista, articulados com o ensino e a pesquisa de forma indissociável e que viabilizem a relação transformadora entre a universidade e outros setores da sociedade, contribuindo para a inclusão social; 
fomentar o interesse em extensão universitária e incentivar novos talentos potenciais entre estudantes de graduação, assim como contribuir para a formação e a qualificação de cidadãos socialmente comprometidos.

A extensão no campus guarda consonância com as orientações da PREx, com o esforço de integração entre ensino, pesquisa e extensão, no sentido de levar conhecimento prático à comunidade. Em 2025, um total de 32 servidores docentes e 12 servidores técnico-administrativos do campus esteve envolvido em 29 ações de extensão, com a participação de 46 alunos.

Para além das bolsas, o apoio ao desenvolvimento de ações de extensão é contemplado por um conjunto de estratégias e mecanismos institucionais. As Ações Curriculares em Comunidades de Saberes (ACCS), por exemplo, preveem um pequeno custeio para ser utilizado nas atividades extensionistas. A infraestrutura do campus, incluindo salas, laboratórios, mobiliário e equipamentos, também é disponibilizada para a realização das ações. Conta com apoio logístico, como o uso de transporte institucional para o deslocamento de discentes e docentes às comunidades atendidas. Adicionalmente, o campus busca ativamente o estabelecimento de parcerias com empresas e outras instituições para a efetivação da interação com a sociedade.

%\subsubsection{Programa de Desenvolvimento Institucional em Tecnologia da Informação}


\subsubsection{Programa de Iniciação à Docência (PID)}

O Programa de Iniciação à Docência (PID) é desenvolvido em duas modalidades, monitoria remunerada e monitoria voluntária. Na primeira, o monitor recebe uma bolsa-auxílio para desempenhar as funções e, por isso, não deve participar de qualquer outra atividade remunerada, seja pública ou privada. Na segunda, o monitor desempenha as atividades de maneira voluntária, sem o recebimento do auxílio.

A carga horária da monitoria é de 12 horas semanais e deve ser cumprida sem afetar as demais atividades acadêmicas do estudante. A duração é de 10 meses, mas a monitoria pode ser renovada uma vez, por igual período, caso o bolsista seja novamente aprovado em processo seletivo. A função de monitor não constitui cargo ou emprego, nem representa vínculo empregatício de qualquer natureza com a universidade, e é uma importante estratégia para o nivelamento dos discentes com mais dificuldades, que têm a oportunidade de reforçar seus estudos com colegas mais experientes.

Algumas das atividades dos bolsistas de PID são: elaborar, juntamente com o professor-orientador, o plano de trabalho da monitoria; participar das tarefas didáticas, inclusive na programação de aulas e em trabalhos escolares; auxiliar o professor-orientador na realização de trabalhos práticos e experimentais, na preparação de material didático e em atividades de classe e/ou laboratório; contribuir, juntamente com o professor-orientador, para a avaliação do andamento da disciplina ou da área. Em 2025, o Campus da UFC em Quixadá teve 18 projetos aprovados, com 25 vagas de bolsas remuneradas e 20 vagas para voluntários.

%Em 2021, 25 bolsistas de PID eram alunos do Campus da UFC em Quixadá.

\subsubsection{Programa Institucional de Bolsas de Iniciação Científica (PIBIC)}

O Programa Institucional de Bolsas de Iniciação Científica (PIBIC) é um programa voltado para o desenvolvimento do pensamento científico e iniciação à pesquisa de estudantes de graduação do ensino superior. Atualmente, o PIBIC é o principal programa de iniciação científica da universidade, resultado de um convênio entre a UFC, o CNPq e a Fundação Cearense de Apoio ao Desenvolvimento Científico e Tecnológico (Funcap).


%Em 2022, o PIBIC contou com 601 cotas de bolsas do CNPq. Essas bolsas são destinadas às várias unidades da graduação. Cada orientador poderá ter, no máximo, dois bolsistas. Pesquisadores que já possuem bolsa BPI da Funcap podem ter, no máximo, uma bolsa. As bolsas BPI são uma espécie de bolsa de produtividade em pesquisa da agência estadual.

Podem concorrer às quotas de bolsa, além de docentes, servidores técnico-administrativos da universidade portadores do título de doutor, em regime de dedicação exclusiva ou 40 horas. São usados critérios de produtividade, produção científica, tecnológica e artística no julgamento dos projetos.

Nos últimos anos, os editais PIBIC reservaram 20\% das cotas de bolsa da UFC para estudantes dos campi no interior do estado. Em 2024, o campus da UFC em Quixadá contou com 11 bolsas PIBIC de diferentes projetos. 

\subsubsection{ Programa Institucional de Bolsas de Iniciação em Desenvolvimento Tecnológico e Inovação (PIBITI)}

O Programa Institucional de Bolsas de Iniciação em Desenvolvimento Tecnológico e Inovação (PIBITI) tem por objetivo principal estimular os jovens do ensino superior nas atividades, metodologias, conhecimentos e práticas próprias ao desenvolvimento tecnológico e processos de inovação. 

Outros objetivos do PIBITI são: i) Contribuir para a formação e inserção de estudantes em atividades de pesquisa, desenvolvimento tecnológico e inovação; ii) Contribuir para a formação de recursos humanos que se dedicarão ao fortalecimento da capacidade inovadora das empresas no país; e iii) Contribuir para a formação do cidadão pleno, com condições de participar de forma criativa e empreendedora na sua comunidade.


%4 - 2022
%4 - 2021
% 2023 - 4 e 4 voluntárias
% 2024 - 1 e 2 volntária

Os responsáveis pela gestão do programa PIBITI são a Coordenadoria de Pesquisa da Pró-Reitoria de Pesquisa e Pós-Graduação (PRPPG) da UFC, com a ajuda do Comitê Interno. Periodicamente, o programa é avaliado pelo Comitê Externo composto por pesquisadores destacados de outras Instituições de Ensino Superior.

Em 2021 e 2022, o Campus Quixadá contou com 4 bolsas PIBITI. Em 2023, foram 4 bolsas remuneradas e 4 voluntários. Em 2024, houve um bolsista remunerado e dois voluntários.

\subsubsection{Programa de Iniciação Acadêmica}

O Programa de Iniciação Acadêmica tem por objetivo  propiciar aos estudantes de cursos de graduação presenciais da UFC – em situação de vulnerabilidade socioeconômica comprovada, especialmente os de semestres iniciais – condições financeiras para sua permanência e desempenho acadêmico satisfatório por meio  da concessão de uma bolsa e integração ao ambiente universitário. O bolsista deve atuar, em caráter de iniciação acadêmica, nas diversas unidades da instituição. Além do auxílio financeiro, o programa também proporciona ao estudante um espaço para conhecer e participar de ações de iniciação à universidade, ensino, pesquisa e extensão, artes, gestão, dentre outros.

O campus Quixadá dispõe anualmente de 80 vagas para Bolsas de Iniciação Acadêmica. Ao longo do ano, é comum que alguns bolsistas se desliguem do programa para assumir outras oportunidades. Nesses casos, novas chamadas são realizadas a partir da lista de espera, o que pode resultar em um número total de beneficiados superior ao número inicial de vagas.


%O Programa Bolsa de Iniciação Acadêmica tem por objetivo propiciar aos estudantes de cursos de graduação presenciais da UFC – em situação de vulnerabilidade socioeconômica comprovada, especialmente os de semestres iniciais – condições financeiras para sua permanência e desempenho acadêmico satisfatório, mediante atuação, em caráter de iniciação acadêmica, nas diversas unidades da instituição. 

%Site O Programa de Iniciação Acadêmica é uma forma de oferecer um espaço para que o estudante possa conhecer e participar de ações de iniciação à universidade, ensino, pesquisa e extensão, artes, gestão, dentre outros; além de ter a oferta direta de uma bolsa

%Em 2021, um total de 88 bolsas foram disponibilizadas para os alunos do campus e 11 (treze) alunos de \nomedocurso participaram da bolsa este ano em projetos.

\subsubsection{Programa de Incentivo ao Desporto}

O Programa Bolsa de Incentivo ao Desporto incentiva os estudantes a incrementarem o seu desempenho desportivo e acadêmico, mediante atuação em atividades relativas à gestão de atléticas, à assessoria desportiva e ao rendimento desportivo. Cada Associação Atlética possui pelo menos um bolsista para organizar eventos e representá-la ante ao Desporto Universitário. Com isso, a Bolsa de Incentivo ao Desporto é organizada nas seguintes modalidades: Gestão de Atléticas; Assessoria Desportiva; e Rendimento Esportivo.

\subsubsection{Programa de Acolhimento e Incentivo à Permanência (PAIP)}

O Programa de Acolhimento e Incentivo à Permanência (PAIP) tem como objetivo reduzir a evasão nos cursos de graduação da UFC por meio da oferta de bolsas a estudantes que participam de projetos voltados à articulação, acompanhamento e avaliação das ações acadêmicas. Desenvolvidos em diversas áreas e unidades da universidade, esses projetos auxiliam na adaptação dos alunos nos semestres iniciais, fortalecem a qualidade do ensino e da aprendizagem e contribuem para a autoavaliação dos cursos.

Há duas modalidades de bolsa PAIP: a Bolsa de Apoio a Projetos de Graduação, que visa promover a articulação, o acompanhamento e a avaliação de ações acadêmicas,  ampliar as iniciativas de atividades alternativas e inovadoras de ensino que contribuam para a redução de fatores determinantes para a reprovação e evasão dos estudantes; e a Bolsa de Apoio à Gestão Acadêmica da Prograd, que visa propiciar oportunidade ao estudante de graduação para o aprimoramento de sua formação, junto à gestão universitária, permitindo-lhe uma ampliação de seus conhecimentos acerca da estrutura e da dinâmica acadêmica. 

%A bolsa tem dois objetivos principais: de propiciar oportunidade ao estudante de graduação para o aprimoramento de sua formação, junto à gestão universitária, permitindo-lhe uma ampliação de seus conhecimentos acerca da estrutura e da dinâmica acadêmica; criar espaço de participação discente nas ações desenvolvidas pelo Gabinete da PROGRAD e pelas coordenadorias que compõem da pró-reitoria. 

A bolsa tem vigência de no máximo dez meses, relativos ao período de março a dezembro de cada ano, com carga horária de 12 horas semanais, nos turnos da manhã e/ou da tarde, de acordo com as indicações dos projetos. Os bolsistas PAIP devem participar do planejamento e da execução das ações da coordenadoria ou projeto ao qual estão vinculados; redigir relatórios parciais e final das atividades desenvolvidas; e apresentar trabalho nos Encontros Universitários da UFC. Em 2025, o Campus de Quixadá contou com 8 vagas remuneradas do Programa PAIP. 

%Em 2021, um total de 13 alunos do campus foram bolsistas desse programa.


%São atividades dos bolsistas PAIP: organizar os dias e turnos de atividades (manhã, tarde ou noite), considerando a disponibilidade de horário do bolsista e, também, as necessidades da coordenadoria ou projeto ao qual está vinculado; participar do planejamento e da execução das ações da coordenadoria ou projeto ao qual está vinculado, bem como redigir relatórios parciais e final das atividades desenvolvidas; e apresentar trabalho nos Encontros Universitários da UFC. Em 2021, um total de 13 alunos do campus foram bolsistas desse programa.

\subsection{ASSISTÊNCIA EM ACESSIBILIDADE} 
\label{sec:AssistenciaEmAcessibilidade}

%Desde agosto de 2010, a UFC conta com um setor exclusivo para elaborar ações rumo à inclusão de pessoas com deficiência, a Secretaria de Acessibilidade UFC Inclui, que, como o nome sugere, busca integrar pessoas cegas, surdas, cadeirantes e com outras limitações de mobilidade no dia a dia da instituição.

Desde agosto de 2010, a UFC conta com um setor exclusivo para elaborar ações rumo à inclusão de pessoas com deficiência, a Secretaria de Acessibilidade UFC-INCLUI. A UFC-INCLUI busca integrar pessoas cegas, surdas, cadeirantes e com outras limitações de mobilidade no dia a dia da instituição.


Com três eixos de atuação – tecnológico, atitudinal e pedagógico – a secretaria trabalha na formulação de uma política central de acessibilidade na UFC, agindo para que esta seja respeitada e implementada nos diversos espaços da universidade.

%Não se trata de um órgão executor – embora ofereça serviços como digitalização de textos, ledores, revisão de projetos arquitetônicos, entre outros – e, sim, de um núcleo de fomentação e acompanhamento de ações intersetoriais.



%Livia: parei aqui


%Não é objetivo da secretaria absorver todas as ações referentes à inclusão, uma vez que a tarefa de acolher pessoas com deficiência diz respeito a toda a sociedade, cabendo ao órgão disseminar a cultura inclusiva e despertar na comunidade universitária o compromisso com o respeito aos direitos desse público. É por isso que a secretaria trabalha na descentralização das iniciativas de acessibilidade, oferecendo suporte e orientação a professores, coordenadores, chefes de departamento, servidores técnico-administrativos e estudantes interessados em fazer sua parte desse desafio.

Uma vez que a tarefa de acolher pessoas com deficiência diz respeito a toda a sociedade, não é objetivo da secretaria absorver todas as ações referentes à inclusão, cabendo ao órgão disseminar a cultura inclusiva e despertar, na comunidade universitária, o compromisso com o respeito aos direitos desse público. A secretaria atua, portanto, na descentralização das iniciativas de acessibilidade, oferecendo suporte e orientação a professores, coordenadores, chefes de departamento, servidores técnico-administrativos e estudantes interessados em fazer sua parte nesse desafio. Além disso, embora ofereça serviços como digitalização de textos, ledores, revisão de projetos arquitetônicos, entre outros, é importante ressaltar que a UFC-INCLUI não se trata de um órgão executor, mas sim de um núcleo de fomentação e acompanhamento de ações intersetoriais.

São atribuições da secretaria: elaborar e gerenciar ações de acessibilidade; oferecer suporte às unidades acadêmicas para a efetivação da acessibilidade na UFC; estimular a inserção de conteúdos sobre acessibilidade nos projetos pedagógicos de cursos de graduação, contribuindo para a formação de profissionais sensíveis ao tema; identificar e acompanhar os alunos com deficiência na UFC; identificar metodologias de ensino que representam barreiras para os alunos com deficiência e propor estratégias alternativas; estimular o desenvolvimento de uma cultura inclusiva na universidade; oferecer serviços de apoio a esse público, como digitalização e leitura de textos acadêmicos, cursos de Língua Brasileira de Sinais (Libras), revisão de processos arquitetônicos com base em critérios de acessibilidade, entre outras ações; promover a formação de recursos humanos em gestão de políticas relacionadas às pessoas com deficiência, qualificando-os para um atendimento adequado; promover eventos para informar e sensibilizar a comunidade universitária; estimular o desenvolvimento de pesquisas de Avaliação Pós-Ocupação nos prédios da UFC; estimular a acessibilidade em ambientes virtuais e nos produtos e eventos de comunicação e marketing; e oferecer orientação e apoio pedagógico a coordenadores e professores, estabelecendo um canal de comunicação entre estes e os estudantes com deficiência.

%O trabalho da secretaria é desenvolvido em três eixos. O eixo atitudinal relaciona-se à ideia de que a inclusão é uma questão de atitude e de sensibilidade. É preciso ajudar a comunidade acadêmica a enfrentar o preconceito e incentivar mudanças de atitude, visando à remoção de barreiras que impedem a acessibilidade. O eixo tecnológico tem a ver com o incentivo de pesquisas e ações em tecnologias assistivas, para o desenvolvimento de equipamentos, serviços e estratégias que permitam o acesso ao conhecimento com autonomia. E o eixo pedagógico concentra-se na ideia de que não basta fazer com que o estudante com deficiência ingresse na Universidade – é preciso oferecer condições para que ele tenha a mesma formação que os colegas. Por isso, a secretaria promove ações que facilitem o ensino-aprendizagem, com alternativas de avaliação.

O trabalho da secretaria é desenvolvido em quatro eixos. O eixo atitudinal relaciona-se à ideia de que a inclusão é uma questão de atitude e de sensibilidade. Assim, a secretaria atua ajudando a comunidade acadêmica a enfrentar o preconceito e a incentivar mudanças de atitude, visando à remoção de barreiras que impedem a acessibilidade. No eixo tecnológico, o objetivo é incentivar pesquisas e ações em tecnologias assistivas, para o desenvolvimento de equipamentos, serviços e estratégias que permitam o acesso ao conhecimento com autonomia. O eixo pedagógico concentra-se na ideia de que não basta fazer com que o estudante com deficiência ingresse na Universidade – é preciso oferecer condições para que ele tenha a mesma formação que os colegas. Por isso, a secretaria promove ações que facilitem o ensino-aprendizagem, com alternativas de avaliação. Por fim, o eixo comunicacional busca oferecer recursos, atividades e bens culturais que promovam independência e autonomia aos indivíduos que necessitam de serviços específicos para acessar o conteúdo proposto, tais como audiodescrição, legendas, janela de Libras, impressões em braille, dublagem, etc.

\subsection{Política de Cotas para Pessoas com Deficiência}
Conforme a Lei n.º 12.711, sancionada em agosto de 2012 (BRASIL, 2012a), a UFC reserva 50\% de suas vagas para alunos que tenham cursado integralmente o Ensino Médio público, em cursos regulares ou da educação de jovens e adultos. Os demais 50\% das vagas permanecem para ampla concorrência. Destas vagas reservadas para a escola pública, metade é destinada a estudantes com renda mensal familiar de até um salário mínimo e meio. O preenchimento das vagas leva em conta ainda critérios de cor ou raça, seguindo dados estatísticos do Instituto Brasileiro de Geografia e Estatística (IBGE).

A partir do SiSU 2018, conforme a Lei n.º 13.409, sancionada em dezembro de 2016 (BRASIL, 2016a), o preenchimento das vagas começou a levar em consideração também uma reserva, em cada modalidade de cota, para pessoas com deficiência, no mínimo igual à proporção na população do estado do Ceará, de acordo com o IBGE.

Somente em 2018, na chamada regular do SiSU, a UFC adotou cotas para pessoas com deficiência. Em 2022, a Universidade recebeu 69 candidatos com deficiência, via SiSU, que ingressaram na instituição. Desse total, 08 vieram para o Campus Quixadá. A importância da adoção de cotas específicas para pessoas com deficiência no processo de democratização do ensino superior é um avanço importante para a universidade, que assume seu papel de atendimento à comunidade e amplia a inclusão de modo mais efetivo, com todas as cotas existentes hoje.


\subsection{Infraestrutura para facilitação da acessibilidade }
Para fazer frente ao esperado aumento do número de pessoas com deficiência na universidade, a secretaria UFC-INCLUI está se preparando para ampliar o atendimento a esse público, mediante a aquisição de novos equipamentos, contratação de intérpretes e ampliação da interlocução com as unidades acadêmicas, a fim de garantir formação de qualidade a todos.

O campus prevê em sua infraestrutura a facilitação da acessibilidade a pessoas com dificuldades de locomoção ou visão, contando com plataformas elevatórias, portas largas de acesso às salas e laboratórios, banheiros com cabine específica para deficientes e plaquetas com inscrições em braille, além de rampas de acesso ou mesmo ausência de degraus desde o estacionamento até todos os ambientes térreos. Além disso, trabalha em estreita ligação com a secretaria no atendimento aos discentes com deficiência.

\section{Gestão do curso com base nos processos de avaliação interna e externa}

A coordenação tem um papel fundamental no processo de avaliação, ao analisar os dados dos relatórios da Autoavaliação Institucional e promover a participação massiva dos discentes, para a promoção da melhoria dos cursos. A coordenação do curso de \nomedocurso busca assim promover uma série de ações para conscientizar os alunos da importância da participação na Autoavaliação Institucional. %É realizada a divulgação comum em sala de aula para que os alunos participem da Autoavaliação Institucional.

Idealmente, a Autoavaliação Institucional busca a participação responsável e efetiva da maioria dos seus agentes, egressos, discentes, docentes e servidores técnico-administrativos. O objetivo é construir uma cultura interna favorável à autoavaliação, que possibilita maior conscientização acerca da missão, bem como das finalidades acadêmica e social da UFC, consolidando assim a noção de que a Autoavaliação Institucional é importante via para a reflexão coletiva e, por conseguinte, para o planejamento institucional participativo.

Além da campanha de sensibilização, esse bom resultado é creditado também à política de divulgação dos resultados do processo. Semestralmente, a coordenação, após a sistematização dos dados, reúne-se com o corpo discente e os apresenta num evento chamado “Seminário de Autoavaliação Institucional”, onde são respondidas questões sobre a importância da avaliação, como os discentes podem participar e as ações a serem tomadas a partir das informações recolhidas.

Durante o seminário, a coordenação do curso de \nomedocurso apresenta a comparação entre os resultados das avaliações dos discentes com o resultado do campus e os da UFC. Esses seminários também são oportunidades para discutir com os alunos questões gerais sobre o curso e sobre o PPC, contando inclusive como atividade complementar na categoria Vivências em Gestão, pois se considera que esse momento aproxima os alunos do acompanhamento e dos processos de tomada de decisão do curso.

A partir da análise dos dados, são estabelecidas metas de resultados positivos e negativos para cada quesito avaliado, e tomadas atitudes em caso de má avaliação. Por exemplo, se um docente tiver uma avaliação negativa na dimensão ``Planejamento pedagógico, didático e domínio do conteúdo'', isso implica que a coordenação analise a situação sob dois cenários: primeiro, quando o professor se autoavalia com ótimo desempenho, tem-se uma situação que exige intervenção imediata da coordenação; já quando o professor se autoavalia reconhecendo suas dificuldades, implica em outro tipo de intervenção por parte da coordenação. A partir da distinção dos diferentes cenários, o coordenador tem informações objetivas que permitem melhor gerenciar as potencialidades do corpo docente e favorecer a melhoria contínua de suas atividades.

A análise conjunta dos diferentes dados da Autoavaliação Institucional proporciona subsídios importantíssimos para a gestão efetiva do curso. A partir desses resultados e das reuniões de apresentação, é elaborado um Plano de Melhorias, acompanhado pela CPA, com ações de melhoria e cronograma de trabalho aprovados pelo Colegiado do curso e pelo NDE.

Na área administrativa, há um esforço contínuo de mapeamento de processos, que naturalmente induz à melhoria das rotinas do curso. Regularmente, também são realizadas pesquisas socioeconômicas com discentes, que apontam as principais questões acadêmicas e sociais que interferem no desempenho ou permanência do aluno no curso.

\subsection{Acompanhamento de egressos do curso}

Para o curso de \nomedocurso, é importante manter o vínculo com os egressos, pois se trata de uma maneira importante de avaliar aspectos diversos do curso. O egresso, estando em plena atuação profissional ou não, tem o potencial de trazer uma percepção atualizada das exigências da sociedade e do mercado, mostrando-se como um elemento relevante para as atividades do curso.

Em 2009, foi criada a Associação dos ex-alunos da UFC (ASSOEX) com o objetivo de congregar aqueles que já passaram pelas salas de aula da instituição, tendo em vista a manutenção de sua proximidade com a universidade, e de criar mecanismos que promovam a sua plena integração à vida acadêmica, política e cultural da instituição. Também é meta da associação despertar nos ex-alunos o interesse pela promoção sociocultural da UFC, garantindo o acesso deles às instalações acadêmicas, esportivas e culturais em iguais condições de tratamento dos atuais alunos e professores.

Além da possibilidade de fazer parte da ASSOEX, listamos abaixo algumas ações planejadas para manter o vínculo com os egressos de \nomedocurso:

\begin{itemize}
    \item Manutenção de cadastro atualizado dos egressos: a cada semestre, os dados de contato dos alunos formados serão coletados, registrados e mantidos pela secretaria acadêmica e Coordenação do curso.
    
    \item Manutenção de uma conta institucional do curso em redes sociais profissionais para o acompanhamento dos egressos.
    
    \item Promoção de eventos com participação de egressos: o Campus da UFC em Quixadá frequentemente promove eventos com a participação de egressos em palestras, minicursos e mesas redondas. Alguns desses eventos são o ExpoSE, WTISC, Flisol, os Encontros Universitários e o InfoGirl.
    
    \item Pesquisa com egressos: desde 2016, tem-se realizado anualmente uma pesquisa de levantamento com ingressantes dos cursos do Campus da UFC em Quixadá. Pretende-se comparar os dados obtidos com os ingressantes anualmente, com os dados que serão coletados quando estes alunos estiverem próximos da formatura, a fim de entender melhor o percurso por eles traçado e compreender suas expectativas profissionais e acadêmicas ao final do curso. Ainda em 2016, a pesquisa de mestrado de uma servidora do Campus da UFC em Quixadá focou na inserção dos egressos de campi do interior no mercado regional, trazendo importantes informações a respeito da realidade desses alunos (NUNES, 2016).
\end{itemize}