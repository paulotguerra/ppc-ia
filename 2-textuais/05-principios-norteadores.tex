\chapter{PRINCÍPIOS NORTEADORES}
\label{cap:principios-norteadores}

Os princípios norteadores definidos para o curso de \nomedocurso procuram estabelecer um equilíbrio entre as necessidades do mercado e as demandas do indivíduo e da  sociedade. Assim, esses princípios foram estabelecidos em conformidade com os princípios institucionais da UFC, expressos no PDI (Plano de Desenvolvimento Institucional) da instituição. Neste capítulo, são apresentados os princípios norteadores do curso.

\section{ÉTICA E CIDADANIA}

As Diretrizes Curriculares Nacionais para a Educação em Direitos Humanos – Parecer CNE/CP 1/2012 \cite{brasil2012dcnedh} destacam a responsabilidade das IES com a formação de cidadãos éticos, comprometidos com a construção da paz, da defesa dos direitos humanos e dos valores da democracia, além da responsabilidade de gerar conhecimento mundial, visando atender aos atuais desafios dos direitos humanos, como a erradicação da pobreza, do preconceito e da discriminação. Assim, o curso assume a ética e a cidadania como princípios fundamentais que orientam a atuação profissional e social de seus estudantes e egressos. %Assim, o curso defende a ética e a cidadania como norteadoras do comportamento profissional e social de seus alunos e egressos.

\section{RESPEITO ÀS DIFERENÇAS E À DIVERSIDADE HUMANA}

As Diretrizes Curriculares Nacionais para a Educação em Direitos Humanos – Parecer CNE/CP 1/2012 \cite{brasil2012dcnedh} recomendam a transversalidade curricular das temáticas relativas aos direitos humanos. O documento define, como princípios da educação em direitos, dentre outros: a dignidade humana, a igualdade de direitos, o reconhecimento e valorização das diferenças e das diversidades, a democracia na educação e a transversalidade. Assim, as atividades do curso de \nomedocurso pautam-se em combater a indiferença, a discriminação, o preconceito, a injustiça e os rótulos em relação a todo e qualquer indivíduo.

\section{EQUILÍBRIO NAS ATIVIDADES DE ENSINO, PESQUISA E EXTENSÃO}

O curso de \nomedocurso segue o princípio da indissociabilidade entre o ensino, a pesquisa e a extensão, estabelecido no Estatuto da UFC \cite{ufc2025estatuto}. Nas ações de ensino, pesquisa e extensão, os indivíduos devem ser sempre considerados como sujeitos integrados e integradores da sociedade. Além das atividades de ensino, os alunos são incentivados e têm a oportunidade de se envolver em atividades de pesquisa e extensão ao longo do curso. Essas experiências permitem a aplicação dos conhecimentos adquiridos em sala de aula na resolução de problemas reais da sociedade. Tais atividades enriquecem o aprendizado, ampliando a produção de conhecimento para além dos muros da universidade, promovendo a participação coletiva na comunidade e fomentando uma interação permanente com a sociedade.
%Além das atividades de ensino, ao longo do curso, os alunos são estimulados e têm a oportunidade de participar ativamente de projetos de pesquisa e extensão, de modo a aplicarem os conhecimentos adquiridos em sala de aula em problemas reais da sociedade, atividades que potencializam o conhecimento que se produz fora do ambiente universitário, bem como dão oportunidade ao aluno adquirir conhecimentos que vão além dos muros da universidade, estimulando a participação coletiva na comunidade e possibilitando a interação permanente com a sociedade. %(UFC, 2018c)

\section{FLEXIBILIDADE NA ESTRUTURAÇÃO CURRICULAR}

A flexibilização curricular é considerada parte essencial na organização dos projetos pedagógicos dos cursos de graduação. O Plano Nacional de Educação \cite{brasil2014pne} define em seus objetivos que se devem estabelecer, em nível nacional, diretrizes curriculares que assegurem a necessária flexibilidade e diversidade nos programas oferecidos pelas diferentes IES, de forma a melhor atender às necessidades distintas de suas clientelas e às peculiaridades das regiões nas quais estão inseridas.

Ao se construir currículos flexíveis, evidencia-se a importância de uma estrutura curricular que permita incorporar outras formas de aprendizagem e formação presentes na realidade social. Segundo \citeonline{cabralneto2004flexibilizacao}, a flexibilização curricular possibilita ao aluno participar do processo de formação profissional; rompe com o enfoque unicamente disciplinar e sequenciado; cria novos espaços de aprendizagem; busca a articulação entre teoria e prática; possibilita ao aluno ampliar os horizontes do conhecimento e a aquisição de uma visão crítica que lhe permita extrapolar a aptidão específica de seu campo de atuação profissional e propicia a diversidade de experiências aos alunos. %Cabral Neto (2004)

Nesse sentido, a organização curricular do curso de \nomedocurso compreende uma quantidade limitada de pré-requisitos entre suas componentes curriculares, além de permitir que o aluno construa seu percurso próprio no curso, estruturando seu currículo de acordo com suas necessidades e interesses pessoais e profissionais, a partir da escolha entre a grande variedade de disciplinas optativas e livres ofertadas pela instituição. Admite-se assim, com o esforço pela construção de um currículo flexível e abrangente, que o aluno é responsável direto na construção de seu próprio itinerário formativo.

\section{DESENVOLVIMENTO DA CAPACIDADE CRÍTICA E DA PROATIVIDADE DO EDUCANDO}

As atividades de ensino, pesquisa e extensão do curso refletem a preponderância da educação sobre a instrução, ou seja, há uma preocupação com a aprendizagem baseada na construção do saber a partir da experiência, prévia ou induzida, do próprio indivíduo, a despeito da simples passagem de informações unidirecionais do professor para o aluno. Os discentes são constantemente estimulados a desenvolver trabalhos e projetos críticos e criativos em que apresentam suas próprias visões a partir do que foi aprendido e discutido nas aulas, e não apenas reproduções mecânicas dos conhecimentos adquiridos.

O curso procura estimular uma postura empreendedora e proativa, de modo que este seja pensado não apenas em nível operacional, como resolução de problemas, mas a partir de uma prática estratégica de gerência dos projetos. %Seguindo a visão de Nogueira e Portinari (2016), o foco não está unicamente na resolução de problemas práticos, mas sim em compreender os problemas inseridos em contextos complexos, que merecem respostas à altura dessa realidade. \textcolor{red}{aqui não achei a referência pessoal..dissertação de 2025 orientadora nogueira....ver no PPC de ES....MUDAR A REFERÊNCIA}.
%

\section{INTERDISCIPLINARIDADE}

A \nomedocurso possui sua base na computação, porém sua aplicabilidade é virtualmente ilimitada, com a necessidade de o desenvolvedor de \nomedocurso entender sobre a área de aplicação. Ao longo do curso, as disciplinas se complementam e mostram o caráter transversal do conhecimento de \nomedocurso.
%O desenvolvimento de software integra o emprego de dois domínios: 1) o de \nomedocurso ligado à computação e; 2) o domínio onde está inserido o problema que motiva a construção do software. No curso de \nomedocurso, a interdisciplinaridade é assegurada desde a concepção do projeto pedagógico, já que são inseridas disciplinas integradoras para permitir um diálogo mais coeso entre as várias disciplinas. %Existe também a possibilidade de participação de projetos no Núcleo de Práticas em Informática, que serve como um ambiente mais próximo do mercado de trabalho do Engenheiro de Software através de diversos projetos com diferentes domínios, aproximando a Universidade ao “mundo real”. 
Além disso, por ser um campus temático, o aluno tem a possibilidade de interagir com docentes, alunos e disciplinas de áreas correlatas ao seu curso, propiciando um ambiente maior de teoria e prática.

\section{INTEGRAÇÃO ENTRE TEORIA E PRÁTICA}

Conforme apresentado no Parecer nº CNE/CP 009/2001 , a integração entre teoria e prática está alinhada com a concepção da prática como componente curricular, que ``implica vê-la como uma dimensão do conhecimento, que tanto está presente nos cursos de formação, nos momentos em que se trabalha na reflexão sobre a atividade profissional, como durante o estágio nos momentos em que se exercita a atividade profissional'' \cite[p. 23]{cne2001parecer9}. Nessa visão, busca-se superar a ideia de que ``o estágio é o espaço reservado à prática, enquanto, na sala de aula se dá conta da teoria'' \cite[p. 23]{cne2001parecer9}. %(BRASIL, 2001b, p. 23)

Desta forma, as atividades do curso buscam contemplar a integração entre teoria-prática, visando proporcionar ao estudante uma educação baseada na reflexão crítica e no fazer. Ao longo do curso, os alunos são desafiados a desenvolver projetos práticos. %a maior parte convergindo para o contexto das disciplinas de Projeto Integrado, cujo objetivo é integrar a participação de alunos e professores nas diversas disciplinas ofertadas em um mesmo semestre letivo, objetivando-se uma maior contextualização do conteúdo a ser aprendido bem como ressaltando a importância do inter-relacionamento dos saberes e dos profissionais envolvidos.

Outra iniciativa de integração teoria-prática é a realização de atividades formativas ao mesmo tempo transversais e paralelas ao curso, como oficinas, exposições, palestras e debates, estabelecidas a partir de parcerias entre alunos e professores de diversas áreas.

Considerando os elementos em referência, o Projeto Pedagógico do Curso de \nomedocurso busca a consolidação de uma identidade própria, orientado por princípios que compreendem que a formação profissional em \nomedocurso envolve uma prática específica, que pressupõe saberes e competências coerentes. Para isso, é preciso que o currículo seja flexível e possibilite não só a formação de competências técnicas, como também o compromisso da ciência com as transformações sociais.