\chapter{PROCEDIMENTOS DE ACOMPANHAMENTO E AVALIAÇÃO DOS PROCESSOS DE ENSINO E APRENDIZAGEM}
\label{cap:procedimentos-de-acompanhamento}
% regis

Com o intuito de oferecer aos alunos de \nomedocurso formação de alto nível e conhecimento extenso e aprofundado nas diversas áreas que compõem essa formação, é realizado um acompanhamento detalhado do processo de ensino-aprendizagem (que contempla os aspectos de acessibilidade, como apresentado na seção \ref{sec:acessibilidade}) nos componentes curriculares e demais atividades do curso. Nesse sentido, através de atividades de avaliações escritas e práticas, os docentes podem acompanhar o amadurecimento dos alunos durante o curso, registrando os resultados individuais e gerais das turmas.

A avaliação do ensino-aprendizagem, dentro do contexto das disciplinas, caracteriza-se tanto formativa quanto somativa. As avaliações formativas são aplicadas para alterar ou aprimorar o que é visto nas aulas enquanto continuam em andamento \cite{russell2014avaliacao}. Assim, espera-se que a avaliação formativa ocorra ao longo do desenvolvimento da disciplina, com a finalidade de proporcionar informações úteis destinadas ao aprimoramento das ações executadas \cite{depresbiteris2017diversificar}. Já as avaliações somativas são úteis para avaliar os resultados do que foi ensinado \cite{russell2014avaliacao}. Espera-se, portanto, que esse tipo de avaliação determine o mérito, o valor final de um programa, com o objetivo de propiciar a tomada de decisões sobre sua continuidade ou não \cite{depresbiteris2017diversificar}. Em resumo, a avaliação formativa possibilita melhorias no processo de ensino e aprendizagem, e a somativa tem caráter mais final, mais conclusivo \cite{depresbiteris2017diversificar}.

Na maior parte das disciplinas do curso de \nomedocurso, a prática didática recorre a avaliações formativas, ou seja, ao longo do semestre letivo, os professores observam e analisam o comportamento e desempenho dos alunos. Ainda assim, muitas vezes é necessário recorrer a avaliações somativas, a fim de se ter uma visão mais sistemática da situação do aluno.

Seja de maneira formativa ou somativa, o processo avaliativo na totalidade exige um conjunto de atividades formais, sistemáticas, que levam o professor a ter condições de apresentar juízo de valor sobre determinado aspecto educacional de interesse \cite{depresbiteris2017diversificar}. Entretanto, esse juízo de valor está frequentemente carregado de certo grau de subjetividade, deixando o processo de análise e tomada de decisão em relação à aprendizagem ainda mais difícil. Para equilibrar esse fator, é interessante, portanto, que o julgamento final sobre o aluno seja emitido baseado em múltiplas situações e múltiplos instrumentos de avaliação \cite{depresbiteris2017diversificar}.

Dessa forma, vê-se o quanto são importantes a escolha e a aplicação de variados instrumentos de avaliação durante o processo de ensino-aprendizagem. Quando se fala em avaliação, costuma-se pensar em provas. Entretanto, é importante reconhecer que nem todas as decisões de avaliação exigem o uso de provas ou mensurações \cite{russell2014avaliacao}. Além disso, historicamente, a avaliação educacional vem sofrendo uma transformação radical com a mudança da cultura da prova (\textit{testing}) para a cultura da avaliação (\textit{assessment}), pois esta pressupõe uma discussão mais ampla, a partir de suas finalidades \cite{depresbiteris2017diversificar}.

Em geral, nas disciplinas do curso, as provas tradicionais são utilizadas apenas como parte do processo avaliativo. Elas não são o único instrumento, sendo, portanto, aplicadas em conjunto com outros formatos, entre os quais se destacam:
\begin{itemize}
    \item Exercícios: podem ser individuais ou em grupo; resolvidos em sala ou em casa; em forma de texto, de imagens ou de códigos; únicos ou em listas.
    \item Discussões e seminários: debates entre professores e alunos a partir de leituras recomendadas ou seminários preparados pelos alunos.
    \item Elaboração de peças e produtos específicos: os alunos são estimulados a desenvolver criativamente peças físicas (papel, argila) ou digitais dentro do contexto do que é aprendido em sala de aula.
    \item Autoavaliação: acontece quando os alunos atribuem notas a si ou ao grupo a que pertencem, contribuindo para a reflexão e crítica do que eles próprios desenvolveram ao longo da disciplina.
    \item Pesquisas: podem ser bibliográficas, de tecnologias, de campo, iconográficas. Nesses casos, os alunos buscam conhecer o que já existe.
    \item Projetos: podem ser projetos de concepção e desenvolvimento de soluções, envolvendo programação ou não; podem ser individuais ou em grupo; podem ser desenvolvidos e apresentados no final da disciplina (avaliação somativa), ou durante o semestre letivo, com o acompanhamento do professor (avaliação somativa).
    %\item Apresentação em bancas: nas disciplinas de Projeto integrado, os alunos apresentam seus projetos a uma banca avaliadora, normalmente formada pelos professores das disciplinas que contribuíram diretamente com o projeto em questão.
\end{itemize}

Vê-se, portanto, que a diversidade de metodologias, técnicas e instrumentos de avaliação contribui para uma visão mais confiável e abrangente do processo de ensino e aprendizagem, já que “a ideia de diversificar os instrumentos de avaliação tem respaldo na necessidade de que se analise a aprendizagem do aluno sob diferentes ângulos e dimensões” \cite[p. 95]{depresbiteris2017diversificar}. Além disso, a variedade de instrumentos e práticas avaliativas favorece o atendimento à diversidade das necessidades dos estudantes, pois, caso algum aluno, especialmente por questões de acessibilidade, não possa realizar uma determinada atividade avaliativa, o professor ainda assim terá alternativas para avaliá-lo.

Os procedimentos de avaliação aplicados no curso estão ainda alinhados a alguns princípios norteadores como “Respeito às diferenças e à diversidade humana”, “Desenvolvimento da capacidade crítica e da proatividade do educando” e “Integração entre teoria e prática” e a estratégias metodológicas como “Metodologias ativas” e “Acessibilidade metodológica”.

É importante esclarecer também que, independentemente do formato adotado pelos professores individualmente em suas componentes curriculares, a avaliação discente segue o Regimento Geral \cite{ufc_regimento_geral_2019} e a Resolução Nº 12/CEPE, de 19 de junho de 2008 \cite{ufc_resolucao_12_cepe_2008}, nos artigos que tratam das regras para aprovação e reprovação por nota ou por falta.

Em geral, são realizadas, no mínimo, uma avaliação no decorrer da primeira metade do semestre letivo e uma segunda avaliação no decorrer da segunda metade do semestre, para cálculo da nota final do aluno, em cada componente curricular, não se restringindo apenas a isso. Os docentes do curso são orientados a discutir os resultados das avaliações, pois esses também são momentos de aprendizado.

A avaliação do rendimento escolar por disciplina abrange a assiduidade e a eficiência, ambas eliminatórias. Com relação à assiduidade, será aprovado o aluno que frequentar 75\% (setenta e cinco por cento) ou mais da carga horária, no caso de disciplina, vedado o abono de faltas. Quando se tratar de componente do tipo atividade (Estágios e TCC), o aluno deverá frequentar 90\% (noventa por cento) ou mais da carga horária. No caso das Atividades Complementares e Atividades de Extensão, o aluno deve apresentar o comprovante adequado, de acordo com os manuais de regulamentação das respectivas atividades.

Na verificação da eficiência, será aprovado por média o aluno que, em cada disciplina, apresentar média aritmética das notas resultantes das avaliações progressivas, igual ou superior a sete. O aluno que apresentar média igual ou superior a quatro e inferior a sete será submetido à avaliação final. O aluno que apresentar a média inferior a quatro está reprovado. Na hipótese de o aluno necessitar da avaliação final, deverá obter uma nota superior ou igual a quatro, e a média dessa avaliação com a média das avaliações progressivas deve resultar em um valor superior ou igual a cinco para ser considerado aprovado. A verificação do rendimento na perspectiva do curso é realizada por meio do Trabalho de Conclusão de Curso I e II e Estágio Supervisionado, atividades obrigatórias para a conclusão do curso de \nomedocurso.

O estudante que contrair duas reprovações por falta, no mesmo componente, ou atingir um total de quatro reprovações por falta em componentes do curso, terá sua matrícula do semestre subsequente bloqueada. Para desbloqueá-la, o aluno precisa preparar um plano de estudos com a Coordenação, que considera horários de aula, estudo, descanso e lazer, além de planejamento de disciplinas a cursar nos semestres seguintes, e assinar um termo de compromisso, obrigando-se a cumprir com aquilo que foi planejado.

Mesmo antes de ser bloqueada, a Coordenação estimula o aluno com mais dificuldades a procurá-la para preparar esse plano de estudos, minimizando ao máximo os efeitos negativos das reprovações no percurso formativo do aluno. Outra ação da Coordenação, no mesmo sentido, é um acompanhamento mais atento durante o período de matrículas, antes do início de cada semestre, solicitando aos alunos que voltem a matricular-se nos componentes curriculares que tenham reprovado anteriormente, e demandando à Coordenação de programas acadêmicos, na medida da necessidade e para evitar represamentos, a abertura de vagas extraordinárias.

Um dos princípios básicos da avaliação da aprendizagem é a transparência, aos sujeitos avaliados, dos elementos passíveis de avaliação, bem como de seus mecanismos e instrumentos. No curso de \nomedocurso, assim como nos demais cursos do campus, essa transparência é estimulada através da publicação, nos primeiros dias de aula, dos planos de ensino das disciplinas. O plano é elaborado pelo professor, preferencialmente no início do semestre. Nele, além das informações básicas do componente curricular, como justificativa, objetivos, ementa e bibliografia, constam também informações específicas do andamento do componente no semestre correspondente, como a metodologia de ensino, as atividades discentes e as formas e cálculos de avaliação. Todos os planos de ensino são, obrigatoriamente, disponibilizados no SIGAA, sistema oficialmente utilizado na UFC.

Para os casos de extraordinário desempenho acadêmico dos discentes, o adiantamento de seus estudos poderá ser realizado mediante Resolução Nº 09/CEPE, de 1º de novembro de 2012 \cite{ufc_resolucao_09_cepe_2012}. Segundo esse documento, é possível conceder abreviação de estudos de componentes curriculares dos cursos de graduação, tendo o aluno de satisfazer todas as exigências preconizadas no texto do documento, bem como obter aprovação em processo avaliativo.