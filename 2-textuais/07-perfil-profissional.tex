\chapter{PERFIL PROFISSIONAL DO EGRESSO}
\label{cap:perfil-profissional}
% regis

% IACG: O perfil profissional do egresso consta no PPC, está de acordo com as DCN (quando houver), expressa as competências a serem desenvolvidas pelo
% discente e as articula com necessidades locais e regionais, sendo ampliado em função de novas demandas apresentadas pelo mundo do trabalho.

O avanço acelerado da Inteligência Artificial (IA) nas últimas décadas tem impulsionado a criação de formações acadêmicas específicas voltadas à preparação de profissionais aptos a lidar com os desafios técnicos, éticos e sociais associados ao desenvolvimento e à aplicação de sistemas inteligentes. Nesse contexto, os cursos de Bacharelado em Inteligência Artificial têm como objetivo formar egressos com um conjunto de competências e habilidades que vão além da base tradicional da Computação, integrando conhecimentos interdisciplinares e uma visão crítica sobre os impactos da tecnologia.

O perfil específico de egressos dos cursos de Bacharelado em Inteligência Artificial em termos de competências e habilidades específicas está definido a partir dos Referenciais de Formação para Cursos de Bacharelado em Inteligência Artificial \cite{sbc2024}. Esses elementos conferem características distintivas à formação em IA e estão alinhados com o que estabelecem os Artigos 4º e 5º das Diretrizes Curriculares Nacionais (DCN16) para a área de Computação \cite{DCN16}.

O egresso do curso de Bacharelado em Inteligência Artificial será um profissional com formação sólida e interdisciplinar, preparado para atuar com competência técnica, responsabilidade ética e visão crítica sobre os impactos sociais, econômicos e ambientais das tecnologias que desenvolve e utiliza. Espera-se que este profissional:

\begin{itemize}

    \item Possua base consistente em Ciência da Computação, Matemática, Estatística e Ciência de Dados, que o capacite a identificar, modelar e resolver problemas complexos por meio da construção de sistemas computacionais, incluindo soluções inovadoras em contextos diversos como sistemas autônomos, embarcados e distribuídos;

    \item Domine os principais paradigmas, técnicas e ferramentas da Inteligência Artificial, como representação do conhecimento, raciocínio automático, aprendizado de máquina, otimização, mineração de dados e visualização interativa, sendo capaz de selecionar, integrar e aplicar tais métodos de forma eficiente e fundamentada;

    \item Seja capaz de analisar e explorar grandes volumes de dados, extraindo conhecimento relevante que subsidie a tomada de decisões em ambientes organizacionais, industriais, científicos ou sociais;
    
    \item Atue de maneira empreendedora e inovadora na criação de soluções computacionais, considerando aspectos técnicos, econômicos e de viabilidade de mercado, além de reconhecer oportunidades de negócio emergentes no campo da Inteligência Artificial;
    
    \item Seja capaz de criar soluções, individualmente ou em equipe, para problemas complexos; reconhecer o caráter fundamental da inovação e da criatividade e compreender as perspectivas de negócios e oportunidades comerciais relevantes na área de Inteligência Artificial;

    \item Seja capaz de agir de forma consciente na construção e no uso de sistemas de IA, compreendendo seus impactos diretos e indiretos sobre a sociedade e os indivíduos no que tange a questões de privacidade, transparência, equanimidade e preconceito implícito em sistemas de software e bases de dados, entre outras.
    
    \item Compreenda e considere os impactos éticos, sociais e legais da Inteligência Artificial, incluindo questões de privacidade, viés algorítmico, transparência, justiça e segurança, adotando uma postura crítica e consciente no desenvolvimento e uso dessas tecnologias;
    
    \item Demonstre habilidades interpessoais e de trabalho em equipe, sendo capaz de colaborar com profissionais de diferentes áreas do conhecimento em projetos multidisciplinares de forma proativa, ética e comunicativa;
    
    \item Seja capaz de comunicar ideias, projetos e resultados de forma clara e objetiva, tanto na forma escrita quanto oral, adaptando a linguagem ao público-alvo;
    
    \item Mantenha atitude investigativa, curiosidade científica e disposição para o aprendizado contínuo, sendo capaz de acompanhar a evolução tecnológica e de atuar na área de Inteligência Artificial;
    
    \item Enfrente com resiliência os desafios impostos por problemas de alta complexidade, demonstrando capacidade de adaptação, análise crítica e pensamento estratégico para propor soluções robustas e contextualizadas na área de IA.

\end{itemize}
