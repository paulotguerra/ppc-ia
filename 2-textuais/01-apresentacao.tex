\chapter{APRESENTAÇÃO}
\label{cap:apresentacao}

A Inteligência Artificial (IA) é a subárea da Ciência da Computação que busca desenvolver sistemas e algoritmos capazes de realizar tarefas que, quando realizadas por humanos, requerem inteligência. Ela envolve a criação de máquinas que podem aprender com dados, tomar decisões, resolver problemas complexos e simular comportamentos inteligentes. De acordo com \citeonline{poole2017ai}, a IA é ``o estudo de agentes computacionais inteligentes, ou seja, sistemas que percebem seu ambiente e tomam decisões para alcançar objetivos específicos''. As abordagens da IA tradicionalmente dividem-se entre paradigmas simbólicos (baseados em regras e lógica formal) e subsimbólicos (baseados em redes neurais e outras técnicas conexionistas), sendo muitas vezes complementares. Tais abordagens são aplicadas a problemas como percepção (visão computacional, compreensão de linguagem natural), raciocínio (busca heurística, planejamento, escalonamento, inferência), ação (robótica, agentes inteligentes) e interação (sistemas conversacionais, interfaces adaptativas), muitas vezes combinadas em arquiteturas complexas como  agentes autônomos e sistemas multiagentes \cite{russell2022ia, poole2017ai} 

Nos últimos anos, a IA tornou-se um dos pilares da transformação digital com impactos econômicos, sociais, políticos e culturais. O \textit{AI Index Report} de 2025 \cite{aiindex2025}, produzido pela \textit{Stanford University}, evidencia esse crescimento da área, desde a pesquisa até sua aplicação na indústria, educação, saúde, justiça e políticas públicas \cite{russell2022ia}. Tal avanço refletiu-se no surgimento de cursos de nível superior de formação de profissionais especializados nesta área tais como: o \textit{B.S. in Artificial Intelligence}, ofertado a partir de 2018 pela \textit{Carnegie Mellon University}, EUA \cite{cmu_bsai}; o Bacharelado em Inteligência Artificial, ofertado pela Universidade Federal de Goiás \cite{ufg_bia} a partir de 2020 e; o Bacharelado em Inteligência Artificial, ofertado pela Universidade Federal do Rio Grande do Norte \cite{ufrn_bia} a partir de 2025. 


A Inteligência Artificial já nasceu interdisciplinar. Desde a conferência de \textit{Dartmouth} \cite{mccarthy2006proposal}, em 1956, considerada o marco inaugural da IA, o desenvolvimento da área ocorreu por contribuições da filosofia, matemática, neurociência, psicologia, teoria do controle e cibernética, linguística, estatística, entre outras \cite{russell2022ia}.  A formação em IA, portanto, requer uma sólida base conceitual e prática em áreas como lógica, algoritmos, teoria da computação, estatística, otimização, sistemas de representação do conhecimento, raciocínio automatizado e aprendizagem de máquina. %(Russell \& Norvig, 2022; Poole \& Mackworth, 2017).
Além disso, o(a) bacharel em Inteligência Artificial deve ser capaz de refletir criticamente sobre os fundamentos, impactos e limites da área, compreendendo os aspectos éticos, sociais, legais e filosóficos associados à automação inteligente, ao viés algorítmico, à explicabilidade dos modelos e à regulação do uso da IA \cite{floridi2018ai4people,mittelstadt2016ethics, eu_ai_act_2021}.



%Modelos computacionais fundamentais para a IA surgiram dessas interações interdisciplinares. Por exemplo, o \textit{perceptron} \cite{rosenblatt1958perceptron}, inspirado em modelos neurobiológicos de aprendizado, estabeleceu as bases para as redes neurais artificiais. Décadas depois, a redescoberta do algoritmo de \textit{backpropagation} \cite{rumelhart1986learning} permitiu avanços no treinamento de redes profundas. Ainda, a evolução do poder computacional e do grande volume de dados disponíveis, impulsionaram o desenvolvimento da técnica de aprendizado profundo, revolucionando áreas como visão computacional, processamento de linguagem natural e reconhecimento de fala \cite{lecun2015deep}.
%(LeCun, Bengio & Hinton, 2015; Wang & Benning, 2020). %Marvin Minsky, John McCarthy, Allen Newell e Herbert Simon, figuras centrais na fundação da IA, já defendiam que o estudo da inteligência exigiria métodos e teorias oriundas de múltiplas disciplinas \cite{russell2022ia}.


%(Floridi et al., 2018; Mittelstadt et al., 2016).


Dessa forma, o curso de Bacharelado em Inteligência Artificial da Universidade Federal do Ceará visa formar profissionais aptos a desenvolver e aplicar soluções baseadas em IA em diversos setores da sociedade, com responsabilidade, criatividade e pensamento crítico. Trata-se de uma formação estratégica para o avanço científico, tecnológico e social do país, considerando o papel central que a IA desempenha na promoção da inovação, na competitividade das empresas e na melhoria da qualidade de vida dos cidadãos.

%Nas últimas décadas, a Inteligência Artificial (IA) passou de uma área de interesse puramente acadêmico para uma área de grande interesse econômico e da sociedade em geral como pode ser visto no AHI Stanford University, 2023, \url{https://aiindex.stanford.edu/report/}. Tal crescimento se refletiu no surgimento de vários cursos de nível superior de formação de profissionais mais especializados na área.  

%A IA já nasce multi e interdisciplinar, em 1958. Ela sempre esteve muito ligada à ciência da computação, mas não só. Desde sua origem ela foi sendo influenciada por outras ciências, por exemplo, a psicologia e a filosofia, com os modelos de conhecimento e as arquiteturas intencionais. Recentemente, passou também a influenciar e ser influenciada por outras áreas do conhecimento, por exemplo, a neurociência (hoje se fala em máquinas que imitam o cérebro e máquinas que aumentam o cérebro). 
%Estudos interdisciplinares entre essas áreas renderam à IA os modelos matemáticos como o \textit{perceptron} (Rosenblatt,1958) e as redes neurais com \textit{backpropagation} (Rumelhart, D.E., Hinton, G.E., Williams, R.J, 1986). 
%Ainda, com a evolução do poder de computação, do volume de dados disponíveis e da pesquisa em neurociência e em ciência cognitiva, impulsionaram o 
%desenvolvimento da técnica de aprendizado profundo - esta denominação de profundo refere-se ao fato de que o algoritmo utiliza uma rede neural com grande quantidade de camadas ocultas e neurônios digitais interconectados, 
%processando milhões de parâmetros – uma arquitetura complexa que alavancou o desenvolvimento da IA nos últimos 10 anos (Wang; Benning, 2020).   

%O bacharel em Inteligência Artificial precisa, além de saber como usar a IA para resolver problemas, também precisa pensar sobre a IA, isto é: entender como a IA resolve os problemas, quais são seus riscos e vantagens e quais são seus limites e potencialidades para a sociedade.   

%Tradicionalmente, a IA inclui uma mistura de abordagens simbólicas e subsimbólicas. As soluções que a IA fornece baseiam-se num amplo conjunto de esquemas de representação de conhecimento geral e especializado, mecanismos de resolução de problemas e técnicas de otimização. Essas abordagens lidam com percepção (por exemplo, reconhecimento de fala, compreensão e geração de linguagem natural, visão computacional), resolução de problemas (por exemplo, busca, planejamento automatizado, otimização), ação (por exemplo, robótica, automação de tarefas, controle) e as arquiteturas necessárias para apoiar estes sistemas (por exemplo, agentes únicos ou sistemas multiagentes). O aprendizado de máquina pode ser usado em cada um desses aspectos e pode até ser empregado de ponta a ponta em todos eles. Ou seja, a IA congrega várias tecnologias para resolver problemas.   

%A formação sólida de bacharéis em Inteligência Artificial influenciará decisivamente na melhoria e na evolução do país e da sociedade como um todo, no que se refere ao atendimento das demandas de inovação, na evolução das empresas e dos cidadãos. 

% >> Ver PPC de Eng. de Software:
% >> https://es.quixada.ufc.br/wp-content/uploads/2023/05/PPC-ES-2023.pdf
%
% 3.1. REALIDADE LOCAL
% 3.1.1. CENÁRIO EDUCACIONAL
% 3.1.2. ASPECTOS SOCIOAMBIENTAIS
% 3.1.3. ASPECTOS REGIONAIS
% 3.2. MISSÃO, VISÃO, VALORES E VOCAÇÃO DO CURSO
% 1.1. JUSTIFICATIVA PARA A CRIAÇÃO DO CURSO
% 1.2. PRINCIPAIS DOCUMENTOS QUE SUBSIDIARAM A ELABORAÇÃO DO PPC
% 1.3. ORGANIZAÇÃO DO DOCUMENTO
