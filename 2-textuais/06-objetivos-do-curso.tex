\chapter{OBJETIVOS DO CURSO}
\label{cap:objetivos-do-curso}

% Os objetivos do curso, constantes no PPC, estão implementados, considerando
% o perfil profissional do egresso, a estrutura curricular, o contexto educacional, características locais e regionais e novas práticas emergentes no campo do conhecimento relacionado ao curso.

O estudo da IA prepara os alunos para determinar quando uma abordagem de IA é apropriada para um determinado problema, identificar modelos e representações e mecanismos de raciocínio apropriados, implementá-los e avaliá-los em relação ao desempenho e ao seu impacto social mais amplo.   

Segundo a proposta da  UNESCO \cite{unesco2024ai}, o letramento em IA consiste nas seguintes vertentes: letramento em dados, letramento em algoritmos e letramento em modelos.  O uso de dados, de algoritmos e de modelos, estatísticos e/ou simbólicos, permite que as máquinas, além de coletar e processar dados durante tarefas de aprendizado, possam também raciocinar sobre os modelos aprendidos. O letramento em IA pode ser alcançado através dos eixos de conteúdos definidos neste referencial curricular.

Os três componentes do letramento em IA apontam as competências específicas da IA que serão abordadas nesta proposta de referencial curricular, que são: 

\begin{itemize}
    \item percepção/atuação: como as máquinas percebem e atuam no mundo. Agentes inteligentes possuem formas de interagir com humanos - ou com outros sistemas - de maneira mais natural, através de textos, da fala, de imagens ou de movimentos. Desta forma, a coleta de dados é feita de maneira automática; 
    \item representação e raciocínio (máquinas usam representações do mundo - projetadas por desenvolvedores - e as utilizam para raciocinar e tomada de decisão); 
    \item aprendizado de máquina: os computadores conseguem aprender modelos para raciocínio e tomada de decisão a partir de dados. Nos últimos anos a IA tem focado ainda mais em novas formas de representação do mundo, isto é, passaram de modelos projetados pelos desenvolvedores para modelos aprendidos a partir dos dados ou híbridos; e
    \item impactos sociais: a IA tem impactos que assumem formas positivas e negativas na sociedade, distinguindo seu uso e desenvolvimento ético e responsável, incluindo ainda questões legais, confiança e explicabilidade. Pensar sobre a IA ajuda a construir o pensamento crítico dos alunos.   
\end{itemize}

Há uma grande intersecção da IA com a computação e a ciência de dados. Enquanto a mineração de dados desenvolve metodologias para explorar os diferentes tipos de dados coletados nos diferentes ambientes, o \textit{learning analytics} refere-se diretamente à medição, à coleta, à análise e à produção de relatórios e \textit{insights} sobre os contextos, permitindo que se possa melhor compreender e otimizar os cenários estudados. Ou seja, a área de conhecimento de IA possui intersecção com a ciência de dados por meio de conexões cruzadas com a área de conhecimento de gestão de dados.

Da mesma forma, existem objetivos explícitos para desenvolver a literacia básica em IA e o pensamento crítico em todos os estudantes de ciências da computação, dada a amplitude das interconexões entre a IA e outras áreas de conhecimento, na prática. É mais fácil falar sobre o que a IA tem para além da computação ao invés do que ela compartilha. A IA vai além da perspectiva da lógica e dos algoritmos e enfatiza aspectos como um conjunto de representações, nem sempre estruturadas, para contextualizar a resolução de problemas diversos, raciocinar por padrões permitindo o processamento de semânticas e contextos e a capacidade de aprendizado dos algoritmos.   Ou   seja,   da   mesma   forma   que   a   IA   dialoga   com   várias ciências, ela possui intimidade com muitas disciplinas da Computação.  

Esses aspectos têm como objetivo garantir uma formação mais holística, levando o egresso a refletir sobre o mundo, a entender e resolver problemas computacionais aplicados em diversas áreas e sabendo agir de forma consciente, ética, empreendedora e inovadora, contribuindo para a evolução e melhoria da sociedade. Para isso, durante o curso de Inteligência Artificial, é importante que o estudante tenha a oportunidade de desenvolver os seguintes aspectos complementares à sua formação:

\begin{enumerate}
    \item Atuar com diferentes profissionais de diferentes áreas para identificar oportunidades do mercado e atender às necessidades da sociedade, sabendo trabalhar em equipe.
    \item Praticar a interdisciplinaridade para atuar em diferentes domínios de sistemas computacionais.
    \item Ter a capacidade de resiliência frente à evolução acelerada da IA.
    \item Realizar ações empreendedoras na busca de soluções mais eficazes, incluindo novas tecnologias, produtos e serviços.
    \item Aprender de forma contínua e autônoma sobre métodos, instrumentos, tecnologias de infraestrutura e domínios de aplicação da computação, além de se adequar rapidamente às mudanças tecnológicas e aos novos ambientes de trabalho.
    \item Exercitar a inovação em IA, por meio de conhecimentos científicos e tecnológicos que vão além dos necessários para suas aplicações tradicionais  (incluindo a possibilidade de criar o seu próprio negócio).
    \item Participar de intercâmbio e internacionalização da ciência e tecnologia.
    \item Envolver-se em pesquisa científica.
    \item Interagir com empresas, por meio, por exemplo, de estágio, laboratórios-empresa e empresa júnior.
    \item As aplicações de IA podem ter impactos significativos na sociedade, afetando tanto indivíduos como a coletividade. Isto traz a necessidade de que os alunos compreendam as implicações do trabalho em IA para a sociedade e que façam escolhas responsáveis sobre quando e como aplicar técnicas de IA.
\end{enumerate}

