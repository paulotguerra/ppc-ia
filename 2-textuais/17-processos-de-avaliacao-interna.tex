\chapter{PROCESSOS DE AVALIAÇÃO INTERNA E EXTERNA DO CURSO}
\label{cap:processos-de-avaliacao-interna}

O planejamento e as ações relacionadas à gestão do curso são regularmente avaliados por meio de processos de avaliação internos e externos, cujos resultados subsidiam ajustes no planejamento. Os processos são todos tratados como ferramentas de Avaliação Institucional, mesmo quando externos ou realizados em outros cursos do campus, com os quais o curso de \nomedocurso compartilha parte de infraestrutura, corpo docente e técnico-administrativo.

A Avaliação Institucional tem como objetivo identificar o perfil e o significado da atuação da IES, através das suas atividades, cursos, programas, projetos e setores. Esse processo é norteado pelo SINAES (BRASIL, 2004b), que adota três macroprocedimentos visando valorar o mérito e a excelência de uma IES, a saber: a Autoavaliação Institucional, a Avaliação das Condições de Ensino dos Cursos de Graduação (ACE) e o ENADE.

Na UFC, a Comissão Própria de Avaliação (CPA) e as Comissões Setoriais de Avaliação (CSA) conduzem o Programa de Autoavaliação Institucional, realizando sua divulgação nas unidades acadêmicas e estimulando as coordenações de curso a realizar a discussão dos seus resultados entre os alunos e professores. No campus, tem como princípio ser um processo contínuo, viabilizado por práticas tanto de pesquisa quanto de gestão do conhecimento.

O Programa de Autoavaliação Institucional da UFC é operacionalizado através do Sistema Integrado de Gestão Acadêmica (SIGAA). A avaliação permite que os alunos expressem, semestralmente, suas opiniões sobre os trabalhos dos docentes, em quatro dimensões com diferentes pesos para o cálculo da nota do docente: planejamento pedagógico, didático e domínio do conteúdo (peso: 40\%); relacionamento e postura com os discentes (peso: 20\%); formas e usos da avaliação do aprendizado discente (peso: 20\%); e pontualidade e assiduidade às aulas (peso: 20\%).

Como parte deste processo, os discentes que cursam disciplinas com carga horária de extensão são convidados a avaliar a qualidade e o êxito dessas práticas. Essa avaliação discente permite analisar os resultados alcançados pelas atividades junto ao público participante, fornecendo subsídios diretos para o aperfeiçoamento das metodologias e da relação com a comunidade externa. Esses dados são utilizados pelo curso, por meio do Núcleo Docente Estruturante (NDE) e do Colegiado, para realizar uma análise contínua quanto as possibilidades, critérios, formas, pertinência, contribuição e resultados esperados das atividades de extensão, bem como sua contribuição para os objetivos do PPC, visando o aperfeiçoamento das características essenciais da extensão e garantindo a sua articulação com o ensino e a pesquisa.

Além da avaliação dos docentes, os discentes realizam autoavaliação e, anualmente, avaliam a infraestrutura e a coordenação do curso. Com relação à infraestrutura, são respondidas questões como se os ambientes de aprendizagem possuem tamanho adequado à quantidade de alunos da turma, se possuem adequada climatização, acústica, iluminação, mobiliários e equipamentos adequados ao ensino, além de laboratórios e acervo bibliográfico. Avalia-se também se os banheiros são limpos e adequados ao uso, e se os espaços comuns, as vias de acesso aos ambientes de aprendizagem e a biblioteca estão adaptados ao atendimento de alunos com deficiências.

Sobre a coordenação, os discentes avaliam se ela é acessível, se presta orientação e os auxilia quando necessário, se promove e divulga o PPC e estimula a participação dos alunos em encontros científicos e nos processos avaliativos do curso, se promove momentos de diálogo com os alunos sobre a formação acadêmica, currículo e mercado de trabalho, finalizando com uma avaliação geral sobre a satisfação com a coordenação do curso.

Quanto à autoavaliação realizada pelos discentes, eles respondem sobre o seu nível de assiduidade, pontualidade, envolvimento e esforço na disciplina, o nível em que os seus conhecimentos prévios contribuíram para o aprendizado e a ampliação dos conhecimentos, e sobre as competências e habilidades deles como resultado do que foi visto na disciplina.

Os docentes também avaliam seu próprio trabalho e os alunos das disciplinas que ministram. Respondem se os alunos foram assíduos e pontuais, se demonstraram motivação para o aprendizado, envolvimento com as atividades de ensino-aprendizado e responsabilidade na execução das atividades acadêmicas solicitadas, se tiveram postura adequada ao processo de ensino e aprendizado e se tinham as competências cognitivas adequadas para cursar a disciplina.

Além destas, o formulário de avaliação possui um campo aberto no qual os estudantes podem fazer comentários diretos, utilizando as próprias palavras, sobre o item que estão avaliando. Todo o processo avaliativo é feito de maneira anônima e nem docente nem coordenação são capazes de identificar o discente avaliador.