\chapter{ATIVIDADES COMPLEMENTARES}
\label{cap:atividades-complementares}

As Atividades Complementares constituem um conjunto de estratégias pedagógico-didáticas que permitem, no âmbito do currículo, a articulação entre teoria e prática e a complementação, por parte do estudante, dos saberes e habilidades necessárias à sua formação. O Programa de Atividades Complementares (PAC) do Campus da UFC em Quixadá, busca qualificar o aluno e desenvolver de forma complementar, nos futuros profissionais, competências valorizadas em sua área de formação, tais como perfil empreendedor, iniciativa, liderança, autoconhecimento, perseverança e habilidade em lidar com obstáculos, mudanças e transformações, além de prestar serviços à comunidade.

O programa possibilita que o aluno realize atividades práticas ligadas à profissão que escolheu, a partir do primeiro semestre do curso, criando um diferencial na formação universitária, oferecendo uma variedade de atividades complementares, agrupadas em sete categorias: atividades de iniciação à docência, à pesquisa e/ou à extensão, atividades artístico-culturais e esportivas, atividades de participação e/ou organização de eventos, experiências ligadas à formação profissional e/ou correlatas, produção técnica e/ou científica, vivências de gestão, outras atividades.

Entre as sete categorias, são permitidos mais de cinquenta tipos diferentes de atividades complementares, abrangendo uma grande diversidade de opções para o aluno, indo desde “publicação de artigos científicos” a “doação de sangue”. As possibilidades de atividades complementares permitidas são aderentes à formação geral do discente, por meio de iniciativas de socialização com colegas e demais membros da comunidade acadêmica, atividades em prol da sociedade, atividades relacionadas à saúde, esporte, cultura, etc.; são também aderentes à formação específica do discente, como experiências técnicas em minicursos, palestras, projetos de pesquisa, etc.

Diversas atividades e eventos de extensão, atualmente, podem ser contabilizados como horas complementares. Entretanto, a partir do atendimento à Resolução CEPE nº 28, de 1º de dezembro de 2017 (UFC, 2017c), que regulamenta a Curricularização da Extensão nos cursos de graduação da UFC, poderão ser integralizadas, como “Atividades Complementares”, apenas as horas excedentes das ações extensionistas.
A forma de aproveitamento das horas varia dependendo da atividade. Em alguns casos, o aluno contabiliza horas complementares com valores diferentes das horas reais. Por exemplo, a participação em congressos nacionais gera uma pontuação de seis horas complementares por dia de evento (hora complementar provavelmente menor que hora real), enquanto a participação como ministrante de um minicurso gera duas horas complementares para cada hora ministrada (hora complementar maior que hora real). Outra variação considerada no aproveitamento das horas é que, em alguns casos, a mesma atividade gera horas complementares diferentes dependendo da natureza da participação do aluno. Por exemplo, para um mesmo minicurso com duração de quatro horas, o aluno que ministrou contabiliza oito horas complementares, o aluno que organizou contabiliza duas horas complementares e o aluno que assistiu contabiliza uma hora complementar.

O cumprimento das atividades complementares é obrigatório à colação de grau. O seu acompanhamento e validação ficam a cargo da Coordenação do Curso. O registro e acompanhamento das Atividades Complementares é realizado através do sistema de acompanhamento SISAC (Sistema de Atividades Complementares) pela Coordenação do Curso, auxiliada pela Secretaria do Curso. O SISAC foi desenvolvido no campus para auxiliar no registro das atividades e no acompanhamento e controle por parte da Coordenação e do próprio aluno, que pode se organizar e planejar o cumprimento das horas complementares durante todo o curso.

Semestralmente, os alunos são orientados a protocolar as atividades complementares realizadas, a serem validadas e lançadas no sistema pela Coordenação e/ou secretaria, mediante a devida comprovação. As atividades protocoladas serão, então, distribuídas entre os itens presentes em cada categoria, de acordo com o Regulamento de Atividades Complementares do curso, observadas as equivalências e os limites de aproveitamento. Recomenda-se que o estudante conclua suas atividades complementares até o semestre anterior àquele em que ele pretende colar grau. O estudante de \nomedocurso que protocolar atividades que contabilizem o mínimo de 192 horas será considerado aprovado nesse componente.

O elenco específico de atividades complementares aproveitáveis, bem como os limites de aproveitamento máximo e mínimo de cada atividade são definidos no Regulamento de Atividades Complementares que atende à Resolução No.07/CEPE, de 17 de junho de 2005, que dispõe sobre estas atividades nos cursos de graduação da UFC (UFC, 2005).