\chapter{TRABALHO DE CONCLUSÃO DE CURSO}
\label{cap:trabalho-de-conclusao-de-curso}

O curso de \nomedocurso possui um Trabalho de Conclusão de Curso (TCC) que envolve todos os procedimentos de uma investigação técnico-científica nos parâmetros acadêmicos. O TCC de \nomedocurso deverá ser desenvolvido pelo estudante ao longo dos dois últimos semestres do curso e envolve as seguintes disciplinas e atividades\footnote{A Seção \ref{sec:integralizacao} apresenta o detalhamento de correquisitos e pré-requisitos relacionados aos três componentes curriculares listados.}:
\begin{itemize}
    \item Disciplina ``Projeto de Pesquisa Científica e Tecnológica'' (PPCT), com 2 créditos (32h), ofertada no 7º semestre;
    \item Atividade ``Trabalho de Conclusão de Curso I'' (TCC I), com 2 créditos (32h), ofertada simultaneamente à disciplina acima citada, no 7º semestre;
    \item Atividade ``Trabalho de Conclusão de Curso II'' (TCC II), com 6 créditos (96h), ofertada no 8º semestre.
\end{itemize}



Cabe ao professor responsável pela disciplina ``Projeto de Pesquisa Científica e Tecnológica'' ministrar o conteúdo de metodologia científica nos seguintes termos:
\begin{itemize}
    \item No início do semestre, apresentar Plano de Ensino, contemplando o cronograma de entrega das versões parciais e da versão final do trabalho realizado pelo discente; no caso do TCC I, auxiliar os alunos na escolha dos temas e do professor-orientador;

    \item Explanar detalhadamente sobre a elaboração, estrutura, redação e apresentação de Trabalhos de Conclusão de Curso, orientando os alunos nessas questões;
    
    \item Reunir-se semanalmente com o grupo de alunos para a construção gradual e colaborativa dos projetos de pesquisa, buscando referencial bibliográfico que fundamente a metodologia empregada em cada projeto;

    \item Verificar a conformidade do TCC com as normas de elaboração de trabalhos acadêmicos da UFC;
    
   % \item Organizar a agenda de defesas e auxiliar na composição das bancas;

   % \item Providenciar, junto à Coordenação do curso, os encaminhamentos administrativos necessários.

\end{itemize}

Enquanto o professor de PPCT contribui com técnicas para elaboração do projeto do TCC, apresentando cada uma de suas etapas de forma conceitual e aplicada, cabe ao orientador as responsabilidades relacionadas ao conteúdo do trabalho e acompanhamento das atividades do aluno em TCC I e TCC II. O TCC I e o TCC II deverão ser entregues em formato acadêmico e defendidos perante uma banca de pelo menos três membros, de acordo com os critérios gerais da UFC, detalhados no Regimento Geral da UFC \cite{ufc_regimento_geral_2019}, serão aprovados com nota igual ou superior a 7,0 e frequência igual ou superior a 90\%.

Assim como acontece com os outros cursos do Campus, para os alunos de \nomedocurso, há a disponibilização dos Trabalhos de Conclusão de Curso em repositórios institucionais próprios da UFC, acessíveis pela internet, para sua devida exposição e consulta pela comunidade em geral\footnote{Os TCCs são armazenados no Repositório Institucional da UFC: \url{http://www.repositorio.ufc.br/} e podem ser acessados também pelo site do campus: \url{https://www.quixada.ufc.br/monografias/}}.

Além disso, para auxiliar a normalização de trabalhos acadêmicos, a Biblioteca Universitária elaborou o Guia de Normalização de Trabalhos Acadêmicos da UFC, tomando como base as normas da ABNT. No site da Biblioteca Universitária constam ainda \textit{templates} em três formatos distintos, já contemplando as recomendações das normas da ABNT: Word, Libre Office e Overleaf\footnote{Todos os documentos sobre normalização de trabalhos acadêmicos estão disponíveis no site da Biblioteca Universitária: \url{https://biblioteca.ufc.br/pt/servicos-e-produtos/normalizacao-de-trabalhos-academicos/}}. Para mais detalhes, observar o Regulamento dos Trabalhos de Conclusão de Curso.

A atividade de conclusão de curso é  intrinsecamente interdisciplinar, onde os conhecimentos adquiridos ao longo do curso são aplicados na criação de soluções científicas ou tecnológicas em \nomedocurso. O desenvolvimento dessa atividade colabora  no alcance das competências previstas no perfil do egresso, como o desenvolvimento do pensamento sistêmico, que permite analisar e entender os problemas e os processos organizacionais, e o desenvolvimento e a gestão de soluções baseadas em tecnologia da informação para os processos de negócio de organizações.