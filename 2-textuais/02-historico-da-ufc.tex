\chapter{HISTÓRICO DA UNIVERSIDADE FEDERAL DO CEARÁ (UFC)}
\label{cap:historico-da-ufc}

\def\ufcAnoReferencia{2024\xspace}

\def\ufcCurGraduacao{118\xspace}
\def\ufcCurGradPresencial{110\xspace}
\def\ufcCurGradEAD{8\xspace}
\def\ufcCurEspecializacao{9\xspace}
\def\ufcCurMestrado{85\xspace}
\def\ufcCurDoutorado{56\xspace}
\def\ufcResidMedica{53\xspace}

\def\ufcVagasGrad{6.278\xspace}

\def\ufcPolosEAD{17\xspace}
\def\ufcMatriculasEAD{412\xspace}

\def\ufcDiscentesPos{8.338\xspace}

\def\ufcCartaPatente{22\xspace}
\def\ufcPedidoPatente{45\xspace}
\def\ufcRegSoftware{9\xspace}
\def\ufcPesqCNPQ{336\xspace}
\def\ufcAcoesExtensao{1.148\xspace}
% \def\ufc{}

% Textos do PDI: https://pdi.ufc.br/wp-content/uploads/2025/01/pdi-2023-2027-3a-revisao-16.12.2024.pdf

A Universidade Federal do Ceará (UFC) foi criada em 16 de dezembro de 1954, pela Lei n.º 2.373, e instalada em 25 de junho do ano seguinte. Com
o lema ``O universal pelo regional'', o reitor Antônio Martins Filho, idealizador e fomentador da fundação da instituição, afirmou o compromisso com o progresso sustentável da nação, associando a importância de uma rede de universidades públicas e gratuitas comprometidas com um projeto de desenvolvimento para o Brasil.

Nessa perspectiva, a UFC, instituição federal de ensino superior, estabelecida como autarquia educacional de regime especial, vinculada ao Ministério da Educação, vem desde então formando profissionais de excelência, gerando e difundindo conhecimentos, preservando e divulgando valores éticos, científicos, artísticos e culturais.%, em conformidade com a sua missão institucional.

Constituída inicialmente a partir de faculdades e de escolas de nível superior já existentes, como a Escola de Agronomia do Ceará, a Faculdade de Direito do Ceará, a Faculdade de Medicina do Ceará e a Faculdade de Farmácia e Odontologia do Ceará, a UFC hoje possui outra dimensão. Com a sua expansão e interiorização ao longo dos últimos anos, a UFC está presente em quase todas as regiões do estado do Ceará, por meio de seus 8 (oito) campi, 3 (três) dos quais localizados no município de Fortaleza, sede da UFC: Campus do Benfica, Campus do Pici e Campus do Porangabuçu; e 5 (cinco) dos quais distribuídos no interior do Ceará, nas cidades Sobral, Quixadá, Crateús, Russas e Itapajé.

São, ao todo, 19 unidades acadêmicas, divididas em centros, faculdades, institutos e campi no interior do estado. São eles: I. Centros – Centro de Ciências; Centro de Ciências Agrárias; Centro de Humanidades e Centro de Tecnologia; II. Faculdades – Faculdade de Direito; Faculdade de Economia, Administração, Atuária e Contabilidade; Faculdade de Educação; Faculdade de Farmácia, Odontologia e Enfermagem; Faculdade de Medicina; III. Institutos – Instituto de Ciências do Mar; Instituto de Cultura e Arte; Instituto de Educação Física e Esportes; e Instituto Universidade Virtual; e Instituto de Arquitetura e Urbanismo e Design; IV. Campi no interior do estado – Campus da UFC em Crateús; Campus da UFC em Quixadá; Campus da UFC em Russas; Campus da UFC em Sobral; e Campus da UFC em Itapajé.

A UFC conta ainda com 1 (um) Centro de Estudos em Aquicultura (CEAC / Labomar Eusébio); 4 (quatro) Fazendas Experimentais: Fazenda Experimental Vale do Curu (Pentecoste), Fazenda Raposa (Maracanaú), Sítio São José (Maracanaú) e Fazenda Lavoura Seca (Quixadá); além dos seguintes equipamentos científicos, tecnológicos e culturais: a Seara da Ciência e o Condomínio de Empreendedorismo e Inovação, localizados no Campus do Pici, a Casa de José de Alencar, localizada no Sítio Alagadiço Novo, a Rádio Universitária FM, o Museu de Arte da UFC (MAUC), a Casa Amarela Eusélio Oliveira (CAEO) e o Teatro Universitário Paschoal Carlos Magno, localizados no Campus do Benfica.

Ao todo a universidade disponibiliza \ufcCurGraduacao cursos de graduação, sendo \ufcCurGradPresencial presenciais e \ufcCurGradEAD na modalidade a distância, oferecidos por meio do Programa Universidade Aberta do Brasil (UAB). Anualmente, a UFC oferta mais de 6 mil novas vagas de ingresso para seus cursos de graduação. Em \ufcAnoReferencia, foram \ufcVagasGrad vagas ofertadas. Os cursos a distância, disponibilizados nos graus de licenciatura, bacharelado e tecnólogo, contaram, no ano de \ufcAnoReferencia, com \ufcMatriculasEAD alunos matriculados e \ufcPolosEAD polos de atendimento distribuídos nos municípios cearenses. Os polos possuem salas de aula, biblioteca, laboratórios de informática, laboratórios para práticas de física, química, dentre outros espaços. Eles também dão suporte aos alunos para sanar dúvidas e resolver problemas.\footnote{A abrangência geográfica dos polos de apoio com alunos ativos ou em matrícula institucional pode ser acessada no endereço \url{https://ead.virtual.ufc.br/}.}

Na pós-graduação, no ano de \ufcAnoReferencia, foram ofertados \ufcCurDoutorado cursos de doutorado, \ufcCurMestrado cursos de mestrado, \ufcCurEspecializacao cursos de especialização, contemplando mais de \ufcDiscentesPos discentes matriculados nesse período, além das \ufcResidMedica residências médicas. 

No âmbito da pesquisa, em \ufcAnoReferencia, a UFC atingiu a marca de \ufcCartaPatente cartas-patente concedidas, \ufcPedidoPatente patentes solicitadas e \ufcRegSoftware registros de softwares depositados no Instituto Nacional da Propriedade Industrial (INPI). Ainda na esfera da pesquisa, a UFC contava, em \ufcAnoReferencia, com \ufcPesqCNPQ pesquisadores CNPq. 

Já na extensão universitária, a UFC realizou \ufcAcoesExtensao ações de extensão em \ufcAnoReferencia, em cinco modalidades: Programas, Projetos, Cursos, Eventos e Prestação de Serviços. Estas ações articulam o conhecimento científico do ensino e da pesquisa com as necessidades da comunidade, interagindo e transformando a realidade social.

A Universidade Federal do Ceará tem a inovação como seu quarto pilar, junto ao ensino, pesquisa e extensão. A partir da Resolução n.º 38/CONSUNI, de 18 de agosto de 2017, a UFC instituiu a sua Política de Inovação, a qual dispõe sobre a definição, geração e gestão de direitos relativos à propriedade intelectual e à inovação tecnológica no âmbito da universidade. 

A estrutura do aparelho administrativo da universidade é composta por seus órgãos colegiados deliberativos – o Conselho Universitário (CONSUNI), o Conselho de Ensino, Pesquisa e Extensão (CEPE) e o Conselho de Curadores –, pelos órgãos de assistência direta e de assessoramento ao reitor, pelos órgãos de planejamento e administração, pelos órgãos de atividades específicas e pelos órgãos suplementares.  
\begin{itemize}
    \item  Órgão de assistência direta e imediata ao reitor: Gabinete e Procuradoria Geral;
 \item  Órgãos de assessoramento ao reitor: Ouvidoria Geral; Coordenadoria Geral de Auditoria, Secretaria Executiva da Comissão de Ética e Comissão Permanente de Processo Administrativo Disciplinar;
 \item  Órgãos de planejamento e administração: Pró-Reitoria de Planejamento e Administração, Pró-Reitoria de Gestão de Pessoas, Superintendência de Infraestrutura e Superintendência de Hospitais Universitários;
 \item   Órgãos de atividades específicas: Pró-Reitoria de Graduação, Pró-Reitoria de Pesquisa e Pós-Graduação, Pró-Reitoria de Extensão, Pró-Reitoria de Assistência Estudantil, PróReitoria de Cultura, Pró-Reitoria de Inovação e Relações Interinstitucionais.
 \item  Órgãos suplementares: Biblioteca Universitária, Superintendência de Tecnologia da Informação, Secretaria de Acessibilidade, Secretaria de Governança, Secretaria de Esportes, Escola Integrada de Desenvolvimento e Inovação Acadêmica, Centro de Excelência em Políticas Educacionais, Centro Estratégico de Excelência em Políticas de Água e Secas, Editora UFC, Secretaria de Comunicação e Marketing, Secretaria de Meio Ambiente, Central Analítica, Colégio de Estudos Avançados.
\end{itemize}

O Conselho Universitário (CONSUNI) é o órgão superior deliberativo e consultivo que concebe a política universitária e arbitra em matéria administrativa, inclusive na gestão econômico-financeira. O Conselho de Ensino, Pesquisa e Extensão (CEPE) é o órgão superior deliberativo e consultivo da universidade, em matéria de ensino, pesquisa e extensão. Já o Conselho de Curadores atua na fiscalização econômico-financeira da universidade. Tais conselhos são compostos por servidores docentes e técnico-administrativos, discentes e representantes da comunidade.

De forma análoga, a administração acadêmica na UFC é exercida por intermédio dos órgãos colegiados deliberativos e por órgãos executivos pertencentes às unidades acadêmicas. Os órgãos executivos que compõem a administração acadêmica na UFC são os centros, as faculdades, os institutos e os campi localizados no interior do estado. Já os órgãos colegiados deliberativos responsáveis pela administração acadêmica são os conselhos das unidades acadêmicas, os departamentos acadêmicos (nos casos de centros e faculdades), as coordenações de cursos de graduação e as coordenações de cursos de pós-graduação (quando houver). Nota-se que os institutos (Fortaleza) e os campi  (Sobral, Quixadá, Russas, Crateús e Itapajé) possuem uma estrutura acadêmica diferente da dos centros e faculdades, dispensando a presença dos departamentos acadêmicos.

Com uma trajetória sólida e em constante expansão, a Universidade Federal do Ceará consolida-se como uma das principais instituições de ensino superior do país, destacando-se pelo compromisso com a excelência acadêmica, a produção científica de relevância e o impacto social de suas ações. Sua presença em diferentes regiões do estado, aliada à diversidade de cursos, programas e iniciativas, reflete o papel estratégico que exerce no desenvolvimento regional e nacional.