\chapter{HISTÓRICO DO CURSO E POLÍTICAS INSTITUCIONAIS NO ÂMBITO DO CURSO}
\label{cap:historico-do-curso}

O Bacharelado em \nomedocurso da Universidade Federal do Ceará (UFC), ofertado no Campus de Quixadá, foi criado com o propósito de formar profissionais altamente capacitados para atuar em uma das áreas mais promissoras e estratégicas da atualidade. Sua criação foi aprovada pelo Conselho Universitário da Universidade Federal do Ceará , por meio da Resolução n$^\circ$ \textcolor{red}{XXX}, de \textcolor{red}{XX} de \textcolor{red}{XXXXXXX} de \textcolor{red}{XXXX}, como parte do compromisso institucional da UFC com a inovação acadêmica e a formação orientada aos desafios contemporâneos.

Vinculado academicamente e administrativamente ao Campus da UFC em Quixadá, o curso se insere em um ambiente que se destaca por sua vocação tecnológica e pela concentração de graduações na área da Computação. A presença do curso reforça o papel estratégico do campus como polo de formação em TI no interior do Ceará, contribuindo para o desenvolvimento científico, econômico e social da região.

A criação do curso está fortemente alinhada às políticas institucionais estabelecidas no Plano de Desenvolvimento Institucional da UFC (PDI 2023–2027), especialmente no que se refere à ampliação da oferta de cursos inovadores, ao estímulo à pesquisa aplicada e ao fortalecimento da extensão universitária. Nesse contexto, o curso promove uma formação interdisciplinar e sólida, integrando ensino, pesquisa e extensão de maneira orgânica e voltada para a realidade social e tecnológica do país.

Entre os objetivos estratégicos destacados no PDI com os quais o curso se articula em seu projeto pedagógico, destacam-se:
\begin{itemize}
    % OE1 Aprimorar a formação discente
    \item {Aprimorar a formação discente}, mediante o uso de metodologias de ensino ativas, projetos interdisciplinares e conteúdos atualizados;

    % OE2 Destacar-se, nacional e internacionalmente, pelo desenvolvimento da ciência, tecnologia, inovação e empreendedorismo.
    \item {Estimular o desenvolvimento da ciência, tecnologia e inovação}, incentivando o envolvimento dos estudantes em projetos de pesquisa e em ecossistemas de inovação;

    % OE3 Fortalecer a extensão universitária na UFC.
    \item {Ampliar e consolidar a extensão universitária}, promovendo a inclusão digital, o desenvolvimento regional e a aplicação prática do conhecimento acadêmico em benefício da sociedade.
    
\end{itemize}

Essas diretrizes são colocadas em prática pelo curso, seguindo programas institucionais que fortalecem a missão institucional, como ``Propostas formativas flexíveis e arranjos curriculares modernos'', ``Inovação tecnológica'' e ``Democratização da extensão universitária''. A incorporação dessas políticas no cotidiano do curso contribui diretamente para a formação de um egresso crítico, tecnicamente preparado e socialmente comprometido para atuar de forma inovadora em diversos contextos.

Dessa forma, o Bacharelado em \nomedocurso representa uma iniciativa estratégica, não apenas por sua aderência a um campo de conhecimento promissor, mas também por reafirmar o compromisso institucional com a excelência acadêmica, a relevância social e com o desenvolvimento sustentável baseado no conhecimento e na inovação.