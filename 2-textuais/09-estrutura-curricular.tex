\chapter{ESTRUTURA CURRICULAR}
\label{cap:estrutura-curricular}

As Diretrizes Curriculares Nacionais para cursos da área de Computação \cite{cne2001parecer9} e a Referenciais de Formação para o Curso de Bacharelado em Inteligência Artificial \cite{sbc2024} serviram como base para a elaboração e organização do currículo do Bacharelado em \nomedocurso. O projeto pedagógico do curso valoriza o desenvolvimento de competências tecnológicas, a capacidade de aprender a aprender, a adaptabilidade às transformações do mundo contemporâneo e a flexibilidade curricular.

A definição dos conteúdos dos componentes curriculares, das metodologias de ensino e aprendizagem, bem como dos instrumentos de avaliação, está alinhada aos princípios norteadores do curso, aos seus objetivos formativos e ao perfil do egresso. Neste capítulo, evidencia-se como as unidades curriculares articulam-se no interior da integralização curricular, favorecendo o desenvolvimento do pensamento crítico e a integração entre teoria e prática, com ênfase na flexibilidade e no uso de metodologias ativas de aprendizagem.

Destaca-se também, no âmbito do curso de \nomedocurso e do Campus da UFC em Quixadá, o compromisso com a acessibilidade. As ações desenvolvidas estão em sintonia com a atuação da \textit{Secretaria de Acessibilidade UFC Inclui}, setor institucional criado em 2010, que elabora e coordena estratégias para promover a inclusão de pessoas com deficiência em seis dimensões: atitudinal, arquitetônica, comunicacional, instrumental, metodológica e programática. A acessibilidade é compreendida de forma ampla, não se restringindo apenas a aspectos físicos, mas considerando um conjunto de elementos complementares essenciais para garantir um processo educacional inclusivo e equitativo.

Outro aspecto relevante é a articulação do curso com o setor produtivo. Parcerias com empresas da área de Tecnologia da Informação e Comunicação (TIC) possibilitam que a coordenação e o corpo docente mantenham uma escuta ativa sobre as demandas locais. Essa interação contínua subsidia a atualização do currículo e a adequação da formação às exigências do mercado de trabalho e da sociedade.

Adicionalmente, as ações de extensão do curso são estruturadas em consonância com as diretrizes da Resolução CNE/CES nº 7/2018. As ações devem necessariamente promover uma interação dialógica com a sociedade, inserindo os estudantes em projetos junto a comunidades externas para a troca de saberes e o enfrentamento de questões complexas. Essa vivência, integrada à estrutura curricular, é um pilar para a formação cidadã dos discentes, que aplicam seus conhecimentos de modo interprofissional e interdisciplinar. Ao construir e aplicar soluções de Inteligência Artificial em cenários reais, busca-se a produção de mudanças que impactem positivamente a comunidade e a própria instituição, materializando a articulação entre ensino, extensão e pesquisa em um processo pedagógico único e indissociável, conforme os princípios deste projeto.

A proposta curricular foi construída com base nas necessidades regionais de formação de profissionais qualificados para atuar em diversas áreas da informática. Considerando o crescimento do mercado de TIC no estado do Ceará, o projeto pedagógico prioriza a formação cidadã e técnica, voltada à promoção do desenvolvimento regional, à inclusão social e à inovação tecnológica.


\section{Conteúdos curriculares}

O currículo do Curso de \nomedocurso da UFC está estruturado para promover o desenvolvimento integral do perfil profissional do egresso, por meio de uma formação sólida em fundamentos da computação, inteligência artificial, ética e inovação. Está organizado de forma a fomentar o comportamento autodidata, a atualização contínua, a criatividade, a experimentação e a reflexão crítica sobre os impactos sociais e ambientais das tecnologias emergentes.

A proposta curricular valoriza a interdisciplinaridade, o que contribui diretamente para a integração entre ensino, pesquisa e extensão, conforme previsto nas Diretrizes Curriculares Nacionais e nas diretrizes institucionais da UFC. A matriz inclui componentes obrigatórios, optativos e de extensão, com cargas horárias expressas em horas-relógio, respeitando os parâmetros normativos e assegurando um equilíbrio entre teoria, prática e formação cidadã.

Além dos componentes que compõem a integralização curricular, o curso contempla temas transversais voltados às políticas de educação ambiental, de educação em direitos humanos, de educação das relações étnico-raciais e ao ensino de história e cultura afro-brasileira, africana e indígena. Tais conteúdos são inseridos especialmente nas disciplinas integradoras, como os projetos extensionistas, nos componentes de formação humanística como Sociedade, Culturas e Tecnologia e Ética em Inteligência Artificial, em componentes específicos sobre cada temática, como Relações Étnico-Raciais e Africanidades, Educação Ambiental e Educação em Direitos Humanos, bem como em atividades complementares promovidas pelo campus.

A educação ambiental é um eixo de atuação presente no Campus da UFC em Quixadá desde 2013, com iniciativas como palestras temáticas, ações de reflorestamento e visitas técnicas. O curso de Inteligência Artificial integra essas ações e oferece a disciplina de Educação Ambiental como optativa, por meio da qual foram realizadas campanhas de conscientização como “Menos um copo” (2019, 2022), além de concursos, palestras e visitas técnicas. Projetos como o EcoPET, promovido pelo PET-TI, integram os discentes em ações contínuas de preservação e educação ambiental, com uso de tecnologias móveis desenvolvidas pelos próprios estudantes do campus.

As questões étnico-raciais e de direitos humanos emergem tanto nos debates em sala de aula quanto em atividades culturais e científicas promovidas no campus. Um exemplo é a “I Mostra de Cinema Africano – Mamma África”, realizada em parceria com a Casa de Saberes Cego Aderaldo, que buscou fomentar o contato com a produção cultural africana e reflexões sobre diversidade, história e identidade. Essas temáticas também surgem em disciplinas Sociedade, Culturas e Tecnologia e Ética em Inteligência Artificial, e nos componentes voltados à atuação social e comunitária.

%A formação oferecida também valoriza a educação empreendedora e a inovação tecnológica. O currículo inclui a disciplina obrigatória de Empreendedorismo em Inteligência Artificial e estimula a participação discente em programas como o INOVE, voltado à pesquisa aplicada, à geração de soluções tecnológicas e à criação de empreendimentos inovadores com impacto regional.

O curso promove ao longo de seu percurso o contato dos estudantes com o fundamento da área, enfatizando o método e o rigor científico que orientam a produção de conhecimento em diferentes temáticas, como Mineração de Dados, Sistemas Multiagentes, Processamento de Linguagem Natural e Visão Computacional. A disciplina Projeto de Pesquisa Científico-Tecnológico desenvolve formalmente a metodologia de elaboração de projetos de pesquisa, fornecendo aos discentes ferramentas conceituais e técnicas necessárias para a definição de problemas, construção de hipóteses, escolha de métodos adequados e estruturação rigorosa de investigações científicas na área do curso. Essa competência se consolida nas disciplinas de Trabalho de Conclusão de Curso I e II, em que os alunos desenvolvem e apresentam pesquisas com base em problemas atuais e aplicação de técnicas aprendidas ao longo do curso. Essa estrutura foi reforçada com a implementação do Mestrado em Ciência da Computação do campus, aprovado pelo MEC em 2019, o que fortalece o vínculo entre graduação e pós-graduação.

A extensão universitária é formalizada no currículo por meio da curricularização prevista na Resolução CEPE nº 28/2017. Diversas disciplinas contemplam ações extensionistas que envolvem comunidades, instituições públicas, escolas e organizações da sociedade civil, ampliando o impacto social da formação acadêmica e a interação com saberes diversos.

Por fim, a estrutura curricular do curso promove um diálogo entre conteúdos atualizados e inovadores, metodologias acessíveis, bibliografias de qualidade e políticas de inclusão e diversidade. Essa articulação qualifica a formação dos egressos, diferenciando o curso na área da Computação e contribuindo para a formação de profissionais críticos, comprometidos com o desenvolvimento tecnológico e com os desafios sociais contemporâneos.


\section{Unidades curriculares}
\label{sec:unidades-curriculares}

As unidades curriculares (UC) do curso de \nomedocurso são áreas de conhecimento que agrupam componentes curriculares afins. Além de cumprirem uma função administrativa — já que um representante de cada unidade compõe o Colegiado do curso — exercem, sobretudo, uma função pedagógica, constituindo-se como espaços específicos de discussão sobre questões didático-pedagógicas de cada área.

Essas unidades visam formar o futuro profissional em \nomedocurso com uma base sólida e interdisciplinar, capacitando-o para atuar em diferentes subáreas da Inteligência Artificial, sempre com consciência crítica, responsabilidade social e comprometimento ético.

Os componentes curriculares estão organizados em seis Unidades Curriculares que cobrem três grandes áreas: Formação Básica; Formação Tecnológica; e Formação Complementar e Humanística.

A formação básica é subdividida nas seguintes Unidades Curriculares: Formação Básica em Matemática; Formação Básica em Ciência da Computação; Formação Básica em Inteligência Artificial; Formação Básica em Aprendizado de Máquina e Ciência de Dados. 
%
A formação tecnológica e realizada por meio da Unidade Curricular Formação Tecnológica em Inteligência Artificial e Aprendizado de Máquina, que contempla os saberes técnicos e práticos para o desenvolvimento e aplicação de soluções em IA. 
%
A formação complementar e humanística é promovida pela Unidade Curricular de mesmo nome, que tem como foco o desenvolvimento das dimensões ética, social, crítica e comunicacional do futuro profissional em \nomedocurso.

De forma mais específica, recomenda-se o seguinte perfil docente para o curso:

\begin{enumerate}
    \item Os docentes da Formação Básica em Matemática e da Formação Complementar e Humanística devem possuir formação nas áreas específicas das disciplinas que ministram. É desejável que tenham vivência acadêmica ou profissional que lhes permita articular os conteúdos desenvolvidos com sua aplicação na área da Inteligência Artificial.

    \item Para a Formação Básica em Ciência da Computação, recomenda-se que os docentes tenham formação em Computação ou áreas afins, com domínio de fundamentos teóricos e práticos. Experiência na aplicação desses conhecimentos à modelagem de problemas e à construção de soluções computacionais é altamente desejável.

    \item Os professores da Formação Básica em Inteligência Artificial e da Formação Básica em Aprendizado de Máquina e Ciência de Dados devem ter formação e experiência em áreas como Inteligência Artificial, Ciência de Dados, Aprendizado de Máquina, Estatística Aplicada ou áreas correlatas. Espera-se que possam proporcionar uma sólida base conceitual e prática, alinhada às tendências contemporâneas da área.

    \item Os docentes da Formação Tecnológica em Inteligência Artificial e Aprendizado de Máquina devem possuir experiência comprovada em projetos de pesquisa, desenvolvimento e inovação envolvendo técnicas e aplicações de IA, como visão computacional, processamento de linguagem natural, sistemas autônomos, entre outros. É recomendável que atuem também na orientação de trabalhos de conclusão de curso e em projetos.

    \item Na Formação Complementar e Humanística, os docentes devem estar preparados para promover reflexões sobre as implicações éticas, sociais, políticas e econômicas do uso da Inteligência Artificial, contribuindo para uma formação cidadã crítica e responsável. %Deve ainda enfatizar a necessidade da realização de estágio profissional e/ou de trabalhos de conclusão de curso.

\end{enumerate}

% Cada Unidade Curricular elege um representante para o Colegiado do curso, juntamente com seu respectivo suplente, com mandato de três anos, permitida uma recondução. As regras de composição e funcionamento das UC seguem as diretrizes da Resolução nº 07/CEPE de 1994 (UFC, 1994), com alterações da Resolução nº 03/CEPE de 2016 (UFC, 2016a).

A distribuição dos componentes curriculares por Unidade Curricular revela o caráter híbrido e inovador do curso de \nomedocurso, evidenciando sua ênfase em fundamentos sólidos, competências técnicas avançadas e formação humanística.  
A seguir, estão descritas em mais detalhes as Unidades Curriculares definidas para o curso, bem como os componentes curriculares que as compõem.

\subsection{Formação Básica em Matemática}

Esta unidade curricular reúne disciplinas obrigatórias e optativas, de caráter prioritariamente teórico, relacionadas com a base matemática necessária para o entendimento e desenvolvimento de técnicas em Inteligência Artificial. A Tabela \ref{tab:ucfbm} apresenta as disciplinas que compõem a \textit{Unidade Curricular Formação Básica em Matemática}.


\begin{table}[h]
  \caption{Componentes curriculares da Unidade Curricular de Formação Básica em Matemática}
    \centering
    \resizebox{0.95\textwidth}{!}{%
    \footnotesize
    \begin{tabular}{|
        >{\centering\arraybackslash}m{1.5cm}|
        >{\centering\arraybackslash}m{6.5cm}|
        >{\centering\arraybackslash}m{1.5cm}|
        >{\centering\arraybackslash}m{1.7cm}|
        >{\centering\arraybackslash}m{1.5cm}|
        >{\centering\arraybackslash}m{1.8cm}| 
    }
    \hline
    \textbf{Código} & \textbf{Componente} & \textbf{Tipo} & \textbf{Caráter} & \textbf{Regime} & \textbf{Unidade responsável} \\ \hline 
QXD0116 & Álgebra Linear / \textit{Linear Algebra} & Disciplina & Obrigatório & Semestral & Campus Quixadá \\ \hline
QXD0006 & Cálculo Diferencial e Integral I / \textit{Calculus I} & Disciplina & Obrigatório & Semestral & Campus Quixadá \\ \hline
QXD0134 & Cálculo Diferencial e Integral II / \textit{Calculus II} & Disciplina & Optativo & Semestral & Campus Quixadá \\ \hline
QXD0135 & Cálculo Diferencial e Integral III / \textit{Calculus III} & Disciplina & Optativo & Semestral & Campus Quixadá \\ \hline
QXD0179 & Estatística Multivariada / \textit{Multivariate Statistics} & Disciplina & Optativo & Semestral & Campus Quixadá \\ \hline
QXD0017 & Lógica para Computação / \textit{Logic for Computer Science} & Disciplina & Obrigatório & Semestral & Campus Quixadá \\ \hline
QXD0056 & Matemática Básica / \textit{Basic Mathematics} & Disciplina & Obrigatório & Semestral & Campus Quixadá \\ \hline
QXD0120 & Matemática Computacional / \textit{Computational Mathematics} & Disciplina & Optativo & Semestral & Campus Quixadá \\ \hline
QXD0008 & Matemática Discreta / \textit{Discrete Mathematics} & Disciplina & Obrigatório & Semestral & Campus Quixadá \\ \hline
QXD0109 & Pré-Cálculo / \textit{Pre-Calculus} & Disciplina & Obrigatório & Semestral & Campus Quixadá \\ \hline
QXD0112 & Probabilidade e Estatística / \textit{Probability and Statistics} & Disciplina & Obrigatório & Semestral & Campus Quixadá \\ \hline
QXD0152 & Teoria dos Grafos / \textit{Graph Theory} & Disciplina & Optativo & Semestral & Campus Quixadá \\ \hline
<27> & Teoria dos Jogos / \textit{Game Theory} & Disciplina & Optativo & Semestral & Campus Quixadá \\ \hline
<32> & Tópicos Especiais em Matemática Computacional / \textit{Special Topics in Computational Mathematics} & Disciplina & Optativo & Semestral & Campus Quixadá \\ \hline
    \end{tabular}
}
  \label{tab:ucfbm}
\end{table}

\subsection{Formação Básica em Ciência da Computação}

Esta unidade curricular reúne disciplinas obrigatórias e optativas, de caráter prioritariamente teórico-prático, relacionadas com a área de formação básica em Ciência da Computação. A Tabela \ref{tab:ucfbcc} apresenta as disciplinas que compõem a \textit{Unidade Curricular Formação Básica em Ciência da Computação}.

\begin{table}[h!]
    \caption{Componentes curriculares da Unidade Curricular de Formação Básica em Ciência da Computação}
    \centering
    \resizebox{0.95\textwidth}{!}{%
    \footnotesize
    \begin{tabular}{|
        >{\centering\arraybackslash}m{1.5cm}|
        >{\centering\arraybackslash}m{6.5cm}|
        >{\centering\arraybackslash}m{1.5cm}|
        >{\centering\arraybackslash}m{1.7cm}|
        >{\centering\arraybackslash}m{1.5cm}|
        >{\centering\arraybackslash}m{1.8cm}| 
    }
    \hline
    \textbf{Código} & \textbf{Componente} & \textbf{Tipo} & \textbf{Caráter} & \textbf{Regime} & \textbf{Unidade responsável} \\ \hline
QXD0005 & Arquitetura de Computadores / \textit{Computer Architecture} & Disciplina & Optativo & Semestral & Campus Quixadá \\ \hline
<07> & Busca e Otimização / \textit{Search and Optimization} & Disciplina & Obrigatório & Semestral & Campus Quixadá \\ \hline
QXD0025 & Compiladores / \textit{Compilers} & Disciplina & Optativo & Semestral & Campus Quixadá \\ \hline
QXD0079 & Computação em Nuvem / \textit{Cloud Computing} & Disciplina & Optativo & Semestral & Campus Quixadá \\ \hline
QXD0183 & Computação Paralela / \textit{Parallel Computing} & Disciplina & Optativo & Semestral & Campus Quixadá \\ \hline
QXD0153 & Desafios de Programação / \textit{Programming Challenges} & Disciplina & Optativo & Semestral & Campus Quixadá \\ \hline
QXD0276 & Desenvolvimento de Software para Dispositivos Móveis / \textit{Software Development for Mobile Devices} & Disciplina & Optativo & Semestral & Campus Quixadá \\ \hline
QXD0253 & Desenvolvimento de Software para Web / \textit{Software Development for the Web} & Disciplina & Optativo & Semestral & Campus Quixadá \\ \hline
QXD0019 & Engenharia de Software / \textit{Software Engineering} & Disciplina & Obrigatório & Semestral & Campus Quixadá \\ \hline
QXD0010 & Estrutura de Dados / \textit{Data Structure} & Disciplina & Obrigatório & Semestral & Campus Quixadá \\ \hline
QXD0115 & Estrutura de Dados Avançada / \textit{Advanced Data Structure} & Disciplina & Obrigatório & Semestral & Campus Quixadá \\ \hline
QXD0011 & Fundamentos de Banco de Dados / \textit{Database Fundamentals} & Disciplina & Obrigatório & Semestral & Campus Quixadá \\ \hline
QXD0001 & Fundamentos de Programação / \textit{Programming Fundamentals} & Disciplina & Obrigatório & Semestral & Campus Quixadá \\ \hline
QXD0254 & Gerência de Projetos / \textit{Project Management} & Disciplina & Optativo & Semestral & Campus Quixadá \\ \hline
QXD0256 & Interação Humano-Computador / \textit{Computer-Human Interaction} & Disciplina & Optativo & Semestral & Campus Quixadá \\ \hline
QXD0016 & Linguagens de Programação / \textit{Programming Languages} & Disciplina & Optativo & Semestral & Campus Quixadá \\ \hline
QXD0040 & Linguagens Formais e Autômatos / \textit{Formal Languages and Automata} & Disciplina & Optativo & Semestral & Campus Quixadá \\ \hline
QXD0181 & Pesquisa Operacional / \textit{Operations Research} & Disciplina & Optativo & Semestral & Campus Quixadá \\ \hline
QXD0188 & Processamento de Imagens / \textit{Image Processing} & Disciplina & Optativo & Semestral & Campus Quixadá \\ \hline
QXD0114 & Programação Funcional / \textit{Functional Programming} & Disciplina & Optativo & Semestral & Campus Quixadá \\ \hline
QXD0007 & Programação Orientada a Objetos / \textit{Object-Oriented Programming} & Disciplina & Obrigatório & Semestral & Campus Quixadá \\ \hline
    \end{tabular}
    }
    \label{tab:ucfbcc}
\end{table}

\begin{table}[h!]
    % \captionsetup{labelformat=empty}
    % \caption{Componentes curriculares da Unidade Curricular de Formação Básica em Ciência da Computação (\textit{continuação})}
    \centering
    \resizebox{0.95\textwidth}{!}{%
    \footnotesize
    \begin{tabular}{|
        >{\centering\arraybackslash}m{1.5cm}|
        >{\centering\arraybackslash}m{6.5cm}|
        >{\centering\arraybackslash}m{1.5cm}|
        >{\centering\arraybackslash}m{1.7cm}|
        >{\centering\arraybackslash}m{1.5cm}|
        >{\centering\arraybackslash}m{1.8cm}| 
    }
    \hline
    \textbf{Código} & \textbf{Componente} & \textbf{Tipo} & \textbf{Caráter} & \textbf{Regime} & \textbf{Unidade responsável} \\ \hline
QXD0041 & Projeto e Análise de Algoritmos / \textit{Design and Analysis of Algorithms} & Disciplina & Obrigatório & Semestral & Campus Quixadá \\ \hline
QXD0021 & Redes de Computadores / \textit{Computer Networks} & Disciplina & Optativo & Semestral & Campus Quixadá \\ \hline
QXD0069 & Segurança / \textit{Security} & Disciplina & Optativo & Semestral & Campus Quixadá \\ \hline
QXD0043 & Sistemas Distribuídos / \textit{Distributed Systems} & Disciplina & Optativo & Semestral & Campus Quixadá \\ \hline
QXD0013 & Sistemas Operacionais / \textit{Operating Systems} & Disciplina & Optativo & Semestral & Campus Quixadá \\ \hline
QXD0046 & Teoria da Computação / \textit{Theory of Computation} & Disciplina & Optativo & Semestral & Campus Quixadá \\ \hline
    \end{tabular}
    }
\end{table}

\subsection{Formação Básica em Aprendizado de Máquina e Ciência de Dados}

Esta unidade curricular abrange disciplinas que apresentam os princípios estatísticos e computacionais do aprendizado de máquina e da análise de dados, proporcionando uma base para a modelagem e interpretação de dados em diferentes contextos. A Tabela \ref{tab:ucfbamcd} apresenta as disciplinas que compõem a \textit{Unidade Curricular Formação Básica em Aprendizado de Máquina e Ciência de Dados}.

\begin{table}[h]
  \caption{Componentes curriculares da Unidade Curricular de Formação Básica em Aprendizado de Máquina e Ciência de Dados}
    \centering
    \resizebox{0.95\textwidth}{!}{%
    \footnotesize
    \begin{tabular}{|
        >{\centering\arraybackslash}m{1.5cm}|
        >{\centering\arraybackslash}m{6.5cm}|
        >{\centering\arraybackslash}m{1.5cm}|
        >{\centering\arraybackslash}m{1.7cm}|
        >{\centering\arraybackslash}m{1.5cm}|
        >{\centering\arraybackslash}m{1.8cm}| 
    }
    \hline
<21> & Análise Exploratória de Dados / \textit{Exploratory Data Analysis} & Disciplina & Optativo & Semestral & Campus Quixadá \\ \hline
<05> & Aprendizado Não Supervisionado / \textit{Unsupervised Learning} & Disciplina & Obrigatório & Semestral & Campus Quixadá \\ \hline
<12> & Aprendizado por Reforço / \textit{Reinforcement Learning} & Disciplina & Obrigatório & Semestral & Campus Quixadá \\ \hline
<09> & Aprendizado Profundo / \textit{Deep Learning} & Disciplina & Obrigatório & Semestral & Campus Quixadá \\ \hline
<06> & Aprendizado Supervisionado / \textit{Supervised Learning} & Disciplina & Obrigatório & Semestral & Campus Quixadá \\ \hline
<22> & Inteligência Artificial Explicável / \textit{Explainable Artificial Intelligence} & Disciplina & Optativo & Semestral & Campus Quixadá \\ \hline
QXD0178 & Mineração de Dados / \textit{Data Mining} & Disciplina & Obrigatório & Semestral & Campus Quixadá \\ \hline
<10> & Processamento de Dados em Larga Escala / \textit{Large-Scale Data Processing} & Disciplina & Obrigatório & Semestral & Campus Quixadá \\ \hline
QXD0177 & Recuperação de Informação / \textit{Information Retrieval} & Disciplina & Optativo & Semestral & Campus Quixadá \\ \hline
<28> & Tópicos Especiais em Aprendizado de Máquina / \textit{Special Topics in Machine Learning} & Disciplina & Optativo & Semestral & Campus Quixadá \\ \hline
<29> & Tópicos Especiais em Ciência de Dados / \textit{Special Topics in Data Science} & Disciplina & Optativo & Semestral & Campus Quixadá \\ \hline
<33> & Visualização de Dados / \textit{Data Visualization} & Disciplina & Optativo & Semestral & Campus Quixadá \\ \hline
    \end{tabular}
    }
  \label{tab:ucfbamcd}
\end{table}

\subsection{Formação Básica em Inteligência Artificial}

Esta unidade curricular compreende disciplinas que introduzem os fundamentos da Inteligência Artificial, suas técnicas clássicas e modernas, abordando conceitos como busca, raciocínio, representação de conhecimento e tomada de decisão. A Tabela \ref{tab:ucfbia} apresenta as disciplinas que compõem a \textit{Unidade Curricular Formação Básica em Inteligência Artificial}.

\begin{table}[h]
  \caption{Componentes curriculares da Unidade Curricular de Formação Básica em Inteligência Artificial}
    \centering
    \resizebox{0.95\textwidth}{!}{%
    \footnotesize
    \begin{tabular}{|
        >{\centering\arraybackslash}m{1.5cm}|
        >{\centering\arraybackslash}m{6.5cm}|
        >{\centering\arraybackslash}m{1.5cm}|
        >{\centering\arraybackslash}m{1.7cm}|
        >{\centering\arraybackslash}m{1.5cm}|
        >{\centering\arraybackslash}m{1.8cm}| 
    }
    \hline
<03> & Agentes Inteligentes / \textit{Intelligent Agents} & Disciplina & Obrigatório & Semestral & Campus Quixadá \\ \hline
<02> & Introdução à Inteligência Artificial / \textit{Introduction to Artificial Intelligence} & Disciplina & Obrigatório & Semestral & Campus Quixadá \\ \hline
<15> & Planejamento Automatizado / \textit{Automated Planning} & Disciplina & Obrigatório & Semestral & Campus Quixadá \\ \hline
<26> & Raciocínio sob Incerteza / \textit{Reasoning Under Uncertainty} & Disciplina & Optativo & Semestral & Campus Quixadá \\ \hline
<04> & Representação do Conhecimento e Raciocínio / \textit{Knowledge Representation and Reasoning} & Disciplina & Obrigatório & Semestral & Campus Quixadá \\ \hline
QXD0076 & Sistemas Multiagentes / \textit{Multiagent Systems} & Disciplina & Obrigatório & Semestral & Campus Quixadá \\ \hline
<31> & Tópicos Especiais em Inteligência Artificial / \textit{Special Topics in Artificial Intelligence} & Disciplina & Optativo & Semestral & Campus Quixadá \\ \hline
    \end{tabular}
    }
  \label{tab:ucfbia}
\end{table}

\subsection{Formação Tecnológica em Inteligência Artificial e Aprendizado de Máquina}

Esta unidade curricular contempla disciplinas voltadas à aplicação prática e desenvolvimento de sistemas baseados em Inteligência Artificial, com foco em tecnologias emergentes, frameworks computacionais e soluções para problemas reais. A Tabela \ref{tab:uctia} apresenta as disciplinas que compõem a \textit{Unidade Curricular Formação Tecnológica em Inteligência Artificial e Aprendizado de Máquina}.

\begin{table}[h]
  \caption{Componentes curriculares da Unidade Curricular de Formação Tecnológica em Inteligência Artificial e Aprendizado de Máquina}
    \centering
    \resizebox{0.95\textwidth}{!}{%
    \footnotesize
    \begin{tabular}{|
        >{\centering\arraybackslash}m{1.5cm}|
        >{\centering\arraybackslash}m{6.5cm}|
        >{\centering\arraybackslash}m{1.5cm}|
        >{\centering\arraybackslash}m{1.7cm}|
        >{\centering\arraybackslash}m{1.5cm}|
        >{\centering\arraybackslash}m{1.8cm}| 
    }
    \hline
    \textbf{Código} & \textbf{Componente} & \textbf{Tipo} & \textbf{Caráter} & \textbf{Regime} & \textbf{Unidade responsável} \\ \hline
<13> & Inteligência Artificial Generativa / \textit{Generative Artificial Intelligence} & Disciplina & Obrigatório & Semestral & Campus Quixadá \\ \hline
<14> & MLOPs / \textit{MLOps} & Disciplina & Obrigatório & Semestral & Campus Quixadá \\ \hline
<23> & Percepção e Ação Robótica / \textit{Robotic Perception and Action} & Disciplina & Optativo & Semestral & Campus Quixadá \\ \hline
<25> & Processamento de Áudio e Voz / \textit{Audio and Speech Processing} & Disciplina & Optativo & Semestral & Campus Quixadá \\ \hline
<11> & Processamento de Linguagem Natural / \textit{Natural Language Processing} & Disciplina & Obrigatório & Semestral & Campus Quixadá \\ \hline
QXD0182 & Visão Computacional / \textit{Computer Vision} & Disciplina & Obrigatório & Semestral & Campus Quixadá \\ \hline
    \end{tabular}
    }
  \label{tab:uctia}
\end{table}

\subsection{Formação Complementar e Humanística}

Esta unidade curricular reúne componentes voltados ao desenvolvimento de competências éticas, sociais, comunicacionais e interdisciplinares, contribuindo para a formação crítica e cidadã do bacharel em Inteligência Artificial. 
Nesta unidade encontram-se também os componentes curriculares de extensão, como as Atividades de Extensão e os Projetos Extensionistas, que atuam como um eixo articulador do currículo. Eles permitem que os estudantes apliquem os saberes técnicos e teóricos, desenvolvidos nas demais unidades curriculares, como a Formação Básica em Ciência da Computação, Aprendizado de Máquina e a Formação Tecnológica em IA, em contextos sociais reais, promovendo a integração entre a formação tecnológica e a humanística e consolidando a indissociabilidade entre ensino, pesquisa e extensão. 
A Tabela \ref{tab:uccfh} apresenta as disciplinas e atividades que compõem a \textit{Unidade Curricular Formação Complementar e Humanística}.

\begin{table}[h]
  \caption{Componentes curriculares da Unidade Curricular de Formação Complementar e Humanística}
    \centering
    \resizebox{0.95\textwidth}{!}{%
    \footnotesize
    \begin{tabular}{|
        >{\centering\arraybackslash}m{1.5cm}|
        >{\centering\arraybackslash}m{6.5cm}|
        >{\centering\arraybackslash}m{1.5cm}|
        >{\centering\arraybackslash}m{1.7cm}|
        >{\centering\arraybackslash}m{1.5cm}|
        >{\centering\arraybackslash}m{1.8cm}| 
    }
    \hline
    \textbf{Código} & \textbf{Componente} & \textbf{Tipo} & \textbf{Caráter} & \textbf{Regime} & \textbf{Unidade responsável} \\ \hline
<01> & Atividades Complementares / \textit{Complementary Activities} & Atividade & Obrigatório & Semestral & Campus Quixadá \\ \hline
<AE> & Atividades de Extensão / \textit{Extension Activities} & Atividade & Obrigatório & Semestral & Campus Quixadá \\ \hline
QXD0232 & Educação Ambiental / \textit{Environmental Education} & Disciplina & Optativo & Semestral & Campus Quixadá \\ \hline
QXD0245 & Educação em Direitos Humanos / \textit{Human Rights Education} & Disciplina & Optativo & Semestral & Campus Quixadá \\ \hline
QXD0029 & Empreendedorismo / \textit{Enterpreneurship} & Disciplina & Optativo & Semestral & Campus Quixadá \\ \hline
<16> & Estágio Curricular Supervisionado I / \textit{Supervised Internship} & Atividade & Obrigatório & Semestral & Campus Quixadá \\ \hline
QXD0508 & Ética e Legislação / \textit{Ethics and Legislation} & Disciplina & Optativo & Semestral & Campus Quixadá \\ \hline
<08> & Ética em Inteligência Artificial / \textit{Ethics in Artificial Intelligence} & Disciplina & Obrigatório & Semestral & Campus Quixadá \\ \hline
QXD0035 & Inglês Instrumental I / \textit{Instrumental English I} & Disciplina & Optativo & Semestral & Campus Quixadá \\ \hline
QXD0036 & Inglês Instrumental II / \textit{Instrumental English II} & Disciplina & Optativo & Semestral & Campus Quixadá \\ \hline
QXD0113 & Língua Brasileira de Sinais - LIBRAS / \textit{Brazilian Sign Language - Libras} & Disciplina & Optativo & Semestral & Campus Quixadá \\ \hline
<24> & Pesquisa e Inovação / \textit{Research and Innovation} & Disciplina & Optativo & Semestral & Campus Quixadá \\ \hline
QXD0110 & Projeto de Pesquisa Científico-Tecnológico / \textit{Scientific and Technological Research Project} & Disciplina & Obrigatório & Semestral & Campus Quixadá \\ \hline
<19> & Projeto Extensionista I / \textit{Extension Project I} & Disciplina & Obrigatório & Semestral & Campus Quixadá \\ \hline
<20> & Projeto Extensionista II / \textit{Extension Project II} & Disciplina & Obrigatório & Semestral & Campus Quixadá \\ \hline
QXD0126 & Psicologia e Percepção / \textit{Psychology and Perception} & Disciplina & Optativo & Semestral & Campus Quixadá \\ \hline
QXD0246 & Relações Étnico-Raciais e Africanidades / \textit{Ethnic-Racial Relationships and Africanities} & Disciplina & Optativo & Semestral & Campus Quixadá \\ \hline
QXD0162 & Sociedade, Culturas e Tecnologia / \textit{Society, Cultures and Technology} & Disciplina & Obrigatório & Semestral & Campus Quixadá \\ \hline
<30> & Tópicos Especiais em Filosofia, Ética e Humanidades / \textit{Special Topics in Philosophy, Ethics and Humanities} & Disciplina & Optativo & Semestral & Campus Quixadá \\ \hline
<17> & Trabalho de Conclusão de Curso I / \textit{Final Project I} & Atividade & Obrigatório & Semestral & Campus Quixadá \\ \hline
<18> & Trabalho de Conclusão de Curso II / \textit{Final Project II} & Atividade & Obrigatório & Semestral & Campus Quixadá \\ \hline
    \end{tabular}
    }
  \label{tab:uccfh}
\end{table}

\clearpage

\section{Integralização curricular} \label{sec:integralizacao}

A estrutura curricular do curso de \nomedocurso foi elaborada de forma a contemplar os objetivos do curso e atingir o perfil profissional proposto, alicerçado nos seus princípios norteadores. A organização do currículo permite a compreensão, o entendimento e o conhecimento necessários para aplicar e desenvolver modelos e métodos baseados em Inteligência Artificial, utilizando tecnologias e metodologias atuais, assegurando as inter-relações com outras áreas do conhecimento. 

Os componentes curriculares podem ser disciplinas ou atividades. As disciplinas do curso são de dois tipos: obrigatórias e optativas. Entre as optativas, os(as) estudantes podem escolher componentes que fazem parte da integralização curricular do curso e também disciplinas livres. As disciplinas livres são aquelas escolhidas fora do elenco específico de componentes curriculares do curso, podendo ser cursadas em qualquer outro curso da universidade. Assim, qualquer código de componente que não pertença ao rol de obrigatórios e optativos elencados aqui, quando cursado pelo(a) estudante, será integralizado como ``livre''. %Dessa forma, a carga horária de disciplinas livres é, necessariamente, parte da carga horária optativa do curso.
%
Na integralização curricular do curso de \nomedocurso, o estudante poderá contabilizar até 192 horas em disciplinas optativas livres, dentro do total de 448 horas exigidas em disciplinas optativas. 

Entre os componentes curriculares do tipo ``atividade'', estão previstas para o curso as seguintes: Estágio Supervisionado, Trabalho de Conclusão de Curso, Atividades de Extensão e Atividades Complementares. Essas atividades são regulamentadas por manuais específicos, anexos a este documento. 

A Tabela~\ref{tab:organizacao_curricular} apresenta a organização curricular por semestre, com informações sobre carga horária, pré-requisitos, correquisitos e equivalências. A Tabela \ref{tab:disciplinas_optativas} apresenta o rol de disciplinas optativas que compõem a integralização curricular do curso. As disciplinas optativas são uma das formas de trabalhar a flexibilidade do currículo.%, onde o aluno poderá escolher 8 disciplinas dentre as disciplinas listadas e que forem ofertadas durante o percurso do aluno no curso.


\begin{table}[h]
\centering
\caption{{Integralização Curricular}}
\renewcommand{\arraystretch}{1.5}
\resizebox{\textwidth}{!}{%
\small
\begin{tabular}{|
>{\centering\arraybackslash}m{1.5cm}|
>{\centering\arraybackslash}m{10.4cm}|
>{\centering\arraybackslash}m{1.4cm}|
>{\centering\arraybackslash}m{1.4cm}|
>{\centering\arraybackslash}m{1.4cm}|
>{\centering\arraybackslash}m{1.5cm}|
>{\centering\arraybackslash}m{1.4cm}|
% >{\centering\arraybackslash}m{1.4cm}|
>{\centering\arraybackslash}m{1.8cm}|
>{\centering\arraybackslash}m{2.4cm}|
>{\centering\arraybackslash}m{2.4cm}|
}
\hline
\centering{\textbf{Código}} & 
\centering{\textbf{Nome do Componente Curricular}\\ (em português e inglês)} & 
\textbf{Carga Horária Teórica} & 
\textbf{Carga Horária Prática} & 
\textbf{Carga Horária EAD} & 
\textbf{Carga Horária Extensão} & 
\textbf{Carga Horária TOTAL} & 
% \textbf{Carga Horária PCC} & 
\textbf{Pré-requisito(s)} & 
\textbf{Correquisito(s)}&  
\textbf{Equivalência(s)} \\ \hline

\multicolumn{10}{|>{\centering\arraybackslash}c|}{\normalsize\textbf{Semestre 1}} \\ \hline
QXD0001 & Fundamentos de Programação / \textit{Programming Fundamentals} & 32 & 64 & 0 & 0 & 96 &  &  &  \\ \hline
<02> & Introdução à Inteligência Artificial / \textit{Introduction to Artificial Intelligence} & 64 & 0 & 0 & 0 & 64 &  &  &  \\ \hline
QXD0056 & Matemática Básica / \textit{Basic Mathematics} & 64 & 0 & 0 & 0 & 64 &  &  &  \\ \hline
QXD0109 & Pré-Cálculo / \textit{Pre-Calculus} & 32 & 0 & 0 & 0 & 32 &  &  &  \\ \hline
QXD0162 & Sociedade, Culturas e Tecnologia / \textit{Society, Cultures and Technology} & 64 & 0 & 0 & 0 & 64 &  &  &  \\ \hline
<AC> & Atividades Complementares / \textit{Complementary Activities} & 32 & 0 & 0 & 0 & 32 &  &  &  \\ \hline

\multicolumn{10}{|>{\centering\arraybackslash}c|}{\normalsize\textbf{Semestre 2}} \\ \hline
QXD0006 & Cálculo Diferencial e Integral I / \textit{Calculus I} & 64 & 0 & 0 & 0 & 64 & QXD0109 &  &  \\ \hline
QXD0010 & Estrutura de Dados / \textit{Data Structure} & 32 & 32 & 0 & 0 & 64 & QXD0001 &  &  \\ \hline
QXD0017 & Lógica para Computação / \textit{Logic for Computer Science} & 64 & 0 & 0 & 0 & 64 & QXD0056 &  &  \\ \hline
QXD0008 & Matemática Discreta / \textit{Discrete Mathematics} & 64 & 0 & 0 & 0 & 64 & QXD0056 &  &  \\ \hline
QXD0007 & Programação Orientada a Objetos / \textit{Object-Oriented Programming} & 32 & 32 & 0 & 0 & 64 & QXD0001 &  &  \\ \hline
<AC> & Atividades Complementares / \textit{Complementary Activities} & 32 & 0 & 0 & 0 & 32 &  &  &  \\ \hline

\multicolumn{10}{|>{\centering\arraybackslash}c|}{\normalsize\textbf{Semestre 3}} \\ \hline
<03> & Agentes Inteligentes / \textit{Intelligent Agents} & 48 & 16 & 0 & 0 & 64 & <02> &  &  \\ \hline
QXD0116 & Álgebra Linear / \textit{Linear Algebra} & 64 & 0 & 0 & 0 & 64 & QXD0056 &  &  \\ \hline
QXD0115 & Estrutura de Dados Avançada / \textit{Advanced Data Structure} & 32 & 32 & 0 & 0 & 64 & QXD0010 &  &  \\ \hline
QXD0011 & Fundamentos de Banco de Dados / \textit{Database Fundamentals} & 32 & 32 & 0 & 0 & 64 & QXD0001 &  &  \\ \hline
QXD0112 & Probabilidade e Estatística / \textit{Probability and Statistics} & 64 & 0 & 0 & 0 & 64 & QXD0056 &  &  \\ \hline
<AE> & Atividades de Extensão / \textit{Extension Activities} & 0 & 0 & 0 & 32 & 32 &  &  &  \\ \hline

\multicolumn{10}{|>{\centering\arraybackslash}c|}{\normalsize\textbf{Semestre 4}} \\ \hline
QXD0019 & Engenharia de Software / \textit{Software Engineering} & 64 & 0 & 0 & 0 & 64 & QXD0007 &  &  \\ \hline
QXD0178 & Mineração de Dados / \textit{Data Mining} & 48 & 16 & 0 & 0 & 64 & QXD0011 &  &  \\ \hline
QXD0041 & Projeto e Análise de Algoritmos / \textit{Design and Analysis of Algorithms} & 32 & 32 & 0 & 0 & 64 & QXD0008, QXD0010 &  &  \\ \hline
<04> & Representação do Conhecimento e Raciocínio / \textit{Knowledge Representation and Reasoning} & 64 & 0 & 0 & 0 & 64 & QXD0017 &  &  \\ \hline
QXD0076 & Sistemas Multiagentes / \textit{Multiagent Systems} & 32 & 32 & 0 & 0 & 64 & <03> &  &  \\ \hline
<AE> & Atividades de Extensão / \textit{Extension Activities} & 0 & 0 & 0 & 32 & 32 &  &  &  \\ \hline

\multicolumn{10}{|>{\centering\arraybackslash}c|}{\normalsize\textbf{Semestre 5}} \\ \hline
<05> & Aprendizado Não Supervisionado / \textit{Unsupervised Learning} & 32 & 32 & 0 & 0 & 64 & QXD0112, QXD0116 &  &  \\ \hline
<06> & Aprendizado Supervisionado / \textit{Supervised Learning} & 32 & 32 & 0 & 0 & 64 & QXD0112, QXD0116 &  &  \\ \hline
<07> & Busca e Otimização / \textit{Search and Optimization} & 32 & 32 & 0 & 0 & 64 & QXD0041 &  &  \\ \hline
<08> & Ética em Inteligência Artificial / \textit{Ethics in Artificial Intelligence} & 64 & 0 & 0 & 0 & 64 & QXD0162 &  &  \\ \hline
<$\ast$> & Optativas / \textit{Elective Courses} & 64 & 0 & 0 & 0 & 64 &  &  &  \\ \hline
<19> & Projeto Extensionista I / \textit{Extension Project I} & 16 & 0 & 0 & 16 & 32 & QXD0178 &  &  \\ \hline
<AE> & Atividades de Extensão / \textit{Extension Activities} & 0 & 0 & 0 & 96 & 96 &  &  &  \\ \hline

\multicolumn{10}{|>{\centering\arraybackslash}c|}{\normalsize\textbf{Semestre 6}} \\ \hline
<09> & Aprendizado Profundo / \textit{Deep Learning} & 32 & 32 & 0 & 0 & 64 & <06> &  &  \\ \hline
<$\ast$> & Optativas / \textit{Elective Courses} & 64 & 0 & 0 & 0 & 64 &  &  &  \\ \hline
<10> & Processamento de Dados em Larga Escala / \textit{Large-Scale Data Processing} & 48 & 16 & 0 & 0 & 64 & QXD0011 &  &  \\ \hline
<11> & Processamento de Linguagem Natural / \textit{Natural Language Processing} & 48 & 16 & 0 & 0 & 64 & <06> &  &  \\ \hline
QXD0182 & Visão Computacional / \textit{Computer Vision} & 48 & 16 & 0 & 0 & 64 & <06> &  &  \\ \hline
<20> & Projeto Extensionista II / \textit{Extension Project II} & 16 & 0 & 0 & 16 & 32 & <19> &  &  \\ \hline
<AE> & Atividades de Extensão / \textit{Extension Activities} & 0 & 0 & 0 & 96 & 96 &  &  &  \\ \hline

\multicolumn{10}{|>{\centering\arraybackslash}c|}{\normalsize\textbf{Semestre 7}} \\ \hline
<12> & Aprendizado por Reforço / \textit{Reinforcement Learning} & 32 & 32 & 0 & 0 & 64 & <09> &  &  \\ \hline
<13> & Inteligência Artificial Generativa / \textit{Generative Artificial Intelligence} & 32 & 32 & 0 & 0 & 64 & <09> &  &  \\ \hline
<14> & MLOPs / \textit{MLOps} & 48 & 16 & 0 & 0 & 64 & <06> &  &  \\ \hline
<$\ast$> & Optativas / \textit{Elective Courses} & 64 & 0 & 0 & 0 & 64 &  &  &  \\ \hline
<15> & Planejamento Automatizado / \textit{Automated Planning} & 48 & 16 & 0 & 0 & 64 & QXD0017 &  &  \\ \hline
QXD0110 & Projeto de Pesquisa Científico-Tecnológico / \textit{Scientific and Technological Research Project} & 16 & 16 & 0 & 0 & 32 & <09> & <17> &  \\ \hline
<17> & Trabalho de Conclusão de Curso I / \textit{Final Project I} & 32 & 0 & 0 & 0 & 32 &  & QXD0110 &  \\ \hline
<AE> & Atividades de Extensão / \textit{Extension Activities} & 0 & 0 & 0 & 32 & 32 &  &  &  \\ \hline

\multicolumn{10}{|>{\centering\arraybackslash}c|}{\normalsize\textbf{Semestre 8}} \\ \hline
<$\ast$> & Optativas / \textit{Elective Courses} & 256 & 0 & 0 & 0 & 256 &  &  &  \\ \hline
<AE> & Atividades de Extensão / \textit{Extension Activities} & 0 & 0 & 0 & 32 & 32 &  &  &  \\ \hline
<16> & Estágio Curricular Supervisionado I / \textit{Supervised Internship} & 0 & 160 & 0 & 0 & 160 & <06> &  &  \\ \hline
<18> & Trabalho de Conclusão de Curso II / \textit{Final Project II} & 64 & 0 & 0 & 0 & 64 & <17> &  &  \\ \hline
\end{tabular}
}
\label{tab:organizacao_curricular}
\end{table}





%Na coluna ``Componente Curricular'', todas as disciplinas nomeadas são obrigatórias, enquanto as disciplinas optativas são representadas pela palavra ``Optativa'' seguida de um número, indicando sua posição na contagem. No curso, recomenda-se que o(a) estudante curse: do primeiro ao terceiro semestre, cinco disciplinas obrigatórias por semestre; no quarto semestre, seis disciplinas obrigatórias; no quinto e sexto semestres, cinco disciplinas obrigatórias e uma optativa; no sétimo semestre, três optativas, além do Estágio Supervisionado I e da primeira parte do TCC; e no oitavo semestre, quatro optativas, o Estágio Supervisionado II e a finalização do TCC. Além disso, ao longo de todo o curso, o(a) estudante deverá cumprir 192 horas de Atividades Complementares e 320 horas de Atividades de Extensão, distribuídas entre disciplinas e a Unidade Curricular Especial de Extensão.



\begin{table}[h]
\centering
\caption{Disciplinas optativas do curso}
\renewcommand{\arraystretch}{1.5}
\resizebox{\textwidth}{!}{%
\small
\begin{tabular}{|
>{\centering\arraybackslash}m{1.5cm}|
>{\centering\arraybackslash}m{10.4cm}|
>{\centering\arraybackslash}m{1.4cm}|
>{\centering\arraybackslash}m{1.4cm}|
>{\centering\arraybackslash}m{1.4cm}|
>{\centering\arraybackslash}m{1.5cm}|
>{\centering\arraybackslash}m{1.4cm}|
>{\centering\arraybackslash}m{1.8cm}|
>{\centering\arraybackslash}m{2.4cm}|
>{\centering\arraybackslash}m{2.4cm}|
}
\hline
% \multicolumn{11}{|>{\centering\arraybackslash}l|}{\normalsize\textbf{Disciplinas optativass}} \\ \hline
\centering{\textbf{Código}} & 
\centering{\textbf{Nome do Componente Curricular}\\ (em português e inglês)} & 
\textbf{Carga Horária Teórica} & 
\textbf{Carga Horária Prática} & 
\textbf{Carga Horária EAD} & 
\textbf{Carga Horária Extensão} & 
\textbf{Carga Horária TOTAL} & 
\textbf{Pré-requisito(s)} & 
\textbf{Correquisito(s)}&  
\textbf{Equivalência(s)} \\ \hline

<21> & Análise Exploratória de Dados / \textit{Exploratory Data Analysis} & 48 & 16 & 0 & 0 & 64 & QXD0112, QXD0178 &  &  \\ \hline
QXD0005 & Arquitetura de Computadores / \textit{Computer Architecture} & 64 & 0 & 0 & 0 & 64 &  &  &  \\ \hline
QXD0134 & Cálculo Diferencial e Integral II / \textit{Calculus II} & 64 & 0 & 0 & 0 & 64 & <01> &  &  \\ \hline
QXD0135 & Cálculo Diferencial e Integral III / \textit{Calculus III} & 64 & 0 & 0 & 0 & 64 & QXD0134 &  &  \\ \hline
QXD0025 & Compiladores / \textit{Compilers} & 32 & 32 & 0 & 0 & 64 & QXD0040 &  &  \\ \hline
QXD0079 & Computação em Nuvem / \textit{Cloud Computing} & 32 & 32 & 0 & 0 & 64 & QXD0011, QXD0021 &  &  \\ \hline
QXD0183 & Computação Paralela / \textit{Parallel Computing} & 48 & 16 & 0 & 0 & 64 & QXD0013, QXD0041 &  &  \\ \hline
QXD0153 & Desafios de Programação / \textit{Programming Challenges} & 32 & 32 & 0 & 0 & 64 & QXD0008, QXD0010 &  &  \\ \hline
QXD0276 & Desenvolvimento de Software para Dispositivos Móveis / \textit{Software Development for Mobile Devices} & 32 & 16 & 0 & 16 & 64 & QXD0007 &  &  \\ \hline
QXD0253 & Desenvolvimento de Software para Web / \textit{Software Development for the Web} & 32 & 16 & 0 & 16 & 64 & QXD0007 &  &  \\ \hline
QXD0232 & Educação Ambiental / \textit{Environmental Education} & 16 & 48 & 0 & 0 & 64 &  &  &  \\ \hline
QXD0245 & Educação em Direitos Humanos / \textit{Human Rights Education} & 64 & 0 & 0 & 0 & 64 &  &  &  \\ \hline
QXD0029 & Empreendedorismo / \textit{Enterpreneurship} & 32 & 0 & 0 & 32 & 64 &  &  &  \\ \hline
QXD0179 & Estatística Multivariada / \textit{Multivariate Statistics} & 48 & 16 & 0 & 0 & 64 & QXD0112, QXD0116 &  &  \\ \hline
QXD0508 & Ética e Legislação / \textit{Ethics and Legislation} & 48 & 0 & 0 & 16 & 64 &  &  &  \\ \hline
QXD0254 & Gerência de Projetos / \textit{Project Management} & 32 & 16 & 0 & 16 & 64 & QXD0019 &  &  \\ \hline
QXD0035 & Inglês Instrumental I / \textit{Instrumental English I} & 64 & 0 & 0 & 0 & 64 &  &  &  \\ \hline
QXD0036 & Inglês Instrumental II / \textit{Instrumental English II} & 64 & 0 & 0 & 0 & 64 & QXD0035 &  &  \\ \hline
<22> & Inteligência Artificial Explicável / \textit{Explainable Artificial Intelligence} & 64 & 0 & 0 & 0 & 64 & <04>, <06> &  &  \\ \hline
QXD0256 & Interação Humano-Computador / \textit{Computer-Human Interaction} & 32 & 16 & 0 & 16 & 64 &  &  &  \\ \hline
QXD0113 & Língua Brasileira de Sinais - LIBRAS / \textit{Brazilian Sign Language - Libras} & 32 & 32 & 0 & 0 & 64 &  &  &  \\ \hline
QXD0016 & Linguagens de Programação / \textit{Programming Languages} & 64 & 0 & 0 & 0 & 64 & QXD0007 &  &  \\ \hline
QXD0040 & Linguagens Formais e Autômatos / \textit{Formal Languages and Automata} & 64 & 0 & 0 & 0 & 64 & QXD0008 &  &  \\ \hline
QXD0120 & Matemática Computacional / \textit{Computational Mathematics} & 48 & 16 & 0 & 0 & 64 & QXD0116 &  &  \\ \hline
<23> & Percepção e Ação Robótica / \textit{Robotic Perception and Action} & 48 & 16 & 0 & 0 & 64 & <03> &  &  \\ \hline
<24> & Pesquisa e Inovação / \textit{Research and Innovation} & 64 & 0 & 0 & 0 & 64 & QXD0162 &  &  \\ \hline
QXD0181 & Pesquisa Operacional / \textit{Operations Research} & 48 & 16 & 0 & 0 & 64 & QXD0116, QXD0041 &  &  \\ \hline
<25> & Processamento de Áudio e Voz / \textit{Audio and Speech Processing} & 48 & 16 & 0 & 0 & 64 & <11> &  &  \\ \hline
QXD0188 & Processamento de Imagens / \textit{Image Processing} & 48 & 16 & 0 & 0 & 64 & QXD0116 &  &  \\ \hline
QXD0114 & Programação Funcional / \textit{Functional Programming} & 32 & 32 & 0 & 0 & 64 &  &  &  \\ \hline
QXD0126 & Psicologia e Percepção / \textit{Psychology and Perception} & 64 & 0 & 0 & 0 & 64 &  &  &  \\ \hline
<26> & Raciocínio sob Incerteza / \textit{Reasoning Under Uncertainty} & 64 & 0 & 0 & 0 & 64 & QXD0017, QXD0112 &  &  \\ \hline
QXD0177 & Recuperação de Informação / \textit{Information Retrieval} & 48 & 16 & 0 & 0 & 64 & QXD0010, QXD0011 &  &  \\ \hline
QXD0021 & Redes de Computadores / \textit{Computer Networks} & 48 & 16 & 0 & 0 & 64 &  &  &  \\ \hline
QXD0246 & Relações Étnico-Raciais e Africanidades / \textit{Ethnic-Racial Relationships and Africanities} & 64 & 0 & 0 & 0 & 64 &  &  &  \\ \hline
QXD0069 & Segurança / \textit{Security} & 32 & 32 & 0 & 0 & 64 & QXD0021 &  &  \\ \hline
QXD0043 & Sistemas Distribuídos / \textit{Distributed Systems} & 32 & 32 & 0 & 0 & 64 & QXD0013 &  &  \\ \hline
QXD0013 & Sistemas Operacionais / \textit{Operating Systems} & 48 & 16 & 0 & 0 & 64 & QXD0005 &  &  \\ \hline
QXD0046 & Teoria da Computação / \textit{Theory of Computation} & 64 & 0 & 0 & 0 & 64 & QXD0040 &  &  \\ \hline
QXD0152 & Teoria dos Grafos / \textit{Graph Theory} & 64 & 0 & 0 & 0 & 64 & QXD0008 &  &  \\ \hline
<27> & Teoria dos Jogos / \textit{Game Theory} & 64 & 0 & 0 & 0 & 64 & QXD0112 &  &  \\ \hline
<28> & Tópicos Especiais em Aprendizado de Máquina / \textit{Special Topics in Machine Learning} & 64 & 0 & 0 & 0 & 64 & <05>, <06> &  &  \\ \hline
<29> & Tópicos Especiais em Ciência de Dados / \textit{Special Topics in Data Science} & 64 & 0 & 0 & 0 & 64 & QXD0178 &  &  \\ \hline
<30> & Tópicos Especiais em Filosofia, Ética e Humanidades / \textit{Special Topics in Philosophy, Ethics and Humanities} & 64 & 0 & 0 & 0 & 64 & <08> &  &  \\ \hline
<31> & Tópicos Especiais em Inteligência Artificial / \textit{Special Topics in Artificial Intelligence} & 64 & 0 & 0 & 0 & 64 & <03> &  &  \\ \hline
<32> & Tópicos Especiais em Matemática Computacional / \textit{Special Topics in Computational Mathematics} & 64 & 0 & 0 & 0 & 64 & QXD0116 &  &  \\ \hline
<33> & Visualização de Dados / \textit{Data Visualization} & 48 & 16 & 0 & 0 & 64 & QXD0178 &  &  \\ \hline

\end{tabular}
}
\label{tab:disciplinas_optativas}
\end{table}

% No âmbito da UFC em Quixadá, é comum a existência de disciplinas semelhantes ofertadas por diferentes cursos, que podem ser cursadas por estudantes de \nomedocurso e aproveitadas conforme sua equivalência registrada.

Um dos princípios norteadores do curso de \nomedocurso é a integração entre teoria e prática. Essa articulação é evidenciada na distribuição da carga horária teórica e prática dos componentes curriculares. As atividades práticas, no âmbito das disciplinas, incluem tanto aulas realizadas em laboratório quanto a execução de trabalhos e projetos experimentais desenvolvidos pelos(as) estudantes. Dessa forma, a contagem das horas teóricas e práticas atribuídas a cada disciplina fornece uma visão concreta de como ocorre essa integração entre teoria e prática ao longo da formação.

%As disciplinas com previsão de atividades em laboratório contam com alocação apropriada em horários e espaços físicos. Além disso, os(as) estudantes têm à sua disposição um laboratório de informática destinado a estudos extraclasse, conforme será detalhado no Capítulo~\ref{cap:infraestrutura-do-curso}. 

\clearpage

A Tabela~\ref{tab:carga_horaria} apresenta a distribuição da carga horária total do curso, abrangendo disciplinas obrigatórias e optativas, bem como atividades complementares e de extensão. %Esses dados são apresentados em créditos e equivalentes em horas. Observa-se que a integralização curricular do curso de \nomedocurso prevê a conclusão de um total de 200 créditos, correspondentes a 3200 horas.

\def\ufcCHObrigatoria{2.144h\xspace}
\def\ufcCHTeorica{1.536h\xspace}
\def\ufcCHPratica{576h\xspace}
\def\ufcCHEAD{0h\xspace}
\def\ufcCHExtensao{32h\xspace}
\def\ufcHorExtensao{288h\xspace}

\begin{table}[h]
\centering
\caption{Distribuição da Carga Horária Total do curso}
\label{tab:carga_horaria}
\renewcommand{\arraystretch}{1.5}
\resizebox{.9\textwidth}{!}{%
\small
\begin{tabular}{|m{4.5cm}|m{7cm}|m{3cm}|}
\hline
\textbf{Tipo do Componente} & \textbf{Componente Curricular} & \textbf{Carga horária} \\ \hline

\multirow{5}{*}{Componentes obrigatórios} 
& Disciplinas obrigatórias & \ufcCHObrigatoria (distribuidas em \ufcCHTeorica teóricas, \ufcCHPratica práticas,  \ufcCHExtensao extensão, \ufcCHEAD EAD)  \\ \cline{2-3}
% & Unidade Curricular Especial de Extensão & \\ \cline{2-3}
& Atividades Complementares &  64h \\ \cline{2-3}
& Atividades Extensionistas & \ufcHorExtensao \\ \cline{2-3}
& Estágio Supervisionado & 160h \\ \cline{2-3}
& Trabalho de Conclusão de Curso & 96h \\ \cline{2-3}
% & Atividades Complementares & \\ 
\hline
\multirow{2}{*}{Componentes optativos}
& Disciplinas optativas & 256h \\ \cline{2-3}
& Disciplinas optativas livres & 192h \\ \hline

\multicolumn{2}{|c|}{\textbf{Total}} & 3200h \\ \hline

\end{tabular}
}
% \vspace{0.5cm}
\end{table}

%Na Universidade Federal do Ceará, cada crédito equivale a 16 horas. O tempo ideal para a conclusão do curso é estimado em 4 (quatro) anos, ou 8 (oito) semestres letivos. Sendo assim, o(a) estudante do curso de \nomedocurso, modalidade Bacharelado, deverá observar o tempo máximo para sua conclusão, estipulado em 6 (seis) anos, ou 12 (doze) semestres letivos.

As cargas horárias mínima, média e máxima por semestre do curso foram definidas seguindo as orientações da Portaria n.º 31/2022, de 20 de abril de 2022. A carga horária mínima é de 192 horas por semestre, correspondendo ao mínimo de 3 disciplinas de 64 horas. A carga horária semestral média corresponde a 400 horas e a carga horária semestral máxima, a 544 horas.

\begin{table}[h]
\centering
\caption{Carga horária por semestre do curso e prazos de conclusão}
\vspace{1em}
\renewcommand{\arraystretch}{1.5}
\resizebox{.75\textwidth}{!}{%
\small
\begin{tabular}{|p{10cm}|c|}
\hline
\textbf{Carga horária por semestre} & \textbf{Horas} \\
\hline
{Carga horária semestral mínima do currículo} & 192h \\
\hline
{Carga horária semestral média do currículo} & 400h \\
\hline
{Carga horária semestral máxima do currículo} & 544h \\ 
\hline
\multicolumn{2}{c}{} \\
\hline
\textbf{Prazos} & \textbf{Semestres} \\
\hline
Mínimo & 7\\
\hline
Médio & 8\\
\hline
Máximo & 12\\
\hline
\end{tabular}
\label{tab:carga_horaria_semestral}
}
\end{table}


\clearpage
\section{Ementário e bibliografias}

As ementas de todos os componentes curriculares que compõem a estrutura curricular do curso, juntamente com as respectivas bibliografias básica e complementar, estão disponíveis no Apêndice \ref{ap:ementario_bibliografia} deste documento. 

Cada bibliografia básica contém pelo menos três títulos, e cada bibliografia complementar inclui pelo menos cinco títulos. As ementas e bibliografias dos componentes curriculares devem revisadas e atualizadas regularmente, levando em consideração os avanços na área de conhecimento de cada componente. Isso visa garantir que o material bibliográfico adotado no curso contribua para a formação definida neste PPC, assegurando abrangência, aprofundamento e coerência teórica.

Conforme detalhado na Seção~\ref{sec:biblioteca}, a Biblioteca do Campus Quixadá (BCQ), em parceria com a Secretaria de Acessibilidade da UFC, oferece serviços especializados, recursos e tecnologia assistiva para atender usuários com deficiência, permitindo-lhes acesso ao material bibliográfico recomendado nas disciplinas. Além disso, todos os títulos físicos indicados possuem exemplares em quantidade adequada no acervo da BCQ, conforme relatório de adequação assinado pelo Núcleo Docente Estruturante (NDE) do curso.
