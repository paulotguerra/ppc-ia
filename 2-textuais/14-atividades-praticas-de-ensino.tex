\chapter{ATIVIDADES PRÁTICAS DE ENSINO / ATIVIDADES DE TUTORIA}
\label{cap:atividades-praticas-de-ensino}

\section{Atividades Práticas de Ensino}

As Atividades Práticas de Ensino (APE) no Bacharelado em Inteligência Artificial têm como objetivo proporcionar aos estudantes experiências concretas de aplicação dos conhecimentos teóricos adquiridos nas disciplinas, fortalecendo a capacidade de resolução de problemas, desenvolvimento de soluções inovadoras e atuação autônoma em contextos reais e simulados.

Essas atividades estão integradas à matriz curricular por meio de componentes como laboratórios, oficinas, projetos integradores, estágios supervisionados, atividades de iniciação científica e desafios de inovação. As APE visam, ainda, estimular o raciocínio crítico e sistêmico, o trabalho em equipe, a comunicação científica e a aprendizagem baseada em projetos.

Dentre as principais formas de APE previstas neste curso, destacam-se:

\begin{itemize}
\item Desenvolvimento de sistemas inteligentes e modelos de aprendizado de máquina, incluindo tarefas como tratamento de dados, experimentação computacional, avaliação de desempenho e interpretação de resultados;
\item Construção de pipelines de engenharia de dados e MLOps, com uso de ferramentas modernas para versionamento, automação e monitoramento de modelos;
\item Elaboração e implantação de APIs com modelos embarcados, utilizando frameworks como FastAPI e Docker;
\item Participação em hackathons, competições e desafios práticos, estimulando a criatividade e o trabalho sob pressão;
\item Projetos interdisciplinares integradores, nos quais os estudantes resolvem problemas reais ou propostos por parceiros externos, utilizando abordagens baseadas em inteligência artificial;
\item Implementação de algoritmos e arquiteturas clássicas e modernas de IA, como redes neurais profundas, \textit{transformers}, sistemas de recomendação e modelos probabilísticos.
\end{itemize}

Essas atividades são registradas, avaliadas e sistematizadas por meio de relatórios técnicos, repositórios de código, apresentações públicas e autoavaliações, permitindo à coordenação pedagógica acompanhar o progresso dos estudantes e garantir a coerência formativa do curso.

\section{Atividades de Tutoria}

As atividades de tutoria desempenham papel fundamental no processo formativo ao promover um acompanhamento mais próximo e individualizado da trajetória dos estudantes, tanto no âmbito acadêmico quanto no desenvolvimento de competências pessoais e profissionais.

A tutoria é exercida por docentes do curso, monitores ou mentores vinculados a programas institucionais, e pode ocorrer em diferentes formatos, como encontros regulares, sessões de orientação, acompanhamento de projetos ou mediação de grupos de estudo.

As principais finalidades das atividades de tutoria incluem:

\begin{itemize}
\item Apoiar o processo de adaptação e permanência dos estudantes no curso, promovendo acolhimento, escuta ativa e orientação acadêmica;
\item Orientar a organização dos estudos e o aproveitamento de oportunidades de aprendizagem, como iniciação científica, estágios, eventos e cursos complementares;
\item Acompanhar o desenvolvimento de projetos práticos, TCCs e atividades de extensão, contribuindo com feedback técnico e metodológico;
\item Estimular a reflexão sobre o papel social da inteligência artificial, por meio de discussões sobre ética, privacidade, justiça algorítmica e impactos sociais das tecnologias emergentes;
\item Fornecer suporte na construção do itinerário formativo e profissional, orientando quanto às trilhas de carreira possíveis no campo da inteligência artificial, bem como à preparação de portfólios, currículos e perfis profissionais.
\end{itemize}

As atividades de tutoria complementam as dimensões técnica e acadêmica do curso, contribuindo para a formação integral dos estudantes e para o alinhamento das ações pedagógicas com os objetivos formativos do bacharelado.
