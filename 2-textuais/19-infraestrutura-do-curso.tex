\chapter{INFRAESTRUTURA DO CURSO}
\label{cap:infraestrutura-do-curso}

%INFRAESTRUTURA
\def\ufcNumEquipEletro{266\xspace}
\def\ufcNumNotebooks{28\xspace}
\def\ufcNumProjetores{33\xspace}
\def\ufcNumDesktops{213\xspace}
\def\ufcNumServidores{6\xspace}
\def\ufcNumVoip{16\xspace}
\def\ufcNumNobreaks{3\xspace}
\def\ufcNumLab{sete\xspace}
\def\ufcNumGabinetes{35\xspace}
% RECURSOS HUMANOS
\def\ufcNumDocentes{59\xspace}
\def\ufcNumDocSubstitutos{1\xspace}
\def\ufcNumDocQuarentaHoras{1\xspace}
\def\ufcNumDocDE{57\xspace}
\def\ufcDocMestres{14\xspace}
\def\ufcDocDoutores{45\xspace}
\def\ufcDocDoutorandos{7\xspace}
\def\ufcDocAfasDoutorado{1\xspace}
\def\ufcNumSTA{29\xspace}
\def\ufcNumSTAMedio{13\xspace}
\def\ufcNumSTASuperior{16\xspace}
\def\ufcSTAEscMedio{2\xspace}
\def\ufcSTAEscGrad{10\xspace}
\def\ufcSTAEscEspec{12\xspace}
\def\ufcSTAEscMestre{5\xspace}
%BIBLIOTECA
\def\ufcBCQAnoReferencia{2025\xspace}
\def\ufcBCQTitulos{1.425\xspace}
\def\ufcBCQExemplares{9.524\xspace}
\def\ufcBCQDocsRepositorio{74 mil\xspace}


% Para o devido aprendizado do corpo discente e de modo a permitir espaços e situações de experimentação, os docentes e discentes dispõem de toda a infraestrutura do campus da UFC de Quixadá, conforme será visto a seguir.

O Campus da UFC em Quixadá conta com uma estrutura física que contempla um estacionamento com vagas para carros de passeio (algumas reservadas para pessoas com deficiência), vagas para motocicletas, vagas para ônibus e bicicletário; espaço de convivência que acomoda a cantina e salas de centro e diretório acadêmico; cinco blocos, sendo quatro blocos didáticos medindo 1.400 m$^2$, cada um, e um bloco administrativo medindo 1.500 m$^2$. Todos os blocos contam com dois pavimentos, térreo e superior. Portanto, o Campus da UFC em Quixadá conta hoje com a infraestrutura arquitetônica descrita abaixo:%, excluindo-se os blocos ainda em construção:

\begin{itemize}
    \item Bloco I (1400 m$^2$): salas de aula, salas de projetos (PET), salas para serviços (secretaria acadêmica, atendimento nutricional, videoconferência), banheiros, gabinetes para professores, laboratórios, copa, plataforma elevatória.

    \item Bloco II (1400 m$^2$): salas de aula, salas de projetos (empresa júnior), salas de estudo em grupo, biblioteca, salão multiuso, banheiros, gabinetes para professores, laboratórios, copa, plataforma elevatória.

    \item Bloco III (1400 m$^2$): salas de aula, salas de desenho, ateliê, salas de estudo em grupo, laboratórios, banheiros, gabinetes para professores, copa, plataforma elevatória.

    \item Bloco IV (1400 m$^2$): salas de aula, salas de projetos (NPI), salas de estudo em grupo, banheiros, gabinetes para professores, laboratórios, copa, plataforma elevatória.

    \item Bloco Administrativo (1500 m$^2$): refeitório universitário, salas para serviços acadêmicos e administrativos (venda de tickets, direção, secretaria da direção, prefeitura, coordenação de programas acadêmicos, núcleo de TIC), salas das coordenações dos cursos de graduação, sala de reuniões, sala do núcleo de inovação, laboratório de redes de alta velocidade, salas de atendimento do NAS (nutrição, serviço social, serviço de psicologia), copa, banheiros.

    \item Área de Convivência (500 m$^2$): cantina, salas de centros e diretórios acadêmicos, área comum coberta, banheiros.
\end{itemize}

O campus conta com \ufcNumEquipEletro equipamentos eletrônicos para atender a demanda interna, sendo estes: \ufcNumNotebooks notebooks; \ufcNumProjetores projetores; \ufcNumDesktops desktops; \ufcNumServidores máquinas servidoras para telemática; \ufcNumVoip aparelhos telefônicos VOIPs; \ufcNumNobreaks nobreaks. Os equipamentos de TI estão distribuídos em salas administrativas e nos laboratórios apresentados adiante.

As seções a seguir descrevem os vários ambientes e condições estruturais que contribuem diretamente para a formação do aluno de \nomedocurso.

\section{SALAS DE AULA}

O campus disponibiliza quatorze salas de aula, distribuídas nos quatro blocos didáticos, onde ocorrem prioritariamente as aulas teóricas. Algumas salas comportam confortavelmente até 60 alunos e outras 30, de maneira a atender às necessidades institucionais e do curso de \nomedocurso. As salas de aula passam por limpeza diária e manutenção periódica, além da disponibilidade de recursos tecnológicos adequados às atividades desenvolvidas: a maior parte dispõe de projetores digitais instalados e duas delas possuem sistema de som integrado. As cadeiras podem ser dispostas em diferentes configurações, oportunizando distintas situações de ensino-aprendizagem. Outras atividades didáticas podem acontecer na sala multiuso, sala de videoconferência, além de nos laboratórios, especificados a seguir.

\section{LABORATÓRIOS DIDÁTICOS}

O Campus da UFC em Quixadá oferece estrutura de laboratórios compreendendo principalmente o uso de ferramentas computacionais. No total, há \ufcNumLab laboratórios de computadores para uso didático, todos com projetores digitais. Um laboratório é utilizado como estrutura permanente de apoio discente, no qual os alunos podem realizar estudos, pesquisas e desenvolvimento de trabalhos em horários extraclasse, não havendo aulas alocadas neste espaço. Os demais laboratórios são destinados às aulas e possuem capacidade variando entre 25 e 30 computadores, podendo assim comportar turmas de até 60 alunos, com dois alunos por computador. Todos os laboratórios possuem acesso à rede Wifi. 

Os computadores são disponibilizados com acesso à internet, além de ferramentas de criação e execução apropriadas para os ambientes de aprendizado, contando com softwares open source condizentes com as práticas efetivadas no mercado de trabalho e destinados a procedimentos como edição de imagem e som, edições vetoriais, diagramação, modelagem tridimensional, arquitetura da informação, prototipagem de telas, ambientes de desenvolvimento e programação, além de pacotes de escritório. As estações de trabalho nos laboratórios são configuradas em dual boot, executando Windows e Linux como sistemas operacionais, de modo que alunos e professores possam lidar com ambientes de acordo com as necessidades de aprendizado.

A manutenção dos laboratórios é feita em dois sentidos: estruturalmente, o controle é realizado por meio de levantamento patrimonial da Universidade, identificando o estado de conservação e local onde se encontra cada unidade registrada. Problemas relacionados ao desgaste físico de materiais são solucionados pontualmente, sob demanda. Já em relação à manutenção do software, anualmente os computadores são formatados e os programas necessários são instalados de acordo com a requisição dos docentes, de modo a manter as máquinas rápidas e atualizadas com as versões mais recentes de sistemas operacionais e programas requeridos.

\section{ESTRUTURAS DE HOSPEDAGEM E COMPUTAÇÃO EM NUVEM}

O campus conta com serviços de hospedagem de subdomínio, para o caso de estudos e projetos que necessitem de páginas web como forma de divulgação ou experimentação. Professores e alunos contam com apoio do Núcleo de Tecnologia da Informação e Comunicação (NTIC) do campus para a configuração dos recursos necessários a essas estruturas e acesso simplificado dentro e fora do campus.

O campus de Quixadá está credenciado no programa AWS Academy, parceria com a Amazon Web Services. Os professores cadastrados podem criar laboratórios virtuais nos quais os alunos têm acesso aos principais serviços da nuvem AWS. Professores e alunos contam com apoio do Núcleo de Tecnologia da Informação e Comunicação (NTIC) do campus para a configuração dos recursos necessários a essas estruturas e acesso simplificado dentro e fora do campus.

\section{ACESSO À INTERNET}

Todos os espaços do Campus da UFC em Quixadá contam com acesso à Internet sem fio (Wifi), estando o campus conectado por um link de 200 Mbps ao Cinturão Digital do Ceará (CDC) e, através deste, com a Rede Nacional de Pesquisa e à Internet. Isso significa que há redes disponíveis nas salas de aula, nos laboratórios, na biblioteca e em demais partes do campus, com acesso livre e descomplicado.

A comunidade acadêmica pode se conectar por meio da rede Eduroam, uma rede internacional segura desenvolvida para a comunidade de ensino e pesquisa. Com ela, estudantes, docentes e técnico-administrativos podem se conectar à internet utilizando suas credenciais institucionais, tanto no campus da UFC quanto em qualquer outra instituição participante da iniciativa, no Brasil e no exterior.

\section{ESPAÇOS DE TRABALHO}

Os docentes do campus desfrutam de vários ambientes para desenvolverem seu trabalho. Além dos espaços didáticos (salas de aula e laboratórios), o campus disponibiliza para os docentes uma sala climatizada, na qual eles podem executar ações acadêmicas extraclasse (planejamento, acompanhamento, orientação, pesquisa). Ao todo, o campus possui \ufcNumGabinetes gabinetes que comportam até dois professores cada. Nos gabinetes, o espaço individual de trabalho de cada docente conta com mesa, cadeira para si e para alunos em atendimento, armário com chave, garantindo privacidade para uso dos recursos, para o atendimento a discentes e orientandos, e para a guarda de material e equipamentos pessoais, com segurança.

Para cada um dos cursos de graduação do campus, há uma sala da coordenação ampla e climatizada, onde trabalham o Coordenador e o Vice-coordenador. Nas coordenações, dispõe-se de mesas, cadeiras para os docentes e para os discentes, armários com chave e telefones VOIP. O ambiente da coordenação possibilita o atendimento individual e em pequenos grupos.

Além das salas de trabalho, os coordenadores e docentes podem usufruir de outros ambientes de apoio às ações acadêmicas como sala de seminários, sala de reunião, sala de videoconferência e sala multiuso. Eles também têm acesso a equipamentos como projetores e notebooks.

\section{BIBLIOTECA}
\label{sec:biblioteca}

A Biblioteca Universitária (BU) é um órgão subordinado à Reitoria, a qual compete prover a UFC de um sistema central de informação, de forma a proporcionar serviços biblioteconômicos e documentais eficientes que possibilitem o desenvolvimento das atividades de ensino, pesquisa e extensão. Este é o órgão responsável por gerenciar todas as bibliotecas da UFC, mantendo acervos especializados tombados e catalogados que visam atender a demanda da comunidade acadêmica em geral. Seus serviços são direcionados ao atendimento de alunos, docentes, servidores técnico-administrativos e pesquisadores. Em 2025, o Sistema de Bibliotecas da UFC conta com 29 bibliotecas distribuídas entre seus campi em Fortaleza e no interior do estado.

A Biblioteca Universitária tem em sua estrutura vários setores (diretorias, secretarias, divisões e seções), com atribuições diversas. Em relação ao registro, catalogação e manutenção do acervo, são executadas ações como: a) classificar o material bibliográfico e documental; b) catalogar, de acordo com as normas vigentes, os conteúdos informacionais do Sistema de Bibliotecas da UFC; c) desenvolver ações educativas voltadas para a sensibilização dos usuários em relação à necessidade de preservação do acervo; d) executar as ações de preservação, conservação preventiva, reparadora e restauração dos acervos documentais e bibliográficos.

Além de cuidar do acervo bibliográfico em si, a BU também auxilia no processo de acompanhamento das bibliografias básicas e complementares dos componentes curriculares dos cursos de graduação da UFC. Neste sentido, são atribuições de setores da BU, por exemplo:
\begin{enumerate}
    \item organizar, manter e atualizar na biblioteca os arquivos dos planos de ensino dos cursos de graduação; e b) analisar as bibliografias do plano de ensino conforme os critérios do Instrumento de Avaliação de Cursos de Graduação Presencial e a Distância do MEC. Este é um dos esforços da biblioteca para, juntamente com a Coordenação, o Colegiado e as UC do curso, manter o acervo das bibliografias básica e complementar adequado e atualizado em relação aos componentes curriculares e aos conteúdos previstos no PPC do curso.
    \item atualização do material bibliográfico e documental através de compras. Para isso, ações como as listadas abaixo são executadas: a) receber e preparar a solicitação de compra de material bibliográfico e documental, mediante indicação das sugestões do corpo docente, discente e técnico-administrativo em educação; b) elaborar o processo de licitação do material bibliográfico; c) acompanhar o processo de compra e recebimento de material bibliográfico.
    \item atividades de gestão que auxiliem os processos de tomada de decisão como: a) coordenar e controlar os relatórios patrimoniais do material bibliográfico do Sistema de Bibliotecas da UFC; b) coordenar o inventário anual do material bibliográfico e documental;
    \item coletar e analisar os dados gerados a partir dos relatórios automatizados, mantendo estatísticas que subsidiem estudos na área.    
\end{enumerate}

Além das atribuições próprias da BU, a administração de cada Biblioteca Setorial, como é o caso da Biblioteca do Campus de Quixadá (BCQ), tem em sua estrutura a Diretoria Setorial, a Seção de Representação Descritiva e Temática da Informação, a Seção de Atendimento ao Usuário, a Seção de Preservação, Conservação e Restauração do Acervo, a Seção de Coleções Especiais e a Seção de Atendimento às Pessoas com Deficiência. Algumas funções básicas de gestão do acervo desempenhadas por algumas das Seções mencionadas são:
\begin{itemize}
    \item Catalogar e classificar, de acordo com as normas vigentes, todo o conteúdo informacional pertinente a área de atuação da biblioteca;
    \item Realizar estudo da bibliografia adotada nos cursos atendidos pela biblioteca setorial e propor a aquisição do material bibliográfico que preencha os requisitos necessários ao pleno desenvolvimento das disciplinas ofertadas;
    \item Acompanhar os relatórios mensais de aquisição de material bibliográfico;
    \item Manter atualizado os repositórios locais, nacionais e internacionais;
    \item Supervisionar os serviços de atendimento ao usuário no que diz respeito à circulação de material bibliográfico;
    \item Orientar os usuários na busca de informações e no uso dos acervos existentes nas Bibliotecas do sistema e fora delas, auxiliando-os em suas necessidades de estudo e pesquisa;
    \item Localizar e fornecer documentos e informações solicitadas;
    \item Desenvolver ações educativas voltadas para a sensibilização dos usuários em relação à necessidade de preservação do acervo;
    \item Zelar pela conservação e funcionamento de equipamentos, máquinas e aparelhos da Seção.
\end{itemize}

%Em relação ao processo de aquisição de material bibliográfico, o campus dispõe de um sistema específico (Gestão de Aquisição de Livros – GAL), para auxiliar no processo de compras. Neste sistema, todas as disciplinas e toda a bibliografia de todos os cursos são cadastrados. Registra-se também os títulos que se deseja para a compra. Baseado nesses dados, o sistema compara a quantidade de exemplares existente no acervo com a quantidade necessária, sugerindo assim a quantidade de exemplares a ser adquirida. O GAL é mais um sistema inovador, desenvolvido no campus, para auxiliar os processos administrativos e acadêmicos dos cursos que aqui estão.

Além destas funções administrativas, a Biblioteca Universitária realiza, periodicamente ou sob demanda, cursos e treinamentos de capacitação no uso otimizado dos recursos informacionais para alunos, professores e técnico-administrativos\footnote{\url{http://www.biblioteca.ufc.br/servicos-e-produtos/cursos-e-treinamentos/}}. Alguns treinamentos específicos fazem parte do calendário oficial da BU: Normalização de Trabalhos Acadêmicos, Referências e Citações, Treinamento em Bases de Dados. Além destes, a Biblioteca promove o projeto “Descobrindo a Biblioteca”, que acontece sempre no início de cada semestre letivo e visa apresentar aos novos alunos os serviços ofertados pelo Sistema de Bibliotecas da UFC. São abordados conteúdos como: regulamento, guia de serviços, acervo, catálogo on-line, meios de acesso do usuário, livros eletrônicos, eventos da Biblioteca Universitária, dentre outros. A iniciativa faz parte da programação de recepção dos recém-ingressos, que tradicionalmente inclui cursos e palestras em suas unidades acadêmicas.

Especificamente, em Quixadá, além de todas as iniciativas listadas acima, desde 2013, a Biblioteca do campus promove a “Maratona do Conhecimento do Sertão Central”, que atualmente é ocorre durante os Encontros Universitários da UFC. O objetivo do evento é difundir ferramentas e fontes de informação acadêmica, promovendo assim o desenvolvimento de habilidades relacionadas à busca, acesso e utilização de informações para construção do conhecimento.

O curso de \nomedocurso dispõe de acervo atualizado, contemplando títulos adotados como bibliografia básica e complementar devidamente indicados nas ementas das disciplinas. Além dos livros físicos, a biblioteca do campus oferece acesso a plataformas online nas quais se encontram livros eletrônicos e artigos de periódicos voltados às principais temáticas abordadas em sala de aula.

Em \ufcBCQAnoReferencia, a BCQ conta com um acervo de \ufcBCQExemplares exemplares físicos, correspondentes a \ufcBCQTitulos títulos, principalmente nas seguintes áreas do conhecimento: Ciências Exatas e da Terra, Engenharias, Ciências Sociais Aplicadas, Ciências Humanas, Linguística, Letras e Artes. Uma descrição mais detalhada das estruturas de conteúdo eletrônico e físico é trazida nas seções a seguir.

Em relação à infraestrutura física, a BCQ conta com salão de estudo climatizado e computadores disponíveis para consulta ao catálogo eletrônico. Além disso, a biblioteca é responsável por gerenciar a ocupação das salas de estudo em grupo disponíveis aos discentes. Vê-se, portanto, que os eventos e treinamentos promovidos, as ações de conscientização e a infraestrutura de acesso aos títulos bibliográficos e de acomodações físicas são exemplos de soluções de apoio à leitura, estudo e aprendizagem disponibilizados pela biblioteca aos discentes e demais membros da comunidade acadêmica.

\subsection{ACERVOS DIGITAIS}

Todo o acervo físico do Sistema de Bibliotecas da UFC está catalogado e disponível digitalmente para a comunidade acadêmica. Através do sistema Pergamum\footnote{\url{https://pergamum.ufc.br/pergamum/biblioteca/index.php}}, os usuários acessam o catálogo online do Sistema de Bibliotecas da UFC, ou seja, os registros de todo o acervo das bibliotecas, inclusive documentos eletrônicos em texto completo, tais como: livros, teses e dissertações, monografias, periódicos, artigos, obras raras e CDs/DVDs, dentre outros.

Além do acervo físico, os usuários das bibliotecas da UFC têm acesso a um enorme conjunto de material digital, conforme descrito a seguir.

\subsection{PORTAL DE ACESSO A CONTEÚDO CIENTÍFICO DIGITAL}

A UFC oferece materiais de estudo por meio de acesso à plataforma Minha Biblioteca\footnote{\url{https://dliportal.zbra.com.br/Login.aspx?key=UFC}} e à coleção da editora Springer\footnote{\url{http://ufc.dotlib.com.br/}}. O acesso interno é feito de qualquer dos campi da UFC. Para acesso remoto, alunos e professores podem realizar configurações de Proxy, por meio de CPF e senha do sistema acadêmico SIGAA/SIGPRH. Assim, todos os alunos matriculados possuem acesso eletrônico a livros e artigos disponibilizados nessas plataformas que contam com mais de 14 mil títulos em texto completo, que podem ser pesquisados pelos usuários no ambiente da UFC ou de qualquer ponto de Internet, sem limites de utilização, visualização ou restrições de usuários simultâneos.

\subsection{PERIÓDICOS}

A biblioteca conta com acesso online às 27 revistas de publicação própria da UFC, por meio do Portal de Periódicos da UFC\footnote{\url{http://periodicos.ufc.br/}}. Além disso, professores e alunos contam com acesso ao Portal Periódicos da CAPES\footnote{\url{https://www.periodicos.capes.gov.br/}}, que disponibiliza documentos periódicos (internacionais e nacionais), livros eletrônicos, bases de dados contendo artigos, referências e resumos de trabalhos acadêmicos e científicos, normas técnicas, patentes, teses e dissertações e outros tipos de materiais, em todas as áreas do conhecimento, totalizando mais de 37 mil títulos. O acesso interno é liberado automaticamente por meio de faixa de IPs da UFC, já o acesso remoto é liberado por meio da Comunidade Acadêmica Federada (CAFe).

\subsection{REPOSITÓRIO INSTITUCIONAL}

Finalmente, há o Repositório Institucional (RI)\footnote{\url{https://repositorio.ufc.br/}} da UFC, de acesso aberto via Internet, que tem como propósito reunir, armazenar, organizar, recuperar, preservar e disseminar a produção científica e intelectual da comunidade universitária (docentes, pesquisadores, técnicos e alunos de pós-graduação stricto sensu) pertencente à UFC. Em \ufcBCQAnoReferencia, este repositório conta com mais de \ufcBCQDocsRepositorio documentos, dentre estes, artigos de periódicos, dissertações, teses, capítulos de livros, artigos publicados em eventos, além das monografias e trabalhos de conclusão dos cursos de graduação da UFC. Dessa forma, o corpo discente tem acesso imediato a produções de alunos situados em outros campi da UFC.

\subsection{OUTROS RECURSOS DIGITAIS}

Além das fontes de bibliografia digital destacadas acima, a Biblioteca Universitária disponibiliza ainda os seguintes recursos:
\begin{itemize}
    \item Portal de Livros Eletrônicos UFC\footnote{\url{http://livros.ufc.br/ojs/}}: acesso aos Livros da Coleção de Estudos da Pós-Graduação da Universidade. Acesso livre.
    \item Coleção de Normas ABNT\footnote{\url{https://www.gedweb.com.br/UFC/}}: acesso a mais de 9 mil Normas Técnicas Brasileiras e Normas Técnicas do Mercosul em texto completo. Acesso remoto via proxy.
\end{itemize}

\subsection{BIBLIOTECA ACESSÍVEL}

A fim de proporcionar ambientes de estudo adequados e um maior acesso à informação aos usuários com deficiência, o Sistema de Bibliotecas da UFC oferece um atendimento pautado na prestação de serviços especializados, na aquisição de equipamentos e tecnologias assistivas desenvolvidas especialmente para esses usuários e na inserção da acessibilidade arquitetônica em suas edificações.

Neste contexto, os alunos com necessidades especiais podem contar com os seguintes especializados, oferecidos em parceria com a Secretaria de Acessibilidade da UFC:
\begin{itemize}
    \item Digitalização e conversão de materiais bibliográficos em formatos acessíveis: neste serviço a bibliografia solicitada pelo professor passa pelo processo de digitalização e/ou edição e é convertida em arquivo digital acessível para posteriormente ser disponibilizada no catálogo da Biblioteca Universitária com acesso restrito aos usuários com deficiência visual.
    \item Orientação à pesquisa bibliográfica para usuários com deficiência visual: serviço realizado mediante treinamentos de uso das bases de dados on-line do Portal da CAPES, Biblioteca Digital de Teses e Dissertações (BDTD) e livros eletrônicos com o auxílio de softwares leitores de tela.
    \item Levantamento bibliográfico para usuários com deficiência visual: o serviço consiste na pesquisa bibliográfica demandada pelos usuários com deficiência visual, transformada posteriormente em arquivo digital acessível para seu uso exclusivo.
\end{itemize}

Em relação a recursos e tecnologias assistivas, a Biblioteca Universitária disponibiliza: a) nos terminais de consulta online, programas leitores de telas; b) recurso de transcrição de textos em Braille, com impressoras Braille à disposição dos alunos (em Fortaleza); c) recurso de acessibilidade linguística, através do projeto “Descobrindo a Biblioteca em Libras”, com a apresentação dos serviços oferecidos pelas bibliotecas para a comunidade acadêmica por uma intérprete da Língua Brasileira de Sinais (Libras).

Especificamente na BCQ, os alunos com deficiência visual têm acesso a um equipamento para leitura autônoma, instalado na biblioteca. Em relação à acessibilidade arquitetônica, a biblioteca tem portas de acesso largas, balcões e mesas de estudo rebaixados, espaço de circulação amplo entre prateleiras, placas com inscrições em Braille e piso tátil.

\section{ACESSIBILIDADE FÍSICA}

O campus possui em sua infraestrutura a facilitação da acessibilidade a pessoas com dificuldades de locomoção ou de visão, contando com plataformas elevatórias, portas largas de acesso às salas e laboratórios, banheiros com cabines específicas para deficientes e plaquetas com inscrições em braille, além de rampas de acesso ou mesmo ausência de degraus desde o estacionamento até todos os ambientes térreos.

\section{RECURSOS HUMANOS}

\def\ufc{\xspace}
\def\ufc{\xspace}

O corpo docente do Campus da UFC em Quixadá, em \ufcAnoReferencia, é formado por \ufcNumDocentes professores, sendo \ufcNumDocSubstitutos substituto, \ufcNumDocQuarentaHoras efetivo em regime de 40h e \ufcNumDocDE efetivos em regime de dedicação exclusiva. Dos efetivos, \ufcDocDoutores são doutores e \ufcDocMestres são mestres. Dos \ufcDocMestres docentes mestres, \ufcDocDoutorandos são doutorandos.\footnote{A lista completa de docentes pode ser encontrada em \url{https://www.quixada.ufc.br/docente/}.}

Em relação ao estímulo à titulação dos docentes, foi aprovada a Resolução nº 01/Conselho do Campus de Quixadá, de 15 de maio de 2014, para disciplinar os afastamentos de docentes para mestrado, doutorado e pós-doutorado. Salienta-se que foi desenvolvido no campus o Sistema de Afastamento (SIAF), criado para gerenciar o afastamento dos docentes. Neste sistema, os docentes fazem uma solicitação de reserva de afastamento e podem acompanhar, em tempo real, sua posição em um ranking construído a partir das regras de afastamento definidas na Resolução. Em \ufcAnoReferencia, há \ufcDocAfasDoutorado professor afastado para o doutorado.

Quanto ao quadro de servidores técnico-administrativos, o campus dispõe, em \ufcAnoReferencia, de \ufcNumSTA servidores. Destes, \ufcNumSTAMedio são cargos de nível médio e \ufcNumSTASuperior de nível superior. Em relação à escolaridade, \ufcSTAEscMedio têm o nível médio, \ufcSTAEscGrad possuem graduação completa, \ufcSTAEscEspec são especialistas e \ufcSTAEscMestre são mestres, demonstrando a qualidade do corpo técnico-administrativo do campus.\footnote{A lista completa dos servidores técnicos-administrativos pode ser encontrada em \url{https://www.quixada.ufc.br/sta/}}

\section{COMITÊ DE ÉTICA EM PESQUISA}

O Comitê de Ética em Pesquisa da UFC (CEP/UFC/PROPESQ), vinculado à Pró-reitoria de Pesquisa e Pós-Graduação, foi instituído em 20 de outubro de 2005. É credenciado junto à Comissão Nacional de Ética em Pesquisa (CONEP) do Ministério da Saúde e constitui um colegiado interdisciplinar, independente e normativo, com ``munus público'', sem fim lucrativo, de caráter consultivo, deliberativo e educativo, criado para defender os interesses dos participantes da pesquisa em sua integridade e dignidade, que obedece aos princípios da Bioética: autonomia, não maleficência, beneficência e justiça, e visa contribuir no desenvolvimento da pesquisa dentro de padrões éticos (Normas e Diretrizes Regulamentadoras da Pesquisa Envolvendo Seres Humanos – Resolução CNS 466/12, II.4). O CEP é responsável pela avaliação e acompanhamento dos aspectos éticos de todas as pesquisas envolvendo seres humanos. No campus, a Coordenadoria de Pesquisa, além de incentivar, apoiar e acompanhar o desenvolvimento das atividades de pesquisa, apoia os contatos entre a comunidade acadêmica e o CEP/UFC/PROPESQ.